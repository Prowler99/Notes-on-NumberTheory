\documentclass{article}

\documentclass{article}
\usepackage{amsmath, amssymb, amsthm, amsbsy, mathrsfs, stmaryrd}
\usepackage{enumitem}
\usepackage[colorlinks,
linkcolor=cyan,
anchorcolor=blue,
citecolor=blue,
]{hyperref}
\usepackage[capitalize]{cleveref}
\usepackage[margin = 1in, headheight = 12pt]{geometry}
\usepackage{bbm}
\usepackage{tikz-cd}

\newtheorem{theorem}{Theorem}

\theoremstyle{definition}
\newtheorem{definition}{Definition}
\newtheorem{exercise}{Exercise}[section]
\newtheorem{problem}{Problem}
\newtheorem{example}{Example}
\newtheorem{proposition}{Proposition}[section]
\newtheorem{lemma}{Lemma}[section]
\newtheorem{corollary}{Corollary}[section]

\theoremstyle{remark}
\newtheorem*{remark}{Remark}

\renewcommand{\Re}{\mathop{\mathrm{Re}}}
\renewcommand{\Im}{\mathop{\mathrm{Im}}}

% 新命令
% 数学对象
    \newcommand{\R}{\mathbb{R}}
    \newcommand{\C}{\mathbb{C}}
    \newcommand{\Q}{\mathbb{Q}}
    \newcommand{\Z}{\mathbb{Z}}
    \DeclareMathOperator{\GL}{GL}
    \DeclareMathOperator{\SL}{SL}
    \newcommand{\p}{\mathfrak{p}}
    \renewcommand{\P}{\mathbb{P}}
    \newcommand{\A}{\mathbb{A}}
% 集合
    \newcommand{\sminus}{\smallsetminus} %(集合)差
% 范畴
    \newcommand{\op}[1]{{#1}^{\mathrm{op}}} %反范畴
    \DeclareMathOperator{\enom}{End} %自态射
    \DeclareMathOperator{\isom}{Isom} %同构
    \DeclareMathOperator{\aut}{Aut} %自同构
    \DeclareMathOperator{\im}{im} %像
    \newcommand{\Set}{\mathbf{Set}} %集合范畴
    \newcommand{\Abel}{\mathbf{Ab}} %群范畴
    \newcommand{\Ring}{\mathbf{Ring}}
    \newcommand{\Cring}{\mathbf{CRing}}
    \newcommand{\Alg}{\mathbf{Alg}}
    \newcommand{\Mod}{\mathbf{Mod}}
    \DeclareMathOperator{\Id}{id}
%向量空间, 矩阵
    \DeclareMathOperator{\rank}{rank} %秩
    \DeclareMathOperator{\tr}{Tr} %迹
    \newcommand{\tran}[1]{{#1}^{\mathrm{T}}} %转置
    \newcommand{\ctran}[1]{{#1}^{\dagger}} %共轭转置
    \newcommand{\itran}[1]{{#1}^{-\mathrm{T}}} %逆转置
    \newcommand{\ictran}[1]{{#1}^{-\dagger}} %逆共轭转置
    \DeclareMathOperator{\codim}{codim} %余维数
    \DeclareMathOperator{\diag}{diag} %对角阵
    \newcommand{\norm}[1]{\left\| #1\right\|} %范数
    \DeclareMathOperator{\lspan}{span} %张成
    \DeclareMathOperator{\sym}{\mathfrak{Y}}
% 群
    \DeclareMathOperator{\inn}{Inn} %(群)内自同构
    \newcommand{\nsg}{\vartriangleleft} %正规子群
    \newcommand{\gsn}{\vartriangleright} %正规子群
    \DeclareMathOperator{\ord}{ord} %元素的阶
    \DeclareMathOperator{\stab}{Stab} %稳定化子
    \DeclareMathOperator{\sgn}{sgn} %符号函数
% 环, 域
    \DeclareMathOperator{\cha}{char} %特征
    \DeclareMathOperator{\spec}{Spec} %素谱
    \DeclareMathOperator{\maxspec}{MaxSpec} %极大谱
    \DeclareMathOperator{\gal}{Gal}
% 微积分
    % \newcommand*{\dif}{\mathop{}\!\mathrm{d}} %(外)微分算子
% 流形
    \DeclareMathOperator{\lie}{Lie}
%代数几何
    \DeclareMathOperator{\proj}{Proj}
%多项式
    \DeclareMathOperator{\disc}{disc} %判别式
    \DeclareMathOperator{\res}{res} %结式

% 结构简写
    \newcommand{\pdfrac}[2]{\dfrac{\partial #1}{\partial #2}} %偏微分式
    \newcommand{\isomto}{\stackrel{\sim}{\rightarrow}} %有向同构
    \newcommand{\gene}[1]{\left\langle #1 \right\rangle} %生成对象
% 文字缩写
    \newcommand{\opin}{\;\mathrm{open\;in}\;}
    \newcommand{\st}{\;\mathrm{s.t.}\;}
    \newcommand{\ie}{\;\mathrm{i.e.,}\;}

% 重定义命令
\renewcommand{\hom}{\mathop{Hom}}
\renewcommand{\vec}{\boldsymbol}
\renewcommand{\and}{\;\text{and}\;}

% 编号
\newcommand{\cnum}[1]{$#1^\circ$} %右上角带圆圈的编号
\newcommand{\rmnum}[1]{\romannumeral #1}


\newcommand{\myit}{$\diamond$}

\title{}
\author{}
\date{}

\begin{document}
\maketitle

\end{document}

\newcommand{\F}{\mathbb{F}}

\title{Notes on Drinfeld Modules and Explicit CFT for Function Fields}

\begin{document}
\maketitle

1) Give a 30min (strict limit !!!) talk. Ideally more like 25min + 5 min for questions.  The talks will be in March. I will try to reserve a room, and will give a more precise time/date when possible.

2) Write an ``extended summary" (meaning around ~5 pages NOT!!! >=10) of you article. It should summarise the article and its main ideas and be accessible to advanced Master students (i.e., the other students in this group).

\section{Review on CFT}



\section{Drinfeld Modules}
Let $F$ be a global function field with field of constants $k = \F_q$.
\subsection{Definition}
Consider the additive group $\Ga_{/L}$ over $L$.
Now the point is, $\Ga_{/L}$ is not only a group scheme, but a $k$-vector space scheme,
and we consider the ring $\End_{k}(\Ga_{/L})$ of all $k$-linear endomorphism of group schemes.
\begin{proposition}
    $\End_{k}(\Ga_{/L}) = k\{\tau\}$, where $\tau$ is the Frobenius-$q$ endomorphism of $F[X]$.
\end{proposition}
\begin{proof}
    An endomorphism $\Ga\to\Ga$ of schemes over $L$ is given by an $L$-algebra homomorphism $\Phi: L[X]\to L[X]$,
    hence it is determined by the image $\varphi(X) = \Phi(X)$\footnote{
        Note that if $\varphi(X) = a_nX^n + \cdots + a_0$,
        then \[\varphi(f(X)) = a_nf(X)^n  + \cdots + a_0\] and
        \[\Phi(f(X)) = f(\Phi(X)) = f(\varphi(X))\] are \textit{\textbf{different}} in general.
    } of $X$.
    It respects the group-scheme structure if it commutes with the co-multiplication map (also an $L$-algebra homomorphism) \[\Delta : F[X]\to F[X]\otimes_L F[X],\quad X\mapsto X\otimes 1 + 1\otimes X.\]
    which amounts to\[(\Phi\otimes\Phi)(\Delta(X)) = (\Phi\otimes\Phi)(X\otimes 1 + 1\otimes X) =  \Phi(X)\otimes 1 + 1\otimes\Phi(X) = 
    \varphi(X)\otimes 1 + 1\otimes\varphi(X)\]
    equals \[\Delta(\Phi(X)) = \Delta(\varphi(X)) = \varphi(\Delta(X)) = \varphi(X\otimes 1 + 1\otimes X).\]
    % Since the tensor is over $L$ and $\varphi$ has coefficients in $L$,
    This is to say that\footnote{
        Recall that the multiplicative structure on $B\otimes_A C$ is given by \[(b\otimes b')\cdot(c\otimes c') = bb'\otimes cc'.\]
    } $\varphi$ is additive, i.e. $\varphi(X+Y) = \varphi(X) + \varphi(Y)$.

    We require furthur that $\Phi$ respects the ``co-$k$-scalar multiplication'', which I don't have the formula right now.
    So let's use the functor point of view.
    Take $c\in k$.
    Yoneda tells us that
    \[\Hom_{[k\text{-}\cat{Alg}^\opp, \cat{Grp}]}(\Ga, \Ga) \simeq \Ga(L[X]), \quad \phi\mapsto \phi(\Id_{L[X]}),\]
    so the co-$c$-multiplication is given by $X\mapsto cX$.
    Therefore $\Phi$ respects this map if $\varphi(cX) = c\varphi(X)$.

    In conclusion,
    \begin{align*}
        \enom_{k}(\Ga_{/L}) &= \{k\text{-linear polynomials in} L[X] \}\\
        &=
    \end{align*}




\end{proof}





\end{document}