\documentclass{article}
\usepackage{amsmath, amssymb, amsthm, amsbsy, mathrsfs, stmaryrd}
\usepackage{enumitem}
\usepackage[colorlinks,
linkcolor=cyan,
anchorcolor=blue,
citecolor=blue,
]{hyperref}
\usepackage[capitalize]{cleveref}
\usepackage[margin = 1in, headheight = 12pt]{geometry}
\usepackage{bbm}
\usepackage{tikz-cd}

\newtheorem{theorem}{Theorem}

\theoremstyle{definition}
\newtheorem{definition}{Definition}
\newtheorem{exercise}{Exercise}[section]
\newtheorem{problem}{Problem}
\newtheorem{example}{Example}
\newtheorem{proposition}{Proposition}[section]
\newtheorem{lemma}{Lemma}[section]
\newtheorem{corollary}{Corollary}[section]

\theoremstyle{remark}
\newtheorem*{remark}{Remark}

\renewcommand{\Re}{\mathop{\mathrm{Re}}}
\renewcommand{\Im}{\mathop{\mathrm{Im}}}

% 数学对象
    \newcommand{\R}{\mathbb{R}}
    \newcommand{\C}{\mathbb{C}}
    \newcommand{\Q}{\mathbb{Q}}
    \newcommand{\Z}{\mathbb{Z}}
    \DeclareMathOperator{\GL}{GL}
    \DeclareMathOperator{\SL}{SL}
    \newcommand{\p}{\mathfrak{p}}
    \renewcommand{\P}{\mathbb{P}}
    \newcommand{\A}{\mathbb{A}}
% 集合
    \newcommand{\sminus}{\smallsetminus} %(集合)差
    \newcommand{\inject}{\hookrightarrow}
    \newcommand{\surject}{\twoheadrightarrow}
% 范畴
    \newcommand{\op}[1]{{#1}^{\mathrm{op}}} %反范畴
    \DeclareMathOperator{\enom}{End} %自态射
    \DeclareMathOperator{\isom}{Isom} %同构
    \DeclareMathOperator{\aut}{Aut} %自同构
    \DeclareMathOperator{\im}{im} %像
    \newcommand{\Set}{\mathbf{Set}} %集合范畴
    \newcommand{\Abel}{\mathbf{Ab}} %群范畴
    \newcommand{\Ring}{\mathbf{Ring}}
    \newcommand{\Cring}{\mathbf{CRing}}
    \newcommand{\Alg}{\mathbf{Alg}}
    \newcommand{\Mod}{\mathbf{Mod}}
    \DeclareMathOperator{\Id}{id}
%向量空间, 矩阵
    \DeclareMathOperator{\rank}{rank} %秩
    \DeclareMathOperator{\tr}{Tr} %迹
    \newcommand{\tran}[1]{{#1}^{\mathrm{T}}} %转置
    \newcommand{\ctran}[1]{{#1}^{\dagger}} %共轭转置
    \newcommand{\itran}[1]{{#1}^{-\mathrm{T}}} %逆转置
    \newcommand{\ictran}[1]{{#1}^{-\dagger}} %逆共轭转置
    \DeclareMathOperator{\codim}{codim} %余维数
    \DeclareMathOperator{\diag}{diag} %对角阵
    \newcommand{\norm}[1]{\left\| #1\right\|} %范数
    \DeclareMathOperator{\lspan}{span} %张成
    \DeclareMathOperator{\sym}{\mathfrak{Y}}
% 群
    \DeclareMathOperator{\inn}{Inn} %(群)内自同构
    \newcommand{\nsg}{\vartriangleleft} %正规子群
    \newcommand{\gsn}{\vartriangleright} %正规子群
    \DeclareMathOperator{\ord}{ord} %元素的阶
    \DeclareMathOperator{\stab}{Stab} %稳定化子
    \DeclareMathOperator{\sgn}{sgn} %符号函数
% 环, 域
    \DeclareMathOperator{\cha}{char} %特征
    \DeclareMathOperator{\spec}{Spec} %素谱
    \DeclareMathOperator{\maxspec}{MaxSpec} %极大谱
    \DeclareMathOperator{\gal}{Gal}
    \DeclareMathOperator{\Frac}{Frac}
% 微积分
    % \newcommand*{\dif}{\mathop{}\!\mathrm{d}} %(外)微分算子
% 流形
    \DeclareMathOperator{\lie}{Lie}
%代数几何
    \DeclareMathOperator{\proj}{Proj}
    \DeclareMathOperator{\Div}{Div}
    \DeclareMathOperator{\PDiv}{PDiv}
%多项式
    \DeclareMathOperator{\disc}{disc} %判别式
    \DeclareMathOperator{\res}{res} %结式

% 结构简写
    \newcommand{\pdfrac}[2]{\dfrac{\partial #1}{\partial #2}} %偏微分式
    \newcommand{\isomto}{\stackrel{\sim}{\rightarrow}} %有向同构
    \newcommand{\gene}[1]{\left\langle #1 \right\rangle} %生成对象
% 文字缩写
    \newcommand{\opin}{\;\mathrm{open\;in}\;}
    \newcommand{\st}{\;\mathrm{s.t.}\;}
    \newcommand{\ie}{\;\mathrm{i.e.,}\;}
    \newcommand{\myand}{\quad\text{and}\quad}

% 重定义命令
\renewcommand{\hom}{\mathop{Hom}}
\renewcommand{\vec}{\boldsymbol}

% 编号
\newcommand{\cnum}[1]{$#1^\circ$} %右上角带圆圈的编号
\newcommand{\rmnum}[1]{\romannumeral #1}
\newcommand{\myit}{$\diamond$}

\title{Elliptic Curves}
\author{LEI Bichang}
\date{2024 Spring - Summer}

\begin{document}
\maketitle

\section{Algebraic Curves}
Let $K$ be a perfect field, $\bar{K}$ a fixed algebraic closure of $K$, and $G_K := \gal(\bar{K}/K)$ the absolute Galois group.
I think there are two main additional features of algebraic curves compared to Riemann surfaces:\begin{itemize}
    \item the Galois group $G_K$ acts on a variety (and many objects relevant to it) over $K$, and
    \item there are inseparable extensions in the positive characteristics.
\end{itemize}

\subsection{Affine and Projective Vartieties over $\bar{K}$}

Let $\bar{K}[\boldsymbol{X}] := \bar{K}[X_1, \dots, X_n]$ or $\bar{K}[X_0, X_1, \dots, X_n]$, $\A^n := \A^n(\bar K)$, and $\P^n:=\P^n(\bar{K})$.

\subsubsection{Vartieties and Local Rings}
An affine variety $V$ is defined as an irreducible algebraic set in $\A^n$; that is, $I(V)\subset \bar{K}[\boldsymbol{X}]$ is a prime ideal.
The affine coordinate ring and the function field of $V$ is \[\bar{K}[V] := \bar{K}[\boldsymbol{X}]/I(V) \myand \bar{K}(V) := \Frac \bar{K}[V].\]
For a point $P\in V$, we define the maximal ideal $\mathfrak{m}_P$ at $P$ to be the ideal of regular functions vanishing at $P$, i.e.,\[\mathfrak{m}_P := \{f\in \bar{K}[V] : f(P) = 0\};\]
and the local ring $\bar{K}[V]_P$ at $P$ to be the localisation of $\bar{K}[V]$ at $\mathfrak{m}_P$.
So we have a chain of function sets \[\mathfrak{m}_P \subset \bar{K}[V] \subset \bar{K}[V]_P \subset \bar{K}(V), \]
and elements in $\bar{K}[V]_P$ are called regular functions at $P$.

The dimension of $V$ is the transcendence degree of $\bar{K}(V)$ over $\bar{K}$. 
Let $P\in V$ and $I(V) = (f_1, \dots, f_m)$. The variety $V$ is said to be nonsingular or smooth at $P$, if the Jacobian matrix \[J_V(P) := \left( \pdfrac{f_i}{X_j}(P) \right)_{\substack{1\le i\le m \\ 1\le j\le n}}\] has rank $n - \dim V$, which is equivalent to \[\dim_{\bar{K}}\mathfrak{m}_P/\mathfrak{m}_P^2 = \dim V.\]
For examples, 
\begin{itemize}
    \item $\dim\A^n = n$, and
    \item $\dim V = n - 1 \iff I(V) = (f)$ for some $f\in \bar{K}[\boldsymbol{X}]$, and $V$ is singular iff \[\pdfrac{f}{X_1} = \dots = \pdfrac{f}{X_n} = 0.\]
\end{itemize}

Now we turn to projective varieties. A projective variety $V$ is a projective algebraic set $V\subset\P^n$ s.t. the homogeneous ideal \[I_+(V) = (f\in K[\boldsymbol{X}] : f\text{ is homogeneous and }f(V) = \{0\})\subset K[X_0, \dots, X_n]\] is prime. The field of rational functions is \[\bar{K}(V) := \left\{ \frac{f}{g} : f, g\in \bar{K}[\boldsymbol{X}]\big/I_+(V)\text{ are homogeneous of the same degree}, g\ne 0 \right\}\]

Let us fix an immersion $\A^n\inject \P^n$, say $\A^n = \{X_0\ne 0\}\subset\P^n$. We have two opposite processes.
\begin{itemize}
    \item For a projective $V\subset \P^n$, $V\cap \A^n$ is an affine variety with ideal \[I(V\cap \A^n) = \big(f(1, X_1, \dots, X_n) : f(X_0, X_1,\dots, X_n)\in I_+(V)\big)\] 
    \item For an affine $V\subset \A^n$, the projective closure $\bar{V}$ has ideal $I_+(\bar{V})$ generated by the homogenisation of $I(V)$ w.r.t. $X_0$.
\end{itemize}
\begin{proposition}
    Let $V\subset\P^n$ be a projective variety.\begin{enumerate}
        \item The affine variety $V\cap \A^n$ is either empty or projective closure equal to $V$. In the latter case, $\bar{K}(V\cap \A^n)\simeq \bar{K}(V)$.
        \item For different choices of $\A^n\inject \P^n$ containing $P\in V$, $\bar{K}[V\cap \A^n]_P$'s are canonically isomorphic as local rings.
    \end{enumerate}
\end{proposition}
Therefore, for $P\in V\subset\P^n$, we define $\mathfrak{m}_P$ and $\bar{K}[V]_P$ to be the corresponding local objects of $V\cap \A^n$, and the functions in $\bar{K}[V]_P$ are regualr functions at $P$. 

\subsubsection{Rational Maps}

\subsection{Affine and Projective Vartieties over $K$}

\subsection{Connection with Schemes}

\subsection{Curves over char $p$ and Frobenius}
In this subsection, assume that $\cha K = p > 0$.


\end{document}