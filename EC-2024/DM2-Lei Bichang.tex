\documentclass{article}
\documentclass{article}
\usepackage{amsmath, amssymb, amsthm, amsbsy, mathrsfs, stmaryrd}
\usepackage{enumitem}
\usepackage[colorlinks,
linkcolor=cyan,
anchorcolor=blue,
citecolor=blue,
]{hyperref}
\usepackage[capitalize]{cleveref}
\usepackage[margin = 1in, headheight = 12pt]{geometry}
\usepackage{bbm}
\usepackage{tikz-cd}

\newtheorem{theorem}{Theorem}

\theoremstyle{definition}
\newtheorem{definition}{Definition}
\newtheorem{exercise}{Exercise}[section]
\newtheorem{problem}{Problem}
\newtheorem{example}{Example}
\newtheorem{proposition}{Proposition}[section]
\newtheorem{lemma}{Lemma}[section]
\newtheorem{corollary}{Corollary}[section]

\theoremstyle{remark}
\newtheorem*{remark}{Remark}

\renewcommand{\Re}{\mathop{\mathrm{Re}}}
\renewcommand{\Im}{\mathop{\mathrm{Im}}}

% 新命令
% 数学对象
    \newcommand{\R}{\mathbb{R}}
    \newcommand{\C}{\mathbb{C}}
    \newcommand{\Q}{\mathbb{Q}}
    \newcommand{\Z}{\mathbb{Z}}
    \DeclareMathOperator{\GL}{GL}
    \DeclareMathOperator{\SL}{SL}
    \newcommand{\p}{\mathfrak{p}}
    \renewcommand{\P}{\mathbb{P}}
    \newcommand{\A}{\mathbb{A}}
% 集合
    \newcommand{\sminus}{\smallsetminus} %(集合)差
% 范畴
    \newcommand{\op}[1]{{#1}^{\mathrm{op}}} %反范畴
    \DeclareMathOperator{\enom}{End} %自态射
    \DeclareMathOperator{\isom}{Isom} %同构
    \DeclareMathOperator{\aut}{Aut} %自同构
    \DeclareMathOperator{\im}{im} %像
    \newcommand{\Set}{\mathbf{Set}} %集合范畴
    \newcommand{\Abel}{\mathbf{Ab}} %群范畴
    \newcommand{\Ring}{\mathbf{Ring}}
    \newcommand{\Cring}{\mathbf{CRing}}
    \newcommand{\Alg}{\mathbf{Alg}}
    \newcommand{\Mod}{\mathbf{Mod}}
    \DeclareMathOperator{\Id}{id}
%向量空间, 矩阵
    \DeclareMathOperator{\rank}{rank} %秩
    \DeclareMathOperator{\tr}{Tr} %迹
    \newcommand{\tran}[1]{{#1}^{\mathrm{T}}} %转置
    \newcommand{\ctran}[1]{{#1}^{\dagger}} %共轭转置
    \newcommand{\itran}[1]{{#1}^{-\mathrm{T}}} %逆转置
    \newcommand{\ictran}[1]{{#1}^{-\dagger}} %逆共轭转置
    \DeclareMathOperator{\codim}{codim} %余维数
    \DeclareMathOperator{\diag}{diag} %对角阵
    \newcommand{\norm}[1]{\left\| #1\right\|} %范数
    \DeclareMathOperator{\lspan}{span} %张成
    \DeclareMathOperator{\sym}{\mathfrak{Y}}
% 群
    \DeclareMathOperator{\inn}{Inn} %(群)内自同构
    \newcommand{\nsg}{\vartriangleleft} %正规子群
    \newcommand{\gsn}{\vartriangleright} %正规子群
    \DeclareMathOperator{\ord}{ord} %元素的阶
    \DeclareMathOperator{\stab}{Stab} %稳定化子
    \DeclareMathOperator{\sgn}{sgn} %符号函数
% 环, 域
    \DeclareMathOperator{\cha}{char} %特征
    \DeclareMathOperator{\spec}{Spec} %素谱
    \DeclareMathOperator{\maxspec}{MaxSpec} %极大谱
    \DeclareMathOperator{\gal}{Gal}
% 微积分
    % \newcommand*{\dif}{\mathop{}\!\mathrm{d}} %(外)微分算子
% 流形
    \DeclareMathOperator{\lie}{Lie}
%代数几何
    \DeclareMathOperator{\proj}{Proj}
%多项式
    \DeclareMathOperator{\disc}{disc} %判别式
    \DeclareMathOperator{\res}{res} %结式

% 结构简写
    \newcommand{\pdfrac}[2]{\dfrac{\partial #1}{\partial #2}} %偏微分式
    \newcommand{\isomto}{\stackrel{\sim}{\rightarrow}} %有向同构
    \newcommand{\gene}[1]{\left\langle #1 \right\rangle} %生成对象
% 文字缩写
    \newcommand{\opin}{\;\mathrm{open\;in}\;}
    \newcommand{\st}{\;\mathrm{s.t.}\;}
    \newcommand{\ie}{\;\mathrm{i.e.,}\;}

% 重定义命令
\renewcommand{\hom}{\mathop{Hom}}
\renewcommand{\vec}{\boldsymbol}
\renewcommand{\and}{\;\text{and}\;}

% 编号
\newcommand{\cnum}[1]{$#1^\circ$} %右上角带圆圈的编号
\newcommand{\rmnum}[1]{\romannumeral #1}


\newcommand{\myit}{$\diamond$}

\title{}
\author{}
\date{}

\begin{document}
\maketitle

\end{document}

\usepackage[ruled, vlined]{algorithm2e}


\newcommand{\F}{\mathbb{F}}
\renewcommand{\O}{\mathcal{O}}
\newcommand{\vol}{\mathrm{vol}}
\newcommand{\s}{\mathrm{s}}

\theoremstyle{definition}
\newtheorem{question}{Question}[section]

\title{DM 1}
\author{Lei Bichang}


\begin{document}
\maketitle

\section{Short vertices}

\begin{question}
    To prove that the bound is tight,
    we will construct a lattice $\Lambda$ with $\lambda_1(\Lambda) = \sqrt{\frac{2}{\sqrt{3}}\vol(\Lambda)}$.

    Consider the lattice \[\Lambda := \Z v_1 + \Z v_2, \quad v_1 = (1, 0),\ v_2 = \left( \frac{1}{2},\frac{\sqrt{3}}{2} \right).\]
    For each $x\in\Z$, we have
    \[\norm{v_2 + xv_1} = \sqrt{\left( x + \frac{1}{2} \right)^2 + \frac{3}{4}}\ge 1 = \norm{v_2} = \norm{v_1}.\]
    So the basis $(v_1, v_2)$ for $\Lambda$ is LG-reduced.
    Therefore $\lambda_1(\Lambda) = \norm{v_1} = 1$ and $\vol(\Lambda) = \frac{\sqrt{3}}{2}$ as desired.
\end{question}

\section{Proving primality}
\begin{question}
    We shall prove by contadiction.
    Suppose that $N$ is not a prime.

    Let $y^2z = x^3  + ax^2z + bz^3$ be an equation that defines $E$, $a, b\in\Z/N\Z$,
    $\Delta(E) = -16(4a^3 + 27b^2) \in(\Z/N\Z)^\times$.
    Let $p \le \sqrt{N}$ be a prime dividing $N$,
    and $E_p$ be the curve over $\F_p$ defined by
    $y^2z = x^3  + (a\bmod p)x^2z + (b\bmod p)z^3$.
    Then $\Delta(E_p) = (\Delta(E)\bmod p)\in\F_p^\times$,
    so $E'$ is an elliptic curve over $\F_p$.
    Let \[E(\Z/N\Z)\to E_p(\F_p),\quad Q = [x : y : z]\mapsto\tilde Q := [\tilde{x} : \tilde{y} : \tilde{z}]\]
    be the reduction modulo $p$ map.
    It sends $O$ to $O$, and points in $E(\Z/N\Z)\sminus E^\s(\Z/N\Z)$ to $E(\F_p)\sminus \{O\}$.
    Since $2P\notin E^\s(\Z/N\Z)$ and $2qP = O$,
    the reduction $\widetilde{2P}\neq O$ and has order $q$.
    For the same reason, $qP\in E_p(\F_p)$ has order $2$.
    Hence $\tilde{P}\in E_p(\F_p)$ has order $2q$.
    By Hasse theorem,
    \[2q\le \# E_p(\F_p) \le (p + 1) + 2\sqrt{p} = (\sqrt{p} + 1)^2.\]
    However,
    \[2q > (N^\frac{1}{4} + 1)^2\ge (\sqrt{p} + 1)^2.\]
    This is a contadiction.
\end{question}
\begin{question}
    Assume that the prime $q\ne 2$, so that $E(\F_p)\simeq \Z/2q\Z$ and there exist $\F_p$-points of order $2q$.
    There are $q - 1$ generators in $\Z/2q\Z$,
    so a uniformly random point in $\Z/2q\Z$ has the chance of $\frac{q-1}{2q}$ to have order $2q$.
    Therefore, we can use the following algorithm.\begin{enumerate}[(1)]
        \item Generate a random point $P\in E(\F_p)$. \textit{Expected cost: $2p/\# E(\F_p) = p/q$ times of a constantly many $\F_p$-operations.}
        \item Check if $P = O$, $P$ has order $2$ (i.e. $P = (x, 0)$ for some $x$) or $P$ has order $q$. \textit{Cost: the time of $q$-multiplication,
        which is $O(\log q)$ times of $\F_p$-operations.}
        \item If one of the three conditions in step (2) is satified, we go back to step (1).
        Otherwise $P$ has order $2q$.
    \end{enumerate}
    The expected time of iterations is $\frac{2q}{q-1}$. So the total complexity is \[\frac{2q}{q-1}\frac{p}{q} O(\log q) = \frac{2p}{q-1}O(\log q)\]
    times of $\F_p$-operations. Since $2q\in [(\sqrt{p} - 1) ^ 2, (\sqrt{p} + 1)^2]$,
    the complexity is about \[\frac{2p}{\frac{p}{2} - \sqrt{p} - \frac{1}{2}} O(\log(\sqrt{p} + 1))\approx O(\log p).\]
\end{question}

\begin{question}
    In the other file, I defined a function \textsf{test\_conjecture}($p, N$)
    that generates $N$ curves over $\F_p$ as in the conjecture and then returns the proportion $P(p)$ of elliptic curves of order $2q$ for some prime $q$.
    Then I generated some primes $p_0 < p_1 < \cdots < p_n$,
    such that $\log(p_{i+1}) - \log(p_i) \approx d$, and repeated the following procedure several times.
    \begin{enumerate}[(1)]
        \item For each prime $p_i$, run \textsf{test\_conjecture}($p_i, N$) and get the proportion $P(p_i)$.
        \item Use linear regression to get $c_1, c_2$ such that \[\log P(p)\approx \log c_1 - c_2\log(\log p).\]
        These constants should be larger than the actual constants if the conjecture is true.
    \end{enumerate}
    After several trials, I would guess that $c_1 = 0.2$, $c_2 = 0.7$.
\end{question}

\begin{question}
    We can use the following algorithm.
    \begin{enumerate}[(1)]
        \item Pick uniformly random $a, b\in\F_p$ and let $E : y^2 = x^3 + ax + b$ over $\F_p$. If $E$ is not an elliptic curve, redo step (1).
        \item Compute $N := \# E(\F_p)$. If $N$ is divided by $2$, go back to step (1). \textit{Cost: polynomial in $\log p$ many $\F_p$-operations.}
        \item Run the Miller-Rabin test $k$-times to check if $q := N/2$ is a prime. If we find $q$ to be composite, go back to step (1).
        \textit{Cost: at most $k\cdot O(\log p)$ many $\F_p$-operations.}
        \item Use the algorithm decribed in Question 2.2 to get a point $P\in E(\F_p)[2q]$. \textit{Expected cost: $O(\log p)$ many $\F_p$-operations.}
    \end{enumerate}
    We run Miller-Rabin test $k$-times in step (3),
    so the probability of misclassifying a composite number as prime is $4^{-k}$,
    and $q$ has the chance of $1 - 4^{-k}$ to be a prime.

    If Conjecture 1 holds, the expected time of iterations is less than $(\log p)^{c_2}/c_1$.
    So the expected times of $\F_p$-operations is polynomial in $\log p$.
\end{question}
\begin{question}
    We can use the following algorithm, which is a variant of the one in Question 2.4.
    \begin{enumerate}[(1)]
        \item Pick uniformly random $a, b\in\Z\cap [0, \dots, N-1]$ and let $\Delta := -16(4a^3 + 27b^2)$. \begin{itemize}
            \item If $\gcd(\Delta, N)\notin \{0, 1\}$, return \textsf{composite}.
            \item If $\gcd(\Delta, N) = 0$, go back to the start of the algorithm step (1).
            \item Otherwise define the elliptic curve $E : y^2 = x^3 + ax + b$ over $\Z/N\Z$.
        \end{itemize}
        % \textit{Cost: }
        \item Pretend that $N$ is a prime and use the algorithm for elliptic curves over finite fields to compute $M := \# E(\Z/N\Z)$. \begin{itemize}
            \item If the program throws an exception, return \textsf{composite}.
            \item Otherwise if $M$ is divided by $4$, or $M\le (N^\frac{1}{4} + 1)^2$, go back to step (1).
        \end{itemize}
        \item Run the Miller-Rabin test $k$-times to check if $q := M/2$ is a prime. If we find $q$ to be composite, go back to step (1).
        \item Pretend that $N$ is a prime and use the algorithm decribed in Question 2.2 to get some $P\in E(\Z/N\Z)[2q]$.
        \begin{itemize}
            \item If the program throws an exception, return \textsf{composite}.
        \end{itemize}
        \item Use the formulae for elliptic curves over fields to compute $2P$ and $qP$.\begin{itemize}
            \item If the program throws an exception, return \textsf{composite}.
            \item Otherwise return \textsf{prime}.
        \end{itemize}
    \end{enumerate}
    Assume first that $N$ is a prime.
    By the analysis in Question 2.4, the algorithm cost polynomial time in $\log N$.
    % Not true: there is an additional condition $q > (...)$, so should cost more time
    If $N$ is composite, then the algorithm will terminate with probability $ > 1 - 4^{-k}$ in step (1) - (3),
    so the cost is smaller than it when $N$ is prime.
    Therefore, the complexity is polynomial in $\log N$ many $\Z/N\Z$-operations.
\end{question}

\begin{question}
    Please see the function \textsf{primality\_test} in other file.
    I have implemented most of the functions, but I failed to realize Schoof's algorithm (the computation of trace modulo primes).
\end{question}

\end{document}
