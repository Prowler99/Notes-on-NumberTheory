\section{Extensions of Local Fields}

\subsection{Simple Extensions of DVRs}
Let $A$ be a local ring with ($\mathfrak{m}$, $k$), $f\in A[X]$ a monic polynomial of deg $n$.
We consider the extension \[A \to B_f := A[X]/f.\]

Let $\bar{f}$ be the image of $f$ in $k[X] \simeq A[X]/\mathfrak{m}$ with decomposition \[\bar{f} = \prod_{i}\bar{g}_i^{e_i},\ g_i\in A[X],\ \bar{g}_i\in k[X]\text{ irreducible.}\]
and \[\bar{B}_f := B_f/\mathfrak{m}B_f \simeq A[X]/(\mathfrak{m}, f) \simeq k[X]/(\bar{f}).\]
\begin{lemma}
    $\mathfrak{m}_i := (\mathfrak{m},\ g_i\bmod f)\subset B_f$ are all the distinct maximal ideals of $B_f$.
\end{lemma}
\begin{proof}
    Denote $\pi : B_f\to\bar{B}_f$. We have $B_f/\mathfrak{m}_i \simeq \bar{B}_f/(\bar{g}_i)$, so $\mathfrak{m}_i$'s are maximal.
    Note that $\mathfrak{m}_i = \pi^{-1}(\bar{g}_i)$.

    Take $\mathfrak{n}\in\spm B_f$.
    If $\mathfrak{n}\supset\mathfrak{m}$, then $\mathfrak{n} = \pi^{-1}\pi\mathfrak{n}$,
    and goes to a maximal ideal in $\bar{B}_f$ (because $\bar{B}_f/\pi\mathfrak{n} \simeq B_f/\mathfrak{n}$),
    so $\mathfrak{n} = \pi^{-1}(\bar{g}_i) = \mathfrak{m}_i$.

    So assume that $\mathfrak{m}\not\subset\mathfrak{n}$, then $\mathfrak{n} + \mathfrak{m}B_f = B_f$.\footnote{In this case $\mathfrak{n}/(\mathfrak{n\cap m})\simeq \bar{B}_f$ as $B_f$-module, and thus $\pi^{-1}\pi\mathfrak{n} = B_f$.}
    Therefore \[\frac{B_f}{\mathfrak{n}} = \frac{\mathfrak{n}+\mathfrak{m}B_f}{\mathfrak{n}} \simeq \frac{\mathfrak{m}B_f}{\mathfrak{n}}.\]
    Since $A$ is local and $B_f$ is a f.g. $A$-mod, by Nakayama's lemma, we see $\mathfrak{n} = B_f$. Contradiction.


\end{proof}

Now take $A$ to be a DVR with $\mathfrak{m} = (\varpi)$ and $K = \Frac A$. Put $L := K[X]/(f)$.
We give two cases where $B_f$ is a DVR.

\subsubsection*{Unramified case}
Let $\bar{f}\in k[X]$ be irreducible. Then $B_f$ is a DVR with maximal ideal $\mathfrak{m}B_f$.
\begin{corollary}\label{simple ext of dvr - unramified - is field}
    $f\in A[X]$ is also irreducible, so $L$ is a field.
    Moreover, $B_f$ is the integral closure of $A$ in $L$, and $L/K$ is unramified if $\bar{f}$ is separable.
\end{corollary}
\begin{proof}
    $L = K[X]/f \simeq \left( A[X]/f \right)\otimes_{A} K = B_f\otimes_A K$.
    As $B_f$ is a domain, $L$ is a field and $L = \Frac B_f$.
    Since $A$ is integrally closed, $B_f$ is also integrally closed, so $B_f$ is the integral closure of $A$ in $L$.
\end{proof}
\subsubsection*{Totally ramified case}
Let $f\in A[X]$ be an \textbf{Eisenstein polynomial}, i.e., \[f = X^n + a_{n-1}X^{n-1} + \cdots  + a_0,\ a_i\in\mathfrak{m},\ a_0\notin\mathfrak{m}^2.\]
\begin{proposition}
    $B_f$ is a DVR, with maximal ideal generated by the image of $X$ and residue field $k$.
\end{proposition}
\begin{proof}
    Let $x$ be the image of $X$ in $B_f$.
    We have $\bar{f} = X^n$, so $B_f$ is a local ring with maximal ideal $(\mathfrak{m}, x)$.
    Because $a_0\in\mathfrak{m\setminus m^2}$, $a_0$ must uniformise $\mathfrak{m}\subset A$, and \[-a_0\bmod f = x^n + \cdots + (a_1\bmod f)\,x,\] Therefore $(\mathfrak{m}, x) = (x)$.
\end{proof}
Similar to \cref{simple ext of dvr - unramified - is field}, $f$ is irreducible and $L$ is a field with $B_f$ the integral closure of $A$ in $L$.

\subsection{Unramified Extensions of Local Fields}
Let $K$ be a local field.
We assume further that both $K$ and its residue field $k = \mathcal{O}_K/\mathfrak{m}$ are perfect.

The slogan is that unramified extensions are just extensions of residue fields.
Using Hensel's lemma, an extension $k(a)/k$ can be lifted to a unique extension $K(\alpha)/K$ over $K$ with \[\gal(K(\alpha)/K)\simeq \gal(k(a)/k).\] Moreover, given an extension $L/K$, there is a maximal unramified subextension $K_0$ in $L$ containing every unramified extensions.

Now we assume $k$ to be finite. Then adjoining roots of unities with order coprime to $p = \cha k$ gives all finite unramified extensions of $K$.

\begin{example}
    Let $K/\Q_p < \infty$ and $k = \mathbb{F}_q$.
    Then the unique extension of $k$ of degree $n$ is the splitting field of $X^{q^n} - X$ over $k$, which equals $k(\mu_{q^n - 1})$ once we fix an algebraic closure of $k$.
    So the unramified extension $K_n/K$ of degree $n$ is the splitting field of $X^{q^n} - X$ over $K$, i.e., \[K_n = K(\mu_{q^{n} - 1}).\] The Galois group $\gal(K_n/K)$ is generated by $\frob_K$, which is determined by \[\frob_K\beta \equiv \beta^q \mod\varpi,\ \forall\beta\in\mathcal{O}_{K_n}\]for any uniformiser $\varpi$ (simultaneously of $K$ and $K_n$).

    What if we adjoin $\zeta_{m}$ to $K$ where $m$ is an arbitary integer prime to $p$?
    The answer is that $K(\mu_m)$ is unramified of degree the smallest positive integer $f$ s.t. $m \mid p^f - 1$, by the following \cref{cyclotomic extension of finite fields} on finite fields.
\end{example}

\begin{lemma}\label{cyclotomic extension of finite fields}
    Let $\zeta_n$ be a primitive $n$-th root of unity over $\F_q$ with $q, n$ coprime. Then $[\F_q(\zeta_n) : \F_q]$ is the smallest integer $f > 0$ s.t. $n \mid q^f - 1$.
\end{lemma}
\begin{proof}
    Because $\cha\F_q\nmid n$, the primitive root $\zeta_n$ exists and $\F_q(\zeta_n)$ is the splitting field of $X^n - 1$ over $\F_q$.
    The degree $f = [\F_q(\zeta_n) : \F_q]$ is the order of $\frob_q$ on $\F_q(\zeta_n)$, i.e., $f$ is the smallest integer s.t. \[\frob_q^f(\zeta_n) = \zeta_n^{q^f} = \zeta_n.\] The definition of primitive root of unity says that \[\zeta_n^{q^f -1} = 1\iff n \mid q^f - 1.\qedhere\]
\end{proof}

\subsection{Newton Polygon}
Let $K$ be a local field with valuation $\val$ extended to $K^\alg$.

For $P = a_0 + a_1 X + \dots + a_dX^d\in K[X]$,
the \textbf{Newton polygon} of $P := \mathrm{NP}(P) := $ convex hull of points \[(0, \val(a_0)), (1, \val(a_1)),\dots, (d, \val(a_d)).\]
\begin{itemize}
    \item $\mathrm{NP}(P)$ is a union of linked segments with increasing slopes.
    \item \textbf{length of a segment} $:=$ its length along $x$-axis.
\end{itemize}

\begin{theorem}
    The number of roots of $P$ in $K^\alg$ with valuation $\lambda$ = the length of $\mathrm{NP}(P)$ with slope $-\lambda$.
\end{theorem}

\subsection{Ramification Groups}

Let $K$ be a CDVF with perfect residue field $k$, $L/K<\infty$ Galois. We will study the Galois group \[G := \gal(L/K)\] by giving filtrations on it.




