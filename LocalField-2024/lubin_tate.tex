\section{Lubin-Tate Theory}

\subsection{Formal Groups}
Let $A$ be a commutative ring.
\begin{itemize}
    \item If $f\in A\llbracket T\rrbracket$ and $g\in A\fring{X_1, \dots, X_n}$, then \begin{align*}
        f \circ g &:= f(g(X_1, \dots, X_n)),\\
        g\circ f &:= g(f(X_1), \dots, f(X_n)).
    \end{align*}
    \item If $F\in A\llbracket X_1, \cdots, X_n  \rrbracket$, we put $F_i := $ the partial derivative of $F$ w.r.t.\! the $i$-th variable $X_i$.    
\end{itemize}
\begin{lemma}\label{power series reversible iff}
    Let $f = \sum_{i\ge 1}a_iT^i\in A\fring{T}$. Then
    \[\exists g\in A\fring{T} \st f\circ g = g\circ f = T\iff a_1 = f'(0)\in A^\times.\]
    Such a power series is called \textbf{reversible}.
\end{lemma}
\begin{proof}
    Use $A\fring{T} = \varprojlim A[T]/T^n$. For details, see the proof of \cref{fund prop of F_varpi}.
\end{proof}


In this section, a \textbf{formal group} means a (commutative) formal group law of dimension one.

A \textbf{homomorphism} $h : F\to G$ between formal groups $F$ and $G$ over $A$
\[:= h\in XA\llbracket X \rrbracket,\ \st h\circ G = F\circ h,\]
that is $h(G(X, Y)) = F(h(X), h(Y))$.
\begin{itemize}
    \item A homomorphism $h : F\to G$ is an isomorphism $\iff h'(0)\in A^\times$.
    \item Every integer $n\in\Z$ gives rise to an endomorphism $[n] = nX + O(X^2) \in \enom(F)$,
    yielding a ring homomorphism $\Z\to\enom(F)$.
\end{itemize}

A \textbf{differential form} on $F$ 
\[:= \omega(X) = p(X)dX\in A\llbracket X \rrbracket dX,\ \st\]
\[\omega(f(X)) = p(f(X))df(X) := p(f(X))f'(X)dX,\ \forall f(X).\]
We say $\omega(X)$ is \textbf{invariant}, if $\omega\circ F(-, Y) = \omega$;
i.e,  \[p(F(X, Y))F_1(X, Y) = p(X).\]
Set $X = 0$, we see that \[p(Y) = p(0)\frac{1}{F_1(0, Y)}.\]
Hence any invariant differential takes the form \[\omega(X) = \frac{a\cdot dX}{F_1(0, X)}.\]
Conversely, we define \[\omega_F := \frac{dX}{F_1(0, X)}\] and call it \textbf{normalized invariant differential}. This name is verified as below.
\begin{proposition}
    $\omega_F$ is invariant for $F$.
\end{proposition}
\begin{proof}
    Take $\left.\frac{d}{dZ}\right|_{Z = 0}$ for \[F(Z, F(X, Y)) = F(F(Z, X), Y),\]
    we get \[F_1(0, F(X, Y)) = F_1(X, Y)F_1(0, X).\qedhere\]
\end{proof}
\begin{itemize}
    \item If $h \in \Hom(F, G)$,
    then \[\omega_G\circ h = h'(0)\cdot\omega_F.\] 
\end{itemize}

\subsection{Formal Groups over local fields}

Let $K$ be an extension of $\Q_p$ inside $\C_p$.

\subsubsection{The Logarithm}
Let $F$ be a formal group over $K$ and $\omega_F$ the normalized invariant differential.
We define \[\log_F(X) := \int\omega_F\in K\fring{X},\quad \st \log_F(0) = 0.\]
\begin{itemize}
    \item If $\omega(X) = (1 + p_1X + p_2X^2 + \cdots)dX$, then \[\log_F(X) = X + \frac{p_1X^2}{2} + \frac{p_2X^3}{3} + \cdots\in X A\llbracket X  \rrbracket.\]
    \item $\log_F(X)\in H_K$ if $F$ is defined over $\O_K$.
\end{itemize}

\begin{proposition}\label{log is additive}
    $\log_F(X + Y) = \log_F(X) + \log_F(Y)$, so $\log_F : F\to_{K} \Ga$ is an isomorphism \textit{over} $K$.
\end{proposition}
\begin{proof}
    Let $E(X) := \log_F(X + Y) - \log_F(X)$.
    Then $dE(X) = \omega_F\circ F - \omega_F = 0$,
    thus $E(X) = E(0) = \log_F(Y)$.
\end{proof}
\begin{example}
    $\log_{\Ga}(X) = X$, $\log_{\Gm}(X) = \log(1 + X)$.
\end{example}
\begin{example}
    $\Ga$ and $\Gm$ are \textit{NOT} isomorphic over $\O_K$, because \[(\m_{\C_p}, +_{\Ga})
     = (\m_{\C_p}, +)
     \not\simeq (1 + \m_{\C_p}, \ \cdot)
     \simeq (\m_{\C_p}, +_{\Ga}),\]
    as the former is torsion-free while the latter has many torsion.
\end{example}
\begin{remark}
    \cref{log is additive} holds for any formal group over a $\Q$-algebra $A$.
    As the proof involves not the axiom of commutativity,
    it shows that any formal group (of dimension $1$) over a $\Q$-algebra is necessarily commutative.
\end{remark}

\subsubsection{The Height}
Let $k$ be a ring of characteristic $p >0$.
If $F, G$ are formal groups over $k$,
and $f\in\Hom(F, G)$, we define the \textbf{height} of $f$ to be
\[\height(f) := \text{largest integer } h\in\Z ,\st f(X) = g\left( X^{p^h} \right) \text{ for some } g\in k\llbracket X \rrbracket.\]
\begin{proposition}
    If $f\in \Hom(F, G)$ and $f(X) = g\left( X^{p^h} \right)$ with $h = \height(f)$,
    then $g'(0)\ne 0$.
\end{proposition}
\begin{proof}
    Two steps.
    \begin{itemize}
        \item If $f\in \Hom(F, G)$ with $f'(0) = 0$,
        then $f(X) = g\left( X^{p^h} \right)$ for some $g$.

        This is because \[0 = f'(0)\omega_F = \omega_G\circ f = \frac{f'(X)dX }{G_1(0, X )},\]
        So $f'(X) = 0$. As $\cha k = p$,
        this leads to the result.
        \item If $F\in\Hom(F, G)$, $f(X) = g\left( X^{p^h} \right)$, then $g\in\Hom(F^{\frob_{p^h}}, G)$.

        Write $F = \sum a_{ij}X^iY^j$, so $F^{\frob_{p^h}}(X) = \sum a_{ij}^{p^h}X^iY^j$. As $\cha k = p$, $F^{\frob_{p^h}}$ is also a formal group over $k$.
        What left is obvious.\qedhere
    \end{itemize}
\end{proof}

\subsubsection{The Torsion of Formal Groups and the Tate Module}
Let $K/\Q_p < \infty$, $ k = \O_K/\pi$ the residue field, $F$ a formal group over $\O_K$.
\begin{itemize}
    \item Note that $F$ can be regarded as a formal group over $K$, and $\bar F := F\bmod \pi\in k\llbracket X \rrbracket$ is a formal group over $k$.
\end{itemize}
We define the \textbf{height} of $F$ to be \[\height(F) := \text{height of } [p]\in\End_k(\bar F).\]
\begin{example}
    For $\Ga$, $[p](X) = 0$ in $k\llbracket X \rrbracket$, so $\height(\Ga_{/{\O_K}}) = \infty$.

    For $\Gm$, $[p](X) = (1 + X)^p - 1 = X^p$ in $k\llbracket X \rrbracket$,
    so $\height(\Gm_{/\O_K}) = 1$.
\end{example}

and consider the $p^n$-torsion points of $F$, namely \[F[p^n] := \{z\in\m_{\C_p}\mid [p^n]_F(x) = 0\}.\]
\begin{itemize}
    \item $F[p^n]$ is a subgroup of $(\m_{\C_p}, +_F)$ and a $\Z/p^n\Z$-module.
    \item $[p] : F[p^{n+1}]\hookrightarrow F[p^n]$
    is a surjective homomorphism of $\Z/p^{n+1}\Z$-module
\end{itemize}
We look at the equation $[p](z) = y$ with $y\in\m_{\bar\Q_p}$ first.
\begin{itemize}
    \item If $h = \height(F) < \infty$,
    then $[p](X)\in \O_K\llbracket X \rrbracket$ has Weierstrass degree $p^h$.\par
    $\implies [p](z) = y$ has $p^h$ solutions in $\m_{\bar\Q_p}$.
    \item From $\omega_F\circ [p] = [p]'(0)\omega_F$, one deduce that $[p]'(X) = p(1 + O(X))$.\par
    $\implies$ all roots of $[p](z) = y$ are simple.
\end{itemize}
Therefore, if $\height(F) < \infty$, then \[\# F[p^n] = p^{hn}.\]
Now define \[T_pF := \varprojlim_n F[p^n].\]
\begin{itemize}
    \item $T_pF$ is a $\Z_p$-module.
    \item If $z = (z_1, z_2, \dots)\in T_pF$,
    then $pz = (0, z_1, z_2, \dots)$.\par
    $\implies T_pF$ is torsion-free. In addition,
    \[\bigcap_{n\ge 0} p^nT_pF = \{0\}.\footnotemark\]
    \footnotetext{We say $T_pF$ is seperated.}
    \item We have an isomorphism
    \begin{align*}
        T_pF/p^nT_pF &\simeq F[p^n]\\ 
        \overline{(z_1, z_2, \dots)} &\mapsto z_n.
    \end{align*}
\end{itemize}
\begin{proposition}
    $T_pF$ is a free $\Z_p$-module of rank $h = \height F$.
\end{proposition}
\begin{proof}
    Let $m_1, \dots, m_h$ be a lift of a $\F_p$-basis of the dimension $h$ vector space $T_pF/pT_pF\simeq F[p]$.
    We claim that $m_1, \dots, m_h$ is a $\Z_p$-basis for $T_pF$.
    \begin{itemize}
        \item (linear independence.)
        Suppose $\lambda_1m_1 + \cdots + \lambda_h m_h = 0$ with $\lambda_i\in\Z_p\setminus \{0\}$.
        $T_pF$ is torsion-free, so $\exists j$ s.t. $p\nmid \lambda_j$.
        Hecen it will give a nontrivial relation modulo $p$.
        \item (generate $T_pF$.)
        % {\color{red} Can I use Nakayama to prove this directly?}
        Use the standard method.
        Obtain \[m = \sum_i \lambda_i^{(k)} m_i + p^kn^{(k)}\] inductively for all $k\ge 1$
        Take $\lambda_i := \lim_k\lambda_i^{(k)}$
        by $\lambda_i^{(k+1)}\equiv \lambda_i^{(k)}\bmod p^{k}$.
        Then \[m - \sum_{i}\lambda_im_i\in\cap_{k\ge 1}p^kT_pF = 0.\qedhere\]
    \end{itemize}
\end{proof}


\subsubsection{Galois representation attached to a formal group}
The Galois group $G_K = \gal(\bar\Q_p/K)$ acts $\Z/p^n$-linearly on $F[p^n]$,\\
$\leadsto G_K$ acts $\Z_p$-linearly on $T_pF$.\\
$\leadsto$ continuous group homomorphism \[\rho_F : G_K\to\aut_{\Z_p}(T_pF)\mathop{\isomto}_{\text{choose basis}} \GL_h(\Z_p).\]
\begin{example}
    For $K = \Q_p$ and $F = \Gm$, $\rho_F = $ cyclotomic charater $\chi_\cyc$.
\end{example}






\subsection{Lubin-Tate formal groups}
From now on, we write $A := \O_K$.

Choose a uniformiser $\varpi$ of $K$. Define
\[\mathcal{F}_\varpi := \left\{f\in\O_K\fring{T}\ \left|\ \begin{aligned}
    &f(T) \equiv \varpi T&\mod {T^2} \\
    &f(T)\equiv T^q&\mod \varpi
\end{aligned}\right.\right\}.\]
For example, $f(T) = T^q + \varpi T\in\mathcal{F}_\varpi$.
The following lemma is a fundamental property of $\mathcal{F}_\varpi$.

\begin{lemma}\label{fund prop of F_varpi}
    Let $f, g\in\mathcal{F}_\varpi$, $\Phi_1$ be a linear form\footnote{A \textbf{linear form} is a homogeneous polynomial of degree 1.} over $\O_K$. Then there is a \textbf{unique} $\Phi\in\O_K\fring{X_1,\dots, X_n}$, s.t.\[\begin{cases}
        \Phi \equiv \Phi_1 \mod (X_1, \dots, X_n)^2,\\
        f(\Phi(X_1, \dots, X_n)) = \Phi(g(X_1), \dots, g(X_n)).
    \end{cases}\]
\end{lemma}
\begin{proof}
    We use a standard method. Finding $\Phi$ is equivalent to finding $\Phi_r\in A[X_1, \dots, X_n]$ s.t. \[\begin{cases}
        \Phi_{r+1} \equiv \Phi_r &\mod (\deg\ge r+1),\\
        f(\Phi_r)\equiv \Phi_r(g(X_1), \dots, g(X_n)) &\mod(\deg\ge r+1). 
    \end{cases}\]
    The second condition is guaranteed because $X\mapsto h(X)$ is $X$-adically continuous for any power series $h$.

    Suppose we have found $\Phi_r$. We look for $\Phi_{r+1}$ of the form $\Phi_{r+1} = \Phi_r + Q$, where $Q$ is homogeneous of degree $r+1$, s.t. \[f(\Phi_{r+1}) \equiv \Phi_{r+1}(g(X_1), \dots, g(X_n)) \mod \deg\ge r+2.\]
    The LHS is
    \[f(\Phi_r) + f(Q)\equiv f(\Phi_r) + \varpi Q\mod\deg\ge r+2,\]
    while the RHS is
    \[\Phi_r\circ g + Q(\varpi X_1, \dots, \varpi X_n)\equiv \Phi_r\circ g + \varpi^{r+1}Q,\]
    so if such a $Q\in A[X_1, \dots]$ exists, it must satisfy 
    \[\varpi(\varpi^r - 1)Q\equiv f\circ \Phi_r - \Phi_r\circ g\mod\deg\ge r+2\]
    and thus being unique.
    This procedure also shows that all $\Phi_r$'s are unique if we require $\Phi_{r+1} - \Phi_r$ to be homogeneous.

    Because $\varpi^r - 1\in A^\times$, it suffices to show \[f(\Phi_r) \equiv \Phi_r\circ g\mod \varpi,\] which is clear.
\end{proof}

By \cref{fund prop of F_varpi}, one may define the \textbf{Lubin-Tate formal groups}.
They are exactly the formal group laws admitting an endomorphism\begin{itemize}
    \item that has derivative at the origin equal to a uniformiser of $K$, and
    \item reduces mod $\m$ to the Frobenius map $T\mapsto T^q$.
\end{itemize}
Moreover, these formal groups admit $\O_K$-actions and are isomorphic as formal $\O_K$-modules.

\begin{proposition}
    For each $f\in \mathcal{F}_\varpi$, there is a unique formal group $F_f$ over $\O_K$ admitting $f$ as an endomorphism.
\end{proposition}
\begin{proof}
    \cref{fund prop of F_varpi} gives $F_f\in A\fring{X, Y}$ s.t. \[\begin{cases}
        F_f = X + Y + \deg \ge 2,\\
        f(F_f(X+Y)) = F_f(f(X), f(Y)).
    \end{cases}\]
    The associativity is proved by showing that both $G_1 = F_f(X, F_f(Y, Z))$ and $G_2 = F_f(F_f(X, Y), Z)$ satisfies 
    \[\begin{cases}
        G = X+Y+Z + \deg\ge 2,\\
        f(G) = G(f(X), f(Y), f(Z)).
    \end{cases}\]
    This is a direct application of \cref{fund prop of F_varpi} and will be used many times.
\end{proof}

So Lubin-Tate formal groups exist. Now we investigate their homomorphisms.
\begin{proposition}
    For each $f, g\in\mathcal{F}_\varpi$ and $a\in \O_K$, there is a unique $[a]_{g, f}\in \O_K\fring{T}$ s.t. \[\begin{cases}
        [a]_{g, f} = aT + \dots,\\
        g\circ [a]_{g, f} = [a]_{g, f} \circ f,
    \end{cases}\]and $[a]_{g, f}\in\hom(F_f, F_g)$, i.e. \begin{align*} F_g\circ [a]_{g, f} = [a]_{g, f}\circ F_f.\end{align*}
    As a corollary of \cref{power series reversible iff}, each $u\in A^\times$ gives an isomorphism $[u]_{g, f} : F_f\isomto F_g$, and there is a unique isomorphism $F_f\simeq F_g$ of the form $T + \cdots$.
    \qed
\end{proposition}

We write $[a]_{f} := [a]_{f, f}\in\enom F_f$.
Note that \[[\varpi]_f = f.\]

\begin{proposition}
    For any $a, b\in\O_K$, \[[a+b]_{g, f} = [a]_{g, f} + [b]_{g, f},\]and\[[ab]_{h, f} = [a]_{h, g}\circ [b]_{g, f}.\]
    
    In particular, $\O_K\hookrightarrow\enom  F_f$ as a ring by $a\mapsto [a]_f$, making $F_f$ a formal $\O_K$-module. The canonical isomorphism $[1]_{g, f}$ is an isomorphism of $\O_K$-modules.
    \qed
\end{proposition}

\subsection{Construction of \texorpdfstring{$K_\varpi$}{}}
Fix an algebraic closure $K^\alg$ of $K$.
Each $f\in\mathcal{F}_\varpi$ associates to $\mathfrak{m}_{K^\alg}$ an $\O_K$-module structure via \[\alpha +_{F_f}\beta := F_f(\alpha, \beta)\]and
\[a\cdot \alpha := [a]_f(\alpha).\]for $|\alpha| < 1, |\beta| < 1$ and $a\in \O_K$.
We denote this $\O_K$-module by $\Lambda_f$.
If $g\in\mathcal{F}_\pi$, then the canonical isomorphism $[1] : F_f\to F_g$ yields an isomorphism of $\O_K$-modules $\Lambda_f\isomto\Lambda_g$.

The $\varpi^n$-torsion part of $\Lambda_f$ is denoted by $\Lambda_{f, n}$ or $F_f[n]$, i.e., \[\Lambda_{f, n} = F_f[n] := \Lambda_f[[\varpi]_f^n].\]
Because $[\varpi]_f = f$, $\Lambda_{f, n}$ is the $\O_K$-module consisting of the roots of $f^{(n)} := f\circ\cdots\circ f$.
If one takes $f$ to be an Eisenstein polynomial, then all the roots of $f^{(n)}$ lie in $\mathfrak{m}_{K^\alg}$, so $\Lambda_{f, n}$ is precisely the set of roots of $f^{(n)}$ equipped with the $\O_K$-module structure from $F_f$.

\begin{lemma}\label{pi^n torsion cyclic of}
    Let $M$ an $\O_K$-module, $M_n = M[\varpi^n]$. If\begin{itemize}
        \item $M_1$ has $q = [\O_K : \varpi]$ elements, and
        \item $\varpi : M \to M$ is surjective,
    \end{itemize}
    then $M_n\simeq \O_K/\varpi^n$.
\end{lemma}
\begin{proof}
    Do induction on $n$. The structure theorem of f.g.\! modules over a PID shows that:
    if $M_1$ having $q$ elements, then $M_1\simeq A/\varpi$.
    Now assume it true for $n-1$.
    Look at the sequence \[0\to M_1\to M_n\stackrel{\varpi}{\to} M_{n-1}\to 0.\] Surjectivity of $\varpi$ implies the exactness of this sequence, and thus $M_n$ has $q^n$ elements. In addition, $M_n$ must be cyclic, otherwise $M_1 = M_n[\varpi^n]$ is not cyclic.
\end{proof}

\begin{proposition}
    The $\O_K$-module $\Lambda_{f, n}$ is isomorphic to $\O_K/\varpi^n$, and hence $\enom(\Lambda_{f, n})\simeq \O_K/\varpi^n$.
\end{proposition}
\begin{proof}
    It suffices to show for a chosen $f$, so let's take $f = \varpi T + \dots + T^q$, an Eisenstein polynomial.
    We use the above \cref{pi^n torsion cyclic of} by the following observations.\begin{itemize}
        \item All roots of an Eisenstein polynomial have valuation $>0$.
        \item If $|\alpha| < 1$, then the Newton polygon of $f(T) - \alpha$ shows that its roots have valuation $>0$, and thus $[\varpi] = f(T)$ is surjective on $\Lambda_f$.\qedhere
    \end{itemize}
\end{proof}

\begin{lemma}\label{galois commutes power series}
    Let $L$ be a finite Galois extension of $K$. Then for every $F\in\O_K\fring{X_1, \dots, X_n}$, $\alpha_1,\dots, \alpha_n\in\mathfrak{m}_L$ and $\tau\in\gal(L/K)$,
    \[\tau F(\alpha_1, \dots, \alpha_n) = F(\tau\alpha_1, \dots, \alpha_n).\]
\end{lemma}
\begin{proof}
    Note that $\tau$ acts continuously on $L$, because the extension of valuation for local fields is unique.
    Therefore writing $F = \lim_{m\to\infty} F_m$ gives the desired result.
\end{proof}

\begin{theorem}\label{construction of K_{varpi, n}}
    Let $K_{\varpi, n} := K(\Lambda_{f, n})\subset K^\alg$.
    These fields are independent to the choice of $f$.\begin{enumerate}
        \item [(a)] $K_{\varpi, n}/K$ is totally ramified of degree $q^{n-1}(q-1)$.
        \item [(b)] The action of $\O_K$ on $\Lambda_{f, n}$ defines an isomorphism \begin{equation}
            \left( \O_K/\mathfrak{m}_K^n \right)^\times \simeq \gal(K_{\varpi, n}/K).
        \end{equation}
        \item [(c)] For all $n$, $\varpi$ is a norm from $K_{\varpi, n}$, i.e., $\exists\alpha_n\in K_{\varpi, n}$ with $N_{K_{\varpi, n}/K}(\alpha_n) = \varpi$.
    \end{enumerate}
\end{theorem}
\begin{proof}
    Since $F_f[n]\simeq_{\O_K} F_g[n]$, the extesnions over $K$ given by them equal.
    Let $f$ be a polynomial $T^q + \dots + \varpi T$.

    Choose a nonzero root $\varpi_1$ of $f(T)$ and, inductively, a root $\varpi_n$ of $f(T) - \varpi_{n-1}$.
    So $\varpi_n\in\Lambda_{f, n}$, and we obtain a tower of extensions \[K_{\varpi, n}\supset K(\varpi_n)\stackrel{q}{\supset} K(\varpi_{n-1}) \stackrel{q}{\supset}\dots\stackrel{q}{\supset} K(\varpi_1)\stackrel{q-1}{\supset} K.\]
    All the extensions with indicated degrees are given by Eisenstein polynomials, and thus Galois and totally ramified.

    The field $K_{\varpi, n} = K(\Lambda_{f, n})$ is the splitting field of $f^{(n)}$ over $K$, hence $\gal(K_{\varpi, n}/K)$ embeds into the permutation group of the set $\Lambda_{f, n}$. By \cref{galois commutes power series}, the action of $\gal(K_{\varpi, n}/K)$ on $\Lambda_n$ preserves its $\O_K$-action, so
    \[\gal(K_{\varpi_n}/K)\hookrightarrow \aut(\Lambda_{f, n})\simeq (\O_K/\varpi^n)^\times.\]
    So $[K_{\varpi, n} : K]\le (q - 1)q^{n-1}$. Comparing the degree gives $K_{\varpi, n} = K(\varpi_n) $.

    Now we prove (c).
    Let $f^{[n]} := (f/T)\circ f\circ\dots\circ f$. Then $f^{[n]}$ is monic with degree $q^{n-1}(q-1)$ and $f^{[n]}(\varpi_n) = 0$, and thus $f^{[n]}$ is the minimal polynomial of $\varpi_n$ over $K$. So we have \[N_{K_{\varpi, n}/K}(\varpi_n) = (-1)^{q^{n-1}(q-1)}\]
    by the following \cref{compute norm and trace from minimal polynomial}.
\end{proof}

\begin{lemma}\label{compute norm and trace from minimal polynomial}
    Let $L/K$ be a finite extension in an algebraic closure $K^\alg$, and $\alpha\in L$ has minimal polynomial $f$ over $K$ of degree $d$. Suppose \[f(X) = (X-\alpha_1)\cdots(X-\alpha_d)\in K^\alg[X],\] and let $e = [L : K(\alpha)]$
    then \[N_{L/K}(\alpha) = \left(\prod_{i = 1}^d \alpha_i\right)^e,\qquad \tr_{L/K}(\alpha) = e\sum_{i = 1}^d \alpha_i.\]
    Moreover, if \[f(X) = a_dX^d + a_{d-1}X^{d-1} + \dots + a_0,\]then \[N_{L/K}(\alpha) = (-1)^{de}a_0^e,\qquad \tr_{L/K}(\alpha) = -ea_{d-1}.\]
\end{lemma}
\begin{proof}
    This follows directly from $N_{L/K} = N_{K(\alpha)/K}\circ N_{L/K(\alpha)}$ and $\tr_{L/K} = \tr_{L/K(\alpha)}\circ \tr_{K(\alpha)/K}$.
    For example,
    \begin{align*}
        N_{L/K}(\alpha) &= N_{L/K(\alpha)}\left( N_{K(\alpha)/K}\alpha \right)\\ &
        = \left( \prod_{\sigma\in\Hom_{K}(K(\alpha), \bar K)}\sigma\alpha \right)^{[L : K(\alpha)]} = \left( \prod_{i=1}^d\alpha_i  \right)^{[L : K(\alpha)]}.\qedhere
    \end{align*}
\end{proof}


Define \[K_\varpi := \bigcup_{n} K_{\varpi, n}.\]
Then $K_\varpi/K$ is totally ramified, Galois, and abelian.
The isomorphisms in \cref{construction of K_{varpi, n}} (b) are
\[(\O_K/\varpi^n)^\times\to \gal(K_{\varpi, n}/K)\quad \bar{u}\mapsto (\Lambda_{f, n}\ni \alpha\mapsto [u]_f(\alpha)),\]
and clearly lift to an continuous isomorphism
\[\O_K^\times\simeq \gal(K_\varpi/K).\]
We call \[\chi_\varpi : G_K\to\gal(K_\varpi/K)\isomto \O_K^\times,\quad g\alpha = [\chi_\varpi(g)]_{f}(\alpha),\forall\alpha\in \Lambda_f = F_f[\pi^\infty]\]
the \textbf{Lubin-Tate charater} attached to $\varpi$.


\subsection{Local Class Field Theory: Statement}
Let $K_\pi = K(F[\pi^\infty])$ be the Lubin-Tate extension. We have $\gal(K_\pi/K)\simeq \O_K^\times$.

Recall that the maximal unramified extension $K^\nr/K$ has Galois group \[\gal(K^\nr/K)\simeq\gal(\bar k/k)\simeq\widehat\Z.\]
If $q = \# k$, then $\frob_q : x\mapsto x^q$ generates a dense subgroup of $\gal(\bar k/k)$.

We define the \textbf{local Artin map} to be the group homomorphism
\[\art_K : K^\times\simeq \pi^\Z\times \O_K^\times\to \gal(K_\pi/K)\times\gal(K^\nr/K)\simeq\footnotemark\gal(K_\pi K^\nr/K)\]
\footnotetext{$K_\pi$ and $K^\nr$ are disjoint.}
s.t.\begin{itemize}
    \item $\pi\mapsto \frob_q$,
    \item $\O_K^\times\ni u\mapsto g\in\gal(K_\pi/K)$ s.t. $\chi_\pi(g) = \chi_\pi(\art_K(u)) = u^{-1}$.
\end{itemize}
\begin{theorem}
    [Local Class Field Theory]
    \begin{enumerate}
        \item [(1)] $K^\ab := K_\pi K^\nr$ is the maximal abelian extension of $K$.
        \item [(2)] $\art_K : K^\times\to K^\ab$ is independent of all choices.
        \item [(3)] If $L/K < \infty$, then the Artin map induces \[K^\times/N_{L/K}(L^\times)\simeq\gal(L/K),\]
        which gives a bijection\footnote{
            In particular, all open subgroups of $K^\times$ are norm of some $L^\times$.
        }\[\{\text{open subgroup of }K^\times\} = \{\text{finite extension of }K\}.\]
        \item [(4)] If $L/K < \infty$,
        then % https://q.uiver.app/#q=WzAsNCxbMCwwLCJMXlxcdGltZXMiXSxbMSwwLCJcXGdhbChMXlxcYWIvTCkiXSxbMSwxLCJcXGdhbChLXlxcYWIvSykiXSxbMCwxLCJLXlxcdGltZXMiXSxbMCwxLCJcXGFydF9LIl0sWzEsMiwiXFxtYXRocm17cmVzfSJdLFswLDMsIk5fe0wvS30iLDJdLFszLDIsIlxcYXJ0X0wiLDJdXQ==
        \[\begin{tikzcd}
            {L^\times} & {\gal(L^\ab/L)} \\
            {K^\times} & {\gal(K^\ab/K)}
            \arrow["{\art_K}", from=1-1, to=1-2]
            \arrow["{N_{L/K}}"', from=1-1, to=2-1]
            \arrow["{\mathrm{res}}\footnotemark", from=1-2, to=2-2]
            \arrow["{\art_L}"', from=2-1, to=2-2]
        \end{tikzcd}\]commutes.
        \footnotetext{Here \[\res : \gal(L^\ab/L)\hookrightarrow\gal(L^\ab/K)\twoheadrightarrow \gal(K^\ab/K).\]}
    \end{enumerate}
\end{theorem}

\begin{corollary}\label{norm of lubin-tate char = cycl char * unramified char}
    $\exists$ unramified charater $\eta : G_K = \gal(\bar\Q_p/K)\to\Z_p^\times$, s.t.\[\forall g\in G_K,\ N_{K/\Q_p}(\chi_\pi(g)) = \chi_\cyc(g)\eta(g).\]
\end{corollary}
We say a charater $\eta$ on $G_K$ is \textbf{unramified}, if it restricts to the trivial charater on the inertia subgroup $I_K = I(\bar\Q_p/K)$.
That is, $\eta$ is lifted from a charater on $\gal(K^\nr/K)\simeq\gal(\bar k/k)\simeq G_K/ I_K$.
\begin{proof}
    We construct this charater $\eta$ on the dense subgroup \[\im(\art_K) = \gene{\frob_q}\times\gal(K_\pi/K)\] first.
    Let $g\in\gal(\bar\Q_p/K)$ with \[g|_{K^\nr} = \frob_q^n\] for $n(g)\in\Z$ so that $g\in \im(\art_K)$.
    Write $q = p^f$, and note that \[\frob_q|_{\Q_p^\nr} = \frob_p^f,\]
    Then we have the commutative diagram% https://q.uiver.app/#q=WzAsNCxbMCwwLCJcXHBpXntuKGcpfVxcY2hpX1xccGkoZyleey0xfSJdLFsyLDAsImc9Il0sWzIsMSwiZ3xfe1xcUV9wXlxcYWJ9PSJdLFswLDEsIlxcbGVmdChOX3tLL1xcUV9wfVxccGlcXHJpZ2h0KV57bihnKX0gTl97Sy9cXFFfcH1cXGxlZnQoXFxjaGlfXFxwaShnKV57LTF9XFxyaWdodCkge1xcY29sb3J7Z3JlZW59ID19ICJdLFsxLDAsIiIsMCx7InN0eWxlIjp7InRhaWwiOnsibmFtZSI6Im1hcHMgdG8ifX19XSxbMSwyLCIiLDIseyJzdHlsZSI6eyJ0YWlsIjp7Im5hbWUiOiJtYXBzIHRvIn19fV0sWzAsM10sWzIsMywiIiwyLHsic3R5bGUiOnsidGFpbCI6eyJuYW1lIjoibWFwcyB0byJ9fX1dXQ==
    \[\begin{tikzcd}
        {\pi^{n(g)}\chi_\pi(g)^{-1}} &&
        {g=\left( \frob_q^{n(g)}, g \right)} \\
        {\left(N_{K/\Q_p}\pi\right)^{n(g)} N_{K/\Q_p}\left(\chi_\pi(g)^{-1}\right)
        {\color{orange} =}
        \ p^{fn(g)}\chi_\cyc(g)^{-1}}
        && {g|_{\Q_p^\ab}= \left( \frob_p^{fn(g)}, g \right)}
        \arrow[from=1-1, to=2-1]
        \arrow[maps to, from=1-3, to=1-1]
        \arrow[maps to, from=1-3, to=2-3]
        \arrow[maps to, from=2-3, to=2-1]
    \end{tikzcd}\]
    and we thereby find \[N_{K/\Q_p}\left( \chi_\pi(g) \right) = \left( 
        \frac{N_{K/\Q_p}\pi }{p^f }
    \right)^{n(g)}\chi_\cyc(g)\]
    and $\eta(g) := N_{K/\Q_p}(\chi_\pi(g))/\chi_\cyc(g)$ indeed defines an unramified character on $\im(\art_K)$.
    Hence it is unramified also on $G_K$.
\end{proof}

\subsection{The Case of \texorpdfstring{$\Q_p$}{}}
Let $K = \Q_p$ and $\varpi = p$. Then $f(T) := (1 + T)^p - 1\in\mathcal{F}_p$.
Note that $f$ is an endomorphism of \[\Gm(X, Y) = X + Y + XY,\] so $F_f = \Gm{}_{/\Z_p}$. Under the isomorphism
\[(\mathfrak{m}, +_{\Gm})\simeq (1 + \mathfrak{m},\ \cdot\ ),\]
the endomorphism $f : a\mapsto (1 + a)^p - 1$ is converted to the Frobenius map $a\mapsto a^p$.

\subsubsection*{The field \texorpdfstring{$(\Q_p)_p = \Q_p(\mu_{p^\infty})$}{Qpp = Qp mup infty}}

For each $r\ge 1$, the $p^r$-torsion part of $\Lambda_f$ is
\[\Lambda_{f, r} = \left\{\alpha\in\Q_p^\alg\left|(1 + \alpha)^{p^r} = 1\right.\right\}\simeq
\left\{\zeta\in(\Q_p^\alg)^{\times}
\left|\zeta^{p^r} = 1\right.\right\} = \mu_{p^r}.\]
The isomorphism is for $\O_K$-modules.
So choose primitive $p^r$-th roots of unity $\zeta_{p^r}$ s.t. $\zeta_{p^r}^p = \zeta_{p^{r-1}}$,
then $\varpi_r := \zeta_{p^r} - 1$ forms a sequence of compatible generators of $\Lambda_{f, r}$.
Therefore \[(\Q_p)_{p, r} = \Q_p(\varpi_r) = \Q_p(\mu_{p^r}),\]
and the Lubin-Tate extension of $\Q_p$ given by uniformiser $p$ is $(\Q_p)_p = \Q_p(\mu_{p^\infty})$,
the cyclotomic extension.

\subsubsection*{The local Artin map \texorpdfstring{$\phi_p : \Q_p^\times\to \gal(\Q_p^\ab/\Q_p)$}{}}

It suffices to look at every \[\phi_p : \Q_p^\times\to \gal(\Q_p(\mu_n)/\Q_p).\]
\begin{itemize}
    \item If $n$ is prime to $p$, then $\Q_p(\mu_n)/\Q_p$ is unramified of degree $f$, where $f$ is the minimum natural number s.t. $m\mid p^f - 1$.
    The map $\phi_p$ sends $up^t$ to the $t$-th power of Frobenius-$p^f$ on $\Q_p(\mu_n) = \Q_p(\mu_{p^f - 1})$, and $\ker\phi_p = (p^{f})^{\Z}\times\Z_p^\times$.
    \item If $n = p^r$, then $\Q_p(\mu_{p^r})/\Q_p$ is totally ramified. The map $\phi_p$ sends $up^t$ to the element sending a root of unity $\zeta$ to $\zeta^{\bar u^{-1}}$, where $\bar u\in\Z$ has the same residue modulo $p^r$ as $u$.
    The kernel is $p^\Z\times (1 + p^r\Z_p)$.
    \item In general, let $n = p^r\cdot m$ with $p\nmid m$. Then $\Q_p(\mu_n) = \Q_p(\mu_{p^r})\Q_p(\mu_m)$, and $\Q_p(\mu_{p^r})\cap\Q_p(\mu_m) = \Q_p$.
\end{itemize}



