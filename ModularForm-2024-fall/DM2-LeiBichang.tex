\documentclass{article}
\usepackage{amsmath, amssymb, amsthm, amsbsy, mathrsfs, stmaryrd}
\usepackage{enumitem}
\usepackage[colorlinks,
linkcolor=cyan,
anchorcolor=blue,
citecolor=blue,
]{hyperref}
\usepackage[capitalize]{cleveref}
\usepackage[margin = 1in, headheight = 12pt]{geometry}
\usepackage{bbm}
\usepackage{tikz-cd}

\newtheorem{theorem}{Theorem}

\theoremstyle{definition}
\newtheorem{definition}{Definition}
\newtheorem{exercise}{Exercise}[section]
\newtheorem{problem}{Problem}
\newtheorem{example}{Example}
\newtheorem{proposition}{Proposition}
\newtheorem{lemma}{Lemma}
\newtheorem{corollary}{Corollary}[section]

\theoremstyle{remark}
\newtheorem*{remark}{Remark}

\renewcommand{\Re}{\mathop{\mathrm{Re}}}
\renewcommand{\Im}{\mathop{\mathrm{Im}}}

% 新命令
% 数学对象
    \newcommand{\R}{\mathbb{R}}
    \newcommand{\C}{\mathbb{C}}
    \newcommand{\Q}{\mathbb{Q}}
    \newcommand{\Z}{\mathbb{Z}}
    \DeclareMathOperator{\GL}{GL}
    \DeclareMathOperator{\SL}{SL}
    \newcommand{\p}{\mathfrak{p}}
    \renewcommand{\P}{\mathbb{P}}
    \newcommand{\A}{\mathbb{A}}
    \newcommand{\F}{\mathbb{F}}
% 集合
    \newcommand{\sminus}{\smallsetminus} %(集合)差
% 范畴
    \newcommand{\op}[1]{{#1}^{\mathrm{op}}} %反范畴
    \DeclareMathOperator{\enom}{End} %自态射
    \DeclareMathOperator{\isom}{Isom} %同构
    \DeclareMathOperator{\aut}{Aut} %自同构
    \DeclareMathOperator{\im}{im} %像
    \newcommand{\Set}{\mathbf{Set}} %集合范畴
    \newcommand{\Abel}{\mathbf{Ab}} %群范畴
    \newcommand{\Ring}{\mathbf{Ring}}
    \newcommand{\Cring}{\mathbf{CRing}}
    \newcommand{\Alg}{\mathbf{Alg}}
    \newcommand{\Mod}{\mathbf{Mod}}
    \DeclareMathOperator{\Id}{id}
%向量空间, 矩阵
    \DeclareMathOperator{\rank}{rank} %秩
    \DeclareMathOperator{\tr}{Tr} %迹
    \newcommand{\tran}[1]{{#1}^{\mathrm{T}}} %转置
    \newcommand{\ctran}[1]{{#1}^{\dagger}} %共轭转置
    \newcommand{\itran}[1]{{#1}^{-\mathrm{T}}} %逆转置
    \newcommand{\ictran}[1]{{#1}^{-\dagger}} %逆共轭转置
    \DeclareMathOperator{\codim}{codim} %余维数
    \DeclareMathOperator{\diag}{diag} %对角阵
    \newcommand{\norm}[1]{\left\| #1\right\|} %范数
    \DeclareMathOperator{\lspan}{span} %张成
    \DeclareMathOperator{\sym}{\mathfrak{Y}}
% 群
    \DeclareMathOperator{\inn}{Inn} %(群)内自同构
    \newcommand{\nsg}{\vartriangleleft} %正规子群
    \newcommand{\gsn}{\vartriangleright} %正规子群
    \DeclareMathOperator{\ord}{ord} %元素的阶
    \DeclareMathOperator{\stab}{Stab} %稳定化子
    \DeclareMathOperator{\sgn}{sgn} %符号函数
% 环, 域
    \DeclareMathOperator{\cha}{char} %特征
    \DeclareMathOperator{\spec}{Spec} %素谱
    \DeclareMathOperator{\maxspec}{MaxSpec} %极大谱
    \DeclareMathOperator{\gal}{Gal}
% 微积分
    % \newcommand*{\dif}{\mathop{}\!\mathrm{d}} %(外)微分算子
% 流形
    \DeclareMathOperator{\lie}{Lie}
%代数几何
    \DeclareMathOperator{\proj}{Proj}
%多项式
    \DeclareMathOperator{\disc}{disc} %判别式
    \DeclareMathOperator{\res}{res} %结式

% 结构简写
    \newcommand{\pdfrac}[2]{\dfrac{\partial #1}{\partial #2}} %偏微分式
    \newcommand{\isomto}{\stackrel{\sim}{\rightarrow}} %有向同构
    \newcommand{\gene}[1]{\left\langle #1 \right\rangle} %生成对象
% 文字缩写
    \newcommand{\opin}{\;\mathrm{open\;in}\;}
    \newcommand{\st}{\;\mathrm{s.t.}\;}
    \newcommand{\ie}{\;\mathrm{i.e.,}\;}

% 重定义命令
\renewcommand{\hom}{\mathop{Hom}}
\renewcommand{\vec}{\boldsymbol}
\renewcommand{\and}{\;\text{and}\;}

% 编号
\newcommand{\cnum}[1]{$#1^\circ$} %右上角带圆圈的编号
\newcommand{\rmnum}[1]{\romannumeral #1}

\newcommand{\prim}{\mathrm{prim}}
\newcommand{\under}{\backslash}
\newcommand{\myit}{$\diamond$}
\DeclareMathOperator{\vol}{vol}

\title{Modular Forms}
\author{Lei Bichang}
\date{}

\begin{document}
\maketitle

\begin{enumerate}
\item It is equivalent to $[\Gamma_\infty : \Gamma_\infty^+]\le 2$. Let \[L_1 := \begin{pmatrix}
    1&\Z\\0&1
\end{pmatrix}, L_2 = \left\{ \left.\pm\begin{pmatrix}
    1&t\\0&1
\end{pmatrix}\right| t\in\Z \right\} = L_1\cup (-1)\cdot L_1,\]
then both $L_1$ and $L_2$ are subgroups of $\SL_2(\Z)$, and thus \[[\Gamma_\infty : \Gamma_\infty^+] = [\Gamma\cap L_2 : \Gamma\cap L_1]\le [L_2 : L_1] = 2.\]

\item 
Let $N > 2$.
Then $-1\not\equiv 1\pmod N$, so $\Gamma_1(N)\cap (-1)\cdot L_1 = \varnothing$ and thus $\Gamma_1(N)_\infty = \Gamma_1(N)_\infty^+$.
Since $-1\in\Gamma_0(N)_\infty$ and $-1\notin\Gamma_0(N)_\infty^+$, we know $[\Gamma_0(N)_\infty : \Gamma_0(N)_\infty^+] \ne 1$, so it equals $2$.


\item If $[\Gamma_\infty : \Gamma_\infty^+] = 2$, then there exists a $t\in\Z$ s.t. \[g := \begin{pmatrix}
    -1&t\\ &-1
\end{pmatrix}\in \Gamma.\]
Let $f\in M_k(\Gamma)$, then \[f(z) = f|_k g(z) = (-1)^{-k}f\left( z-t \right) = -f(z-t).\]
If $f = \sum_{n\ge 0}a_nq_N^n$ is the Fourier expansion of $f$ at infinity, then \[\sum_{n\ge 0}a_ne^{\frac{2\pi i n}{N}z} = -\sum_{n\ge 0}a_ne^{-\frac{2\pi i nt}{N}}e^{\frac{2\pi i n}{N}z}.\]
Comparing the terms gives \[f(\infty) = a_0 = 0.\]

\item Let \[\Z^2_\prim := \left\{(c, d)\in\Z^2\setminus \{(0, 0)\} | \gcd(c, d) = 1\right\}\] Take $g = \begin{pmatrix}
    a & b \\ c & d
\end{pmatrix}\in\Gamma$. Since $\det g = ad - bc = 1$, the integers $c$ and $d$ are coprime.
Then because \[\begin{pmatrix}
    1 & t \\ & 1
\end{pmatrix}g = \begin{pmatrix}
    a + tc&b+td\\c&d
\end{pmatrix},\] the map $\Gamma_\infty^+\backslash\Gamma\to \Z^2_\prim$ is well-defined.

If $g' = \begin{pmatrix}
    a'&b'\\c&d
\end{pmatrix}\in \SL_2(\Z)$,
then $a'd - b'c = 1$, so \[(a'-a)d = (b'-b)c.\]
Since $c, d$ are coprime, we have $c \mid (a'-a)$ and $d \mid (b'-b) $. Hence, \[t' := \frac{a'-a}{c} = \frac{b'-b}{d}\in\Z.\]
If $g'\in\Gamma$, then \[\begin{pmatrix}
    1& t'\\ &1
\end{pmatrix} = g'g^{-1}\in\Gamma_\infty^+,\]
i.e., $g'\in \Gamma_\infty^+ g$.
So the map $\Gamma_\infty^+\backslash\Gamma\to \Z^2_\prim$ is injective.


\item Let $G = G_{k,\Gamma, \infty}$. For all $g\in\Gamma$ and $z\in\mathcal{H}$,
\begin{align*}
    (G|_k g)(z) &= j(g, z)^{-k}G(gz)\\
    &= \sum_{h\in\Gamma_\infty^+\under\Gamma}j(g, z)^{-k}j(h, gz)^{-k}\\
    &= \sum_{h\in\Gamma_\infty^+\under\Gamma} j(hg, z)^{-k} = G(z).
\end{align*}

\item Let $G = G_{k,\Gamma, \infty}$.
If $[\Gamma_\infty : \Gamma_\infty^+] = 2$,
then we can write $\Gamma_\infty = \Gamma_\infty^+\sqcup\Gamma_\infty^+\gamma$ with \[\gamma = \begin{pmatrix}
    -1 & t \\ & -1
\end{pmatrix}\]for some $t\in \Z$.
Hence \[\Gamma = \bigsqcup_h \Gamma_\infty h = \bigsqcup_h \left(\Gamma_\infty^+h\sqcup\Gamma_\infty^+\gamma h\right),\]
and 
\begin{align*}
    G(z) &= \sum_{g\in\Gamma_\infty^+\under\Gamma} j(g, z)^{-k}\\
    &= \sum_{h\in\Gamma_\infty\under\Gamma} \left( j(h, z)^{-k} + j(\gamma h, z)^{-k} \right)\\
    &= \sum_{h\in\Gamma_\infty\under\Gamma} (1 + j(\gamma, hz)^{-k}) j(h, z)^{-k}.
\end{align*}
Since $j(\gamma, \tau) = - 1$ for all $\tau\in\mathcal{H}$, we get $G(z) = 0$ for all $z\in\mathcal{H}$ once $k$ were odd.


\item For $g = \begin{pmatrix}
    a&b\\c&d
\end{pmatrix}$, as $z\to i\infty$, $j(g, z) \to \infty$ if $c\ne 0$ and $j(g, z) = d = \pm 1$ if $c = 0$.
If $g\in\Gamma$, then $c = 0$ if and only if $g\in\Gamma_\infty$.
Hence \begin{align*}
    \lim_{z\to i\infty}G_{k, \Gamma, \infty}(z)
    &= \sum_{g\in\Gamma_\infty^+\under\Gamma} \lim_{z\to i\infty} j(g, z)^{-k}
    = \sum_{g\in \Gamma_\infty^+\under\Gamma_\infty} \lim_{z\to i\infty}j(g, z)^{-k}
    \\ &= \begin{cases}
        1, & [\Gamma_\infty : \Gamma_\infty^+] = 1,\\
        0, & [\Gamma_\infty : \Gamma_\infty^+] = 2 \text{ and } k \text{ is odd};\\
        2, & [\Gamma_\infty : \Gamma_\infty^+] = 2 \text{ and } k \text{ is even}.
    \end{cases}
\end{align*}
So $G_{k, \Gamma, \infty}$ is bounded at infinity.

\item We have\begin{align*}
    G_{k, \Gamma, \infty}|_kg(z)
    &= \sum_{h\in\Gamma_\infty^+\under\Gamma} j(hg, z)^{-k}.
\end{align*}
As we see in \textbf{7.}, $\lim_{z\to i\infty}j(hg, z)^{-k} = 0$ if and only if the matrix $hg$ has nonzero bottom-left term.
For each $h\in\Gamma$, since $hg\infty\in c = \Gamma\cdot g\infty$ and the cusp $c\ne \infty$, we know that $hg\infty\ne\infty$. Therefore $hg$ has nonzero bottom-left term, and thus
\[G_{k, \Gamma, \infty}|_kg(\infty)
= \sum_{h\in\Gamma_\infty^+\under\Gamma}
\lim_{z\to i\infty} j(hg, z)^{-k} = 0.\]



\item This follows from \textbf{5.}\! ($G_{k, \gamma, \infty}$ is a weak modular form of weight $k$), \textbf{7.}\! ($G_{k, \gamma, \infty}$ is bounded at infinity) and \textbf{8.}\! ($G_{k, \gamma, \infty}$ is bounded at all the cusps different from infinity).

\item 
To begin with, we note that:
\begin{lemma}
    If $f\in M_k(\Gamma)$ and $g\in\SL_2(\Z)$,
    then $f|_kg\in M_k(g^{-1}\Gamma g)$.
\end{lemma}
\begin{proof}
\begin{itemize}
    \item For all $\gamma\in\Gamma$, $(f|_kg)|_k(g^{-1}\gamma g) = f|_k(\gamma g) = (f|_{k}\gamma) |_k g = f|_kg.$
    \item For all $h\in\SL_2(\Z)$,$(f|_k g)|_k h = f|_k(gh)$ is bounded at infinity.
\end{itemize}
Hence $f|_kg\in M_k(g^{-1}\Gamma g)$.
\end{proof}
For simplicity, we use the following notation.
\begin{definition}
    For every $f\in M_k(\Gamma)$ and $g\in\SL_2(\Z)$, define \[f(g\infty)
    := (f|_k g)(\infty) = \lim_{z\to i\infty} f|_k g(z).\]
\end{definition}
We can verify some basic properties.
\begin{lemma}\label{value at cusp}
    Let $f\in M_k(\Gamma)$ and $g, h\in\SL_2(\Z)$.
    \begin{enumerate}
        \item $(f|_kg)(h\infty) = f(gh\infty)$.
        \item If $g\infty$ and $h\infty$ represent the same cusp of $\Gamma$,
        then $f(g\infty)$ and $f(h\infty)$ only differ by a sign, which is independent of $f$.
        In particular, if $\{g_1\infty, \dots, g_r\infty\}$ is a set of representatives of the cusps of $\Gamma$, then $f\in S_k(\Gamma)$ if and only if $f(g_1\infty) = \cdots = f(g_r\infty) = 0$.
    \end{enumerate}
\end{lemma}
\begin{proof}
    Property (a) is straightforward.
    For (b), suppose that $g\infty = \gamma h\infty$ for some $\gamma\in\Gamma$. Then $g^{-1}\gamma h\in\SL_2(\Z)_\infty$, so there is a $t\in \Z$ s.t. \[T := g^{-1}\gamma h = \begin{pmatrix}
        \pm1 & t \\ & \pm1
    \end{pmatrix}.\]
    Now\begin{align*}
        (f|_kh)(z) &= (f|_k(\gamma^{-1}gT))(z) = ((f|_kg)|_kT)(z)\\
        &= (\pm 1)^{-k}(f|_k g(z\pm t)).
    \end{align*}
    So $f(g\infty) = \pm f(h\infty)$, and the sign is determined by $g$ and $h$.
\end{proof}


Now let $\{g_1\infty, \dots, g_r\infty\}$ be fixed representatives of all the different cusps of $\Gamma$, where $g_1, \dots, g_r\in\SL_2(\Z)$.
For each $i \in \{1, \dots, r\}$,
the function $G_{k, g_i^{-1}\Gamma g_i, \infty}\in M_k(g_i^{-1}\Gamma g_i)$, so \[G_i := G_{k, g_i^{-1}\Gamma g_i, \infty}|_{k}g_i^{-1}\in M_k(\Gamma).\]
If $j\ne i$, then the cusp reprensented by $g_i^{-1}g_j\infty$ is not infinity, and thus
\begin{equation}\label{G_i g_j}
    G_i(g_j\infty) = \left( G_{k, g_i^{-1}\Gamma g_i, \infty}|_k (g_i^{-1}g_j) \right)(\infty) = 0
\end{equation} by \textbf{8.}

Now take $f\in M_k(\Gamma)$. 
We claim that \begin{equation}\label{minus Eisensteins gives cusp form}
    f_0 := f - \sum_{\stackrel{1\le i\le r}{G_i(g_i\infty)\ne 0}}\frac{f(g_i\infty)}{G_i(g_i\infty)} G_{i}\in S_k(\Gamma),
\end{equation}
and thereby proving that $S_k(\Gamma)$ together with $G_1, \dots, G_r$ generates $M_k(\Gamma)$.
By \cref{value at cusp}, it suffices to show for $1\le i\le r$,
\[f_0(g_i\infty) = f(g_i\infty) - \sum_{\stackrel{1\le j\le r}{G_j(g_j\infty)\ne 0}}\frac{f(g_j\infty)}{G_j(g_j\infty)} G_{j}(g_i\infty) = 0.\]
By \cref{G_i g_j}, this is true if \[f(g_i\infty)\ne 0\implies G_i(g_i\infty)\ne 0.\]
Since $f|_kg_i\in M_k(g_i^{-1}\Gamma g_i)$, then by \textbf{3.} and \textbf{7.}, \begin{align*}
    G_i(g_i\infty) = \left( G_{k, g_i^{-1}\Gamma g_i, \infty} \right)(\infty) = 0
    &\iff k\text{ is odd and } [(g_i^{-1}\Gamma g_i)_\infty : (g_i^{-1}\Gamma g_i)_\infty^+] = 2 \\
    &\implies f(g_i\infty) = (f|_kg_i)(\infty) = 0,
\end{align*}
which completes the proof.

\item 
Keep our notations in \textbf{10.}
Consider the $\C$-linear map \[ \iota:   M_k(\Gamma)\to \C^{|C_\Gamma|}
\] given by \begin{equation}\label{embed M_k(Gamma) in to S_k(Gamma) times somthing}
    f\mapsto (f(g_1\infty), \dots, f(g_r\infty)).
\end{equation}
From \cref{minus Eisensteins gives cusp form}, we deduce that $\ker\iota = S_k(\Gamma)$ and $\im\iota$ is generated by $\iota(G_1), \dots, \iota(G_r)$ because $M_k(\Gamma)$ is generated by $S_k(\Gamma)$ and $G_1, \dots, G_r$.

If $k$ is even, then $G_i(g_i\infty) \ne 0$ for all $i\in\{1, \dots, r\}$, and thus $\iota(G_i)$ is the vector in $\C^{|C_\Gamma|}$ whose $i$-th element is nonzero and other elements are zero. Therefore, $\iota(G_1), \dots, \iota(G_r)$ form a basis of $\C^{|C_\Gamma|}$. Hence $\dim M_k(\Gamma) = \dim S_k(\Gamma) + |C_\Gamma|$.

\item
Keep the notations in \textbf{11.}
When $k$ is odd, the image of $\iota$ is still generated by $\iota(G_i)$'s,
but \[\iota(G_i) \ne 0\iff [(g_i^{-1}\Gamma g_i)_\infty : (g_i^{-1}\Gamma g_i)_\infty^+] = 1,\]
and those nonzero $\iota(G_i)$'s are linearly-independent.
Therefore $\dim(\im\iota) = |C_{\Gamma}'|$,
and $\dim M_k(\Gamma) = \dim S_k(\Gamma) + |C_\Gamma'|$.

\item
Since the series $G_{k,\Gamma,\infty}$ is normally convergent on any $X_{A, B}$,
\begin{align*}
    \vol(\Gamma\under\mathcal{H})\left<f, G_{k, \Gamma, \infty}\right>
    &= \int_{\Gamma\under\mathcal{H}}f(z)\sum_{g\in\Gamma_\infty^+  \under\Gamma} \overline{j(g, z)^{-k}}y^{k-2}dxdy\\
    &= \sum_{g\in\Gamma_\infty^+  \under\Gamma}\int_{\Gamma\under\mathcal{H}} f(z)\overline{j(g, z)^{-k}}y^{k-2}dxdy,
\end{align*}
where we write $z = x+iy$.
Take a fundamental domain $D_\Gamma$ for $\Gamma$.
For each $g\in\Gamma$, since the volume form \[d\mu(z) := \frac{dxdy}{y^2}\] is $\SL_2(\R)$-invariant, so under the change of variable $z\mapsto g^{-1}\tau$,
\begin{align*}
    \int_{\Gamma\under\mathcal{H}} f(z)\overline{j(g, z)^{-k}}y^{k-2}dxdy
    &=\int_{D_\Gamma} f(z)\overline{j(g, z)^{-k}}(\Im z)^kd\mu(z)\\
    &= \int_{gD_\Gamma} f(g^{-1}\tau)\overline{j(g, g^{-1}\tau)^{-k}}(\Im g^{-1}z)^kd\mu(\tau)\\
    &= \int_{gD_\Gamma} f(\tau)j(g^{-1}, \tau)^k \overline{j(g^{-1}, \tau)^k} |j(g^{-1}, \tau)|^{-2k} (\Im\tau)^kd\mu(\tau)\\
    &= \int_{gD_\Gamma} f(\tau) (\Im\tau)^kd\mu(\tau),
\end{align*}
where we used $1 = j(1, \tau) = j(g, g^{-1}\tau)j(g^{-1}, \tau)$.
Because $\bigcup_{g\in\Gamma_\infty^+\under\Gamma} gD_\Gamma$ is a fundamental domain for $\Gamma_\infty^+$, \begin{align*}
    \left<f, G_{k, \Gamma, \infty}\right> &=
    \frac{1}{\vol(\Gamma\under\mathcal{H})}\sum_{g\in \Gamma_\infty^+\under\Gamma}\int_{gD_\Gamma} f(\tau)(\Im\tau)^kd\mu(\tau)\\
    &= \frac{1}{\vol(\Gamma\under\mathcal{H})}\int_{\Gamma_\infty^+\under\mathcal{H}} f(z)y^{k-2}dxdy.
\end{align*}
The group $\Gamma_\infty^+$ is a subgroup of $\begin{pmatrix}
    1&\Z\\ & 1
\end{pmatrix}$, so it is generated by $\begin{pmatrix}
    1&t\\ &1
\end{pmatrix}$ for some $t\in \Z$,
and therefore $\{z\in\mathcal{H} | 0\le\Re(z)\le t\}$ is a fundamental domain for $\Gamma_\infty^+$.
So \begin{align*}
    \int_{\Gamma_\infty^+\under\mathcal{H}} f(z)y^{k-2}dxdy
    &=\int_{0}^\infty y^{k-2} dy \int_0^N f(z)dx\\\
    &= \int_0^\infty y^{k-2} a_0 = 0,
    % &=\int_0^N dx\int_0^\infty f(z)y^{k-2}dy.
\end{align*}
where $a_0 = 0$ is the constant term of the $q$-expansion of $f\in S_k(\Gamma)$.
Hence $\left<f, G_{k,\Gamma,\infty}\right> = 0$.


\item 
The injective map \[\SL_2(\Z)_\infty^+\under\SL_2(\Z)\to\Z^2_\prim\]
is surjective, because for each $(c, d)\in\Z_\prim$, we can find $a, b\in\Z$ s.t. $ac - bd = 1$, which gives a matrix $\begin{pmatrix}
    a&b\\c&d
\end{pmatrix}\in\SL_2(\Z)$.
Therefore,\[G_{k, \SL_2(\Z), \infty} = \sum_{g\in\SL_2(\Z)_\infty^+\under\SL_2(\Z)}j(g, z)^{-k} = \sum_{(c, d)\in\Z^2_\prim} (cz+d)^{-k}.\]
% Every pair $(c, d)\in\Z^2\setminus\{0, 0\}$
Note that the map \[\Z^2_\prim\times \Z_{\ge1} \to \Z^2\setminus\{0, 0\}\quad ((c, d), u)\mapsto(cu, du)\] is bijective, whose inverse is given by $(c, d)\mapsto ((c/\gcd(c, d), d/\gcd(c, d)), \gcd(c, d))$. Hence,
the Eisenstein series\begin{align*}
    G_k(z) &= \sum_{(c, d)\in\Z^2 \setminus\{0, 0\}}(cz+d)^{-k}
    = \sum_{n\ge 1}n^{-k}\sum_{(c, d)\in\Z^2_\prim} (cz+d)^{-k}\\
    &= \zeta(k) G_{k, \SL_2(\Z), \infty}.
\end{align*}
So $G_{k, \SL_2(\Z), \infty} = 2E_k(z)$.




\end{enumerate}

\end{document}