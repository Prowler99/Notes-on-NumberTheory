%如果你有任何建议和疑问请发送邮件至qrx_math@ruc.edu.cn,协助我做的更好。
% 重要提示:
%   1. 请确保使用 UTF-8 编码保存
%   2. 请使用 XeLaTeX编译
%   3. 修改、使用、发布本文档请务必遵循 LaTeX Project Public License
%   4. 不需要的注释可以尽情删除
%*********************************************************************
\documentclass[12pt,UTF8]{ctexart}
\usepackage{amsmath, amssymb, amsthm, amsbsy, mathrsfs, stmaryrd}
\usepackage{tikz-cd}
\usepackage{float}
\usepackage{graphicx}
\usepackage{epstopdf}
\usepackage{booktabs}
\usepackage{multirow}
\usepackage{geometry}
\usepackage{appendix}
\usepackage{hyperref}
\usepackage[nameinlink]{cleveref}
\usepackage{enumerate}
\usepackage{threeparttable}
\usepackage{fancyhdr}
\usepackage{setspace}
\usepackage[T1]{fontenc}
% \usepackage{mathptmx}%Times New Roman字体
\usepackage{titletoc}
\usepackage{pdfpages}
\usepackage{tikz}
\usepackage{etoolbox}
\usepackage{xcolor}
\usepackage{caption}
\usepackage{array}
\usepackage{enumitem}
\usepackage{titlesec}



\usepackage{natbib}
\bibliographystyle{alpha}


\captionsetup{font={small},labelfont={bf}} 
\captionsetup[table]{skip=3pt}
\captionsetup[figure]{skip=3pt}
%圆形脚注
\usepackage{pifont}
\usepackage[perpage,symbol*]{footmisc}
\DefineFNsymbols{circled}{{\ding{192}}{\ding{193}}{\ding{194}}
{\ding{195}}{\ding{196}}{\ding{197}}{\ding{198}}{\ding{199}}{\ding{200}}{\ding{201}}}
\setfnsymbol{circled}

%图标编号按章节排序(可选)
% \renewcommand {\thetable} {\thesection{}.\arabic{table}}
% \renewcommand {\thefigure} {\thesection{}.\arabic{figure}}

%修改目录

\titlecontents{section} % set formatting for \section -
                        % \subsection must be formatted separately 
[2.3em]                 % adjust left margin
{\heiti}             % font formatting
{\contentslabel{2.3em}} % section label and offset
{\hspace*{-2.3em}}
{\titlerule*[1pc]{.}\contentspage}

\titleformat{\section}{\centering \zihao{3} \heiti \bfseries}{\thesection}{1em}{}

\titleformat{\subsection}{\flushleft \zihao{4} \heiti \bfseries}{\thesubsection}{1em}{}
\titleformat{\subsubsection}{\flushleft \zihao{-4} \heiti \bfseries}{\thesubsubsection}{1em}{}

\geometry{a4paper,left = 2cm,right = 1.5cm,top = 2cm,bottom= 2cm}


% 页眉:学校标志(教务处主页提供下载):
% 高度为0.98 cm,宽度为4.13 cm,居中放置。
% 页码采用10.5号宋体字,居中放置,格式为:第1页。
\pagestyle{fancy}
\lhead{}
\chead{\includegraphics[width = 4.13cm,height = 0.98cm]{head.png}}
\rhead{}
\lfoot{}
\rfoot{}

%目录和引用超链接
\hypersetup{
colorlinks=true,
linkcolor=black,
citecolor=cyan
}

% \tikzcdset{}

% 新定理

\newtheorem{theorem}{定理}[section]

\theoremstyle{definition}
\newtheorem{definition}{定义}[section]
\newtheorem{defprop}[definition]{定义-命题}
\newtheorem{example}[definition]{例}
\newtheorem{proposition}{命题}[section]
\newtheorem{lemma}[proposition]{引理}
\newtheorem{corollary}[proposition]{推论}

\theoremstyle{remark}
\newtheorem{remark}{注记}

\crefname{definition}{定义}{定义}
\crefname{proposition}{命题}{命题}
\crefname{example}{例}{例}
\crefname{lemma}{引理}{引理}
\crefname{corollary}{推论}{推论}
\crefname{defprop}{定义-命题}{定义-命题}
\crefname{theorem}{定理}{定理}
\crefname{equation}{}{}
\crefname{remark}{注记}{注记}

% 新命令
% 数学对象
    % 基础空间
    \newcommand{\N}{\mathbb{N}}
    \newcommand{\R}{\mathbb{R}}
    \newcommand{\Q}{\mathbb{Q}}
    \newcommand{\Z}{\mathbb{Z}}
    \newcommand{\C}{\mathbb{C}}
    % 常量
    % \newcommand{\e}{\mathrm{e}} %自然底数
    % 群
    \DeclareMathOperator{\GL}{GL}
    \DeclareMathOperator{\SL}{SL}
% 算符&映射&函子
    % 集合
    \newcommand{\sminus}{\smallsetminus} %(集合)差
    \newcommand{\inject}{\hookrightarrow} %单射
    \newcommand{\surject}{\twoheadrightarrow} %满射
    % 范畴
    \DeclareMathOperator{\id}{id}
    \newcommand{\op}{\mathrm{op}} %反范畴
    \DeclareMathOperator{\enom}{End} %自态射
    \DeclareMathOperator{\isom}{Isom} %同构
    \DeclareMathOperator{\aut}{Aut} %自同构
    \DeclareMathOperator{\im}{im} %像
    \newcommand{\set}{\mathbf{Set}} %集合范畴
    \newcommand{\abel}{\mathbf{Ab}} %群范畴
    \newcommand{\cring}{\mathbf{CRing}}
    \DeclareMathOperator{\Id}{id}
    %向量空间, 矩阵
    \DeclareMathOperator{\rank}{rank} %秩
    \DeclareMathOperator{\tr}{Tr} %迹
    \newcommand{\tran}[1]{{#1}^{\mathrm{T}}} %转置
    \newcommand{\ctran}[1]{{#1}^{\dagger}} %共轭转置
    \newcommand{\itran}[1]{{#1}^{-\mathrm{T}}} %逆转置
    \newcommand{\ictran}[1]{{#1}^{-\dagger}} %逆共轭转置
    \DeclareMathOperator{\codim}{codim} %余维数
    \DeclareMathOperator{\diag}{diag} %对角阵
    \newcommand{\norm}[1]{\left\| #1\right\|} %范数
    \DeclareMathOperator{\spec}{Spec} %谱
    \DeclareMathOperator{\lspan}{span} %张成
    \DeclareMathOperator{\sym}{Sym}
    % 群
    \DeclareMathOperator{\inn}{Inn} %(群)内自同构
    \newcommand{\nsg}{\vartriangleleft} %正规子群
    \newcommand{\gsn}{\vartriangleright} %正规子群
    \DeclareMathOperator{\ord}{ord} %元素的阶
    \DeclareMathOperator{\stab}{Stab} %稳定化子
    \DeclareMathOperator{\sgn}{sgn} %符号函数
    \DeclareMathOperator{\gal}{Gal}
    % 环, 域
    \DeclareMathOperator{\cha}{char} %特征
    % \DeclareMathOperator{\spec}{Spec} %素谱
    \DeclareMathOperator{\maxspec}{MaxSpec} %极大谱
    % 微积分
    % \newcommand*{\dif}{\mathop{}\!\mathrm{d}} %(外)微分算子
    % 流形
    \DeclareMathOperator{\lie}{Lie}
    % 代数几何
    \DeclareMathOperator{\spf}{Spf}
    % \DeclareMathOperator{\sp}{Sp}
    \DeclareMathOperator{\proj}{Proj}
% 结构简写
    \newcommand{\pdfrac}[2]{\dfrac{\partial #1}{\partial #2}} %偏微分式
    \newcommand{\isomto}{\stackrel{\sim}{\rightarrow}} %有向同构
    \newcommand{\gene}[1]{\left\langle #1 \right\rangle} %生成对象
% 文字缩写
    \newcommand{\opin}{\;\mathrm{open\;in}\;}
    \newcommand{\st}{\;\mathrm{s.t.}\;}
    \newcommand{\ie}{\;\mathrm{i.e.,}\;}

% 重定义命令
\renewcommand{\and}{\;\text{and}\;}

% 编号
\newcommand{\cnum}[1]{$#1^\circ$} %右上角带圆圈的编号
\newcommand{\rmnum}[1]{\romannumeral #1}

\renewcommand{\O}{\mathcal{O}}
\newcommand{\e}{\mathrm{e}}
\newcommand{\frp}{\mathfrak{p}}
\newcommand{\compl}{\mathbf{Compl}_\O}
\newcommand{\nilp}{\mathbf{Nilp}_\O}
\newcommand{\nilpnr}{\mathbf{Nilp}_{\O^\nr}}
\renewcommand{\P}{\mathbb{P}}
\newcommand{\A}{\mathbb{A}}
\DeclareMathOperator{\Hom}{Hom}
\newcommand{\cont}{\mathrm{cont}}
\newcommand{\F}{\mathbb{F}}
\newcommand{\alg}{\mathbf{Alg}}
\newcommand{\Mod}{\mathbf{Mod}}
\newcommand{\comaug}{\mathbf{ComplAug}}
\DeclareMathOperator{\rig}{rig}
\DeclareMathOperator{\Sp}{Sp}
\newcommand{\symm}[1]{\mathfrak{S}_{#1}}

\newcommand{\nil}{\mathbf{Nil}}
\newcommand{\nilaug}{\mathbf{NilAug}}
\newcommand{\E}{\mathbb{E}}
\newcommand{\fra}{\mathfrak{a}}

\newcommand{\myit}{$\diamond$}
\renewcommand{\Re}{\mathop{\mathrm{Re}}}
\renewcommand{\Im}{\mathop{\mathrm{Im}}}


\begin{document}
\renewcommand{\bar}{\overline}
\renewcommand{\tilde}{\widetilde}
\renewcommand{\hat}{\widehat}
\newcommand{\nr}{\mathrm{nr}}

\newcommand\dunderline[3][-1pt]{{%
  \setbox0=\hbox{#3}
  \ooalign{\copy0\cr\rule[\dimexpr#1-#2\relax]{\wd0}{#2}}}}

% 2.3.1首页
% 2.3.1.1论文编码: 12号黑体字,加粗,置顶,居右。
% 2.3.1.2文头:“中国人民大学本科毕业论文(设计)”,在论文编码下一行, 28号黑体字,加粗,居中。
% 2.3.1.3论文题名:论文文头下,隔一行,28号黑体字,加粗,居中。
% 2.3.1.4论文副题名:居中排印在论文题名下,20号黑体字,加粗,副题名前加特殊符号中“长划线”。
% 2.3.1.5作者、学院、专业、年级、学号、指导教师、论文成绩、日期:论文标题下隔六行,依次排印在论文副题名下,各占一行,距左端空5格,名称后用“:”, 20号黑体字,加粗,内容下需要加下划线,内容置于下划线中部,两端对齐。

\begin{titlepage}
\thispagestyle{fancy}
\cfoot{}
  \begin{flushright}
    \heiti \textbf{论文编码:RUC-BK-070701-2020201699}
  \end{flushright}

  \fontsize{28pt}{\baselineskip}\textbf{\heiti{中国人民大学本科毕业论文(设计)}}

  \vspace{24mm}
  \centering
  \fontsize{28pt}{\baselineskip}\textbf{\heiti $p$-进半平面与Drinfeld定理}

  \vspace{3mm}

  \begin{spacing}{1.2}
    \LARGE\selectfont{\textbf{\heiti }}
  \end{spacing}

  \vspace{64mm}

  \flushleft
  \begin{spacing}{1.3}
    \hspace{27mm}\heiti\LARGE\selectfont{\textbf{作\hspace{14mm}者:}\dunderline[-10pt]{1pt}{\makebox[78mm][c]{雷笔畅}}}
  
    \hspace{27mm}\heiti\LARGE\selectfont{\textbf{学\hspace{14mm}院:}\dunderline[-10pt]{1pt}{\makebox[78mm][c]{数学学院}}}

    \hspace{27mm}\heiti\LARGE\selectfont{\textbf{专\hspace{14mm}业:}\dunderline[-10pt]{1pt}{\makebox[78mm][c]{数学拔尖人才实验班}}}

    \hspace{27mm}\heiti\LARGE\selectfont{\textbf{年\hspace{14mm}级:}\dunderline[-10pt]{1pt}{\makebox[78mm][c]{2020级}}}

    % \hspace{27mm}\heiti\LARGE\selectfont{\textbf{学\hspace{14mm}号:}\dunderline[-10pt]{1pt}{\makebox[78mm][c]{2020201699}}}
    

    \hspace{27mm}\heiti\LARGE\selectfont{\textbf{指导教师:}\dunderline[-10pt]{1pt}{\makebox[78mm][c]{王善文}}}
    
    \hspace{27mm}\heiti\LARGE\selectfont{\textbf{论文成绩:}\dunderline[-10pt]{1pt}{\makebox[78mm][c]{A-(87)}}}
    
    \hspace{27mm}\heiti\LARGE\selectfont{\textbf{完成日期:}\dunderline[-10pt]{1pt}{\makebox[78mm][c]{2024年5月25日}}}
  \end{spacing}

  \vspace{25mm}

\end{titlepage}

\newpage
\begin{spacing}{1.625}
    \begin{center}
    {
        \heiti\zihao{3}
        \hspace*{\fill}
        
        \hspace*{\fill}
        
        中国人民大学学位论文原创性声明和使用授权说明
    }
    
        \zihao{4} \textbf{原创性声明}
    \end{center}
    \zihao{-4}
    
    本人郑重声明:所呈交的学位论文,是本人在导师的指导下,独立进行研究工作所取得的成果.除文中已经注明引用的内容外,本论文不含任何其他个人或集体已经发表或撰写过的作品或成果.对本文的研究做出重要贡献的个人和集体,均已在文中以明确方式标明.
    
    \hspace*{\fill}
    
    \hspace*{\fill}

    \begin{spacing}{2}
        \hspace{23em}论文作者签名:
    
        \hspace{27em}日期:\hspace{2em}年\hspace{2em}月\hspace{2em}日
    \end{spacing}

    \begin{center}
        \zihao{4}
        \hspace*{\fill}

        \hspace*{\fill}

        \hspace*{\fill}

        \textbf{学位论文使用授权说明}
    \end{center}
    \zihao{-4}

    \hspace*{\fill}
    
    本人完全了解中国人民大学关于收集、保存、使用学位论文的规定,即:
    \begin{itemize}
        \item[\ding{108}] 按照学校要求提交学位论文的印刷本和电子版本;
        \item[\ding{108}] 学校可以公布论文的全部或部分内容,可以采用影印、缩印或其他复制手段保存论文.
    \end{itemize}
      
    \hspace*{\fill}
    
    \hspace*{\fill}
    
    \begin{spacing}{2}
    \hspace{23em}论文作者签名:

    \hspace{23em}指导教师签名:
    
    \hspace{27em}日期:\hspace{2em}年\hspace{2em}月\hspace{2em}日
    \end{spacing}
    
\end{spacing}
\cfoot{}


% 中文摘要及关键词:
% “摘要”,单倍行距,段前1行,段后1行,16号黑体字,加粗,居中。
% 摘要内容, 12号宋体字,起行空两格,回行顶格, 1.5倍行距。
% “关键词”,摘要下隔一行,左端顶格,12号黑体字,加粗,后面加“:”。
% 关键词内容,直接放在“:”之后,词间间隔3格,12号宋体字。

\newpage
\renewcommand{\abstractname}{\textbf{\Large \heiti 摘要}}
\renewenvironment{abstract}{%
    \par\small
    \noindent\mbox{}\hfill{\bfseries \abstractname}\hfill\mbox{}\par
    \vskip 2.5ex}{\par\vskip 2.5ex} 
%中文摘要及关键词放在扉页一、外文摘要及关键词放在扉二,页码编排为Ⅰ,Ⅱ,设置页眉
\begin{abstract}
%1.5倍行距
\begin{onehalfspace}
设$D$是在素数$p$处分歧的$\Q$上的四元数代数. Cerednik-Drinfeld定理给出了相应于$D(\A_f)^\times$的紧开子群$U$的志村曲线的$p$-进单值化.
此定理最初由Cerednik证明; 而Drinfeld利用他的 ``基本定理'', 即$p$-进半平面$\Omega = \P^1_{\Q_p} - \P^1(\Q_p)$的一个形式模型$\hat{\Omega}$参数化了一族$p$-可除群,
重新导出了Cerednik的原始结果. 
为了证明Drinfeld的定理, 首先需要利用Deligne与Drinfeld的函子给出$\hat{\Omega}$的一个模诠释, 然后利用形式模的Cartier理论构造出Drinfeld定理中所需的同构.
这篇文章将主要跟随\cite{BC91}, 构造$p$-进半平面及其形式模型, 并补充其中部分细节, 随后参照\cite{Zi84}给出Cartier理论的主要内容, 最后陈述Drinfeld定理.
\\[12pt]
\textbf{\textbf{\heiti 关键词:}}$p$-进半平面\quad Cartier理论 \quad Drinfeld定理
\end{onehalfspace}
\end{abstract}
\setcounter{page}{1}
\pagenumbering{Roman}
\cfoot{\footnotesize \thepage}
\newpage


% 外文摘要及关键词:
% Abstract,单倍行距,段前1行,段后1行,16号Times New Roman字,加粗,居中。
% 内容使用12号Times New Roman字,起行空两格,回行顶格,两倍行距。
% Key Words,英文摘要内容下隔一行,左端顶格,12号Times New Roman字,
% 加粗,后面加“:”。
% 关键词内容,直接放在“:”之后,互相之间间隔3格,12号Times New Roman字。

\newcommand{\enabstractname}{\textbf{\Large Abstract}}
\newenvironment{enabstract}{%
    \par\small
    \noindent\mbox{}\hfill{\bfseries \enabstractname}\hfill\mbox{}\par
    \vskip 2.5ex}{\par\vskip 2.5ex}  
\begin{enabstract}
\begin{doublespace}
Let $D$ be a quaternion algebra over $\Q$, ramified at a prime $p$.
The Cerednik-Drinfeld theorem gives the $p$-adic uniformisation of Shimura curves associated to compact open subgroups $U$ of $D(\A_f)^\times$. This theorem was first proved by Cerednik. Then Drinfeld reinterpreted Cerednik's original result in another way using his theorem using his ``fundamental theorem'', which states that a formal model $\hat{\Omega}$ of the $p$-adic half plane $\Omega = \P^1_{\Q_p} - \P^1(\Q_p)$ parameterised a family of $p$-divisible groups.

To prove the Drinfeld's  theorem, the first step is to give a modular discription of $\hat{\Omega}$ through Deligne's and Drinfeld's functor. Then with the help of the Cartier theory on formal modules, one can construct the isomorphism in Drinfeld's theorem.
Following mainly to \cite{BC91}, this article will construct the $p$-adic half plane and its formal model with more details depicted. Then, the main contents of Cartier theory will be given with reference to \cite{Zi84}.
Finally, we state Drinfeld's theorem.
\\[12pt]
\textbf{Keywords:}$p$-adic half plane \quad  Cartier theory  \quad Drinfeld's theorem
\end{doublespace}
\end{enabstract}
\cfoot{\thepage}
\newpage
\cfoot{}

\renewcommand\contentsname{目录}
% \renewcommand\listfigurename{插\ 图}
% \renewcommand\listtablename{表\ 格}
\tableofcontents
%不需要图表目录可以删去下面两行
% \listoffigures
% \listoftables
\newpage
\setcounter{page}{1}
\pagenumbering{arabic}
\pagestyle{fancy}
\cfoot{\zihao{5} 第 \thepage 页}
% 按照自然段依次排列,每段起行空两格,回行顶格。12号宋体字,(重点文句,12号宋体字,加粗),1.25倍行距。
%正文、注释双面打印,编排页码,自第1页起,设置页眉。
\begin{spacing}{1.25}
\section{绪论}
对复上半平面$\mathcal{H} = \{\tau \in\C : \Im\tau > 0\}\subset \P^1(\C)-\P^1(\R)$的研究由来已久. 
作为复上半平面的$p$-进类比, $p$-进半平面$\Omega = \P^1_{\Q_p} - \P^1(\Q_p)$对于数论而言是同样重要的研究对象, 其应用之一是给出所谓的$p$-进单值化.

在复几何中, 某个或某类空间的单值化(uniformisation)通常指给出其万有覆叠.
例如, 熟知的黎曼单值化定理指出, 任何连通黎曼面一定同构于复平面$\A^1(\C)$, 复射影平面$\P^1(\C)$, 或者复上半平面$\mathcal{H}$的商. 在$p$-进的情形, 类似的结论被称为$p$-进单值化. 

为了研究$p$-进半平面, 首先需要建立一种$p$-进域上的解析理论: 刚性解析几何.
事实上, 最初正是Tate在研究$\Q_p$上的椭圆曲线及其单值化时发现了这种理论.
粗略地讲, 刚性解析几何理论关心刚性解析空间(rigid analytic spaces)范畴. 正如复流形由$\C^n$中的高维圆盘(polydisk)粘合而成, 刚性解析空间是由仿射胚子集(affnoid subset)粘合而成的; 这种子集是$\C_p^n$或$\Q_p^n$中的高维圆盘的推广. 例如, 射影空间$\P^1(\C_p)$中的仿射胚子集就是$\P^1$挖去有限个开圆盘. 具体的理论可以参考\cite{BS14}和\cite{FvdP}.

Raynaud发展了源自Tate的刚性解析几何, 并将其与形式概型的理论联系了起来. 对于环$A$及其理想$I$, 考虑$I$给出的进制拓扑以及$A$关于$I$-进拓扑的完备化$\hat{A} := \varprojlim A/I^n$.
我们可以在Zariski拓扑空间$\spec A/I$上装备环层\[D(f)\mapsto A\left<f^{-1}\right> := \varprojlim A/I^n[f^{-1}],\] 所得环层空间(ringed space)记作$\spf \hat{A}$.所谓的形式概型(formal scheme)便是那些局部上形如$\spf \hat{A}$的环层空间.
形式概型范畴与刚性解析空间范畴由函子$\rig$联系起来:
取$\Z_p$上形式概型$X$的刚性泛在纤维(rigid generic fibre)可以得到$\Q_p$上的刚性解析空间$X^{\rig}$.
% 可以在此放jjd的绝好图
如果形式概型$X$的刚性泛在纤维是刚性解析空间$Y$, 我们就称$X$是$Y$的形式模型(formal model).
在\cite{raynaud1974geometrie}中, Raynaud证明了每个可容许刚性解析空间(admissible rigid analytic space)都有形式模型, 并且在可容许爆破(admissible blow-up)的意义下唯一.
具体的理论可以参考\cite{BS14}.
% 而$\Q_p$上的刚性解析空间能够特殊化(specialise)到$\F_p$上的


1976年, \v{C}erednik在\cite{vcerednik1976uniformization}中利用$p$-进半平面给出了一族志村曲线的单值化.
设$D$是$\Q$上的四元数代数, $U$是$D(\A_f)^\times$的紧开子群, 其中$\A_f$是$\Q$的有限ad\'{e}le.
\v{C}erednik证明了以下结果: 如果$U$在$p$位置的分量$U_p$为极大紧子群,
则相应的志村曲线$S_U$基变换到$\Q_p$上所得曲线$S_U\otimes\Q_p$是一些$p$-进半平面关于$\mathrm{PGL}_2(\Q_p)$的离散子群的商之并; 这就是$S_U$的$p$-进单值化.
其后, Drinfeld在\cite{drinfel1976coverings}给出了他的 ``基本定理'', 以形式群的模空间重现了$p$-进半平面$\Omega$; 更准确地说, 他证明了$\hat{\Omega}\hat{\otimes}_{\Z_p}\hat{\Z}_p^\nr$参数化了一族带有$D$的整数环的作用的高度$4$的形式群,
其中$\hat{\Omega}$为$\Omega$的一个形式模型, $\hat{\Z}_p^\nr$是$\Z_p$的极大非分歧扩张的完备化.
利用此结果, Drinfeld重新证明了\v{C}erednik的单值化定理, 并且揭示了其后更丰富的结构. 这一单值化结果就此被称为\v{C}erednik-Drinfeld定理.
然而, Drinfeld的论文\cite{drinfel1976coverings}过于简短而难以阅读. 于是, Boutot与Carayol在\cite{BC91}中对一维情形更具体地解释了Drinfeld对\v{C}erednik-Drinfeld定理的证明.

本文主要关心的是Drinfeld的基本定理, 即$p$-进半平面的模诠释.
首先, Drinfeld使用了Deligne的函子来局部地描述$\hat{\Omega}$, 以此给出了$\hat{\Omega}$的一个模诠释. 随后, 利用形式群的Cartier理论, Drinfeld构造出了函子$\hat{\Omega}$到该模问题的自然变换, 并且证明其为同构, 从而完成基本定理的证明.

本文主要参考了\cite{BC91}. 
首先, 本文具体地构造和描述了$p$-进局部域$K$上的$p$-进半平面及其形式模型$\hat{\Omega}$. 随后, 本文回顾了形式群以及形式群的Cartier理论的主要结果. 最后, 本文陈述了Drinfeld定理并给出其中的主要构造.



% \section{绪论}


\subsection*{符号说明}
在这篇文章中, 固定素数$p$以及特征零而剩余类域特征$p$的非阿局部域$K$, 即$\Q_p$的有限扩张$K$.
记$\O := K^\circ$为$K$的整数环, $\varpi$为一个选定的素元(uniformizer), $k := \O/\varpi$为剩余类域, $q := \#k$为剩余类域的阶.
选定$K$的代数闭包$\bar{K}$及其完备化$C := \hat{\bar{K}}$.
局部域$K$上的范数由$|\varpi| = q^{-1}$规范, 并延拓至$C$上.


除非特别说明, 我们约定环为含幺环.

记集合范畴为$\set$, Abel群范畴为$\abel$, 交换环范畴为$\cring$. 交换环$A$上的模范畴记作$\Mod_A$, 含幺代数范畴记作$\alg_A$.

对环$A$, 记$A^\times$为其单位群, $A^\op$为其反环. 如果$A$是交换环(相应地, 分次环), $M$是$A$-模(相应地, 分次$A$-模), 由$M$的给出的$\spec A$ (相应地, $\proj A$)上拟凝聚层记作$\tilde{M}$.

对拓扑空间$X$, 集合(或群, 环, 代数等) $G$给出的$X$上常值层记作$\underline{G}_X$或在底空间明确时略去下标.

对环层空间(ringed space) $X$, 记其结构层(structure sheaf)为$\mathscr O_X$, 底空间为$|X|$(或在含义清楚时以$X$代替).

对于范畴$\mathfrak{C}$, 以$x\in\mathfrak{C}$表示$x$是$\mathfrak{C}$的对象.
记$x\in\mathfrak{C}$的恒等态射为$\Id_x$.

对于集合$X$上的等价关系$\sim$, 记商映射$X\surject X/\sim$为$x\mapsto [x]$.
\section{$p$-进半平面的构造}

设$K$是$p$-进局部域, $C$是$K$的代数闭包的完备化.
$K$上的$p$-进半平面是一个$K$上的刚性解析空间$\Omega$,
其$C$-点在集合意义上等于$\P^1(C)-\P^1(K)$.
在这一节中, 我们将首先构造$\mathrm{PGL}_2(K)$的Bruhat-Tits树$I$及其几何实现$I_\R$, 以此构造出$\Omega(C) = \P^1(C) - \P^1(K)$上的刚性解析结构.
然后, 我们通过粘合局部信息, 构造出$\Omega$的一个形式模型$\hat{\Omega}$.

\subsection{$\mathrm{PGL}_2(K)$的Bruhat-Tits树}

\subsubsection{定义}
有限维$K$-向量空间$V$中的\textbf{格(lattice)}指$V$的满秩自由子$\O$-模.
同一向量空间中的两个格$M$与$M'$称为是\textbf{位似的(homothetic)}, 如果存在$\lambda\in K^\times$, 使得$M' = \lambda M$.
位似是一个等价关系.
\begin{definition}
    群$\mathrm{PGL}_2(K)$的\textbf{Bruhat-Tits树(Bruhat-Tits tree)}是无向图$I$, 定义如下:
    \begin{enumerate}
        \item [\myit] 顶点之集合为全体$K^2$中格的位似类. 格$M\subset K^2$对应的顶点记作$[M]$.
        \item [\myit] 顶点$s$与$s'$被一条边$[s, s']$连接, 当且仅当存在$s$的代表元$M$和$s'$的代表元$M'$, 满足$\varpi M \subsetneq M'\subsetneq M$.
    \end{enumerate}
\end{definition}

设$s' = [M']$与$s = [M]$相邻, 则选取代表元可以使得$\varpi M\subsetneq M' \subsetneq M$,
即\[0\subsetneq M'/\varpi M\subsetneq M/\varpi M\simeq k^2,\]
因此每个与$s$相邻的顶点对应着二维$k$-向量空间中$k^2$的一条直线,
或$\P^1(k)$中的一个点. 容易看出这是一个双射, 因此与一个顶点邻接的顶点数或与其相连的边数总是$q+1$.


\subsubsection{$I$的几何实现}
按定义, 图$I$的几何实现$I_\R$是向$I$的每一条边$[s, s']$指定一条线段\[\{ts + (1-t)s' : 0\le t\le 1\}\]所得到的对象; 上式中的加法看作形式和.
我们可以将$I_\R$与二维$K$-向量空间$K^2$中$K$-范数的$K^\times$-数乘等价类等同起来;
这里我们称$K^2$中的两个$K$-范数$|\cdot|$与$|\cdot|'$是$K^\times$-数乘等价的, 如果存在$\lambda\in K^\times$, 使得$|\cdot| = \lambda|\cdot|'$.

\begin{enumerate}
    \item 对于顶点$s = [M]$, 定义范数$|\cdot|_M$为以$M$为单位球的$K^2$中范数.
    具体地, 如果$M = \O e_1 + \O e_2$, 则\[|a_1e_1 + a_2e_2|_M := \max\{|a_1|, |a_2|\}.\]
    对于不同的代表元, 这样定义出的范数自然也相差一个$K^\times$中元素的数乘.
    \item 设顶点$s = [M]$与$s' = [M']$相邻, 且$\varpi M\subset M'\subset M$.
    通过取$M/\varpi M$的$k$-基再提升回$M$中, 我们总是可以取得$M$的一组$\O$-基$e_1$, $e_2$, 使得$M' = \O e_1 + \O \varpi e_2$.
    于是对于$v = a_1e_1 + a_2e_2\in K^2$, \begin{align*}
        |v|_M &= \max\{|a_1|, |a_2|\},\\
        |v|_{M'} &= \max\{|a_1|, q|a_2|\}.
    \end{align*}
    对于边$[s, s']$中的点$x = (1-t)s + ts'$, $0\le t\le 1$, 我们定义$K^2$上的范数$|\cdot|_x$为\[|v|_x = |v|_t := \max\{|a_1|, q^t|a_2|\}.\]
    基于$K$上赋值的离散性, 我们看到\[\{v\in K^2 : |v|_t\le\lambda\} = \begin{cases}
        M, & q^t\le \lambda < q,\\
        M', & 1\le \lambda < q^t.
    \end{cases}\]
\end{enumerate}

反之, 设$|\cdot|$是$K^2$上的范数. 
首先注意到如果$|\cdot|$是$K^2$中的范数, 则闭球$M_\lambda := \{v\in K : |v|\le \lambda\}$对任何正实数$\lambda$都是$K^2$中的格,
% 并且$|\cdot|$由其单位球$\{v\in K : |v|\le 1\}$决定.
并且\[M_{\lambda'}\subset M_{\lambda}\iff \lambda' \le \lambda,\ \lambda,\lambda'\in\R_{>0}.\]
又因为$\varpi M_{\lambda} = M_{q^{-1}\lambda}$, 所以格$M_\lambda$的位似类对于不同正实数$\lambda$至多取两个值, 并且$\lambda\mapsto[ M_{\lambda}]$是周期的.
\begin{enumerate}
    \item 如果$[M_\lambda] = s$恒成立, $|\cdot|$自然对应着$|\cdot|_s$.
    \item 如果$[M_\lambda]$或者等于$s$, 或者等于$s'$, 则乘以适当的$K^\times$中元素后, \[M_\lambda = \begin{cases}
        M, & q^t\le \lambda < q,\\
        M', & 1\le \lambda < q^t.
    \end{cases}\]于是$|\cdot|$对应于$(1-t)s + ts'\in [s, s']$.
\end{enumerate}

\subsection{刚性解析空间$\Omega$}

记$\Omega := \P^1(C) - \P^1(K)$. 全体$K$-线性同态$K^2\to C$组成的空间
% \[\Hom_K(K^2, C)\isomto  C^2,\ f \mapsto (f(1, 0), f(0, 1)),\]
\[\Hom_K(K^2, C)\simeq\Hom_K(K^2, K)\otimes_K C\]
是二维的$C$-线性空间; 并且在此同构下, $K^2\subset C^2$的原像正是那些满足$f(0, 1)$与$f(1, 0)$在$K$上线性相关的同态$f$之集合.
因此存在自然的双射\[(\Hom_K(K^2, C) - \{0\})/C^\times\simeq \P^1(C),\]
并且$\P^1(K)$在此双射下的原像为秩为$1$的同态之集合.
于是$\Omega$作为集合可以与$K$-线性嵌入$K^2\inject C$之集合的$K^\times$-数乘等价类等同; 这里的数乘等价与范数的定义相似: 称$z, z':K^2\inject C$是$K^\times$-数乘等价的, 如果存在$\lambda\in K^\times$, 使得$z = \lambda z'$.
对于每个这样的嵌入$z : K^2\inject C$, 可以定义出$K^2$上的范数$|\cdot|_z:= |z(\cdot)|$,
由此定义出映射\[\lambda : \Omega\to I_\R,\ [z]\mapsto [|\cdot|_z].\]

\begin{proposition}\label{preimages of lambda are affinoid}
    固定$I$中相邻的顶点$s = [M]$和$s' = [M']$, 并且固定$M$的基$e_1, e_2$使得$M' = \O e_1 + \O \varpi e_2 $.
    对$\Omega$中的每个嵌入的等价类, 选取代表元$z : K^2\inject C$使得$z(e_2) = 1$, 则$z(e_1)\in C - K$;
    以$z\mapsto z(e_2) = \zeta$将$\Omega$与$C - K$等同.

    在上述选取下, 我们有:
    \begin{align*}
        \lambda^{-1}(s) &= B(0, 1) - \bigcup_{a\in\O/\varpi\O} B^\circ(a, 1),\\
        \lambda^{-1}(s') &= B(0, q^{-1}) - \bigcup_{b\in\varpi\O/\varpi^2\O} B^\circ(b, q^{-1}),\\
        \lambda^{-1}(x) &= \{\zeta \in C : |\zeta| = q^{-t}\},\ x = (1-t)s + ts',\ 0\le t\le 1,\\
        &\\
        \lambda^{-1}([s, s']) &= B(0, 1) - \bigcup_{a\in(\O/\varpi\O)^\times} B^\circ(a, 1) - \bigcup_{b\in\varpi\O/\varpi^2\O} B^\circ(b, q^{-1}). 
    \end{align*}
    其中$B(x, r) = \{x\in C : |x|\le r\}$, $B^\circ(x, r) = \{x\in C : |x| < r\}$.
\end{proposition}
\begin{proof}
    参见\cite[Chapter I, (2.3)]{BC91}.
\end{proof}
此命题说明任何Bruhat-Tits树的顶点和边在$\lambda$下的原像都是$\P^1_K(C)$中的仿射胚子集,
即$\P^1_K(C)$中有限个开圆盘的补集; 并且$\lambda^{-1}(s)$与$\lambda^{-1}(s')$均为$\lambda^{-1}([s, s'])$的开子集.
将所有边在$\lambda$下的原像沿相应顶点的原像粘合, 我们就得到了一个$K$上的刚性解析空间的$C$-点集, 它作为集合等于$\Omega(C)$.
我们称刚性解析空间$\Omega$为$K$上的\textbf{$p$-进半平面($p$-adic half plane)}.


\subsection{形式概形$\hat{\Omega}$}


% \subsubsection{形式概形$\hat{\Omega}_s$, $\hat{\Omega}_{[s, s']}$和$\hat{\Omega}$}
对于$K^2$中的格$M$, 我们可以定义相应的射影空间$\P(M)$. 选取$M$的一组基等价于固定同构$M\simeq\O^2$, 从而诱导同构$\P(M)\simeq\P^1_\O$.
而格之间的位似$M' = \lambda M$决定出唯一的同构$\P(M)\simeq\P(M')$, 因而我们可以任意选取$s = [M]$的代表元$M$, 定义$\P_s := \P(M)$.

令$\Omega_s$为$\P_s$去除其特殊纤维的有理点得到的开子概形, $\hat{\Omega}_s$为$\Omega_s$沿其特殊纤维的形式完备化.

\begin{proposition}\label{hatOmega - vertex}
    我们有形式概形的同构\[\hat{\Omega}_s \simeq \spf\O\left<T,\ \frac{1}{T^q - T}\right>.\]
\end{proposition}

\begin{proof}
    首先, $\Omega_s$的特殊纤维是$(\Omega_s)_k\simeq\mathbb{P}_{k}^1-\P_{k}^1(k) = \spec k[T, 1/(T^q-T)]$.
    其次, 选取$\P^1_s$的仿射开覆盖$U_0 = \spec \O[T]$和$U_1 = \spec \O[1/T]$, 使得$\infty\in\P^1_k(k)$对应到$(p, 1/T)\in U_1$.
    尽管\[\Omega_{s0} := U_0\cap \Omega_s = \A^1_\O - \A^1_k(k) = \A^1_K\cup (\A^1_k-\A^1_k(k))\]也不是仿射概形, 但随着$\A^1_K$在模可逆元$\varpi$时被消灭,\[\Omega_{s0}/\varpi^n = \Omega_{s0} \times_\O \O/\varpi^n = \spec \O/\varpi^n[T, 1/(T^q-T)]. \]
    故$\hat{\Omega_{s0}} = \spf \O\left<T, 1/(T^q-T)\right>$. 同理$\hat{\Omega_{s1}} = \spf\O\left<1/T, T^q/(1-T^{q-1})\right>$.
    这两片仿射空间相等, 并通过恒等映射粘合成$\hat{\Omega}_s = \spf\O\left<T, 1/T, 1/(T^{q-1}-1)\right>$.
\end{proof}

因此$\hat{\Omega}_s$的刚性泛在纤维$\Omega_s^{\rig}$同构于$\Sp K\left<T, 1/(T^q-T)\right>$,
其$C$点正是$\P^1_K(C) = \P^1_\O(\O_C)$去除那些不特殊化到$\P^1_k(k)$的点.
在\cref{preimages of lambda are affinoid}\;中的选取下, $\Omega_s^{\rig}(C) = \lambda^{-1}(s)$.

然后, 考虑邻接$s$的顶点$s' = [M']$, 它按下述方式定出$\P^1_s$的特殊纤维上的一个有理点:
选取代表元使得$\varpi M\subset M'\subset M$, 则满射\[M\otimes_\O k = M/\varpi M\surject M/M'\simeq k\]给出$\P(M) = \P_s$的特殊纤维的一个$k$点.
不妨将此闭点记作$s'$.
具体来说, 选取基使得\[M = \O e_1+\O e_2,\ M' = \O e_1 + \O \varpi e_2.\]
在等同\[\P_s(k) = (\P_s)_k(k) \isomto \P(k\bar{e_1}+k\bar{e_2}) = \P^1(k) = \left\{[a : b] = \frac{b}{a}\in \P^1(k) = k\cup{\infty}\right\}\]下,
$M'/\varpi = k\bar{e_1}\in\P_s(k)$对应到$[1 : 0] = 0$, 即$I := (\varpi, T_0)\in\proj\O[T_0, T_1]\simeq\P_s$.

令$\P_{[s, s']}$为$\P_s$沿$s'$的爆破.
命$\Omega_{[s, s']}$为$\P_{[s, s']}$去除其特殊纤维中$s'$以外的有理点所得开子概形, 再定义$\hat{\Omega}_{[s, s']}$为$\Omega_{[s, s']}$沿其特殊纤维的形式完备化.
% 参考\cref{eg: blow up of A^1_Zp}得到\[\P_{[s, s']} = \proj \bigoplus_{n\ge 0} I^n = ? \to \P_s.\]
% % 直观上, $\P_{[s, s']}$是两条分别对应$\P^1_s$和$\P^1_{s'}$的射影直线$\P^1_\O$交于一闭点.



% 另一方面, 沿$\P_{s'}$中由$s$定义的闭点爆破, 所得概形同构于$\P_{[s, s']}$. 

% 仿照\cref{hatOmega - vertex}的证明, 容易看出
\begin{proposition}\label{hatOmega - edge}
    我们有形式概形的同构\[\hat{\Omega}_{[s, s']} \simeq \spf\O\left.\left< T_0, T_1, \frac{1}{T_0^{q-1}-1}, \frac{1}{T_1^{q-1}-1} \right>\right/ (T_0T_1 - \varpi),\]
    并且$T_0$, $T_1$分别给出$\hat{\Omega}_s$和$\hat{\Omega}_{s'}$到$\hat{\Omega}_{[s, s']}$的开浸入.
\end{proposition}
\begin{proof}
    参考\cref{eg: blow up of A^1_Zp}\;和\cref{hatOmega - vertex}, 容易证明.
\end{proof}

最终, 沿\cref{hatOmega - edge}\;中的浸入粘合所有$\hat{\Omega}_{[s, s']}$, 我们就得到了$\O$上的形式概型$\Omega$, 其刚性泛在纤维的$C$-点等于$\Omega(C)$.

\section{$\hat{\Omega}$的模诠释}

\subsection{Deligne的函子}
记$\alg_\O$为交换$\O$-代数范畴. 我们考虑$\alg_\O$的以下两个子范畴:
\begin{enumerate}
    \item [\myit] $\varpi$-幂零$\O$-代数范畴$\nilp$, 其对象为$\varpi$在其中幂零的交换$\O$-代数, 态射为$\O$-同态;
    \item [\myit] 完备$\O$-代数范畴$\compl$, 其对象为$\varpi$-进完备交换$\O$-代数, 态射为连续$\O$-同态.
\end{enumerate}
易见$\nilp$是$\compl$的全子范畴, 而$\compl$可以看作$\nilp$的完备化.
我们将形式概型看作$\compl$上的函子, 详见附录\;\ref{sec: formal scheme as functor}.

对$I$的顶点$s = [M]$, 定义$\compl$上取值在集合范畴$\set$的函子$\mathcal{F}_s$如下. 对$R\in\compl$, 命$\mathcal{F}_s(R)$为二元对$(L, \alpha)$的同构类, 其中:
\begin{enumerate}
    \item [\myit] $L$为秩$1$的自由$R$-模, $\alpha : M\to L$为$\O$-模同态.
    \item [\myit] 对每个$x\in\spec R/\varpi$, 
    由于$\varpi$属于$x$对应的$R$中素理想$\frp$, 故从$\mathbf{k}(x) = R_\frp/\frp R_\frp$看出$\varpi M$落在$M\stackrel{\alpha}{\to} L\to L\otimes \mathbf{k}(x)$的核中,
    于是可以定义$\alpha_x : M/\varpi\to L\otimes_R \mathbf{k}(x)$; 我们要求$\alpha_x$为单射.
\end{enumerate}

\begin{proposition}\label{hatOmega eq F - vertex}
    函子$\mathcal{F}_s$由$\hat{\Omega}_s$表出.
\end{proposition}
\begin{proof}
    同态$\alpha$的条件表明对任意$u\in M - \varpi M$, $\alpha(u)\in L$不等于$0$, 从而是$L$的生成元.
    特别地, 这说明$\alpha\otimes\id_R : M\otimes_\O R \to L$为满射.
    因此\[(L, \alpha)\mapsto \alpha\otimes\id_R : M\otimes_\O R \twoheadrightarrow  L\]给出了函子的嵌入\[\mathcal{F}_s(R)\inject\hat{\mathbb{P}}_s(R) = \mathbb{P}_s(R) = \{\tilde{M}\otimes_{\O}\mathscr{O}_{\spec R}\surject\mathscr{L} : \mathscr{L}\text{ 为可逆}\ \mathscr{O}_{\spec R}\text{ -模}\}/\simeq.\]

    为了描述这个函子, 取$M$的一组基$e_1, e_2$. 则$\alpha(e_1), \alpha(e_2)$都是$L$的生成元,
    故$(L, \alpha)$的同构类由唯一的$\zeta\in R$使得$\alpha(e_1) = \zeta\alpha(e_2)$决定; 事实上还立刻看出$\zeta\ne 0$.
    不妨命$\alpha(e_2) = 1$.
    定义等价于\[M/\varpi = ke_1\oplus ke_2\to L\otimes_R\mathbf{k}(x)\simeq R\otimes_R\mathbf{k}(x) = \mathbf{k}(x),\quad e_1\mapsto \bar{\zeta},\ e_2\mapsto 1\]
    为单射, 即对所有$a\in k$, $\bar{\zeta} - a\cdot 1\ne 0\in \mathbf{k}(x)$.

    另一方面, 由\cref{fomal completion = restriction}\,和\cref{hatOmega - vertex}\,知, \[\hat{\Omega}_s(R) = \Hom_\O(\spf R, \hat{\Omega}_s)\simeq\Hom_\O(\spec R, \spec \O[T, 1/(T^q-T)]).\]
    注意到给出态射$\spec R\to \spec \O[T, 1/(T^q-T)]$等价于给出态射$\spec R\to\spec \O[T]$, 使得$\spec R$中的每个点$x$都不被映到$\A^1_k(k)\inject \spec\O[T]$中;
    即对于对所有$x\in \spec R$和$a\in k$, 不存在交换图\[
    \begin{tikzcd}
    R \arrow[d]   & {\O[T]} \arrow[d] \arrow[l] \\
    \mathbf{k}(x) & {k[T]/(T-a);} \arrow[l]     
    \end{tikzcd}\]
    因此, 给出态射$\spec R\to\spec\O[T]$等价于决定$T$在$\O[T]\to R$下的像$\zeta$, 而上述交换图不存在等价于$\zeta$在$\mathbf{k}(x)$中的像$\bar{\zeta}$满足$\bar{\zeta} - a\cdot 1\ne 0$, 对任何$a\in k$成立.
    
    如果$x\in \spec R[1/\varpi]\inject\spec R$, 则作为泛在纤维中的点, 其剩余类域$\mathbf{k}(x)$是$K$的扩张, 从而是特征零的域, 因此不存在态射$k\to\mathbf{k}(x)$.
    所以只需考察$\spec R/\varpi\inject \spec R$中的点; 而$x$在$\spec R$与$\spec R/\varpi$中的剩余类域相等.
    这正说明$\hat{\Omega}_s(R) = \mathcal{F}_s(R)$.
\end{proof}

对Bruhat-Tits树$I$的边$[s, s']$, 定义$\compl$上的函子$\mathcal{F}_{[s, s']}$如下. 取$s$与$s'$的代表元$M, M'$, 使得$\varpi M\subset M'\subset M$.
命$\mathcal{F}_{[s, s']}(R)$为六元组$(L, L', \alpha, \alpha', c, c')$的同构类, 其中:
\begin{enumerate}
    \item [\myit] $L$, $L'$为秩$1$的自由$R$-模; $\alpha : M \to L$, $\alpha' : M \to L'$为$\O$-模同态; $c : L\to L'$, $c' : L' \to L$为$R$-模同态.
    \item [\myit] 图\begin{equation}\label{diag : def edge functor}
    \begin{tikzcd}
    \varpi M \arrow[d, "\alpha/\varpi"] \arrow[r, hook] & M' \arrow[r, hook] \arrow[d, "\alpha'"] & M \arrow[d, "\alpha"] \\
    L \arrow[r, "c"]                                    & L' \arrow[r, "c'"]                      & L                    
    \end{tikzcd}\end{equation} 交换.
    \item [\myit] 对每个$x\in \spec R/\varpi$, \begin{align*}
        \ker[\alpha_x : M/\varpi\to L\otimes_R \mathbf{k}(x)]&\subset M'/\varpi M,\\
        \ker[\alpha_x' : M'/\varpi \to L\otimes_R \mathbf{k}(x)] &\subset \varpi M/\varpi M';
    \end{align*}
\end{enumerate}

\begin{proposition}\label{hatOmega eq F - edge}
    函子$\mathcal{F}_{[s, s']}$由$\hat{\Omega}_{[s, s']}$表出.
\end{proposition}
\begin{proof}
    选取基使得$M = \O e_1+\O e_2$, $M'=\O e_1+\O\varpi e_2$.
    % 条件\[\ker[\alpha_x : M/\varpi\to L\otimes_R \mathbf{k}(x)]\subset M'/\varpi M\]
    条件\begin{align*}
        \ker[\alpha_x : M/\varpi\to L\otimes_R \mathbf{k}(x)]&\subset M'/\varpi M,\\
        \ker[\alpha_x' : M'/\varpi \to L\otimes_R \mathbf{k}(x)] &\subset \varpi M/\varpi M'.
    \end{align*}等价于\begin{align*}
        \alpha_x : M/M' \simeq ke_2&\inject L\otimes_R \mathbf{k}(x)\simeq \mathbf{k}(x),\\
        \alpha'_x : M'/\varpi M\simeq ke_1 &\inject L\otimes_R \mathbf{k}(x)\simeq \mathbf{k}(x).
    \end{align*}
    即$\alpha(e_2)$为$L$的生成元, $\alpha'(e_1)$为$L'$的生成元.
    于是二元组$(L, \alpha)$和$(L', \alpha')$的同构类分别由唯一的$\zeta, \eta\in R$,
    使得\[\alpha(e_1) = \zeta\alpha(e_2),\ \eta\alpha(e_1) = \alpha'(e_2)\]决定.
    不妨命$\alpha(e_2) = 1\in L$, $\alpha'(e_1) = 1\in L'$.
    为了得到六元组, 只需添入交换图\cref{diag : def edge functor}的信息; 直接的验算表明它等价于\[c = \eta,\ c' = \zeta,\ \zeta\eta = \eta\zeta = \varpi.\]
    于是, 给出六元组的同构类归结为给出$\zeta, \eta\in R$, 使得$\zeta\eta = \varpi$, 且对所有$a, b\in k$, $\bar{\zeta} - a\cdot 1\ne 0\in\mathbf{k}(x)$, $\bar{\eta} - b\cdot 1\ne 0\in\mathbf{k}(x)$.
    而\[\hat{\Omega}_{[s, s']}(R) = \Hom_\O\left( \O\left.\left[ T_0, T_1, \frac{1}{T_0^{q-1}-1}, \frac{1}{T_1^{q-1}-1} \right]\right/(T_0T_1-\varpi) ,\ R\right).\]
    类似于\cref{hatOmega eq F - vertex}, 按定义展开即可看出$\hat{\Omega}_{[s, s']}(R) = \mathcal{F}_{[s, s']}(R)$.
\end{proof}

\subsection{Drinfeld的函子: 定义和陈述}
对任何$\O$-代数$R$, 定义\[R[\Pi] := R[X]/(X^2-\varpi)\]
并装备$\Z/2\Z$-分次: $R[\Pi]_0 = R$, $R[\Pi]_1 = R\Pi$.
\begin{definition}\label{def: functor F}
    定义$\nilp$上取值在集合范畴的函子$\mathcal{F}$如下.
    任取$B\in\nilp$, 记$S = \spec B$. 定义$\mathcal{F}(B) = \mathcal{F}(S)$为四元组$(\eta, T, u, r)$的同构类, 其中:\begin{enumerate}
        \item [\myit] $\eta = \eta_0\oplus\eta_1$为$S$上Zariski-可构造的平坦$\Z/2\Z$-分次$\O[\Pi]$-模.
        \item [\myit] $T = T_0\oplus T_1$为$\Z/2\Z$-分次$\mathscr{O}_S[\Pi]$-模, 满足齐次分支$T_0$和$T_1$皆为$S$上可逆层.
        \item [\myit] $u : \eta\to T$为$0$次$\O[\Pi]$-线性态射, 满足$u\otimes_\O\mathscr{O}_S : \eta\otimes_\O\mathscr{O}_S \inject T$为单射.
        \item [\myit] $r : \underline{K}^2\to\eta_0\otimes_\O K$为$K$-线性同构.
    \end{enumerate}
    
    并且这些资料被以下条件限制. 记$S_i$为$\Pi : T_i\to T_{i+1}$的零点集(zero locus).\begin{enumerate}
        \item [C1] $\eta_i|_{S_i} = \underline{\O}^2$.
        %,  即$\eta_i$在$S_i$上带有平凡的$\Pi$作用.
        \item [C2] 对每个$S$的几何点$x$, $u$诱导的映射$\eta_x/\Pi\eta_x\inject T(x)/\Pi T(x)$为单射.
        \item [C3] $\left.\bigwedge^2\eta_i\right|_{S_i} = \varpi^{-i}\left.\left( \bigwedge^2 \left( \Pi^i\, r\, \underline{\O}^2 \right) \right)\right|_{S_i}$.
    \end{enumerate}
\end{definition}

让我们初步观察该定义.\begin{enumerate}
    % \item $\O[\Pi]$是主理想整环, 因为多项式环$\O[X]$中的理想必为主理想或者形如$(\varpi, f(X))$,
    % 而后者在$\O[\Pi]$中的像等于$(X^2, f(X)) = (\gcd(X^2, f(X)))$的像, 为主理想.
    % \item 主理想整环上的模平坦当且仅当无挠. 于是由$\eta$为$\O[\Pi]$-平坦知, $\eta_i$是$\O$-平坦且无挠的,
    % 并且$\Pi : \eta_i\to \eta_{i+1}$为单射.
    % 同构$r$的存在则说明$\eta_0$的茎是有限生成$\O$-模, 故$\eta_0$的茎总是秩$2$的自由$\O$-模
    % \footnote[1]{这不蕴涵$\eta_0$为常值层, 因为茎上的态射不能保证粘合为层的态射.};
    % $\Pi : \eta_1\inject\eta_0$保证$\eta_1$的茎也同构于$\O^2$.
    % % \textbf{这是否说明C1总是成立? 同构的层是 ``同一个层'' 吗?}
    \item 给出层$\eta$与$T$上的$\Pi$作用和态射$u$等价于给出周期$2$的$\O$-模范畴中交换图\[
    \begin{tikzcd}
    \cdots \arrow[r, "\Pi"] & \eta_0 \arrow[r, "\Pi"] \arrow[d, "u_0"] & \eta_1 \arrow[r, "\Pi"] \arrow[d, "u_1"] & \eta_0 \arrow[r, "\Pi"] \arrow[d, "u_0"] & \cdots \\
    \cdots \arrow[r, "\Pi"] & T_0 \arrow[r, "\Pi"]                     & T_1 \arrow[r, "\Pi"]                     & T_0 \arrow[r, "\Pi"]                     & \cdots
    \end{tikzcd}\]
    \item 取$S$的仿射开覆盖$\{U_j = \spec R_j\}_j$使得可逆层$T_0$与$T_1$限制在$U_j$上同构于$\mathscr{O}_{U_j}$.
    于是$\Pi : T_i|_{U_j}\to T_1|_{U_j}$由元素$f_i\in R_j$给出, $i = 0, 1$. 按定义, 限制在$U_j$上, $\Pi = f_i$的零点集为\begin{align*}
        \{\frp\in\spec R_j : [{R_j}_{\frp}\ni 1\mapsto f_i\in {R_j}_{\frp}] = 0\} = \{\frp\in\spec R_j : f_i\in\frp\} = V(f_i),
    \end{align*}
    即$S_i\cap U_j = V(f_i)$. 作为态射, $\Pi^2 = \varpi$; 故作为元素, $f_0f_1 = \varpi$.
    由于$R_j$是$R$-代数, $\varpi$在$R$中幂零说明$\varpi$也在$R_j$中幂零, 所以\[(S_0\cap U_j)\cup(S_1\cap U_j) = V(f_0)\cup V(f_1) = V(\varpi) = \spec R_j.\]
    因此, $S = S_0\cup S_1$.
    % \item 取$S = \spec R$的几何点等价于取$\spec R/\varpi$的几何点.
\end{enumerate}

\begin{theorem}\label{hatOmega eq F}
    函子$\mathcal{F} : \nilp\to\set$由形式概形$\hat{\Omega}$表出.
\end{theorem}

\subsection{自然变换$\hat{\Omega}\to\mathcal{F}$的构造}

首先注意到以下事实.
取Bruhat-Tits树$I$的顶点$s = [M]$. 由于$\bigwedge^2 M$是$\bigwedge^2 K = K$的$\O$-子模, 一定存在$n\in\Z$使得$\bigwedge^2 M = \varpi^n\O$.
如果$\lambda M$是$s$的另一个代表元, 其中$\lambda = u\varpi^m$, $u\in\O^\times$, 则\[\bigwedge^2 \lambda M = \lambda^2\bigwedge^2 M = \varpi^{2m+n}\O,\]
故整数$n$的奇偶性无关代表元$M$的选取.
我们称顶点$s = [M]$是\textbf{奇的}(相应地, \textbf{偶的}), 如果上述整数$n$是奇数(相应地, 偶数).
又注意到, 如果$s' = [M']$是邻接$s$的顶点, 且$\varpi M\subset M'\subset M$, 则\[\varpi^2\bigwedge M\subset\bigwedge^2 M'\subset\bigwedge M^2,\]
因此$s'$的奇偶性与$s$相反.

此后我们总是选取代表元$M$, 使得$\bigwedge^2 M = \varpi^{-1}\O$或者$\bigwedge^2 M = \O$,
并且固定边$[s, s']$的定向, 使得$s$为奇而$s'$为偶.

\subsubsection{嵌入$\mathcal{F}_s\to\mathcal{F}$}
取$I$的顶点$s$和$B\in\nilp$.
先考虑$s$为奇顶点的情形. 对每个点$(L, \alpha)\in \mathcal{F}_s(R)$, 我们定义交换图\[
\begin{tikzcd}
{} \arrow[r] & \eta_0=\underline{M} \arrow[r, "\Pi=1"] \arrow[d, "u_0=\alpha"] & \eta_1=\underline{M} \arrow[r, "\Pi=\varpi"] \arrow[d, "u_1=\alpha"] & \eta_0=\underline{M} \arrow[r] \arrow[d, "u_0=\alpha"] & {} \\
{} \arrow[r] & T_0=\tilde{L} \arrow[r, "\Pi=1"]                         & T_1=\tilde{L} \arrow[r, "\Pi=\varpi"]                         & T_0=\tilde{L} \arrow[r]                         & {}
\end{tikzcd}\]
嵌入$M\inject K^2$诱导出同构$r : \underline{K}^2\isomto\underline{M}\otimes_\O K = \eta_0\otimes_{\O} K$.
显然四元组$(\eta, T, u, r)$适合\cref{def: functor F}\,中的类型要求;
我们来验证剩余的三个条件.
\begin{enumerate}
    \item [C1]
    层$\eta_0$和$\eta_1$均为常值层$\underline{M}$, 条件显然成立. 
    \item [C2]
    取$S$的几何点$x$.
    % 由于$\varpi$在$R$中幂零, 闭浸入$\spec R/\varpi\to \spec R$在集合上为满射.
    在$\eta_x = \eta_{0, x} \oplus\eta_{1, x} = M\oplus M$上, $\Pi$的作用由\[\Pi : M^2\to M^2,\ (m_0, m_1)\mapsto (\varpi m_1, m_0),\]
    给出, 于是商\[\eta_x/\Pi\eta_x = \frac{M\oplus M}{\varpi M\oplus M}\simeq M/\varpi M.\]
    类似地, $T(x) = T_x\otimes_{R}\mathbf{k}(x) = L^2\otimes_R\mathbf{k}(x)$, \[\Pi : T(x) \to T(x),\ (l_0, l_1)\otimes a\mapsto (\varpi l_1, l_0)\otimes a,\]
    商\[T(x)/\Pi T(x) = \frac{(L\oplus L)\otimes_R\mathbf{k}(x)}{(\varpi L\oplus L)\otimes_R\mathbf{k}(x)}\simeq \frac{\mathbf{k}(x)}{\varpi\mathbf{k(x)}}.\]
    因为$\mathbf{k}(x)$是$R\in\nilp$上的代数, $\varpi$也在域$\mathbf{k}(x)$中幂零, 故$\varpi \mathbf{k}(x) = 0$, $T(x)/\Pi T(x)\simeq \mathbf{k}(x)$.
    
    态射$u$诱导出映射\[u_x : \eta_x\to T(x),\ (m_0, m_1)\mapsto (\alpha(m_0), \alpha(m_1))\otimes 1,\]
    进而诱导出$M/\varpi M\simeq\eta_x/\Pi\eta_x \to \mathbf{k}(x)\simeq T(x)/\Pi T(x)$,
    这正是$\mathcal{F}_s$定义中的单射$\alpha_x : M/\varpi\inject L\otimes_R\mathbf{k}(x)\simeq\mathbf{k}(x)$.
    
    \item [C3]
    显然$S_0 = \varnothing$, 故$S_1 = S$. 观察$S$上任意一点$x$处的茎.
    由定义, $r(\O^2) = M = \eta_{0, x}$, 而$\Pi|_{\eta_0} = 1$; 由奇顶点的定义立刻看到C3成立.
\end{enumerate}

若$s$为偶顶点, 则对每个点$(L, \alpha)\in \mathcal{F}_s(R)$, 我们定义交换图\[
\begin{tikzcd}
{} \arrow[r] & \eta_0=\underline{M} \arrow[r, "\Pi=\varpi"] \arrow[d, "u_0=\alpha"] & \eta_1=\underline{M} \arrow[r, "\Pi=1"] \arrow[d, "u_1=\alpha"] & \eta_0=\underline{M} \arrow[r] \arrow[d, "u_0=\alpha"] & {} \\
{} \arrow[r] & T_0=\tilde{L} \arrow[r, "\Pi=\varpi"]                         & T_1=\tilde{L} \arrow[r, "\Pi=1"]                         & T_0=\tilde{L} \arrow[r]                         & {}
\end{tikzcd}\]
嵌入$M\inject K^2$诱导出同构$r : \underline{K}^2\isomto\underline{M}\otimes_\O K$.
验证与奇顶点的情形类似, 略去不表.

\subsubsection{嵌入$\mathcal{F}_{[s, s']}\to \mathcal{F}$}

取$I$的边$[s, s']$, $B\in\nilp$和$\mathcal{F}_{[s s']}(B)$中的点
\begin{equation}\label{embed-eq: pt of F edge}
\begin{tikzcd}
\varpi M \arrow[d, "\alpha/\varpi"] \arrow[r, hook] & M' \arrow[r, hook] \arrow[d, "\alpha'"] & M \arrow[d, "\alpha"] \\
L \arrow[r, "c"]                                    & L' \arrow[r, "c'"]                      & L.                   
\end{tikzcd}
\end{equation}
我们逐次构造如下.

以图\[\begin{tikzcd}
{} \arrow[r] & T_0=\tilde{L} \arrow[r, "\Pi=c"] & T_1=\tilde{L}' \arrow[r, "\Pi=c'"] & T_0=\tilde L \arrow[r] & {}
\end{tikzcd}\]定义$T$及其上的$\Pi$-作用.
于是$S_0$为$c : L'\to L$的零点集, $S_1$为$c' : L\to L'$的零点集. 令$U_0\subset S_0$收集所有使得$c'_x$可逆的$x\in S$, $U_1\subset S_1$收集所有使得$c_x$可逆的$x\in S$.

在$U_0$和$U_1$上, \cref{embed-eq: pt of F edge}分别退化为$\mathcal{F}_s(U_0)$和$\mathcal{F}_{s'}(U_1)$的点, 从而对应到$\mathcal{F}(U)$和$\mathcal{F}(U')$的点.
在$V := S - (U_0\cup U_1) = S_0\cap S_1$上, 我们以资料\[
\begin{tikzcd}
\underline{M}' \arrow[d] \arrow[r, hook] & \underline M \arrow[r, "\Pi=\varpi"] \arrow[d] & \underline M' \arrow[d] \\
\tilde L' \arrow[r, "c'"]                & \tilde L \arrow[r, "c"]                        & \tilde L'              
\end{tikzcd}
.\]
和$M'\inject K^2$定义$\mathcal{F}(V)$的一个点.

然后, 我们证明这三个点粘合为$\mathcal{F}(S)$的点$(\eta, T, u, r)$, 其中$T$已经被定义. 我们以交换图\[
    \begin{tikzcd}
        {M|_{U_0}} & {M|_{U_0}} & {M|_{U_0}} \\
        {M'|_V} & {M|_V} & {M'|_V} \\
        {M'|_{U_1}} & {M'|_{U_1}} & {M'|_{U_1}}
        \arrow[Rightarrow, no head, from=1-1, to=1-2]
        \arrow["\varpi", from=1-2, to=1-3]
        \arrow[hook, from=2-1, to=1-1]
        \arrow[hook, from=2-1, to=2-2]
        \arrow["\Id", from=2-1, to=3-1]
        \arrow["\Id", from=2-2, to=1-2]
        \arrow["\varpi", from=2-2, to=2-3]
        \arrow["\varpi", from=2-2, to=3-2]
        \arrow[hook, from=2-3, to=1-3]
        \arrow["\Id", from=2-3, to=3-3]
        \arrow["\varpi", from=3-1, to=3-2]
        \arrow[Rightarrow, no head, from=3-2, to=3-3]
    \end{tikzcd}
\]
定义\[\begin{tikzcd}
	{\eta_0} & {\eta_1} & {\eta_0}
	\arrow["\Pi", from=1-1, to=1-2]
	\arrow["\Pi", from=1-2, to=1-3]
\end{tikzcd}.\] 
特别地, $\eta_0|_{S_0} = M$而$\eta_1|_{S_1} = M'$. 上图的交换性又表明三个点中的$r$粘合为$r : \underline{K}^2\simeq \eta_0\otimes_\O K$.
最后, 定义$u_0|_{S_0} := \alpha'$, $u_0|_{U_1} := c^{-1}\alpha$, 并类似地定义$u_1$.
这就给出了嵌入$\mathcal{F}_{_[s, s']}\inject\mathcal{F}$.
粘合所有这些信息, 便得到$\hat{\Omega}\to\mathcal{F}$;
\cite[I, 5.6]{BC91}证明了此自然变换为同构.


% \subsection{$\hat{\Omega}$上的$\mathrm{PGL}_2(K)$作用}

\section{形式群与Cartier理论}

形式群在有些文献中被定义为满足类似群乘法性质的形式幂级数, 即形式群律;
有些文献中则将形式群看作一类群函子. 
本节将主要参考\cite{Zi84}, 采用函子的观点建立形式群的Cartier理论.
囿于篇幅限制, 陈述的大多数结论将不给出证明.

% 于是, 一个$n$维形式群律$G$给出一个态射\[\mu : A[[X_1,\dots, X_n]]\to A[[X_1, \dots, X_n, Y_1, \dots, Y_n]],\]
% 将$X_i$映至$G_i(X, Y)$;
% \cref{def: formal group}\;中的三条公理分别对应以下三张交换图: \[
% \begin{tikzcd}
%                                                     & {A[[X, Y]]}                                                    &                                           \\
%                                                     &                                                                &                                           \\
% {A[[X]]} \arrow[ruu, "1\otimes\epsilon"]            &                                                                & {A[[X]]} \arrow[luu, "\epsilon\otimes 1"] \\
%                                                     &                                                                &                                           \\
%                                                     & {A[[X]]} \arrow[ruu, "1"] \arrow[luu, "1"] \arrow[uuuu, "\mu"] &
% \end{tikzcd}\]\[
% \begin{tikzcd}
%                                        & {A[[X, Y, Z]]}                                 &                                        \\
%                                        &                                                &                                        \\
% {A[[X, Y]]} \arrow[ruu, "1\otimes\mu"] &                                                & {A[[X, Y]]} \arrow[luu, "\mu\otimes1"] \\
%                                        &                                                &                                        \\
%                                        & {A[[X]]} \arrow[luu, "\mu"] \arrow[ruu, "\mu"] &                                       
% \end{tikzcd}\]\[
% \begin{tikzcd}
% {A[[X, Y]]} \arrow[rr, "{X\mapsto Y,\ Y\mapsto X}"] &                                                  & {A[[X, Y]]} \\
%                                                     &                                                  &             \\
%                                                     &                                                  &             \\
%                                                     & {A[[X]]} \arrow[luuu, "\mu"] \arrow[ruuu, "\mu"] &            
% \end{tikzcd}\]

% 类似地, 形式群律的态射$\varphi : G\to H$定出坐标环之间的反向态射\[\varphi^* : A[[Y_1, \dots, Y_m]]\to A[[X_1, \dots, X_n]],\]
% 将$Y_i$映到$\varphi_i(X)$, 要求$\varphi^*$保持余态射: \[% 
% \begin{tikzcd}
% {A[[X, X']]}                  &  & {A[[Y, Y']]} \arrow[ll, "\varphi^*\otimes\varphi^*"]  \\
%                               &  &                                                       \\
%                               &  &                                                       \\
% {A[[X]]} \arrow[uuu, "\mu_G"] &  & {A[[Y]]} \arrow[uuu, "\mu_H"] \arrow[ll, "\varphi^*"]
% \end{tikzcd}\]

% \begin{definition}
%     设$\mu : A[[X]]\to A[[X, Y]]$是形式群律$G$的余态射.
%     其\textbf{不变导数(invariant derivation)}指坐标环$A[[X]]$的一个导数$D : A[[X]]\to A[[X]]$, 使得
%     \[% https://tikzcd.yichuanshen.de/#N4Igdg9gJgpgziAXAbVABwnAlgFyxMJZABgBoBmAXVJADcBDAGwFcYkQBBZZADUspABfUuky58hFACYK1Ok1bsuvfkJEgM2PASJlichizaJO3HqQAEATVXDRWiURn6ahxSeXnrtuTCgBzeCJQADMAJwgAWyQyEBwIJBl5I3YAHVTI5jVQiOjEAEYaeJjXBWMQABFskHCopEK4hMRyUpSTdMzq2ryWxsTW9xB89Ig8SPgLKsFKQSA
%     \begin{tikzcd}
%     {A[[X, Y]]}                 &  & {A[[X, Y]]} \arrow[ll, "1\otimes D"]        \\
%                                 &  &                                             \\
%                                 &  &                                             \\
%     {A[[X]]} \arrow[uuu, "\mu"] &  & {A[[X]]} \arrow[ll, "D"] \arrow[uuu, "\mu"]
%     \end{tikzcd}\]交换.
% \end{definition}

\subsection{形式群: 函子观点}
我们以幂零代数上的函子定义形式群, 并指出形式群律与作为函子的形式群的联系.

\subsubsection{形式群}
环$R$上的一个\textbf{幂零代数(nilpotent algebra)}指$R$-代数$N$, 使得存在某个自然数$r$, $N^r = 0$.
记$\nil_R$为$R$上幂零代数构成的范畴.

任何非零的幂零代数都不含幺, 但是我们可以将幂零$R$-代数范畴嵌入含幺$R$-代数的范畴中: 设$N$为幂零$R$-代数, 我们在$R\oplus N$上定义自然的乘法:\[(r_1, n_1)\cdot(r_2, n_2) := (r_1r_2, r_1n_2 + r_2n_1 + n_1n_2).\]
于是$R\oplus N\in\alg_R$. 注意到投影$R\oplus N\surject R$与嵌入$R\inject R\oplus N$的复合等于$\Id_R$,
于是我们可以具体描述$\nil_R$在$\alg_R$中的像.
\begin{definition}
    一个\textbf{增广$R$-代数(augmented $R$-algebra)}指含幺的$R$代数$A$并装备以\textbf{增广同态(augmentation)} $\epsilon : A\to R$, 使得同态的复合$R\to A\stackrel{\epsilon}{\to} R$为恒等同态$\Id_R$,
    其中第一个箭头为$A$的结构映射(structure map). 记$A^+ := \ker\epsilon$为增广$R$-代数$A$的\textbf{增广理想(augmentation ideal)}.
    称增广代数$A$是幂零的, 如果其增广理想$A^+$是幂零的. 记$\nilaug_R$为幂零的增广$R$-代数范畴.
\end{definition}
结合以上讨论, 容易看出范畴$\nil_R$与范畴$\nilaug_R$等价.

\begin{definition}
    一个$R$上的\textbf{光滑交换形式群(smooth commutative formal group)}, 简称\textbf{形式群(formal group)}, 指保持无穷直和的正合函子$G : \nil_R\to\abel$.
\end{definition}
最简单的两个例子是加法群\[\mathbb{G}_\mathrm{a} : N\mapsto (N, +)\]和乘法群\[\mathbb{G}_\mathrm{m} : N\mapsto (1 + N)^\times,\]
其中$(N, +)$表示$N$的加法群, 而$(1 + N)^\times$表示形如$1 + n, n\in N$的元素组成的集合连同显然的乘法.
下面的例子是$\mathbb{G}_\mathrm{m}$的推广.
\begin{example}
    设$S$为增广$R$代数. 我们定义$\nil_R$上的函子\[\mathbb{G}_\mathrm{m}S : N\mapsto (1 + S^+\otimes_R N)^\times.\]
    此函子保持直和, 且当$S$在$R$上平坦(flat)时正合, 从而是形式群.
    特别地, \[\Lambda_R :=\mathbb{G}_\mathrm{m}R[t] : N \mapsto \Lambda(N) = (1 + tN[t])^\times\]是形式群,
    并且在Cartier理论中发挥至关重要的作用.
\end{example}

现在我们考虑$\nil_R$范畴的 ``完备化''.
一个完备的增广$R$-代数指增广$R$-代数$A$连同一列理想降链$\{\mathfrak{a}_n\}$, 满足$\mathfrak{a}_1 = A^+$,
且$A$对这组理想给出的拓扑完备, 即$A\simeq\varprojlim A/\mathfrak{a_n}$.
完备的增广$R$代数连同其间的连续同态组成一个范畴, 记作$\comaug_R$.
我们有显然的嵌入$\nil_R\inject\comaug_R$, 且任何$\nil_R$上的函子$H$都能延拓到$\comaug_R$上:
\[H(A) := H(A^+) := \varprojlim H(A^+/\mathfrak{a}_n).\]
例如, $\Lambda_R$在$R[[X]]$上的取值$\Lambda_R(R[[X]])$为幂级数环$R[[X, t]]$中形如\[1 + \sum_{m,n\ge 1}b_{mn}X^mt^n\]的元素组成的集合,
装备以$R[[X, t]]$中的乘法, 其中$b_{mn}\in R$, 且对固定的$m$, 当$n$充分大时$b_{mn} = 0$.

% \subsubsection{可表性与投射可表性}

% \begin{lemma}
%     设$A\in\comaug_R$, $N\in\nil_R$, 则\[\Hom_{R, \cont}(A, R\oplus N)\simeq \varinjlim\Hom_R(\fra/\fra_n, N).\]
% \end{lemma}
% \begin{proof}
%     首先, \[\Hom_{R, \cont}(A, R\oplus N) = \Hom_{R, \cont}(\varprojlim \fra/\fra_n, N).\]
%     记$\hat{\fra} := \varprojlim \fra/\fra_n$.
%     我们有自然的映射\[\Hom_{R}(\fra/\fra_n, N)\to \Hom_{R, \cont}(\hat{\fra}, N)\footnote{前者带离散拓扑, 因而其中元素总为连续同态.};\]
%     如果$g\in\Hom_{R}(\fra/\fra_m, N)$满足$[f] = [g]$, 则存在$k$使得图\[\begin{tikzcd}
%         & {\fra/\fra_n} \\
%         {\hat{\fra}} & {\fra/\fra_k} & N \\
%         & {\fra/\fra_m}
%         \arrow["f", from=1-2, to=2-3]
%         \arrow[from=2-1, to=1-2]
%         \arrow[from=2-1, to=2-2]
%         \arrow[from=2-1, to=3-2]
%         \arrow[from=2-2, to=1-2]
%         \arrow[from=2-2, to=3-2]
%         \arrow["g"', from=3-2, to=2-3]
%     \end{tikzcd}\]交换,
%     因而我们得到良定的映射\[\varinjlim\Hom_{R}(\fra/\fra_n, N)\to \Hom_{R, \cont}(\hat{\fra}, N),\; [f]\mapsto f\circ (\hat{\fra}\to \fra_n).\]
%     反之, 由于$N$带有离散拓扑, 对任何连续同态$f : \fra\to N$都存在某个$n$使得$\fra_n = \ker f$, 因而$f$穿过$\fra\to \fra/\fra_n$, 从而给出$[\fra/\fra_n\to N]\in\varinjlim\Hom_R(\fra/\fra_n, N)$.
% \end{proof}

\subsubsection{切空间}
通过定义平凡的乘法, 我们可以将$R$上的模范畴嵌入$R$上的幂零代数范畴, 即对$x, y\in M\in\Mod_R$定义$xy := 0$.
称函子$H : \nil_R\to\set$在$\Mod_R\inject\nil_R$上的限制为$H$的\textbf{切函子(tangent functor)}, 记作$t_H$.

注意到如果函子$t : \Mod_R\to\set$保持有限直积, 则$t$将透过忘却函子$\Mod_R\to\set$分解;
即对任何$M\in\Mod_R$, $t(M)$容许典范的$R$-模结构.
而且, 如果$t : \Mod_R\to\abel$保持有限直积, 则$t(M)$由此获得的加法与其Abel群结构的加法相同.

对于函子$t : \Mod_R\to\Mod_R$, 我们可以构造自然变换$(-)\otimes_R t(R)\to t$如下:
设$M\in\Mod_R$, 每个$m\in M$给出$R$线性同态\[c_m : K\to M,\ 1\mapsto m;\]
于是定义\begin{equation}\label{cart-eq: tensor t(R) to t}
    M\otimes_R t(R)\to t(M),\ m\otimes\xi\mapsto t(c_m)\xi.
\end{equation}
\begin{defprop}\label{def: tangent space}
    如果$t : \Mod_R\to \Mod_R$是保持无穷直和的右正合函子, 则自然变换(\ref{cart-eq: tensor t(R) to t})为同构.
    特别地, 任何$R$上的形式群$G$的切函子$t_G$都透过这样的函子分解, 因此$t_G$由$t_G(R) = G(R)$决定;
    记$\lie(G) := G(R)$, 称为$G$的\textbf{切空间(tangent space)}.
\end{defprop}
\begin{proof}
    两侧的函子皆右正合, 因而取模的展示便将问题划归为对自由模$R^{(I)}$证明(\ref{cart-eq: tensor t(R) to t})为同构; 由于两侧的函子保持无穷直和, 问题再次划归为证明(\ref{cart-eq: tensor t(R) to t})对$R$成立; 这是显然的.
\end{proof}

\begin{definition}
    如果形式群$G$的切空间$\lie G$是秩为$d$的有限生成射影模, 我们就称$G$的维度有限, 并记$\dim G = d$.
\end{definition}

正如实李群的情形,
形式群之间的态射如果诱导出切空间的同构, 则此态射本身也是同构.
为此, 我们需要利用函子与自然变换的光滑性.
\begin{definition}
    设$H, G : \nil_R\to\set$. 自然变换$\xi : H\to G$称为是\textbf{光滑的(smooth)},
    如果对任何$\nil_R$中的满射$M\surject N$, \[H(M)\to H(N)\times_{G(N)}G(M)\]也是满射.
    称函子$H$光滑, 如果典范的态射$H\to\Hom(R, -)$光滑.
\end{definition}
注意到函子$H$光滑当且仅当$H$保持满射, 所以特别地, 形式群皆光滑.

\begin{theorem}\label{isom of tangent isom}
    \cite[Theorem 2.30]{Zi84}
    设$H, G : \nil_R\to\set$正合. 若自然变换$\alpha : H\to G$诱导出切函子的同构$\alpha|_{\Mod_R} : t_H\to t_G$,
    则当$H$或$\alpha$光滑时, $\alpha : H\to G$为同构.
\end{theorem}

\subsubsection{形式群律}
\begin{definition}\label{def: formal group law}
    交换环$R$上的一个\textbf{维数$n$的形式群律(formal group law of dimension $n$)}指幂级数的$n$元组$G = (G_1, \dots, G_n)$,
    其中$G_i(X, Y)\in R[[X_1, \dots, X_n, Y_1, \dots, Y_n]]$, 满足以下公理.\begin{enumerate}
        \item [\myit] $G_i(X, 0) = G_i(0, X) = X_i$; 特别地, 这说明$G_i(X, Y) = X_i + Y_i + {}$\!同时包含$X_i$与$Y_i$的高阶项.
        \item [\myit] $G_i(G(X, Y), Z) = G_i(X, G(Y, Z))$.
        \item [\myit] $G_i(X, Y) = G_i(Y, X)$.
    \end{enumerate}
    称环$R[[X]] = R[[X_1, \dots, X_n]]$为$G$的坐标环(coordinate ring).
    
    若$G$是$R$上$n$维的形式群律, $H$是$R$上$m$维的形式群律, 其间的态射$\varphi : G \to H$定义为$m$个$n$元形式幂级数\[\varphi(X) = \varphi_i(X_1, \dots, X_n)\in A[[X_1, \dots, X_n]],\quad 1\le i\le m,\]
    满足\[\varphi(G(X, Y)) = H(\varphi(X), \varphi(Y)),\]
    即\[\varphi_i(G_1(X, Y),\dots, G_n(X, Y)) = H_i(\varphi_1(X), \dots, \varphi_m(X), \varphi_1(Y), \dots, \varphi_m(Y)),\ 1\le i\le m.\]
\end{definition}
加法群律$\mathbb{G}_{\mathrm{a}}(X, Y) = X + Y$和乘法群律$\mathbb{G}_{\mathrm{m}}(X, Y) = X + Y + XY$是最简单的一维形式群律, 可以定义在任何交换环上.
后者的表达式$\mathbb{G}_{\mathrm{m}}(X, Y) = (1 + X)(1+Y)-1$更清楚地显示出$\mathbb{G}_{\mathrm{m}}$表示着乘法.

设$G$为维数$n$的形式群律, $N$为幂零$R$-代数. 在$N^n$上, $G$定义出运算\[(a_n)_n +_G (b_n)_n := (G_n(a, b)).\]
形式群律的定义和\cite[Corollary 1.5]{Zi84}表明$+_G$赋予了$N^n$一个新的群结构.
容易验证\[\tilde{G} : \nil_R\to\abel,\ N\mapsto (N^n, +_G)\]是$n$维的形式群.
不仅如此, 在态射层面, 如果$H$是形式群律, 则\[\Hom(G, H)\simeq\Hom(\tilde{G}, \tilde{H}).\]

注意到$\tilde{G}$的切空间$\lie(G) = R^n$的Abel群结构仍是直和$R^n$的群结构.
反过来, 观察切空间即可确定形式群是否来自形式群律.

\begin{theorem}\label{from formal group law if free of finite rank}
    \cite[Corollary 2.32]{Zi84}
    设$H$为$R$上的形式群. 若切空间$\lie(H)$为$R$上有限秩的自由模, 则$H$来自形式群律, 即存在形式群律$G$使得$H = \tilde{G}$.
\end{theorem}

\subsection{Cartier理论的主定理}

\subsubsection{第一主定理与Cartier环}

回忆$n$元对称群$\symm{n}$在$n$元多项式环$A[X_1, \dots, X_n]$上以重排变元作用着, 其中$A$为环;
并且\[A[X_1, \dots, X_n]\to A[X_1, \dots, X_n]^{\symm{n}},\ X_i\mapsto \sigma_i(X) \]
为环同构\footnote{参见\cite[定理5.8.5]{Li19}}, 其中$\sigma_i(X)$为初等对称多项式.

\begin{definition}\label{def: weak symm}
    称函子$H : \nil_R\to\set$是\textbf{弱对称}的, 如果对任何$n\ge 1$和$A\in\nilaug_R$,
    嵌入\[A[[X_1, \dots, X_n]]^{\symm{n}}\inject A[[X_1, \dots, X_n]]\]
    诱导出的映射\[H(A[[X_1, \dots, X_n]]^{\symm{n}})\to H(A[[X_1, \dots, X_n]])^{\symm{n}}\]为同构.
\end{definition}

\begin{example}
    % 如果要证, 写这里
    左正合函子弱对称. 特别地, 形式群弱对称.
\end{example}

\begin{theorem}[Cartier第一主定理]\label{Hom(Lambda H) isomto H(R[[X]])}
    \cite[Theorem 3.5]{Zi84}
    设函子$H : \nil_K\to\abel$是弱对称的, 则我们有Abel群的同构
    \begin{align*}
        \lambda_H : \Hom(\Lambda_R, H)&\stackrel{\sim}{\longrightarrow} H(R[[X]])\\
                    \Phi&\longmapsto\Phi_{R[[X]]}(1-Xt).
    \end{align*}
\end{theorem}
% 因此, 对于任何形式群$H$, 存在同构$\lambda_H : \Hom(\Lambda, H)\simeq H(R[[X]])$.
\begin{definition}
    命$\E_R := \left( \enom \Lambda_R \right)^\op$, 称为$R$的\textbf{Cartier 环}.
    对任何函子$H$, 群$\Hom(\Lambda, H)$带有$\enom(\Lambda)$自然的右作用, 相应的左$\E$-模记作$M_H$,
    称为$H$的\textbf{Cartier 模}.
\end{definition}

我们考虑Cartier环$\E$中的一些特殊元素. 透过同构$\lambda_\Lambda : \E_R\simeq \Lambda(R[[X]])\subset R[[X, t]]$, 我们定义:
    \begin{align*}
    V_n &:= \lambda_\Lambda^{-1}(1-X^nt),\qquad n\in\mathbb{N},\\
    F_n &:= \lambda_\Lambda^{-1}(1-Xt^n),\qquad n\in\mathbb{N},\\
    [c] &:= \lambda_\Lambda^{-1}(1-cXt),\qquad c\in R.
\end{align*}
利用这些元素, 我们可以具体地描述Cartier环中的元素.

\begin{theorem}
    \cite[Theorem 3.12]{Zi84}每个$\xi\in\E$具有唯一的展开式\[x = \sum_{m, n\ge 0} V_m [a_{m, n}] F_n,\]
    其中$a_{m, n}\in B,$ 且对固定的$m$, 当$n\gg 0$时$a_{m, n} = 0$.
\end{theorem}

\subsubsection{既约Cartier模与第二主定理}

\begin{definition}
    一个\textbf{$V$-既约Cartier模($V$-reduced Cartier module)}是一个左$\E$-模$M$装备以一族Abel群的滤过
    \[M = M^1\supset M^2 \supset \cdots,\]
    满足以下条件:
    \begin{enumerate}
        \item $V_m[c]M^n\subset M^{mn}$, $\forall m, n\in\mathbb{N}$, $c\in K$;
        \item $F_m$是连续自同态, 即对任何$n$, 存在$r$, $F_mM^r\subset M^n$;
        \item $V_m : M/M^2\to M^m/M^{m+1}$为双射;
        \item $M$完备, 即$M = \varprojlim M/M^n$.
    \end{enumerate}
\end{definition}

例如, \cite[Example 3.10]{Zi84}\;指出正合函子$H$的Cartier模$M_H$是既约的,
其滤过由\[M_H^n := \im \left[ H(X^nR[[X]])\to H(XR[[X]]) \right]\]给出.
对于$M_\Lambda \simeq\E$, 我们命\[\E_n := M_\Lambda^n.\]

设$M$是$V$-既约Cartier模.
我们将对每个既约Cartier模构造一个$\nil_R$上的右正合函子.

设$Q$是右$\E$-模. 对每个自然数$n$, 置\[Q_n := \{x\in Q : x\E_n = 0\}.\]
于是$\{Q_n\}$构成$Q$的子模升链. 称$Q$是\textbf{扭}的右$\E$-模, 如果存在$n$使得$Q = Q_n$
\begin{defprop}
    设$M$为既约左$\E$-模, $Q$为右$\E$-模. 对自然数$n$,
    记\[Q_n\circ M^n := \im \left[ Q_n\otimes_\Z M^n\to Q\otimes_\E M \right].\]
    成立$Q_n\circ M^n\subset Q_{n+1}\circ M^{n+1}$,
    于是可以定义\[(Q\otimes_\E M)_\infty := \varinjlim Q_n\circ M^n\]
    和\textbf{既约张量积(reduced tensor product)}\[Q\bar{\otimes}_\E M := \frac{Q\otimes_\E M}{(Q\otimes_\E M)_\infty}.\]
\end{defprop}

\begin{lemma}\label{exact seq of torsion module is exact after tensor}
    \cite[Theorem 3.21]{Zi84}
    如果\[Q_1\to Q_2\to Q_3\to 0\]是扭的右$\E$-模的正合列, 则\[Q_1\bar{\otimes}_\E M\to Q_2\bar{\otimes}_\E M\to Q_3\bar{\otimes}_\E M\to 0\]正合.
\end{lemma}

\begin{defprop}\label{def: functor - nil tensor V-reduced module}
    设$N\in\nil_R$.
    定义函子\[\Lambda\bar{\otimes}_\E M : N\mapsto \Lambda(N) \bar{\otimes}_\E M,\]则$\Lambda\bar{\otimes}_\E M$是$\nil_R\to\abel$的右正合函子.
\end{defprop}
\begin{proof}
    取$\Phi\in\E$和$x\in N$, 右作用按\[x\cdot\Phi := \Phi_N(n)\]定义并延拓至$\Lambda(N)$. \cite[Theorem 3.22]{Zi84}证明了$\Lambda(N)$为扭, 故\cref{exact seq of torsion module is exact after tensor}给出右正合性.
\end{proof}

% \begin{example}[一些既约张量积]
%     \begin{enumerate}
%     \item $\E/\E_n\bar{\otimes}_\E M = M/M^n$.
%     \end{enumerate}
% \end{example}

\begin{definition}
    称$V$-既约Cartier模$M$是\textbf{$V$-平坦的($V$-flat)}如果$M/M^2$是平坦的$R$-模.
    % 如果
\end{definition}

\begin{theorem}[Cartier第二主定理]\label{formal group equiv V-flat V-reduced Cartier module}
    环$R$上的形式群范畴与$V$-平坦$V$-既约Cartier模范畴等价, 相应的函子分别由\[H\longmapsto M_H\]
    和\[\Lambda\bar{\otimes}_\E M\longmapsfrom M\]给出.
\end{theorem}

\subsection{局部Cartier理论}
记$\Z_{(p)}$为$\Z$在素理想$(p) = p\Z$处的局部化.
从现在起, 我们设$R$为$\Z_{(p)}$-代数.

\subsubsection{$p$-典型元素}

设$H$为$R$上的形式群. 如果$n$是与$p$互素的整数,
则乘以$n$的自同态$n : H\to H$为同构; 因为根据\cref{isom of tangent isom}\,和\cref{def: tangent space},
只需要验证$n$在$t_H(R) = H(R)$上为同构. 特别地, $n\in\E^\times$.

我们定义\[\epsilon_1 := \prod_{\ell}\left( 1-\frac{1}{\ell}F_\ell V_\ell \right)\in\E,\]
其中$\ell$取遍不等于$p$的素数. 对于与$p$互素的整数$n$, 定义\[\epsilon_n := \frac{1}{n}V_n\epsilon_1 F_n\in\E_n.\]
\begin{lemma}
    $\epsilon_n$构成$\E$的一组投影子(projector), 即\[\epsilon^2 = \epsilon,\ \epsilon_n\epsilon_m=0(m\ne n),\ \sum_{p \nmid n}\epsilon_n = 1.\]
\end{lemma}
\begin{proof}
    归结到$R$为$\Q$-代数的情形. 参见\cite[Lemma 4.11]{Zi84}.
\end{proof}

因此对于任何幂零$R$-代数$N$, 有分解\[\Lambda(N) = \bigoplus_{p\nmid n}\Lambda(N)\epsilon_n.\]
置$\Lambda_n(N) := \Lambda(N)\epsilon_n$.
由于$\epsilon_n$为投影子, $\Lambda_n$均为形式群.
记$\hat{W} := \Lambda_1$, 称为\textbf{Witt向量的形式群(formal group of Witt vectors)}.

注意到$F_nV_n = n$, 故右乘$V_n$与右乘$\frac{1}{n}F_n$给出互逆的态射$\Lambda_n\to\hat{W}$和$\hat{W}\to\Lambda_n$.
因此上述分解可以重写为同构\[\Lambda\simeq\bigoplus_{p\nmid n}\hat{W}.\]
此同构能够转移到所有既约Cartier模上.
\begin{definition}
    设$M$为既约Cartier模. 子群$\epsilon_1M\subset M$中的元素称为是\textbf{$p$-典型的($p$-typical)}.
\end{definition}
\begin{theorem}
    元素$m\in M$是$p$-典型的当且仅当\[F_nm = 0,\;\forall n > 1,\, p\nmid n.\]
    每个元素$m\in M$能唯一地分解为\[m = \sum_{p\nmid n} V_nm_n,\]
    其中$m_n$为$p$-典型元素.
\end{theorem}
\begin{proof}
    由于$F_{\bullet}$对下标具乘性, 只要考虑素数$\ell\neq p$即可验证$p$-典型性的判别. 直接计算得证.

    分解取$m_n := \frac{1}{n}\epsilon_1 F_nm$.
\end{proof}

\subsubsection{主定理的局部版本}

设$H$为$R$上的形式群.
结合\cref{Hom(Lambda H) isomto H(R[[X]])}, 我们得到同构
\[\Hom(\hat{W}, H) = \Hom(\Lambda\epsilon_1, H) \simeq \epsilon_1\Hom(\Lambda, H)\simeq\epsilon_1 H(R[[X]]).\]

\begin{defprop}
    Witt向量的形式群之自同态环$\enom \hat{W}\simeq \epsilon_1\E\epsilon_1$.
    定义关联于素数$p$的\textbf{局部Cartier环(local Cartier ring)}为
    $\E_p := \epsilon_1\E\epsilon_1$. 命\[V := \epsilon_1V_p = V_p\epsilon_1,\ F := \epsilon_1F_p = F_p\epsilon_1,\ [a]_p := \epsilon_1[a] = [a]\epsilon_1\, (a\in R),\]
    则$\E_p$的每个元素具有唯一的分解\[x = \sum_{m, n\ge 0} V^m [a_{m, n}] F^n,\] 其中$a_{m, n}\in B$, 且对固定的$m$, 有$n\gg 0\implies a_{m, n} = 0$.
\end{defprop}
\begin{proof}
    参见\cite[Definition and Theorem 4.17]{Zi84}.
\end{proof}

\begin{definition}
    称左$\E_p$-模$M$是\textbf{$V$-既约的}, 如果:
    \begin{enumerate}
        \item $V : M\to M$为单射,
        \item $M$为$V$-进完备, 即$M\simeq\varprojlim M/V^nM$.
    \end{enumerate}
\end{definition}

% \begin{theorem}
%     若$M$为$V$-既约Cartier模, 则存在典范同构\[\hat{W}\otimes_{\E_p}\epsilon_1M\simeq\Lambda\bar{\otimes}_\E M.\]
% \end{theorem}

\begin{theorem}[Cartier第二主定理, 局部版本]\label{formal group equiv V-reduced Cartier module - local}
    设$H:\nil_R\to\abel$为形式群.
    \begin{enumerate}
        \item Cartier模$M_H = H(R[[X]])$的$p$-典型元素之集$\epsilon_1H(R[[X]])$具有$V$-既约$\E_p$-模结构,
        记作$M_{p, H}$.
        \item 存在典范同构\[\hat{W}\otimes_{\E_p} M_{p, H}\simeq H.\]
        \item 函子$H\mapsto M_{p,H}$给出了$R$上形式群与$V$-既约$\E_p$-模中那些$M_p/VM_p$为平坦$R$-模的元素组成的子范畴之间的等价,
        且$H$的切空间$\lie(H)$在此等价下被映到$M_{p, H}/VM_{p, H}$.
    \end{enumerate}
\end{theorem}

\subsubsection{Witt向量}
回顾对于素数$p$, 取Witt向量环给出了交换环范畴到自身的函子$W : \cring\to\cring$;
它由以下性质刻画: 作为集合, $W(R) = R^\N$, 装备以环结构使得\[W(R)\to R^\N,\ (a_n)_n\mapsto (w_n(a_0, \dots, a_n))_n\]
为环同态, 其中多项式\[w_n(X_0, \dots, X_n) = X_0^{p^n} + pX_1^{p^{n-1}} + \dots + p^nX_n\in \Z[X_0, \dots, X_n].\]
本小节中, 我们将以另一种方式刻画Witt向量的形式群$\hat{W}$, 并建立它与Witt向量的一些联系.

\begin{lemma}
    取$N\in\nil_R$.
    群$\Lambda(N)$的每个元素可以唯一地表示为有限乘积\[\prod_{i=1}^n \left( 1- x_it^i \right),\; x_i\in N;\]
    而$\hat{W}(N)$的每个元素可以唯一地表示为有限乘积\[\prod_{i=1}^n \left( 1- y_it^{p^i} \right)\epsilon_1,\; y_i\in N.\]
\end{lemma}
\begin{proof}
    考虑一般并非群同态的映射\begin{equation}\label{cart-eq: direct sum to Lambda}
        \bigoplus_{i=1}^\infty N\to\Lambda(N),\; (x_i)\mapsto\prod_i(1-x_it^i).
    \end{equation}
    注意到右边的乘积有限, 且此映射对于$N$呈函子性;
    我们断言{\cref{cart-eq: direct sum to Lambda}}是自然同构, 从而说明每个$\Lambda(N)$中元素可唯一地表作有限积.
    由\cref{isom of tangent isom}, 只要在$N^2 = 0$时证明;
    此时$\prod_i(1-x_it^i) = 1 - \sum_i x_it^i$. 按定义, 右边的求和唯一, 是故\cref{cart-eq: direct sum to Lambda}为同构.

    因为$F_\ell\epsilon_1 = 1$在素数$\ell\neq p$时成立, 所以每个$\hat{W}$中元素都可写作
    \[\prod(1-x_it^i)\epsilon_1 = \prod(1-x_it)F_i\epsilon_1 = \prod(1-x_{p^j}t)F_{p^j}\epsilon_1.\]
    我们证明将$\Lambda\to\Lambda\epsilon_1$限制到$\bigoplus_{i = p^n} N$在同构\cref{cart-eq: direct sum to Lambda}下的像上为到$\hat{W}$的同构.
    仍然只需在$N^2 = 0$处检验: 此时对任何$m > 1$, 当$(m, p) = 1$时
    \[(1-y_nt^{p^n})V_m = (1-y_nt)F_{p^n}V_m = (1-y_n^mt)F_{p^n} = 1\cdot F_{p^n} = 1,\]
    故\[(1-y_nt^{p^n})\epsilon_m = (1-y_nt^{p^n})\frac{1}{m}V_m\epsilon_1F_m = 1\cdot\frac{1}{m}\epsilon_1F_m  = 1;\]
    因此当$\prod (1-y_nt^{p^n}) = 1$时, \[\prod(1-y_nt^{p^n}) = \prod\sum_{p\nmid m}(1-y_nt^{p^n})\epsilon_m = 1,\]
    明所欲证.
\end{proof}

\begin{theorem}
    多项式族$\{w_n\}$定出了群函子同态\begin{align*}
        \hat{W}(N)&\longrightarrow \bigoplus_{n= 0}^\infty\mathbb{G}_{\mathrm{a}}(N)\\
        \prod (1-y_nt^{p^n})&\longmapsto (w_n(y_0, \dots, y_n))_n.
    \end{align*}
    是故嵌入$\bigoplus_{n\ge0}N\to N^\N$诱导出群同态\begin{align*}
        \hat{W}(N)&\longrightarrow W(N)\\
        \prod (1-y_nt^{p^n})&\longmapsto (y_n)_n.
    \end{align*}
\end{theorem}
\begin{proof}
    由于$N$幂零, 上述映射良定. 同态性参见\cite[Theorem 4.2.5]{Zi84}
\end{proof}
借助Witt环, 我们可以给出局部Cartier环的另一种描述.
\begin{corollary}
    映射\begin{align*}
        W(R)&\longrightarrow\E_p\\
        (a_n)&\longmapsto\sum_{n}V^n[a_n]F^n
    \end{align*}
    是环的嵌入. 透过此嵌入将$W(R)$视为局部Cartier环$\E_p$的子环,
    则$\E_p$同构于$W(R)[V, F]$关于右理想滤过$\{(V^n)\}_n$的完备化.
\end{corollary}

\subsubsection{高度}
高度是形式群的一个重要不变量. 为此, 我们要先定义同源的概念. 
\begin{definition}
    设$G, H$为来自形式群律的$R$上形式群. 态射$\varphi : G\to H$称为一个\textbf{同源(isogeny)},
    如果$\ker\varphi : \nil_R\to \set$可表.
\end{definition}

设$\varphi : G\to H$为同源, 则$\ker\varphi$由有限生成的射影$R$-代数$A$表出. 
考虑$A$的素理想$\mathfrak{p}$的剩余类域$\mathbf{k}(\mathfrak{p}) = A_\mathfrak{p}/\mathfrak{p}A_\mathfrak{p}$. \cite[Theorem 5.3]{Zi84}表明存在自然数$h(\mathfrak{p})$使得$\dim_{\mathbf{k}(\mathfrak{p})} A\otimes_R\mathbf{k}(\mathfrak{p}) = p^{h(\mathfrak{p})}$.
交换代数的结果指出$\mathfrak{p}\mapsto h(\mathfrak{p})$是局部常值函数.
\begin{definition}
    称同源$\varphi$的\textbf{高度(height)}为$h\in\N$, 如果$h(\mathfrak{p}) = h$对所有$\mathfrak{p}\in\spec A$成立.
    称形式群$G$的高度为$h$, 如果$p\in\enom G$为同源且高度为$h$.
\end{definition}

例如, 设$R$的特征为$p$, 我们考察$\mathbb{G}_{\mathrm{m}/R}$的乘$p$同态\[\mathbb{G}_{\mathrm{m}}(N)\to\mathbb{G}_{\mathrm{m}}(N),\ 1 + n\mapsto (1+n)^p = 1 + n^p.\]
则$\ker\varphi$由$R[X]/X^p$表出, 因此$p : \mathbb{G}_{\mathrm{m}}\to \mathbb{G}_{\mathrm{m}}$是高度为$1$的同源, $\mathbb{G}_{\mathrm{m}}$的高度为$1$.
\section{Drinfeld定理}
从现在起, 让我们考虑$K$上的一个四元数代数除环$D$, 其整数环记作$\O_D$.
记$\O^\nr$为$K$的极大非分歧扩张(maximal unramified extension)的整数环, $\hat{\O}^\nr$为$\O^\nr$的$\varpi$-进完备化.
% (应该加点东西.)
% 具体地, 我们考虑高度为$4$的特殊形式$\O_D$-模.
% 在$k$的一个代数闭包$\bar{k}$上, 这些形式模落在同一个同源类中;
% 选择并固定其中之一, 记作$\Phi$.
Drinfeld的``基本定理''称形式$\O$-概形$\hat{\Omega}\hat{\otimes}_{\O}\hat{\O}^{\nr}$参数化了一族幂零$\O$-代数上高度为$4$的形式$\O_D$-模.

\subsection{形式模}
\subsubsection{形式$\O$-模的Cartier理论}
% 本节内容详见\cite{Zi84}.
\begin{definition}
    一个$B$上的\textbf{形式$\O$-模(formal $\O$-module)}指$B$上的一个来自形式群律的光滑形式群$X$连同一个$\O$-作用,
    即环同态$i : \O\to \enom X$;
    并且, 我们要求此$\O$-作用在切空间$\lie X$上诱导的$\O$-作用与$\lie X$的$B$-代数结构诱导的$\O$-作用相同;
    即对任何$a\in \O$, 相应的$B$上幂级数$i(a)(T)\equiv aT\pmod{T^2}$.
    特别地, 对于$\O = \Z_p$, $B$上的形式$\Z_p$-模范畴与$B$上的形式群范畴等价.
\end{definition}

% {\color{red}  why????}

% 因此对于$\O = \Z_p$, $B$上的形式$\Z_p$-模范畴与$B$上的形式群范畴等价.

设$B$是一个$\O$-代数. 仿照Witt向量环的定义, 容易证明$B^\N$容许唯一的$\O$-代数结构, 记作$W_\O(B)$,
使得鬼映射(ghost map) $w : W_\O(B)\to B^\mathbb{N}$为$\O$-代数同态, 其中$w = (w_n)_n$的分量由多项式映射\[w_n : (a_n)_n\mapsto a_0^{q^n} + \varpi a_1^{q^{n-1}} + \dots + \varpi^{n}a_n\]
给出.
同样地, 我们考虑$W_\O(B)$上的移位映射(Verschiebung map) \[\tau : (a_0, a_1, a_2,\cdots)\mapsto (0, a_0, a_1, \cdots),\]
由关系\[w_n\sigma = w_{n+1},\ \forall n\in\mathbb{N}\]决定的Frobenius同态$\sigma$, 和Teichm\"uller提升\[[\cdot] : a \mapsto (a, 0, 0, \cdots).\]

记$a\in \O$在结构映射$\O\to W_\O(B)$下的像为$a = a\cdot 1$, 则其鬼分量显然为$w_n(a) = a$.

% \begin{remark}
%     注意区分$W_\O(B)$作为$\O$-代数的结构映射\[\O\to W_\O(B),\ a\mapsto a := a\cdot 1\]与Techi\"muller提升在$\O$上的限制
%     \[\O\to B\to W_\O(B),\ a\mapsto [a].\]
% \end{remark}

\begin{definition}\label{def: Cartier ring}
    相应于$B$的Dieudonn\'e环被定义为非交换的$\O$-代数$W_\O(B)[F, V]$, 其中$F$和$V$满足: 对任何$x\in W_\O(B)$,\begin{align*}
        Fx &= \sigma(x)F,\\
        xV &= V\sigma(x),\\
        VxF &= \tau(x),\\
        FV &= \varpi
    \end{align*}
    我们在Dieudonn\'e环上装备由右理想$(V)$定出的$V$-进滤过,
    并定义\textbf{Cartier 环}$E_\O(B)$为Dieudonn\'e环$W_\O(B)[F, V]$关于$V$-进拓扑的完备化.
\end{definition}
当$\O = \Z_p$时, $E_{\Z_p}(B)$正是上一节定义的局部Cartier环$\E_{B, p}$.
因此, 我们可以平行于上一节的结论, 建立起形式$\O$-模的Cartier理论.

每个$E_\O(B)$中元素$x$可以典范地写作\[x = \sum_{m, n\ge 0} V^m [a_{m, n}] F^n,\ a_{m, n}\in B,\ n\gg 0\implies a_{m, n} = 0.\]
特别地, 嵌入$W_\O(B)\inject E_\O(B)$由\[(a_0, \cdots)\mapsto \sum_{n\ge 0} V^n[a_n]F^n\]给出.

\begin{definition}
    一个$B$上的\textbf{Cartier $\O$-模}指一个左$E_\O(B)$-模, 满足\begin{enumerate}
        \item $M/VM$是有限秩自由$B$-模,
        \item $V$在$M$上为单射,
        \item $M$关于$V$-进拓扑分离且完备, 即$M \simeq\varprojlim M/V^nM$且$\bigcap_{n}V^nM = 0$.
    \end{enumerate}这样的模也称为\textbf{既约Cartier $\O$-模}.
\end{definition}

仿照\cref{formal group equiv V-reduced Cartier module - local}\;的证明并结合\cref{from formal group law if free of finite rank},
我们得到:
\begin{theorem}
    $B$上的形式$\O$-模范畴与$B$上的Cartier $\O$-模范畴等价.
    而且, 如果$M$是相应于形式$\O$-模$X$的Cartier $\O$-模, 则$M/VM = \lie(X)$.\qedhere
\end{theorem}

% \subsubsection{Cartier模中元素的具体形式}
% 设$M$是$B$上的Cartier模. 

\subsubsection{形式$\O_D$-模的Cartier理论}
回忆$D$是$K$上的四元数代数, $\O_D$为其整数环.
% {\color{red}(so $D$ is required to be division?)}
由\cite[Theorem 13.3.11]{Vo21}, $K$的二次非分歧扩张$K'$唯一地嵌入$D$.
记$\O'$为$K'$的整数环, $\sigma\in\gal(K'/K)$为其Galois群中的非平凡元素.
同样由\cite[Theorem 13.3.11]{Vo21}, 存在$\Pi\in\O_D$, 使得$\Pi^2 = \varpi$, 且对任何$a\in\O'$, $\Pi a = \sigma(a)\Pi$. 固定一个这样的$\Pi$.

\begin{definition}
    一个$B$上的\textbf{形式$\O_D$-模}指$B$上的形式$\O$-模$X$连同一个$\O_D$-作用$i : \O_D\to\enom X$延拓了$X$本身的$\O$-作用.
    称形式$\O_D$-模$X$是\textbf{特殊(special)}的, 如果其$\O'$-作用使$\lie X$成为秩$1$自由$B\otimes_\O\O'$模.
\end{definition}

Cartier环$E_\O(B)$带有自然的$\Z/2\Z$-分次, 由$\deg F = \deg V = 1$定义;
齐次分量为\begin{align*}
    E_\O(B)_i = \left\{ \sum V^m[a_{m, n}]F^n : m + n\equiv i\!\!\pmod 2,\ \forall m, n \right\},\ i = 0, 1.
\end{align*}
特别地, $W_\O(B)\subset E_\O(B)_0$.

\begin{definition}
    一个\textbf{分次Cartier $\O[\Pi]$-模}指一个$\Z/2\Z$-分次Cartier $\O$-模$M = M_0\oplus M_1$连同一个$1$次$E_\O(B)$线性自同态$\Pi$,
    满足$\Pi^2 = \varpi$. 此时$M_0$与$M_1$自动成为$W_\O(B)$-模. 称分次Cartier $\O[\Pi]$-模$M$是\textbf{特殊}的, 如果$M_0/VM_1$和$M_1/VM_0$皆为秩$1$自由$B$模.
\end{definition}

\begin{theorem}
    \cite[II, 2.3]{BC91}
    设$B$是$\O'$-模,
    则$B$上的形式$\O_D$-模范畴与$B$上的分次Cartier $\O[\Pi]$-模范畴等价, 且此范畴等价保持特殊性.
\end{theorem}

% \subsubsection{形式模的高度}
% 对于$\O$上的形式模而言, 一个重要的不变量是高度.
% \begin{defprop}
%     设$\phi(T)$是$\O$上作为形式群律的形式$\O$-模之间的态射$F\to G$, 且$\phi\neq 0$,
%     则$\phi$具有形式\[,\]
%     其中$h\ge 0$为自然数. 我们称$h$为$\phi$的\textbf{高度(height)}.
%     定义形式$\O$-模$X$的高度为$\varpi\in\enom X$的高度.
% \end{defprop}
% \begin{proof}
%     记$\phi = (\phi_1, \dots, \phi_m)$, 则\[\phi_i(F_1(X, Y), \dots, F_n(X, Y)) = G(\phi_1(X),\dots, \phi_m(X)).\]
    
% \end{proof}

\subsection{Drinfeld定理: 陈述}


% 本节将考虑代数闭包$\bar{k}$上形式模的一些概念和性质.
回忆$\bar{k}$为$k$的代数闭包, 其Witt向量环为$W_\O(\bar{k}) = \hat{\O}^\nr$.
根据\cite[II, 5.2]{BC91}, $\bar{k}$上高度$4$的特殊形式$\O_D$-模具有唯一的同源类.
固定一个$\bar{k}$上高度$4$的特殊形式$\O_D$-模$\Phi$.

\begin{definition}
    定义$\nilp$上的函子$G$如下: 对$B\in\nilp$, $G(B)$为三元组$(\psi, X, \rho)$的同构类, 其中\begin{enumerate}
        \item [\myit] $\psi : \bar{k}\to B/\varpi B$为$k$-同态,
        \item [\myit] $X$为$B$上高度$4$的特殊形式$\O_D$-模,
        \item [\myit] $\rho : \psi_*\Phi\to X_{B/\varpi B}$为高度$0$的拟同源.
    \end{enumerate}
\end{definition}

\begin{theorem}[Drinfeld]\label{Drinfeld: G}
    函子$G : \nilp\to\set$由形式$\O$-概形$\hat{\Omega}\hat{\otimes}_{\O}\hat{\O}^\nr$表出.
\end{theorem}

根据Witt向量环的泛性质, 给出一个$k$-同态$\psi:\bar{k}\to B/\varpi B$, 等价于给出一个$\O$-同态$\tilde{\psi} : \O^\nr\to B$.
于是, 给出$G(B)$中的一个点等价于给出$\O$-同态\footnote{由于$\varpi$在$B$中幂零, 这等价于给出连续的$\O$-同态$\hat{\O}^\nr\to B$.}
$\tilde{\psi} : \O^\nr\to B$连同$G(B_{\tilde{\psi}})$中的一个点,
其中$B_{\tilde{\psi}}$表示赋予$B$以来自$\tilde{\psi}$的$\O^\nr$-代数结构,
而$\bar{G}$定义如下.

\begin{definition}
    令$\nilpnr$为$\varpi$在其中幂零的$\O^\nr$-代数组成的范畴\footnote{自然也等于$\varpi$在其中幂零的$\hat{\O}^\nr$-代数组成的范畴.}.
    定义$\nilpnr$上的函子$\bar{G}$如下: 对$B\in\nilpnr$, $\bar{G}(B)$为二元对$(X, \rho)$的同构类, 其中\begin{enumerate}
        \item [\myit] $X$为$B$上高度$4$的特殊形式$\O_D$-模,
        \item [\myit] $\rho : \Phi_{B/\varpi B}\to X_{B/\varpi B}$为高度$0$的拟同源.
    \end{enumerate}
\end{definition}


注意到给出$\Hom_{\hat{\O}^\nr}(B, \hat{\Omega}\hat{\otimes}_{\O}\hat{\O}^\nr)$中的一个点等价于给出交换图
\[\begin{tikzcd}
    \spf B \arrow[r, "f"] \arrow[rd, "\psi^*"] & \hat{\Omega}\hat{\otimes}_{\O}\hat{\O}^\nr \arrow[d] \\
                                               & \spf \hat{\O}^\nr                                   
    \end{tikzcd}\]
其中$f\in\Hom_{\O}(B, \hat{\Omega}\hat{\otimes}_{\O}\hat{\O}^\nr)$, $\psi\in\Hom_{\O, \cont}(\hat{\O}^\nr, B) = \Hom_{\O}(\O^\nr, B)$.
因此为了证明\cref{Drinfeld: G}, 只需证明下述定理.

\begin{theorem}[Drinfeld]\label{Drinfeld: G bar}
    函子$\bar{G} : \nilpnr\to\set$由形式$\hat{\O}^\nr$-概形$\hat{\Omega}\hat{\otimes}_{\O}\hat{\O}^\nr$表出.
\end{theorem}

记$\bar{H} : \nilpnr\to \set$为\cref{def: functor F}\,中的函子$F : \nilp\to\set$在$\nilpnr$上的限制.
因为$F$由形式$\O$-概形$\hat{\Omega}$表出, 所以$\bar{H}$由形式$\hat{\O}^\nr$-概形$\hat{\Omega}\otimes_\O\hat{\O}^\nr$表出.
\cref{Drinfeld: G bar}\,也就是说函子$\bar{G}$同构于$\bar{H}$.
我们将在下一小节中构造自然变换$\xi : \bar{G}\to\bar{H}$; $\xi$实为同构的证明请参阅\cite{BC91}或\cite{drinfel1976coverings}.
% 下一步便是构造自然变换$\xi :\bar{G}\to \bar{H} $, 再证明$\xi$为同构.
\subsection{自然变换$\xi : \bar{G}\to\bar{H}$的构造}

设$B\in\nilpnr$, $M$是$B$上的分次Cartier $\O[\Pi]$-模.

回忆Frobenius同态$\sigma : W_\O(B)\to W_\O(B)$为$\O$-代数同态.
将$M$透过$\sigma$进行系数限制(restriction of scalars)得到一个$W_\O(B)$-模, 记作$M^\sigma$.

我们定义$N(M)$为$\O$-模同态\[M\to M\oplus M^\sigma,\ m\mapsto (Vm, -\Pi m)\]的余核,
其上带有来自$M\oplus M$的$\Z/2\Z$-分次.
由于$m\mapsto (Vm, -\Pi m)$对于$V$和$\Pi$的作用等变,
$N(M)$上也带有$V$和$\Pi$的1次作用.

定义映射\[\lambda_M : N(M)\to M,\ [(m, m')]\mapsto \Pi m + Vm'.\]
\begin{proposition}
    存在唯一的映射$L_M : M\to N(M)$, 满足:\begin{enumerate}
        \item $\lambda_M\circ L_M = F$,
        \item $L_M$对$B$呈函子性: 对任何$\O$-同态$B\to B'$, 图\[% https://tikzcd.yichuanshen.de/#N4Igdg9gJgpgziAXAbVABwnAlgFyxMJZABgBpiBdUkANwEMAbAVxiRAFkQBfU9TXfIRQBGclVqMWbAHIAKdgEpuvEBmx4CRMsPH1mrRBwDkyvusFFRO6nqmG57I0q7iYUAObwioAGYAnCABbJDIQHAgkUQl9NgAZAH1OHl8A4MRQ8KQAJmSQfyDs6kzEAGYbSQMQBKSVfLSo4pKXLiA
        \begin{tikzcd}
        M \arrow[r, "L_M"] \arrow[d] & N(M) \arrow[d] \\
        M' \arrow[r, "L_{M'}"]          & N(M')         
        \end{tikzcd}\]交换, 其中$M' = M\hat{\otimes}_{E_\O(B)} E_\O(B')$.
    \end{enumerate}
\end{proposition}

定义\[\phi_M : N(M)\to N(M),\ [(m, m')]\mapsto L_M(m) + [(m', 0)].\]
再取$\phi_M$的不动点\[\eta_M := N(M)^{\phi_M} = \{z\in N(M) : \phi(z) = z\}.\]
则$\eta_M$继承了来自$N(M)$的分次$\O[\Pi]$-模结构.

有了以上的准备, 我们就能够对$(X, \rho)\in \bar{G}(B)$定义四元组$\xi(X, \rho) = (\eta_X, T_X, u_X, r_{X, \rho})$如下. 记$M(Y)$为形式$\O_D$-模$Y$的分次Cartier $\O[\Pi]$-模, $S = \spec B$.
\begin{enumerate}
    \item [\myit] $\eta_X$为$X$上的层, 在每个仿射开集$\spec A\subset S$上取值$\eta_X(\spec A) = \eta_{M(X_A)}$.
    \item [\myit] $T_X$为$X$上的层, 在每个仿射开集$\spec A\subset S$上取值$T_X(\spec A) = \lie X_A = M_{X_A}/VM_{X_A}$.
    \item [\myit] $u_X : \eta_X\to T_X$, 在每个仿射开集$\spec A\subset S$由$[(m, m')]\mapsto m\bmod V$定义.
    \item [\myit] $r_{X, \rho} : \underline{K}^2\to \eta_{X, 0}\otimes_\O K$由$\rho$根据\cite[II, 7.5]{BC91}\;诱导而出.
\end{enumerate}
这便是我们寻求的自然变换$\xi : \bar{G}\to\bar{H}$.
% \subsubsection{$N(M)$与$\eta_M$上的滤过}

% \subsubsection{刚性化}

% \subsection{}

\newpage
\addcontentsline{toc}{section}{\heiti 附录}
\section{附录}
本节考虑的概形统一认为是Noether的.
\subsection{射影丛}
考虑概形$X$上的一个分次$\O_X$-代数$\mathscr{B}$, 即带有$\O_X$-代数结构的拟凝聚分次$\O_X$-模.
% 我们进一步假设$\mathscr{B}$的一次齐次部分$\mathscr{B}_1$为凝聚层, 并且$\mathscr{B}_1^n = \mathscr{B}_n$.
任何$X$的仿射开子概形$U$都给出其上的概形$\proj \mathscr{B}(U) \to U$.
如果$V$是$U$的仿射开子概形, 则
\[\mathscr{B}(V) = \mathscr{B}(U)\otimes_{\mathscr{O}_X(X)}\mathscr{O}_X(V),\]
因此$\proj\mathscr{B}(V) = \proj\mathscr{B}(U)\times_X V$.
于是, 取$X$的仿射开覆盖$U_i$, $\proj \mathscr{B}(U_i)$可以粘合成为$X$上的概形, 记作$\proj \mathscr{B}\to X$.
\begin{example}
    取$\mathscr{B} = \mathscr{O}_X[T_0, \dots, T_n]$为多项式代数, 则$\proj \mathscr{B} = \P^n_X = \P^n_\Z\times_{\Z} X$.
\end{example}

对于$X$上的拟凝聚层$\mathscr{E}$, 我们定义$X$-概形范畴上取值在集合范畴$\set$中的函子$\P(\mathscr{E})$,
将$h : Y\to X$映到二元对$(\mathscr{L},\ h^*\mathscr{E}\surject\mathscr{L})$的集合, 其中$\mathscr{L}$为$Y$上可逆层.
由\cite[II, Proposition 7.12]{Ha77}知, 此函子由射影概形$\proj(\sym(\mathscr{E}))$表出, 因而为$X$上射影概形, 称为相应于$\mathscr{E}$的\textbf{射影丛(projective bundle)}.

\begin{example}\label{eg: P(O^2)}
    取$X = \spec \O$. 自由模$M = \O^2$给出$X$上的凝聚层$\tilde{M}$,
    于是给出$\P(M) := \P(\tilde{M})$.

    设$A$为$\O$-代数, 则$\P(M)$的$A$-点由\begin{align*}
        \P(M)(A) &= \{M\otimes_\O A\surject L : L\text{ 为秩1的自由}\ A\text{ 模}\}\\
        &=\{N\subset_A M\otimes_\O A : M\otimes_\O A/N \text{ 为秩1的自由}\ A\text{ 模}\}.
    \end{align*}
    如果$A = F$是一个域, 那么$\P(M)(F)$中可以实现为$F^2$中余维数$1$的$F$-子空间之集合, 因此$\P(M)(F) = \P^1(F)$.
    特别地, $\P(M)(K) = \P^1(K)$, $\P(M)(k) = \P^1(k)$.
    此外, 存在双射\[% https://tikzcd.yichuanshen.de/#N4Igdg9gJgpgziAXAbVABwnAlgFyxMJZABgBpiBdUkANwEMAbAVxiRAB12AFACgFkAlD04B5ASAC+pdJlz5CKAEzkqtRizadegngGlxE1TCgBzeEVAAzAE4QAtkjIgcEJAEZqcABZZLOR9QARjBgUEgAzE70zKyIIABynHZ0aHAuAASJ7BB4dvAA+qLpupLSIDb27tQuASDBoRFO3r7+iE4MdMEMXLJ4BGwMMH4g1NEacfEA5JwAximiAHqKnGY4mZOSFBJAA
    \begin{tikzcd}
    \P(M)(\O) \arrow[rr, "N\mapsto N\otimes_\O K", bend left] &  & \P(M)(K) \arrow[ll, "N'\cap\O^2\mapsfrom N'", bend left]
    \end{tikzcd}.\] 
\end{example}

\subsection{爆破}
\begin{definition}
    设$\mathscr{I}$为$X$上的凝聚理想层.
    称$\tilde{X} := \proj\left( \bigoplus_{n\ge 0} \mathscr{I}^n\right)\to X$为$X$沿理想层$\mathscr{I}$或闭子概形$Z := V(\mathscr{I})$的\textbf{爆破(blow up)}.
\end{definition}
由$\proj$的构造, 我们可以在仿射开集上作爆破再粘合.
所以不妨设$X = \spec A$, 于是存在$A$的有限生成的理想$I = (f_1, \dots, f_n)$使得$\mathscr{I} = \tilde{I}$, 爆破$\tilde{X} = \proj B$,
$B := \bigoplus_{d\ge 0}I^d$的分次$A$-代数结构给出态射$\tilde{X}\to X$.
为了区分$B$的一次部分$I = B_1$和零次部分的子集$I\subset A = B_0$, 记$t_i = f_i\in B_1$, 而$f_i\in B_0$.
考虑满同态\[\phi : A[T_1, \dots, T_n]\to B,\ T_i\mapsto t_i.\]
这是分次代数同态, 因而$\tilde{X} = \proj B\simeq\proj A[T_1,\dots, T_n]/\ker\phi$是$\P_A^n$的闭子概形.
注意到多项式$P(T_1, \dots, T_n)\in\ker\phi$当且仅当$P(f_1, \dots, f_n) = 0\in A$.
% 让我们先考察两个简单的特殊情形.
% \begin{enumerate}
%     \item 理想$I$由单个正则元\footnote{regular element, 即非零因子}$f$生成, 则$\phi$是单射, 故$\tilde{X}\simeq \P^0_A\simeq X$.
%     \item 理想$I$幂零, 则由\cite[Lemma 2.3.35]{Liu02}知, $\tilde{X} = \varnothing$.
% \end{enumerate}

\begin{proposition}\label{blow up eq proj if integral}
    令$J := (f_iT_j-f_jT_i)_{1\le i, j\le n}$, 则$J\subset\ker\phi$.
    如果$Z:=V_+(J)\subset\P_A^{n-1}$是整的(integral), 则$\tilde{X}\simeq Z$.
\end{proposition}
\begin{proof}
    参见\cite[Lemma 8.1.2]{Liu02}.
\end{proof}

% \begin{theorem}\label{blow up of regular}
%     设$X$是正则局部Noether概形, $Y = V(\mathscr{I})$为其正则闭子概形, $\tilde{X}$为$X$沿$Y$的爆破.
%     则$\tilde{X}$正则, 且对任何点$x\in Y$, 纤维$\tilde{X}_x$同构于$\P^{r-1}_{\mathbf{k}(x)}$, 其中$r = \dim_x X-\dim_x Y$.
% \end{theorem}
% \begin{proof}
%     参见\cite[Lemma 8.1.19]{Liu02}.
% \end{proof}

% \begin{example}\label{eg: blow up of A^n over a field}
%     考虑$X = \A^n_k$沿原点$o$的爆破$\tilde{X}$, 其中$k$为域.
%     记$A = k[T_1, \dots, T_n]$, $X = \spec A$, $I = (T_1,\dots, T_n)\subset A$, $B := \bigoplus_{d\ge 0}I^d$, 则$\tilde{X} = \proj B\to X$.
% \end{example}

\begin{example}\label{eg: blow up of A^1_Zp}
    取$X = \A^1_{\Z_p} = \spec\Z_p[T]$. 我们考虑$X$沿极大齐次理想$I = (p, T)$定出的闭点${x} = V(I)$的爆破$\tilde{X}$.
    记$A = \Z_p[T]$, $B = \bigoplus_{d\ge 0}I^d$.
    我们考虑满同态$\phi : A[S, W]\to B$和理想$J = (TS - pW)\subset\ker\phi$.
    由于$TS-pW$不可约, \cref{blow up eq proj if integral}\,导出\[\tilde{X}\simeq\proj\frac{A[S, W]}{TS-pW} = V_+(J)\subset \P^1_{A}.\]

    开子概形\[D_+(W) = \spec \frac{\Z_p\left[ T, s \right]}{Ts-p},\ s = S/W\]
    和\[D_+(S) = \spec\frac{\Z_p[T, w]}{T-pw}\simeq\spec\Z_p[w] \simeq \spec\Z_p[T/p],\ w = W/S\]
    组成了$\tilde{X}$的一个仿射开覆盖; 它们透过同构
    \[D_+(W)_S = \spec \Z_p[T, s, s^{-1}] = \spec \Z_p[T, w^{-1}, w] = D_+(S)_W\]
    粘合成为$\tilde{X}$.

    % 我们还关心$\tilde{X}$的特殊纤维.

    % 首先, $\tilde{X}\to X$在爆破点$x\in X$上的纤维$\tilde{X}_x\simeq \P^1_{\F_p}$. 由\cref{blow up of regular}\,知,
    % 只需证明$\dim_xX = 2$.
    % 为此, 注意到以下事实: 若$X_0\subsetneq X_1\subsetneq\dots\subsetneq X_n$是$X$的不可约闭子集链, $x\in X_0$, 则对$X$的任何开子集$U$, $U\cap X_i$是$U$的不可约闭子集, 且
    % \[X_0\cap U\subsetneq \dots\subsetneq X_n\cap U,\]
    % 从而推出$\dim_x X\ge n$.
    % 于是, 由$\{x\}\subsetneq \A^1_{\F_p}\subsetneq X$得到$\dim_xX \ge 2$; 而$\dim_x X\le \dim X = 2$, 故$\dim_xX = 2$.
\end{example}

\subsection{形式概形}

\subsubsection{形式概形作为环层空间}
如前所述, 形式概型是那些局部上形如$\spf A$的环层空间, 其中环$A$关于其理想$I$定出的进制拓扑完备.
对于一般的概型$X$, 我们定义其沿其闭子概型$Y$的形式完备化(formal completion)为\[\hat{X} := \varinjlim X/\mathscr{I}^n = \left( Y, \varprojlim\mathscr{O}/\mathscr{I}^n\right),\]
其中$\mathscr{I}$是截出$Y$的理想层. 这样的空间是形式概型. 本文中, 我们主要考虑的离散赋值环上的概形沿其特殊纤维的形式完备化.
\begin{example}
    考虑射影直线$\P^1_{\Z_p}= \proj\Z_p[T_0, T_1]$沿其特殊纤维$\P^1_{\F_p}$的形式完备化.
    闭浸入$i : \P^1_{\F_p}\to\P^1_{\Z_p}$在仿射开集$D_+(T_0)$和$D_+(T_1)$上由模$p$给出,
    因而$\P^1_{\Z_p}$沿其特殊纤维的形式完备化$\hat{\P}^1_{\Z_p} = \left(|\P^1_{\F_p}|, \varprojlim \mathscr{O}_{\P^1_{\Z_p}}/(\ker i^\#)^n\right)$
    确为两片$\hat{\A}^1_{\Z_p}$透过$\spf\O\left< T, 1/T\right>$的自同构$T\mapsto 1/T$粘合而成, 是形式概型.
\end{example}

% \begin{example}
%     考虑仿射概形$X = \spec\Z_p[T_0, T_1]/(T_0T_1 - p)$.
% \end{example}

\subsubsection{形式概形作为函子}\label{sec: formal scheme as functor}

任何$\O$上的概形$X$定出$\O$-代数范畴$\alg_\O$上的函子$R\mapsto X(R)$, 其沿特殊纤维的完备化$\hat{X}$定出$\varpi$-进完备$\O$-代数范畴$\compl$上的函子\[R\mapsto \hat{X}(R) = \Hom_\O(\spf R, X).\]

\begin{proposition}\label{fomal completion = restriction}
    成立函子同构$\hat{X} \simeq X|_{\compl}$.
\end{proposition}

\begin{proof}
    只需对仿射概形$X = \spec A$验证. 由于任何完备$\O$-代数都是$\varpi$-幂零$\O$-代数的逆向极限, 而逆向极限与$\Hom(A, -)$交换, 我们只要验证上述函子限制在$\nilp$上成立,
    即\[\Hom_{\O}(A, R) \simeq \Hom_{\O, \cont}(\varprojlim A/\varpi^n, R),\ \forall R\in\nilp.\]

    记$\hat{A} = \varprojlim A/\varpi^n$. 同态$\hat{A}\to R$自然给出同态$A \to\hat{A}\to R$.
    反之, 给定$\O$-同态$A\to R$, 由于$\varpi$在$R$中幂零, 对充分大的自然数$n$有交换图\[
    \begin{tikzcd}
    A \arrow[d] \arrow[r]         & R, \\
    A/\varpi^n \arrow[ru, dashed] &  
    \end{tikzcd}\]
    从而诱导出$\hat{A}\to R$. 可以直接验证这两个对应互逆.
\end{proof}

% \section{刚性解析几何}


% \section{四元数代数}
% \begin{definition}
%     设$K$是特征不等于$2$的域, $a, b\in K^\times$. 定义$(a, b)_K$为由基$\{1, i, j, ij\}$生成的$K$-代数, 其乘法由\[i^2 = a,\ j^2 = b,\ ij = -ji\]给出.
%     称$(a, b)_K$是$K$上的一个\textbf{四元数代数(quaternion algebra over $K$)}, $\{1, i, j, ij\}$是$(a, b)_K$的一组\textbf{四元数基(quaternion basis)}.
% \end{definition}
% Hamilton四元数即为$R$上的四元数代数$(-1, -1)_\R$.
% \subsection{局部域上的四元数代数}
% \subsection{$\Q$上的四元数代数}

\vspace{10em}
\noindent{\heiti{\zihao{4} \textbf{作者签名:}}\dunderline[-2pt]{0.5pt}{\hspace{20em}}}


\newpage
\end{spacing}
\nocite{*}
\addcontentsline{toc}{section}{\heiti 参考文献}
\bibliography{references}


\newpage
\addcontentsline{toc}{section}{\heiti 致谢}
\section*{致谢}

首先, 我想感谢亲人们长久以来在物质上和精神上对我的支持. 因为他们, 我才得以来到中国人民大学, 并在此学习数学; 他们始终如一的支持让我得以成长为今天的样貌.

感谢老师们在过去四年对我的指导和帮助. 其中王善文老师不仅指导了我的本科毕业论文, 而且一直在数学上指引着我. 正是在他的引导下, 我产生了对数论与代数几何的强烈兴趣, 并立志在未来继续钻研.  

感谢我的同学和朋友们, 他们的陪伴在学术和生活方面于我不可或缺.
特别地, 我要感谢姜杰东师兄一次又一次耐心而详尽地解答我na\"ive的数学问题.

本文至此结束, 但这只是一条长路的起点, 而我希望自己有意志与能力长久地走下去.

P.S. 嘿,我知道已经晚了。但是至少在这个私人版本里,我可以感谢陆老师。您使我发生的改变超过了一切数学或文章的影响之总和。谢谢。

\end{document}