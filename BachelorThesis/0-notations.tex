% \section{绪论}


\subsection*{符号说明}
在这篇文章中, 固定素数$p$以及特征零而剩余类域特征$p$的非阿局部域$K$, 即$\Q_p$的有限扩张$K$.
记$\O := K^\circ$为$K$的整数环, $\varpi$为一个选定的素元(uniformizer), $k := \O/\varpi$为剩余类域, $q := \#k$为剩余类域的阶.
选定$K$的代数闭包$\bar{K}$及其完备化$C := \hat{\bar{K}}$.
局部域$K$上的范数由$|\varpi| = q^{-1}$规范, 并延拓至$C$上.


除非特别说明, 我们约定环为含幺环.

记集合范畴为$\set$, Abel群范畴为$\abel$, 交换环范畴为$\cring$. 交换环$A$上的模范畴记作$\Mod_A$, 含幺代数范畴记作$\alg_A$.

对环$A$, 记$A^\times$为其单位群, $A^\op$为其反环. 如果$A$是交换环(相应地, 分次环), $M$是$A$-模(相应地, 分次$A$-模), 由$M$的给出的$\spec A$ (相应地, $\proj A$)上拟凝聚层记作$\tilde{M}$.

对拓扑空间$X$, 集合(或群, 环, 代数等) $G$给出的$X$上常值层记作$\underline{G}_X$或在底空间明确时略去下标.

对环层空间(ringed space) $X$, 记其结构层(structure sheaf)为$\mathscr O_X$, 底空间为$|X|$(或在含义清楚时以$X$代替).

对于范畴$\mathfrak{C}$, 以$x\in\mathfrak{C}$表示$x$是$\mathfrak{C}$的对象.
记$x\in\mathfrak{C}$的恒等态射为$\Id_x$.

对于集合$X$上的等价关系$\sim$, 记商映射$X\surject X/\sim$为$x\mapsto [x]$.