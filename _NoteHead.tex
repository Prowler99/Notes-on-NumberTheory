% \documentclass{article}
\usepackage{fontspec}
\usepackage{amsmath, amssymb, amsthm, amsbsy, mathrsfs}
\usepackage{stmaryrd}
\usepackage{enumerate}

\usepackage{xr-hyper} % 支持跨文件的超链接
\usepackage[colorlinks,
linkcolor=cyan,
anchorcolor=blue,
citecolor=blue,
]{hyperref}
\usepackage[capitalize]{cleveref}

\externaldocument[UND-]{../Undergrad/something}
\externaldocument[ANT-]{../ANT-2022-summer/note_summer_ANT}
\externaldocument[EC-]{../EC-2024/notes_elliptic_curves}
\externaldocument[LF-]{../LocalField-2024/notes_cft}
\externaldocument[MF-]{../ModularForm-2024-fall/note_modularForm}

\usepackage[margin = 1in, headheight = 12pt]{geometry}
\usepackage{bbm}
\usepackage{tikz-cd}

% \setmainfont{Arial}

\linespread{1.2}

\theoremstyle{definition}

\newtheorem{theorem}{Theorem}

% This environment is terrible
\newenvironment{theoremprime}[1]
  {\renewcommand{\thetheorem}{\ref{#1}$'$}%
   \addtocounter{theorem}{-1}%
   \begin{theorem}}
  {\end{theorem}}

\newtheorem{definition}{Definition}
\newtheorem{defthm}[definition]{Def-Thm}
\newtheorem{propdef}[definition]{Prop-Def}
\newtheorem{exercise}{Exercise}[section]
\newtheorem{problem}{Problem}[section]
\newtheorem{example}{Example}[section]
\newtheorem{proposition}{Proposition}[section]
\newtheorem{lemma}{Lemma}[section]
\newtheorem{corollary}{Corollary}[section]

\theoremstyle{remark}
\newtheorem*{remark}{Remark}

\renewcommand{\Re}{\mathop{\mathrm{Re}}}
\renewcommand{\Im}{\mathop{\mathrm{Im}}}
\renewcommand{\bar}{\overline}
\renewcommand{\tilde}{\widetilde}
\renewcommand{\hat}{\widehat}

% 新命令
% 数学对象
    \newcommand{\R}{\mathbb{R}}
    \newcommand{\C}{\mathbb{C}}
    \newcommand{\Q}{\mathbb{Q}}
    \newcommand{\Z}{\mathbb{Z}}
    \DeclareMathOperator{\GL}{GL}
    \DeclareMathOperator{\SL}{SL}
    \newcommand{\p}{\mathfrak{p}}
    \renewcommand{\P}{\mathbb{P}}
    \newcommand{\A}{\mathbb{A}}
    \newcommand{\Ga}{{\mathbb{G}_{\mathrm{a}}}}
    \newcommand{\Gm}{{\mathbb{G}_{\mathrm{m}}}}
% 集合
    \newcommand{\sminus}{\smallsetminus} % (集合)差
    \newcommand{\inject}{\hookrightarrow} % 单射
    \newcommand{\surject}{\twoheadrightarrow}
% 范畴
    \newcommand{\op}[1]{{#1}^{\mathrm{op}}} % 反范畴
    \DeclareMathOperator{\Hom}{Hom}
    \DeclareMathOperator{\enom}{End} % 自态射
    \DeclareMathOperator{\End}{End} % 自态射
    \DeclareMathOperator{\isom}{Isom} % 同构
    \DeclareMathOperator{\aut}{Aut} % 自同构
    \DeclareMathOperator{\im}{im} % 像
    \newcommand{\Set}{\mathbf{Set}} % 集合范畴
    \newcommand{\Abel}{\mathbf{Ab}} % 群范畴
    \newcommand{\Ring}{\mathbf{Ring}}
    \newcommand{\Cring}{\mathbf{CRing}}
    \newcommand{\Alg}{\mathbf{Alg}}
    \newcommand{\Mod}{\mathbf{Mod}}
    \DeclareMathOperator{\Id}{id}
%向量空间, 矩阵
    \DeclareMathOperator{\mat}{Mat}
    \DeclareMathOperator{\rank}{rank} % 秩
    \DeclareMathOperator{\tr}{Tr} % 迹
    \newcommand{\tran}[1]{{#1}^{\mathrm{T}}} % 转置
    \newcommand{\ctran}[1]{{#1}^{\dagger}} % 共轭转置
    \newcommand{\itran}[1]{{#1}^{-\mathrm{T}}} % 逆转置
    \newcommand{\ictran}[1]{{#1}^{-\dagger}} % 逆共轭转置
    \DeclareMathOperator{\codim}{codim} % 余维数
    \DeclareMathOperator{\diag}{diag} % 对角阵
    \newcommand{\norm}[1]{\left\| #1\right\|} % 范数
    \DeclareMathOperator{\lspan}{span} % 张成
% 群
    \DeclareMathOperator{\inn}{Inn} % (群)内自同构
    \newcommand{\nsg}{\vartriangleleft} % 正规子群
    \newcommand{\gsn}{\vartriangleright} % 正规子群
    \DeclareMathOperator{\ord}{ord} % 元素的阶
    \DeclareMathOperator{\stab}{Stab} % 稳定化子
    \DeclareMathOperator{\sgn}{sgn} % 符号函数
% 环, 域
    \DeclareMathOperator{\cha}{char} % 特征
    \DeclareMathOperator{\spec}{Spec} % 素谱
    \DeclareMathOperator{\maxspec}{MaxSpec} % 极大谱
    \newcommand{\spm}{\maxspec} %极大谱
    \DeclareMathOperator{\gal}{Gal} % Galois群
    \DeclareMathOperator{\Gal}{Gal} % Galois群
    \DeclareMathOperator{\Frac}{Frac} % 分式环
% 同调代数
    \DeclareMathOperator{\ext}{Ext} 
    \DeclareMathOperator{\Tor}{Tor}
% 微积分
    \newcommand*{\dif}{\mathop{}\!\mathrm{d}} %(外)微分算子
% 流形
    \DeclareMathOperator{\lie}{Lie}
%代数几何
    \DeclareMathOperator{\proj}{Proj} % 射影谱
%多项式
    \DeclareMathOperator{\disc}{disc} % 判别式
    \DeclareMathOperator{\res}{res} % 结式
% 结构简写
    \newcommand{\pdfrac}[2]{\dfrac{\partial #1}{\partial #2}} % 偏微分式
    \newcommand{\isomto}{\stackrel{\sim}{\rightarrow}} % 有向同构
    \newcommand{\gene}[1]{\left\langle #1 \right\rangle} % 生成对象
% 文字缩写
    \newcommand{\opin}{\;\mathrm{open\;in}\;}
    \newcommand{\st}{\;\mathrm{s.t.}\;}
    \newcommand{\ie}{\;\mathrm{i.e.,}\;}

% 重定义命令
\renewcommand{\hom}{\mathop{\mathrm{Hom}}}
\renewcommand{\vec}{\boldsymbol}
\renewcommand{\and}{\;\text{and}\;}

% 编号
\newcommand{\cnum}[1]{$#1^\circ$} %右上角带圆圈的编号
\newcommand{\rmnum}[1]{\romannumeral #1}
\newcommand{\myit}{$\diamond$}

% \tikzcdset{scale cd/.style={every label/.append style={scale=#1},
%     cells={nodes={scale=#1}}}}
