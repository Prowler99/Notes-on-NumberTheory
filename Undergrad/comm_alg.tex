\section{Commutative Algebra}

% \subsection{Exact Functors}
\subsection{Nakayama Lemma}

\subsection{Flatness}
Recall:
\begin{center}
    right-adjoint
    $\implies$ preserve $\varprojlim$
    $\implies$ left-exact
    $\iff$ right-derivative
    $\iff$ preserve \textit{finite} $\varprojlim$
\end{center}

\subsubsection{Definition}
Let $A$ be a commutative ring, $M$ an $A$-module.
We say $M$ is \textbf{flat} over $A$,
if the tensor-with-$M$ functor $(-)\otimes_A M$ is exact;
i.e., the tensor-with-$M$ functor preserves injections:
\[N\hookrightarrow N'\implies
N\otimes_A M\hookrightarrow N'\otimes_A M'.\]
\begin{proposition}
[Basic properties of flat modules]\label{basic property of flat}
Let $A$ be a commutative ring, $B$ an $A$-algebra.
\begin{enumerate}
    \item [(a)] free $\implies$ flat.
    \item [(b)] (Tensor) $M$ flat over $A$ \& $N$ flat over $A\implies$
    $M\otimes_A N$ flat over $A$.
    \item [(c)] (Base change)
    $M$ flat over $A\implies M\otimes_A B$ flat over $B$.
    \item [(d)] (Transitivity)
    $B$ flat over $A$ \& $M$ flat over $B\implies M$ flat over $A$. 
\end{enumerate}
\end{proposition}

\begin{theorem}
    An $A$-module $M$ is flat if and only if for every ideal $I$ of $A$,
    $I\otimes_A M\to IM$ is an isomorphism.
\end{theorem}
\begin{corollary}
    Over a PID, flat $\iff$ torsion-free.
\end{corollary}

\subsubsection{Local Nature of Flatness}

\begin{corollary}
    Over a Dedekind domain, flat $\iff$ torsion-free.
\end{corollary}



\subsection{Cyclotomic Extensions}

Fix an algebraic closure $\bar{F}$ of a field $F$.
An $n$-th root of unity is $\zeta\in F$ s.t. $\zeta^n = 1$.
A \textbf{primitive $n$-th root of unity} is an $n$-th root of unity $\zeta\in\mu_n(\bar{F})$ s.t. \[\zeta^d = 1\iff n\mid d.\]
\begin{proposition}\label{basic property of mu-n when char not divide n}
    Assume $\cha F\nmid n$, then:
\begin{itemize}
    \item $\mu_n(\bar{F})\simeq \Z/n\Z$ as group, and the generatos of $\mu_n(\bar{F})$ are precisely the $n$-th \textit{premitive} roots of unity.
    \item $F(\mu_n)$ is the splitting field of $X^n - 1$ over $F$, and $F(\mu_n)/F$ is Galois with an embedding \[\chi_n : \gal(F(\mu_n)/F)\hookrightarrow (\Z/n\Z)^\times\]
    defined by \[\sigma(\zeta) = \zeta^{\chi_n(\sigma)},\quad \forall \zeta\in \mu_n,\ \sigma\in\gal(F(\mu_n)/F).\]
\end{itemize}
\end{proposition}

\subsubsection*{Cyclotomic Polynomials}

\begin{definition}
    The $n$-th \textbf{cyclotomic polynomial} is \[\Phi_n(X) := \prod_{d\mid n}\left( X^d - 1 \right)^{\mu(n/d)},\]where $\mu : \Z_{\ge 1}\to \{0, \pm 1\}$ is the Mobi\"us function.
\end{definition}

\begin{example}
    If $p\in\Z$ is a prime,
    then \[\Phi_{p^n}(X) = \frac{X^{p^n} - 1}{X^{p^{n-1}} - 1},\quad\forall n\in\Z_{\ge 1}.\]
\end{example}

\begin{theorem}\label{characterise cyclotomic polynomials}
    The polynomial $\Phi_n(X)\in\Z[X]$ is monic
    with integral coefficients of degree $\varphi(n) = \# \Z/n\Z$. These polynomials are characterised by\[\prod_{d\mid n}\Phi_d(X) = X^n - 1,\quad \forall n\ge 1.\]
    In addition, $\Phi_n(X)$ is irreducible over $\Q$.
\end{theorem}



















