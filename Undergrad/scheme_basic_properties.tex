\section{Schemes}
In this section, all rings are assumed to be commutative.

\subsection{Some Properties of Schemes}

\subsubsection{Noetherian Schemes}
A scheme $X$ is said to be \textbf{Noetherian} if it is a \textit{finite} union of affine open $X_i$'s such that every $\O_X(X_i)$ is a Noetherian ring.

\begin{proposition}
    Let $X$ be a Noetherian scheme.
\begin{enumerate}
\item [(1)] Every open or closed subscheme of $X$ is Noetherian.
\item [(2)] $\O_{X, x}$ is Noetherian for all $x\in X$.
\item [(3)] $\O_X(U)$ is Noetherian for all $U\subset X$ affine open.
\end{enumerate}
\end{proposition}
\begin{proof}
    The statement (1) and (2) essentially follows from that localization and taking quotient preserves Noetherianity.
    
    Let $\{X_i\}_i$ be a finite affine Noetherian covering of $X$, and $U\subset X$ be affine open. We need to show that $\O_X(U)$ is Noetherian.
    By (1), $U_i := U\cap X_i$ is an affine Noetherian for each $i$. Let $I$ be an ideal of $\O_X(U)$.
    For every $i$, the ideal $I\O_X(U_i)$ of $\O_X(U_i)$ is finitely generated,
    so there is an ideal $J$ of $\O_X(U)$ s.t. $J\O_X(U_i) = I\O_X(U_i)$ for all $i$.
    Since $\{U_i\}$ covers $U$,
    we hvae $J\O_{X, x} = I\O_{X, x}$ for all $x\in U$, i.e.,
    $I/J\otimes_{\O_X(U)} \O_X(U)_\p$ for every $\p\in\spec \O_X(U) \simeq U$.
    Hence $I/J = 0$, and $I$ is finitely generated.
\end{proof}

\subsubsection{Reduced Schemes}

A scheme $X$ is said to be \textbf{reduced at} $x\in X$,
if the local ring $\O_{X, x}$ is reduced\footnote{
    A ring $A$ is reduced if its nilradical
$\nil(A) = 0$.
}.
We say that $X$ is \textbf{reduced} if it is reduced at every point.

\begin{proposition}\label{reduced is Zariski local}
    Let $X$ be a scheme. Then TFAE:
    \begin{enumerate}
    \item [(a)] $X$ is reduced.
    \item [(b)] $\forall U\subset X$ open, $\O_X(U)$ is reduced.
    \item [(c)] $\exists$ affine cover $\{U_i\}$ of $X$, s.t. all $\O_X(U_i)$ are reduced.
    \end{enumerate}
\end{proposition}

To prove this, we consider an ideal sheaf $\mathcal{N}_X$ on $X$ defined by
\[\mathcal{N}_X(U) := \left\{s\in \O_X(U)\mid s_x\in\nil(\O_{X, x})\right\},\]
so that $X$ is reduced $\iff\mathcal{N}_X = 0$


\begin{proposition}\label{reduced subscheme structure on closed subset}
    Let $Z$ be a closed subset of the scheme $X$.
    Then there is a unique structure of reduced scheme on $Z$, making it a closed subscheme of $X$.

    In particular, there is a unique closed subscheme $i : X_{\red}\hookrightarrow X$
    with the same topological space as $X$.
    More precisely, $X_{\red} = V(\mathcal{N}_X) = (X, \O_X/\mathcal{N}_X)$.
    This subscheme $X_{\red}$ verifies the following property:
    \begin{itemize}
        \item If $Y$ is a reduced scheme,
        then any morphism $Y\to X$ factors through $Y\to X_{\red}\to X$.
    \end{itemize}
\end{proposition}

\begin{proof}[Proof of \cref{reduced is Zariski local} and \cref{reduced subscheme structure on closed subset}]
    Let $U\subset X$ be an affine open.
    Then since taking radical commutes with localization,
    we have $\mathcal{N}_X(U) = \nil(\O_X(U))$.
    This proves \cref{reduced is Zariski local}.
    We also see that $|V(\mathcal{N}_X)| = |X|$.

    Next, we verify that the closed ringed subspace $X_{\red} := (X, \O_X/\mathcal{N}_X)$
    of $X$ is a scheme (and, of course, reduced); that is, it is locally affine. So assume $X = \spec A$.
    Let $N = \nil(A)$ and $i : \spec A/N\to \spec A$ the closed embedding.
    For any $g\in A$,
    we know that the kernel of
    \[A_g = \O_{\spec A}(D(g))\to \O_{\spec A/N}(D(\bar g)) = (A/N)_{\bar g}\]
    is $N\otimes_A A_g$,
    which $ = \nil(A_g) = \mathcal{N}_X(D(g))$;
    i.e, $\ker i^\# = \mathcal{N}_X$.
    Hence $X_{\red} \simeq \spec A/N$ is a scheme.

    Now let $Z$ be a closed subset of $X$.
    We first show that the reduced closed subscheme $(Z, \O_Z)$ is unique if it exists.
    For an affine open $U\subset X$,
    $(Z\cap U, \O_Z|_{Z\cap U})$ is a closed subscheme of $U$,
    so it is isomorphic to $\spec \O_X(U)/I$
    for some ideal $I\subset \O_X(U)$.
    As a subset of $U = \spec \O_X(U)$,
    $Z\cap U = V(I)$.
    Since $Z$ is reduced, $\nil(\O_X(U)/I) = 0$,
    namely $I$ is radical.
    If there is another reduce closed subscheme structure of $X$ on $Z$,
    it must also be given by a radical ideal $J\subset\O_X(U)$ on $U$,
    and $V(I) = V(J)$. But for radical ideals, this means $I = J$.

    To make $Z$ a reduced closed subscheme,
    take affine cover $\{U_i = \spec A_i\}_i$ of $X$
    and ideals $I_i\subset A_i$ s.t. $Z\cap U_i = V(I_i)$.
    We endow $Z\cap U_i$ with the reduced scheme structure of $\spec A_i/\sqrt{I_i}$ and use uniqueness to glue them.
\end{proof}

\subsection{Connected Schemes}

\subsection{Irreducible Schemes}
A topological space $X$ is \textbf{irreducible}
if it is \textit{not} the union of proper closed subsets, or equivalently, any two nonempty open subsets intersects.
The set of irreducible subsets in a space $X$ admits maximal w.r.t.\! inclusions,
called \textbf{irreducible components}.
\begin{proposition}
    Let $X$ be a topological space.
\begin{itemize}
    \item [(1)] The closure of a irreducible subset is irreducible.
    \item [(2)] If $X$ is irreducible, and $U\subset X$ is a non-empty open set, then $U$ is irreducible and dense in $X$.
    \item [(3)]
\end{itemize}
\end{proposition}
\begin{proof}
    (1) Trivial.\par
    (2) Easy to check that $\bar U = X$.
    Any two nonempty open subsets of $U$ is still an open subset of $X$, and hence it is dense in $X$. In particular, this intersection is nonempty.\par
\end{proof}

A scheme $X$ is said to be irreducible if the underlying topological space $|X|$ is irreducible.


\subsection{Integral Schemes}
A scheme $X$ is said to be \textbf{integral at} $x\in X$, if the local ring $\O_{X, x}$ is an integral domain.
We say that $X$ is \textbf{integral} if it is reduced and irreducible.


\subsection{Normal Schemes}

A scheme $X$ is said to be \textbf{normal at}
$x\in X$, if the local ring $\O_{X, x}$ is (an integral domain and) normal\footnote{
    An integral domain $A$ is normal if it is integrally closed in its fraction field.
}.

\subsection{Regular Schemes}



\section{Some Properties of Morphisms}

\subsection{Seperated (in the sense of topology) Morphisms}
A morphism $f : X\to Y$ of schemes is said to be
\textbf{seperated}, if the diagonal map \[\Delta_{X/Y} : X\to X\times_Y X\]
is a closed immersion.
\begin{proposition}
    Any morphism of affine schemes is seperated.

    If $f : X\to Y$ is a morphism, then $f$ is seperated if and only if $\Delta(X)\subset X\times_YX$ is a closed subset.
\end{proposition}
\begin{proof}
    Let $\spec B\to \spec A$.
    The diagonal map $\spec B\to \spec (B\otimes_A B)$
    is induced by
    \[B\otimes_A B\to B\quad b\otimes b'\mapsto bb',\]
    which is surjective,
    meaning that it is a quotient map and thus gives a closed immersion.

    The second assertion comes from this lemma:
    \begin{lemma}\label{closed immersion iff 1.locally closed immersion 2.image closed}
        Let $f : X\to Y$ be a morphism of schemes (or is ringed space enough?).
        If\begin{enumerate}
        \item [(\rmnum{1})] $\exists Y_i\subset Y$ s.t. $X = \bigcup_{i}f^{-1}(X_i)$, and $f|_{f^{-1}Y_i} : f^{-1}(Y_i)\to Y_i$ are closed immersions, and
        \item [(\rmnum{2})] $f(X)\subset Y$ is a closed subset,
        \end{enumerate}
        then $f : X\to Y$ is a closed immersion.
    \end{lemma}
    \noindent\textit{Proof of the lemma.}
    (T.B.C.)
\end{proof}

\begin{proposition}\label{seperated is Zariski global}
    Let $X$ be a scheme. Write $\Delta = \Delta_{X/\Z}$.
    Then TFAE:
    \begin{enumerate}
\item [(a)] $X$ is seperated\footnote{If $\mathcal{P}$ is a property for morphisms, we say an $S$-scheme $X$ has $\mathcal{P}$ if the structure map $X\to S$ has $\mathcal{P}$.}.
\item [(b)] $\forall U, V\subset X$ affine open,
the map $\O_X(U)\otimes_\Z \O_X(V)\to \O_X(U\cap V)$ is surjective;
or equivalently, the map
$U\cap V = \Delta^{-1}(U\times_\Z V)\stackrel{\Delta}{\to} U\times_\Z V$ is a closed immersion.
\item [(c)] $\exists$ affine cover $\{U_i\}$ of $X$,
s.t. the map $\O_X(U_i)\otimes_\Z\O_X(U_j)\to \O_X(U_i\cap U_j)$ is surjective for all $i, j$.
% \item [(d)] Any morphism $X\to Y$ of schemes is seperated.
\end{enumerate}
\end{proposition}
\begin{proof}
    The implication $(\mathrm{a})\implies(\mathrm{b})\implies(\mathrm{c})$ is clear.
    To prove that (c) $\implies$ (a), assume (c), namely all $\Delta^{-1}(U_i\times_\Z U_j)\to U_i\times_\Z U_j\subset X\times_\Z Y$ are closed immersions.
    By \cref{closed immersion iff 1.locally closed immersion 2.image closed}, $X$ is seperated.

    % It is left to prove (a) $\implies$ (d).

\end{proof}

\subsection{Quasi-Compact Morphisms}
A morphism $f: X\to Y$ of schemes is said to be
\textbf{quasi-compact}, if the inverse image $f^{-1}(V)$ of any affine open $V\subset Y$ is quasi-compact.
\begin{proposition}
    Let $f : X\to Y$ be a morphism of schemes.
    \begin{enumerate}
    \item [(1)] If $f$ is a closed immersion, then $f$ is quasi-compact.
    \item [(2)] If $f$ is an open immersion, and $Y$ is locally Noetherian, then $f$ is quasi-compact.
    \end{enumerate}
\end{proposition}


\subsection{Morphisms of Finite Type}
A ring homomorphism $A\to B$ is of finite type if it make $B$ a finitely generated $A$-algebra. 

A morphism $f: X\to Y$ of schemes is said to be
\textbf{of finite type}, if
\begin{itemize}
    \item $f$ is quasi-compact, and
    \item For any affine open $V\subset Y$ and affine open $U\subset f^{-1}(V)$, the map $\O_Y(V)\to\O_X(f^{-1}V)\to\O_X(U)$ is of finite type.
\end{itemize}

\begin{proposition}
    Let $f : X\to Y$ be a morphism of schemes.
    If $\exists$ affine cover $\{V_i\}_i$ of $Y$, 
    s.t.\begin{enumerate}
        \item [(\rmnum{1})] $f^{-1}(V_i)\subset X$ is a \textit{finite} union of affine open $U_{ij}$ for all $i$, and
        \item [(\rmnum{2})] $\O_Y(V_i)\to \O_X(U_{ij})$ is of finite type for all $i, j$,
    \end{enumerate}
    then $f$ is of finite type.
\end{proposition}

\subsection{Proper Morphisms}

\subsection{Flat Morphisms}
A morphism $f : X\to Y$ of schemes is said to be
\textbf{flat at} $x\in X$, if $f^\#_x : \O_{Y, f(x)}\to \O_{X, x}$ is flat.
We say that $f$ is \textbf{flat} if it is flat at every point.

\subsection{Unramified Morphisms}

\subsection{\'Etale Morphism}

\subsection{Smooth Morphisms}

