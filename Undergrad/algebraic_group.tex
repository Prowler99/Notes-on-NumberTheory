\section{Linear Algebraic Groups}

\subsection{First Definitions and Properties}
In this note, a variety over $k$ is
a geometrically reduced seperated (rellay?) scheme of finite type over $k$.
and a group scheme (resp. algebraic group over $k$)
is a group object in the category of schemes
(resp.\! varieties over $k$).

There are two particular types of algebraic groups:
\begin{itemize}
    \item a \textbf{linear group} is an algebraic group that is affine, and
    \item an \textbf{abelian variety} is an algebraic group that is connected and complete.
\end{itemize}

\begin{theorem}[Chevalley]
    For an algebraic group $G$,
    there is a maximal linear subgroup $G_\mathrm{aff}$ of $G$, which is normal and the quotient $A(G) := G/G_{\mathrm{aff}}$ is an abelian variety.
\end{theorem}

This note focuses mainly on linear algebraic groups $G$.
In this case, $G = \spec A$ for a reduced (maybe geometrically?) $k$-algebra of finite type, and the group structure on $G$ is equivalently to a \textbf{Hopf algebra} structure on $A$.

\begin{theorem}
    Any linear algebraic group is a closed subgroup of some $\GL_n$.
\end{theorem}