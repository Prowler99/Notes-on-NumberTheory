\section{Group Cohomology}
Recall that for a group $G$ and a $G$-mod $M$,
we define \begin{align*}
    H^1(G, M) = \frac{Z^1(G, M )}{B^1(G, M )}
    = \frac{\{f  :G\to M\mid f(ab) = af(b) + f(a)\}}{\{g\mapsto gm - m\mid m\in M \}}.
\end{align*}
We apply this construction to:\begin{itemize}
    \item $G = $ a congruence subgroup $\Gamma < \SL_2(\Z)$,
    \item $M = V_n(R)$ as follows. Let $R$ be a ring, $n\in\Z_{\ge 1}$.
    Define \[R[X, Y]_n := \{\text{homogeneous polynomials of degree }n\},\]
    a free $R$-module of rank $n+1$.
    The monoid $\mathrm{M}_2(\Z)\cap\GL_2(\Q)^+$ acts on $R[X, Y]_n$ by
    \[\left( \begin{pmatrix}
        a & b\\ c & d
    \end{pmatrix}P \right)(X, Y) := P\left( \begin{pmatrix}
        X & Y 
    \end{pmatrix} \begin{pmatrix}
        a & b \\ c & d
    \end{pmatrix}\right) = P(aX + cY, bX + dY);\]
    this is the \textit{left} action on $R[X, Y]\hookrightarrow \{\text{function } R\times R\to R\}$
    induced by the \textit{right} action on $R\times R$.
    \par $\leadsto V_n(R) := R[X, Y]_n$ with its $\SL_2(\Z)$-action.

    Note that $V_n(R)\simeq \mathrm{Sym}^nR^2$, where $R^2$ is equipped with the standard $\SL_2(\Z)$-action.
\end{itemize}
We will show that $H^1(\Gamma, V_n(\C))$ ``resembles''
a space of modular forms. It has an integral structure
\[H^1(\Gamma, V_n(\Z))\hookrightarrow H^1(\Gamma, V_n(\C)),\]
which could give rise to the $\Z$-lattice we used in the last section.
\begin{proposition}
    If $S$ is flat over $R$, then as $S$-modules,
    \[H^1(\Gamma, V_n(S))\simeq H^1(\Gamma, V_n(R))\otimes_R S. \]
\end{proposition}
% \begin{proof}
% Flatness of $R\to S$ yields an exact sequence
% \[0\to B^1_R\otimes_RS\to Z^1_R\otimes_RS\to H^1_R\otimes_RS\to 0.\]
% \begin{itemize}
% \item $B^1_R\otimes_RS\simeq B^1_S$.\par
% For $P\in V_n(R)$, define $\alpha_P\in B^1_R$ by
% \[\alpha_P(g) := gP - P,\quad g\in\Gamma.\]
% The embedding $B^1_R\hookrightarrow C^1_R$ gives an embedding $B^1_R\otimes_R S\hookrightarrow C^1_R\otimes_RS$.
% As symmetric powers are preserved by base change,
% $V_n(R)\otimes_R S \simeq V_n(S)$ as $S$-modules.
% Since $R\to S$ is flat,
% \[C^1_R\otimes_R S =\left( \prod_\Gamma V_n(R) \right)\otimes_R S
% \simeq \prod_\Gamma\left( V_n(R)\otimes_R S  \right) \simeq C^1_S.\]
% Finally, the image of $B_R^1\otimes S\to C^1_R\otimes_R S \simeq C^1_S$
% is the set of finite sum\[\sum_{i}s_i\alpha_{P_i} = \alpha_{\sum_i s_iP_i},\quad s_i\in S, P_i\in V_n(R),\]
% which is exactly $B^1_S$.

% \item $Z^1_R\otimes_R S \simeq Z^1_S$.\par
% Again, this is equivalent to
% $\im (Z^1_R\otimes_RS\to C_S^1) = Z_S^1$.
% Let $f\in Z_S^1$.
% Since $\Gamma$ is finitely generated, the cocycle $f$ is determined by its values on a finite set of generators $\{g_1, \dots, g_r\}$ of $\Gamma$.
% Now for $g\in \Gamma$,
% write \[f(g) = \sum_{i=0}^{n}s_i(g)X^iY^{n-i}.\]






% \end{itemize}


% \end{proof}

\subsection{The Eichler-Shimura map}
Define the space of \textbf{anti-holomorphic cusp forms}
\[\overline{S_k(\Gamma)} := \{z\mapsto \overline{f(z)}\mid f\in S_k(\Gamma)\}.\]

\begin{definition}
    For $n\ge 0$, $u, v\in \mathcal{H}$, $f\in M_{n+2}(\Gamma)$,
    define \begin{align*}
        I_f(u, v) &:= \int_u^v f(z) (Xz + Y)^n dz\\
        I_{\bar{f}}(u, v) &:=
        \int_u^v \overline{f(z)}(X\bar{z} + Y)^n dz.
    \end{align*}
    These integrals are in $V_n(\C)$.
\end{definition}
\begin{lemma}\label{lem: property of If(u v)}
    Let $f\in M_{n+2}(\Gamma)$ or $S_{n+2}(\Gamma)$, $u, v, w\in\mathcal{H}$.\begin{itemize}
        \item $I_f(u, w) = I_f(u, v) + I_f(v, w)$.
        \item If $\gamma\in \mathrm{M}_2(\Z)\cap\GL_2(\Q)^+$,
        then \[I_f(\gamma u, \gamma v) = \left( \det g \right)^{-n}\gamma I_{f\mid_{n+2}\gamma}(u, v).\]
        In particular, if $\gamma\in\Gamma$, then \[I_f(\gamma u, \gamma v) = \gamma I_f(u, v).\]
    \end{itemize}
\end{lemma}
\begin{proof}
    The first identity is a part of definition of integral.
    We compute the second.
    \begin{align*}
        I_f(\gamma u, \gamma v) = 
    \end{align*}
\end{proof}

\begin{theorem}
    The map
% https://q.uiver.app/#q=WzAsNCxbMCwwLCJNX3tuKzJ9KFxcR2FtbWEpXFxvcGx1cyBcXG92ZXJsaW5le1Nfe24rMn0oXFxHYW1tYSl9Il0sWzEsMCwiSF4xKFxcR2FtbWEsIFZfbihcXEMpKSJdLFswLDEsIihmLCBcXGJhcntnfSkiXSxbMSwxLCJcXGxlZnQoIFxcZ2FtbWFcXG1hcHN0byBJX2YoYSwgXFxnYW1tYSBhKSArIElfe1xcYmFye2d9fShiLCBcXGdhbW1hIGIpIFxccmlnaHQpIl0sWzAsMV0sWzIsMywiIiwwLHsic3R5bGUiOnsidGFpbCI6eyJuYW1lIjoibWFwcyB0byJ9fX1dXQ==
\[\begin{tikzcd}
	{M_{n+2}(\Gamma)\oplus \overline{S_{n+2}(\Gamma)}} & {H^1(\Gamma, V_n(\C))} \\
	{(f, \bar{g})} & {\left( \gamma\mapsto I_f(a, \gamma a) + I_{\bar{g}}(b, \gamma b) \right)}
	\arrow[from=1-1, to=1-2]
	\arrow[maps to, from=2-1, to=2-2]
\end{tikzcd}\]
where $a, b\in\mathcal{H}$ are arbitarily chosen, is a well-defined isomorphism, called the \textbf{Eichler-Shimura map}.
\end{theorem}
It won't be proved in this course that this is an isomorphism.
\begin{proof}
    [Proof that this is well defined]
\end{proof}

















