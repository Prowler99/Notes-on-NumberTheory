\documentclass{article}
\documentclass{article}
\usepackage{amsmath, amssymb, amsthm, amsbsy, mathrsfs, stmaryrd}
\usepackage{enumitem}
\usepackage[colorlinks,
linkcolor=cyan,
anchorcolor=blue,
citecolor=blue,
]{hyperref}
\usepackage[capitalize]{cleveref}
\usepackage[margin = 1in, headheight = 12pt]{geometry}
\usepackage{bbm}
\usepackage{tikz-cd}

\newtheorem{theorem}{Theorem}

\theoremstyle{definition}
\newtheorem{definition}{Definition}
\newtheorem{exercise}{Exercise}[section]
\newtheorem{problem}{Problem}
\newtheorem{example}{Example}
\newtheorem{proposition}{Proposition}[section]
\newtheorem{lemma}{Lemma}[section]
\newtheorem{corollary}{Corollary}[section]

\theoremstyle{remark}
\newtheorem*{remark}{Remark}

\renewcommand{\Re}{\mathop{\mathrm{Re}}}
\renewcommand{\Im}{\mathop{\mathrm{Im}}}

% 新命令
% 数学对象
    \newcommand{\R}{\mathbb{R}}
    \newcommand{\C}{\mathbb{C}}
    \newcommand{\Q}{\mathbb{Q}}
    \newcommand{\Z}{\mathbb{Z}}
    \DeclareMathOperator{\GL}{GL}
    \DeclareMathOperator{\SL}{SL}
    \newcommand{\p}{\mathfrak{p}}
    \renewcommand{\P}{\mathbb{P}}
    \newcommand{\A}{\mathbb{A}}
% 集合
    \newcommand{\sminus}{\smallsetminus} %(集合)差
% 范畴
    \newcommand{\op}[1]{{#1}^{\mathrm{op}}} %反范畴
    \DeclareMathOperator{\enom}{End} %自态射
    \DeclareMathOperator{\isom}{Isom} %同构
    \DeclareMathOperator{\aut}{Aut} %自同构
    \DeclareMathOperator{\im}{im} %像
    \newcommand{\Set}{\mathbf{Set}} %集合范畴
    \newcommand{\Abel}{\mathbf{Ab}} %群范畴
    \newcommand{\Ring}{\mathbf{Ring}}
    \newcommand{\Cring}{\mathbf{CRing}}
    \newcommand{\Alg}{\mathbf{Alg}}
    \newcommand{\Mod}{\mathbf{Mod}}
    \DeclareMathOperator{\Id}{id}
%向量空间, 矩阵
    \DeclareMathOperator{\rank}{rank} %秩
    \DeclareMathOperator{\tr}{Tr} %迹
    \newcommand{\tran}[1]{{#1}^{\mathrm{T}}} %转置
    \newcommand{\ctran}[1]{{#1}^{\dagger}} %共轭转置
    \newcommand{\itran}[1]{{#1}^{-\mathrm{T}}} %逆转置
    \newcommand{\ictran}[1]{{#1}^{-\dagger}} %逆共轭转置
    \DeclareMathOperator{\codim}{codim} %余维数
    \DeclareMathOperator{\diag}{diag} %对角阵
    \newcommand{\norm}[1]{\left\| #1\right\|} %范数
    \DeclareMathOperator{\lspan}{span} %张成
    \DeclareMathOperator{\sym}{\mathfrak{Y}}
% 群
    \DeclareMathOperator{\inn}{Inn} %(群)内自同构
    \newcommand{\nsg}{\vartriangleleft} %正规子群
    \newcommand{\gsn}{\vartriangleright} %正规子群
    \DeclareMathOperator{\ord}{ord} %元素的阶
    \DeclareMathOperator{\stab}{Stab} %稳定化子
    \DeclareMathOperator{\sgn}{sgn} %符号函数
% 环, 域
    \DeclareMathOperator{\cha}{char} %特征
    \DeclareMathOperator{\spec}{Spec} %素谱
    \DeclareMathOperator{\maxspec}{MaxSpec} %极大谱
    \DeclareMathOperator{\gal}{Gal}
% 微积分
    % \newcommand*{\dif}{\mathop{}\!\mathrm{d}} %(外)微分算子
% 流形
    \DeclareMathOperator{\lie}{Lie}
%代数几何
    \DeclareMathOperator{\proj}{Proj}
%多项式
    \DeclareMathOperator{\disc}{disc} %判别式
    \DeclareMathOperator{\res}{res} %结式

% 结构简写
    \newcommand{\pdfrac}[2]{\dfrac{\partial #1}{\partial #2}} %偏微分式
    \newcommand{\isomto}{\stackrel{\sim}{\rightarrow}} %有向同构
    \newcommand{\gene}[1]{\left\langle #1 \right\rangle} %生成对象
% 文字缩写
    \newcommand{\opin}{\;\mathrm{open\;in}\;}
    \newcommand{\st}{\;\mathrm{s.t.}\;}
    \newcommand{\ie}{\;\mathrm{i.e.,}\;}

% 重定义命令
\renewcommand{\hom}{\mathop{Hom}}
\renewcommand{\vec}{\boldsymbol}
\renewcommand{\and}{\;\text{and}\;}

% 编号
\newcommand{\cnum}[1]{$#1^\circ$} %右上角带圆圈的编号
\newcommand{\rmnum}[1]{\romannumeral #1}


\newcommand{\myit}{$\diamond$}

\title{}
\author{}
\date{}

\begin{document}
\maketitle

\end{document}

\usepackage[ruled, vlined]{algorithm2e}


\newcommand{\F}{\mathbb{F}}
\renewcommand{\O}{\mathcal{O}}
\newcommand{\vol}{\mathrm{vol}}
\newcommand{\s}{\mathrm{s}}

\title{Note on Computational Elliptic Curves}
\author{Lei Bichang}

% \newtheorem{problem}{Problem}


\begin{document}
\maketitle

These notes are based the course taught by Prof. Benjamin Wesolowski at ENS Lyon in 2025.

\section*{Basics}

In $\Z$ and $\F_p$\footnote{
    Any $x\in\F_p$ can be represented by $\log(p)$ bits.
}, addition and multiplication are ``easy''\footnote{$\exists$ polynomial (in length of input) time algorithm.}.
\subsection*{Extended GCD algorithm}
\begin{algorithm}[H]
    \caption{Extended GCD}\label{alg: ExtGCD}
    \KwIn{$a, b : \Z_{\ge 0}$}
    \KwOut{$\gcd(a, b), u, v : \Z$ s.t. $\gcd(a, b) = ua + vb$}
    $(x, u, v, y, u', v')\gets (a, 1, 0, b, 0, 1)$\\
    \While{$y > 0$}{
        $(x, y)\gets \left(x\bmod y, \floor{\frac{x}{y}}\right)$\\
        \eIf{$r < y - r$}{
            $(x, u, v, y, u', v')\gets (y, u', v', r, u - qu', v - qv')$
        }{
            $(x, u, v, y, u', v')\gets (y, u', v', y -r, (q + 1)u' - u, (q + 1)v' - v)$
        }
    }\Return{$(x,u,v)$}
\end{algorithm}
\begin{itemize}
    \item Complexity = $O(\log(a)\log(b))$.
    \item It applies to $k[X]$, so it applies to general $\F_{p^r}$; the output is up to a factor in $\F_p^\times$.
\end{itemize}
\subsection*{Exponentiation}
Let $G$ be a group.

\begin{algorithm}[H]
    \caption{Square and Multiply}\label{alg: exponential}
    \KwIn{$g : G$, $m = (m_im_{i-1}\dots m_{0})_2 : \Z_{\ge 0}$}
    \KwOut{$g^m : G$}
    $z\gets x$
    \For{$j = i - 1,\ i - 2,\ \dots,\ 0$}{
        $z\gets z^2$
        \If{$m_j = 1$}{$z\gets z\cdot x$}
    }\Return{$z$}
\end{algorithm}
\begin{itemize}
    \item Complexity = $O(\log(m))$, depending on the complexity of multiplication in $G$.
\end{itemize}

\subsection*{Elliptic Curves}


\section{Factoring polynomials over a finite field}
\begin{problem}
Factorize $f(X)\in\F_q[X]$.
\end{problem}

Let's start with a simpler question: how to find linear factors of $f$?
This is equivalent to factorize \[g(X) := \gcd(f(X), X^q - X) = \prod_{\substack{r\in \F_q\\ f(r) = 0}} (X-r).\]
Note that the map $(\cdot)^{\frac{q-1}{2}}$ on $\F_q^7\times $ is the Legendre symbol $\left( \frac{\cdot}{q} \right)$.
Hence for $u(X)\in \F_q[X]$,
we have \[g_u(X) := \gcd(g(X), u(X)^{\frac{q-1}{2}} - 1) = \prod_{\substack{f(r) = 0 \\ u(r)\in (\F_q^\times)^2}} (X - r).\]
If we take a linear polynomial $u(X) = X + \delta$,
then $g_u(X)$ is a nontrivial factor of $g(X)$
if and only if there are roots $\alpha\ne\beta$ of $f(X)$ such that $\alpha + \delta\in(\F_q^\times)^2$ and $\beta + \delta\notin(\F_q^\times)^2$.

\begin{theorem}[Robin]
    If $\alpha\ne\beta\in\F_q$,
    then \[\#\{\delta\in\F_q\mid \alpha + \delta\ne 0, \beta+\delta\ne 0; \text{ one is a square, the other isn't}\} = \frac{q-1}{2}.\]
\end{theorem}
\begin{proof}
    Let \[\psi : \F_q\sminus\{-\beta\}\to \F_q\sminus \{1\}\quad \delta\mapsto \frac{\alpha + \delta}{\beta + \delta}.\]
    $\psi$ is injective, hence bijective.
    Meanwhile,
    so the condition on LHS is equivalent to \[\psi(\delta)^{\frac{q-1}{2}} = -1.\]
    There are $\frac{q-1}{2}$ elements in $\F_q\sminus \{1\}$ has this property.  
\end{proof}

\begin{corollary}
    For a uniformly random $\delta\in\F_q$,
    \[\Pr\left[ \gcd(g(X), (X + \delta)^{\frac{q-1}{2}} - 1) \text{ is a nontrivial factor of } g(X)\right] \ge \frac{(q-1)/2}{q} \ge \frac{1}{3}.\]
\end{corollary}

\begin{algorithm}[H]
    \caption{Partial Factorization}
    \KwIn{$g(X) : \F_q[X]$, $\deg(g) > 1$, only different linear factors over $\F_q$}
    \KwOut{one nontrivial factorization of $g(X)$}
    \Repeat{}{
        $\delta\gets$ uniformly random in $\F_q$\\
        $g_1(X)\gets \gcd(g(X), (X + \delta)^{\frac{q-1}{2}} - 1)$
        \If{$0 < \deg g_1(X) < \deg g$}{
            \Return{$\left(g_1(X), \frac{g(X)}{g_1(X)}\right)$}
        }
    }
\end{algorithm}

\begin{algorithm}[H]
    \caption{Factorization}
    \KwIn{$g(X) : \F_q[X]$, only different linear factors over $\F_q$}
    \KwOut{Complete factorization of $g(X)$}
    \eIf{$\deg g = 1$}{
        \Return{$g$}
    }{
        $(g_1, g_2)\gets$ PartialFactorization($g$)\\
        \Return{(Factorization($g_1$), Factorization($g_2$))}
    }
\end{algorithm}

\begin{itemize}
    \item \# of calls to ``PartialFactorization'' is $\deg(g) - 1$.
    \item It is hard to analyze the precise complexity, as it is affected by the pattern of $g$.
\end{itemize}

How to get higher degree factors?
We can get the multiplicity of linear  factors\footnote{
    A na\"ive way: divide the factor until it is not a root.
}
then consider \[f_{\ge 2}(X) := \frac{f(X)}{\prod\text{linear factor}}.\]
Now \[\gcd(f_{\ge 2}(X), X^{\frac{q^2 - 1}{2}} - 1)\]is the product of all quadratic factors,
because quadratic factors are linear factors over $\F_{q^2}$.
Continuely raise the degree until we get the factorization.


\section{Counting rational points of an elliptic curve over a finite field}
\begin{problem}
    Calculate $\#E(\F_q)$ for an elliptic curve $E$ over $\F_q$.
\end{problem}

\end{document}
