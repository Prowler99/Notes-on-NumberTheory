\documentclass{article}
\usepackage{fontspec}
\usepackage{amsmath, amssymb, amsthm, amsbsy, mathrsfs, stmaryrd}
\usepackage{enumitem}
\usepackage[colorlinks,
linkcolor=cyan,
anchorcolor=blue,
citecolor=blue,
]{hyperref}
\usepackage[capitalize]{cleveref}
\usepackage[margin = 1in, headheight = 12pt]{geometry}
\usepackage{bbm}
\usepackage{tikz-cd}


\theoremstyle{definition}
\newtheorem{theorem}{Theorem}
\newtheorem{definition}{Definition}
\newtheorem{exercise}{Exercise}[section]
\newtheorem{problem}{Problem}
\newtheorem{example}{Example}
\newtheorem{proposition}{Proposition}[section]
\newtheorem{lemma}{Lemma}[section]
\newtheorem{corollary}{Corollary}[section]

\theoremstyle{remark}
\newtheorem*{remark}{Remark}

\renewcommand{\Re}{\mathop{\mathrm{Re}}}
\renewcommand{\Im}{\mathop{\mathrm{Im}}}
\renewcommand{\bar}{\overline}
\renewcommand{\tilde}{\widetilde}
\renewcommand{\hat}{\widehat}

% 新命令
% 数学对象
    \newcommand{\R}{\mathbb{R}}
    \newcommand{\C}{\mathbb{C}}
    \newcommand{\Q}{\mathbb{Q}}
    \newcommand{\Z}{\mathbb{Z}}
    \DeclareMathOperator{\GL}{GL}
    \DeclareMathOperator{\SL}{SL}
    \newcommand{\p}{\mathfrak{p}}
    \renewcommand{\P}{\mathbb{P}}
    \newcommand{\A}{\mathbb{A}}
% 集合
    \newcommand{\sminus}{\smallsetminus} %(集合)差
% 范畴
    \newcommand{\op}[1]{{#1}^{\mathrm{op}}} %反范畴
    \DeclareMathOperator{\enom}{End} %自态射
    \DeclareMathOperator{\isom}{Isom} %同构
    \DeclareMathOperator{\aut}{Aut} %自同构
    \DeclareMathOperator{\im}{im} %像
    \newcommand{\Set}{\mathbf{Set}} %集合范畴
    \newcommand{\Abel}{\mathbf{Ab}} %群范畴
    \newcommand{\Ring}{\mathbf{Ring}}
    \newcommand{\Cring}{\mathbf{CRing}}
    \newcommand{\Alg}{\mathbf{Alg}}
    \newcommand{\Mod}{\mathbf{Mod}}
    \DeclareMathOperator{\Id}{id}
%向量空间, 矩阵
    \DeclareMathOperator{\rank}{rank} %秩
    \DeclareMathOperator{\tr}{Tr} %迹
    \newcommand{\tran}[1]{{#1}^{\mathrm{T}}} %转置
    \newcommand{\ctran}[1]{{#1}^{\dagger}} %共轭转置
    \newcommand{\itran}[1]{{#1}^{-\mathrm{T}}} %逆转置
    \newcommand{\ictran}[1]{{#1}^{-\dagger}} %逆共轭转置
    \DeclareMathOperator{\codim}{codim} %余维数
    \DeclareMathOperator{\diag}{diag} %对角阵
    \newcommand{\norm}[1]{\left\| #1\right\|} %范数
    \DeclareMathOperator{\lspan}{span} %张成
    \DeclareMathOperator{\sym}{\mathfrak{Y}}
% 群
    \DeclareMathOperator{\inn}{Inn} %(群)内自同构
    \newcommand{\nsg}{\vartriangleleft} %正规子群
    \newcommand{\gsn}{\vartriangleright} %正规子群
    \DeclareMathOperator{\ord}{ord} %元素的阶
    \DeclareMathOperator{\stab}{Stab} %稳定化子
    \DeclareMathOperator{\sgn}{sgn} %符号函数
% 环, 域
    \DeclareMathOperator{\cha}{char} %特征
    \DeclareMathOperator{\spec}{Spec} %素谱
    \DeclareMathOperator{\maxspec}{MaxSpec} %极大谱
    \newcommand{\spm}{\maxspec}
    \DeclareMathOperator{\gal}{Gal}
    \DeclareMathOperator{\Frac}{Frac}
% 微积分
    % \newcommand*{\dif}{\mathop{}\!\mathrm{d}} %(外)微分算子
% 流形
    \DeclareMathOperator{\lie}{Lie}
%代数几何
    \DeclareMathOperator{\proj}{Proj}
%多项式
    \DeclareMathOperator{\disc}{disc} %判别式
    \DeclareMathOperator{\res}{res} %结式

% 结构简写
    \newcommand{\pdfrac}[2]{\dfrac{\partial #1}{\partial #2}} %偏微分式
    \newcommand{\isomto}{\stackrel{\sim}{\rightarrow}} %有向同构
    \newcommand{\gene}[1]{\left\langle #1 \right\rangle} %生成对象
% 文字缩写
    \newcommand{\opin}{\;\mathrm{open\;in}\;}
    \newcommand{\st}{\;\mathrm{s.t.}\;}
    \newcommand{\ie}{\;\mathrm{i.e.,}\;}

% 重定义命令
\renewcommand{\hom}{\mathop{\mathrm{Hom}}}
\renewcommand{\vec}{\boldsymbol}
\renewcommand{\and}{\;\text{and}\;}

% 编号
\newcommand{\cnum}[1]{$#1^\circ$} %右上角带圆圈的编号
\newcommand{\rmnum}[1]{\romannumeral #1}

\newcommand{\fring}[1]{\llbracket #1 \rrbracket}

\newcommand{\myit}{$\diamond$}
\newcommand{\Gm}{\mathbb{G}_{\mathrm{m}}}
\renewcommand{\O}{\mathcal{O}}
\newcommand{\nr}{\mathrm{nr}}
\newcommand{\alg}{\mathrm{alg}}
\newcommand{\ab}{\mathrm{ab}}
\DeclareMathOperator{\frob}{Frob}

\title{Notes on CFT}
\author{}
\date{}

\begin{document}
\maketitle

\section{Review: Galois theory}
\subsection{Field Extensions}
Let $L/K$ be an algebraic extension. It is called: \begin{enumerate}
    \item [$\diamond$]\textbf{normal}, if every polynomial $f\in K[T]$ with a root in $L$ splits in $L$, $\iff$ $L$ is the splitting field of a bunch of polynomials over $K$;
    \item [$\diamond$]\textbf{separable}, if for every element in $L$, its minimal polynomial over $K$ has no multiple roots in its splitting field, $\iff$ $\gcd(f, f') = 1$;
    \item [$\diamond$]\textbf{Galois}, if it is normal and separable, i.e., $L$ is the splitting field of a bunch of \textit{seperable} polynomials over $K$. We put $\gal(L/K) := \aut_K(L)$.
\end{enumerate}
\begin{remark} {}
\begin{enumerate}
    \item For a finite \textit{normal} extension $L/K$, $|\aut_K(L)| \le [L:K]$, where the equality holds $\iff L/K$ is separable, i.e. Galois. This is because a $K$-automorphism of $L = K[T]/(f)$ just permutes the roots of $f$.
    \item Normality is NOT transitive. As an example, take $\Q\subset\Q(\sqrt{2})\subset\Q(\sqrt[4]{2})$.
\end{enumerate} 
\end{remark}

% We introduce a convenient notion here.
% \begin{definition}
%     Let $\Omega/F$ be a field extension and $\mathcal{C}$ a family of subextensions in $\Omega/F$. We say $\mathcal{C}$ is \textbf{distinguished}, if it satisfies:\begin{enumerate}
%         \item [\textbf{D1}] $\forall L/E/F$, \[L/F\in\mathcal{C}\iff L/E\in\mathcal{C}\ \&\ E/F\in\mathcal{C};\]
%         \item [\textbf{D2}] $\forall L, M$, \[L/F\in\mathcal{C}\implies LM/M\in\mathcal{C}.\]
%     \end{enumerate}    
% \end{definition}
% \begin{remark}
%     Let $\mathcal{C}$ be a distinguished family of subextensions.
% \begin{enumerate}
%     \item The conditions implies that $\mathcal{C}$ is closed under \textit{finite} composition.
%     \item The \textit{union} of all fields in $\mathcal{C}$ is a field, and thus equal to the composition of all fields in $\mathcal{C}$.
%     \item Finite, algebraic, seperable and purely inseperable extensions are distinguished.
% \end{enumerate}
% \end{remark}


\subsection{Galois theory}
Now let $L/K$ be a Galois extension. Equip $\gal(L/K)$ with the following \textbf{Krull topology}: $\forall\sigma\in\gal(L/K)$, a basis of nbhd around $\sigma$ is given by\[\sigma\gal(L/F),\quad\text{where } L/F/K,\; F/K < \infty\text{ \& Galois}.\]
\begin{itemize}
    \item Two elements $\sigma, \tau\in\gal(L/K)$ are ``close'' to each other, if $\sigma|_F = \tau|_F$ for sufficiently large finite Galois subextensions $F/K$.
    \item Both multiplication and inverse on $\gal(L/K)$ are continuous for Krull topology.
    \item The Krull topology is profinite for $L/K$ infinite, whence \[\gal(L/K) \simeq \lim_{\stackrel{\longleftarrow}{F/K < \infty\text{ \& Galois}}}\gal(F/K). \]
    When $L/K < \infty$, this is the discrete topology.
    \item If there is a tower \[K\subset L_1\subset L_2\subset\dots\subset L,\] where all $L_n/K$'s are Galois, and \[L = \bigcup_{n} L_n,\]
    then \[\gal(L/K) = \varprojlim_n\gal(L_n/K).\]

\end{itemize}

Galois theory says that the intermediate fields of $L/K$ corresponds to the closed subgroups of $\gal(L/K)$ bijectively and $\gal(L/K)$-equivariantly.
\begin{enumerate}
    \item [$\rightarrow$:] For an intermediate field $F$, it gives $\gal(L/F)\subset \gal(L/K)$. Note that $L/F$ is Glaois, but $F/K$ is NOT always Galois.
    The Galois group acts on $\{\text{intermediate field of } L/K\}$ via $(\sigma, F) \mapsto \sigma F = \sigma(F)$.
    \item [$\leftarrow$:] For a closed subgroup $H < G$, it fixes a subfield $L^H \subset L$. The Galois group acts on $\{H : H < \gal(L/K)\}$ by conjugation, i.e., $(\sigma, H) \mapsto \sigma H\sigma^{-1}$.
\end{enumerate}
In particular,\begin{enumerate}
    \item [$\diamond$] \textit{Galois} extensions correspond to \textit{normal closed} subgroups, and
    \item [$\diamond$] \textit{finite} extensions correspond to \textit{open} subgroups.
\end{enumerate}

\subsubsection*{Base change}
\begin{proposition}\label{field extension base change}
\[\begin{tikzcd}[sep = tiny]
	&& LM \\
	&&& M \\
	L \\
	& K
	\arrow[no head, from=2-4, to=1-3]
	\arrow[no head, from=3-1, to=1-3]
	\arrow["{\text{Galois}}", no head, from=4-2, to=3-1]
	\arrow[no head, from=4-2, to=2-4]
\end{tikzcd}\]    Let $L/K$ be Galois. If $M/K$ is any extension, and both $L$ and $M$ are subextensions of $\Omega/K$, then $LM/M$ is Galois, and
    \begin{align*}
        \gal(LM/M) &\stackrel{\sim}{\longrightarrow}\gal(L/L\cap M)\\
        \sigma&\longmapsto \sigma|_L.
    \end{align*}
\end{proposition}
As a corollary, if $L, L'$ are Galois subextensions of $\Omega/K$, then $LL'/K$ is also Galois, and \begin{align*}
    \gal(LL'/K)&\hookrightarrow \gal(L/K)\times \gal(L'/K)\\
    \sigma &\mapsto (\sigma|_L, \sigma|_{L'}).
\end{align*}
This embedding is an isomorphism if $L\cap L' = K$.





\section{Extensions of Local Fields}

\subsection{Simple Extensions of DVRs}
Let $A$ be a local ring with ($\mathfrak{m}$, $k$), $f\in A[X]$ a monic polynomial of deg $n$.
We consider the extension \[A \to B_f := A[X]/f.\]

Let $\bar{f}$ be the image of $f$ in $k[X] \simeq A[X]/\mathfrak{m}$ with decomposition \[\bar{f} = \prod_{i}\bar{g}_i^{e_i},\ g_i\in A[X],\ \bar{g}_i\in k[X]\text{ irreducible.}\]
and \[\bar{B}_f := B_f/\mathfrak{m}B_f \simeq A[X]/(\mathfrak{m}, f) \simeq k[X]/(\bar{f}).\]
\begin{lemma}
    $\mathfrak{m}_i := (\mathfrak{m},\ g_i\bmod f)\subset B_f$ are all the distinct maximal ideals of $B_f$.
\end{lemma}
\begin{proof}
    Denote $\pi : B_f\to\bar{B}_f$. We have $B_f/\mathfrak{m}_i \simeq \bar{B}_f/(\bar{g}_i)$, so $\mathfrak{m}_i$'s are maximal.
    Note that $\mathfrak{m}_i = \pi^{-1}(\bar{g}_i)$.

    Take $\mathfrak{n}\in\spm B_f$.
    If $\mathfrak{n}\supset\mathfrak{m}$, then $\mathfrak{n} = \pi^{-1}\pi\mathfrak{n}$,
    and goes to a maximal ideal in $\bar{B}_f$ (because $\bar{B}_f/\pi\mathfrak{n} \simeq B_f/\mathfrak{n}$),
    so $\mathfrak{n} = \pi^{-1}(\bar{g}_i) = \mathfrak{m}_i$.

    So assume that $\mathfrak{m}\not\subset\mathfrak{n}$, then $\mathfrak{n} + \mathfrak{m}B_f = B_f$.\footnote{In this case $\mathfrak{n}/(\mathfrak{n\cap m})\simeq \bar{B}_f$ as $B_f$-module, and thus $\pi^{-1}\pi\mathfrak{n} = B_f$.}
    Therefore \[\frac{B_f}{\mathfrak{n}} = \frac{\mathfrak{n}+\mathfrak{m}B_f}{\mathfrak{n}} \simeq \frac{\mathfrak{m}B_f}{\mathfrak{n}}.\]
    Since $A$ is local and $B_f$ is a f.g. $A$-mod, by Nakayama's lemma, we see $\mathfrak{n} = B_f$. Contradiction.


\end{proof}

Now take $A$ to be a DVR with $\mathfrak{m} = (\varpi)$ and $K = \Frac A$. Put $L := K[X]/(f)$.
We give two cases where $B_f$ is a DVR.

\subsubsection*{Unramified case}
Let $\bar{f}\in k[X]$ be irreducible. Then $B_f$ is a DVR with maximal ideal $\mathfrak{m}B_f$.
\begin{corollary}\label{simple ext of dvr - unramified - is field}
    $f\in A[X]$ is also irreducible, so $L$ is a field.
    Moreover, $B_f$ is the integral closure of $A$ in $L$, and $L/K$ is unramified if $\bar{f}$ is separable.
\end{corollary}
\begin{proof}
    $L = K[X]/f \simeq \left( A[X]/f \right)\otimes_{A} K = B_f\otimes_A K$.
    As $B_f$ is a domain, $L$ is a field and $L = \Frac B_f$.
    Since $A$ is integrally closed, $B_f$ is also integrally closed, so $B_f$ is the integral closure of $A$ in $L$.
\end{proof}
\subsubsection*{Totally ramified case}
Let $f\in A[X]$ be an \textbf{Eisenstein polynomial}, i.e., \[f = X^n + a_{n-1}X^{n-1} + \cdots  + a_0,\ a_i\in\mathfrak{m},\ a_0\notin\mathfrak{m}^2.\]
\begin{proposition}
    $B_f$ is a DVR, with maximal ideal generated by the image of $X$ and residue field $k$.
\end{proposition}
\begin{proof}
    Let $x$ be the image of $X$ in $B_f$.
    We have $\bar{f} = X^n$, so $B_f$ is a local ring with maximal ideal $(\mathfrak{m}, x)$.
    Because $a_0\in\mathfrak{m\setminus m^2}$, $a_0$ must uniformise $\mathfrak{m}\subset A$, and \[-a_0\bmod f = x^n + \cdots + (a_1\bmod f)\,x,\] Therefore $(\mathfrak{m}, x) = (x)$.
\end{proof}
Similar to \cref{simple ext of dvr - unramified - is field}, $f$ is irreducible and $L$ is a field with $B_f$ the integral closure of $A$ in $L$.

\subsection{Unramified Extensions of Local Fields}
Let $K$ be a CDVF\footnote{CDVF stands for complete discrete valuation field.}. We assume further that both $K$ and its residue field $k = \mathcal{O}_K/\mathfrak{m}$ are perfect.

The slogan is that unramified extensions are just extensions of residue fields.
Using Hensel's lemma, an extension $k(a)/k$ can be lifted to a unique extension $K(\alpha)/K$ over $K$ with \[\gal(K(\alpha)/K)\simeq \gal(k(a)/k).\] Moreover, given an extension $L/K$, there is a maximal unramified subextension $K_0$ in $L$ containing every unramified extensions.

Now we assume $k$ to be finite. Then adjoining roots of unities with order coprime to $p = \cha k$ gives all finite unramified extensions of $K$.

\begin{example}
    Let $K/\Q_p < \infty$ and $k = \mathbb{F}_q$.
    Then the unique extension of $k$ of degree $n$ is the splitting field of $X^{q^n} - X$ over $k$, which equals $k(\mu_{q^n - 1})$ once we fix an algebraic closure of $k$.
    So the unramified extension $K_n/K$ of degree $n$ is the splitting field of $X^{q^n} - X$ over $K$, i.e., \[K_n = K(\mu_{q^{n} - 1}).\] The Galois group $\gal(K_n/K)$ is generated by $\frob_K$, which is determined by \[\frob_K\beta \equiv \beta^q \mod\varpi,\ \forall\beta\in\mathcal{O}_{K_n}\]for any uniformiser $\varpi$ (simultaneously of $K$ and $K_n$).

    What if we adjoin $\zeta_{m}$ to $K$ where $m$ is an arbitary integer prime to $p$?
    The answer is that $K(\mu_m)$ is unramified of degree the smallest positive integer $f$ s.t. $m \mid p^f - 1$, by the following \cref{cyclotomic extension of finite fields} on finite fields.
\end{example}

\begin{lemma}\label{cyclotomic extension of finite fields}
    Let $\zeta_n$ be a primitive $n$-th root of unity over $\F_q$ with $q, n$ coprime. Then $[\F_q(\zeta_n) : \F_q]$ is the smallest integer $f > 0$ s.t. $n \mid q^f - 1$.
\end{lemma}
\begin{proof}
    Because $\cha\F_q\nmid n$, the primitive root $\zeta_n$ exists and $\F_q(\zeta_n)$ is the splitting field of $X^n - 1$ over $\F_q$.
    The degree $f = [\F_q(\zeta_n) : \F_q]$ is the order of $\frob_q$ on $\F_q(\zeta_n)$, i.e., $f$ is the smallest integer s.t. \[\frob_q^f(\zeta_n) = \zeta_n^{q^f} = \zeta_n.\] The definition of primitive root of unity says that \[\zeta_n^{q^f -1} = 1\iff n \mid q^f - 1.\qedhere\]
\end{proof}

\subsection{Ramification Groups}

Let $K$ be a CDVF with perfect residue field $k$, $L/K<\infty$ Galois. We will study the Galois group \[G := \gal(L/K)\] by giving filtrations on it.





\section{Lubin-Tate Theory}

\subsection{Formal Groups}
In this section, a formal group means a commutative formal group law of dimension one. If $f\in A\llbracket T\rrbracket$ and $g\in A\fring{X_1, \dots, X_n}$, then \begin{align*}
    f \circ g &:= f(g(X_1, \dots, X_n)),\\
    g\circ f &:= g(f(X_1), \dots, f(X_n)).
\end{align*}

\begin{lemma}\label{power series invertible iff}
    Let $f = \sum_{i\ge 1}a_iT^i\in A\fring{T}$. Then
    \[\exists g\in A\fring{T} \st f\circ g = g\circ f = T\iff a_1\in A^\times.\]
\end{lemma}
\begin{proof}
    Use $A\fring{T} = \varprojlim A[T]/T^n$. For details, see the proof of \cref{fund prop of F_varpi}.
\end{proof}

\subsection{Lubin-Tate formal groups}
From now on, we write $A := \O_K$.

Choose a uniformiser $\varpi$ of $K$. Define
\[\mathcal{F}_\varpi := \left\{f\in\O_K\fring{T}\ \left|\ \begin{aligned}
    &f(T) \equiv \varpi T&\mod {T^2} \\
    &f(T)\equiv T^q&\mod \varpi
\end{aligned}\right.\right\}.\]
For example, $f(T) = T^q + \varpi T\in\mathcal{F}_\varpi$.
The following lemma is a fundamental property of $\mathcal{F}_\varpi$.

\begin{lemma}\label{fund prop of F_varpi}
    Let $f, g\in\mathcal{F}_\varpi$, $\Phi_1$ be a linear form\footnote{A \textbf{linear form} is a homogeneous polynomial of degree 1.} over $\O_K$. Then there is a \textbf{unique} $\Phi\in\O_K\fring{X_1,\dots, X_n}$, s.t.\[\begin{cases}
        \Phi \equiv \Phi_1 \mod (X_1, \dots, X_n)^2,\\
        f(\Phi(X_1, \dots, X_n)) = \Phi(g(X_1), \dots, g(X_n)).
    \end{cases}\]
\end{lemma}
\begin{proof}
    We use a standard method. Finding $\Phi$ is equivalent to finding $\Phi_r\in A[X_1, \dots, X_n]$ s.t. \[\begin{cases}
        \Phi_{r+1} \equiv \Phi_r &\mod (\deg\ge r+1),\\
        f(\Phi_r)\equiv \Phi_r(g(X_1), \dots, g(X_n)) &\mod(\deg\ge r+1). 
    \end{cases}\]
    The second condition is guaranteed because $X\mapsto h(X)$ is $X$-adic continuous for any power series $h$.

    Suppose we have found $\Phi_r$. We look for $\Phi_{r+1}$ of the form $\Phi_{r+1} = \Phi_r + Q$, where $Q$ is homogeneous of degree $r+1$, s.t. \[f(\Phi_{r+1}) \equiv \Phi_{r+1}(g(X_1), \dots, g(X_n)) \mod \deg\ge r+2.\]
    The LHS is
    \[f(\Phi_r) + f(Q)\equiv f(\Phi_r) + \varpi Q\mod\deg\ge r+2,\]
    while the RHS is
    \[\Phi_r\circ g + Q(\varpi X_1, \dots, \varpi X_n)\equiv \Phi_r\circ g + \varpi^{r+1}Q,\]
    so if such a $Q\in A[X_1, \dots]$ exists, it must satisfy 
    \[\varpi(\varpi^r - 1)Q\equiv f\circ \Phi_r - \Phi_r\circ g\mod\deg\ge r+2\]
    and thus being unique. This procedure also shows that all $\Phi_r$'s are unqie if we require $\Phi_{r+1} - \Phi_r$ to be homogeneous.

    Because $\varpi^r - 1\in A^\times$, it suffices to show \[f(\Phi_r) \equiv \Phi_r\circ g\mod \varpi,\] which is clear.
\end{proof}

By \cref{fund prop of F_varpi}, one may define the \textbf{Lubin-Tate formal groups}.
They are exactly the formal group laws admitting an endomorphism\begin{itemize}
    \item that has derivative at the origin equal to a uniformiser of $K$, and
    \item reduces mod m to the Frobenius map $T\mapsto T_q$.
\end{itemize}
Moreover, these formal groups admit $\O_K$-actions and are isomorphic as formal $\O_K$-modules.

\begin{proposition}
    For each $f\in \mathcal{F}_\varpi$, there is a unique formal group $F_f$ over $\O_K$ admitting $f$ as an endomorphism.
\end{proposition}
\begin{proof}
    \cref{fund prop of F_varpi} gives $F_f\in A\fring{X, Y}$ s.t. \[\begin{cases}
        F_f = X + Y + \deg \ge 2,\\
        f(F_f(X+Y)) = F_f(f(X), f(Y)).
    \end{cases}\]
    The associativity is proved by showing that both $G_1 = F_f(X, F_f(Y, Z))$ and $G_2 = F_f(F_f(X, Y), Z)$ satisfies 
    \[\begin{cases}
        G = X+Y+Z + \deg\ge 2,\\
        f(G) = G(f(X), f(Y), f(Z)).
    \end{cases}\]
    This is a direct application of \cref{fund prop of F_varpi} and will be used many times.
\end{proof}

So Lubin-Tate formal groups exist. Now we investigate their homomorphisms.
\begin{proposition}
    For each $f, g\in\mathcal{F}_\varpi$ and $a\in \O_K$, there is a unique $[a]_{g, f}\in \O_K\fring{T}$ s.t. \[\begin{cases}
        [a]_{g, f} = aT + \dots,\\
        g\circ [a]_{g, f} = [a]_{g, f} \circ f,
    \end{cases}\]and $[a]_{g, f}\in\hom(F_f, F_g)$, i.e. \begin{align*} F_g\circ [a]_{g, f} = [a]_{g, f}\circ F_f.\end{align*}
    As a corollary of \cref{power series invertible iff}, each $u\in A^\times$ gives an isomorphism $[u]_{g, f} : F_f\isomto F_g$, and there is a unique isomorphism $F_f\simeq F_g$ of the form $T + \cdots$.
    \qed
\end{proposition}

We write $[a]_{f} := [a]_{f, f}\in\enom F_f$.
Note that \[[\varpi]_f = f.\]

\begin{proposition}
    For any $a, b\in\O_K$, \[[a+b]_{g, f} = [a]_{g, f} + [b]_{g, f},\]and\[[ab]_{h, f} = [a]_{h, g}\circ [b]_{g, f}.\]
    
    In particular, $\O_K\hookrightarrow\enom  F_f$ as a ring by $a\mapsto [a]_f$, making $F_f$ a formal $\O_K$-module. The canonical isomorphism $[1]_{g, f}$ is an isomorphism of $\O_K$-modules.
    \qed
\end{proposition}

\subsection{Construction of \texorpdfstring{$K_\varpi$}{}}
Fix an algebraic closure $K^\alg$ of $K$.
Each $f\in\mathcal{F}_\varpi$ associates to $\mathfrak{m}_{K^\alg}$ an $\O_K$-module structure via \[\alpha +_{F_f}\beta := F_f(\alpha, \beta)\]and \[a\cdot \alpha := [a]_f(\alpha)\footnote{These power serieses converges because they actually falls in a finite extension of $K$.}.\]for $|\alpha| < 1, |\beta| < 1$ and $a\in \O_K$.
We denote this $\O_K$-module by $\Lambda_f$.
If $g\in\mathcal{F}_\pi$, then the canonical isomorphism $[1] : F_f\to F_g$ yields $\Lambda_f\isomto\Lambda_g$.

The $\varpi^n$-torsion part of $\Lambda_f$ is denoted by $\Lambda_{f, n}$, i.e., $\Lambda_{f, n} := \Lambda_f[[\varpi]_f^n]$. Because $[\varpi]_f = f$, $\Lambda_{f, n}$ is the $\O_K$-module consisting of the roots of $f^{(n)} := f\circ\cdots\circ f$.
If one takes $f$ to be an Eisenstein polynomial, then all the roots of $f^{(n)}$ lie in $\mathfrak{m}_{K^\alg}$, so $\Lambda_{f, n}$ is precisely the set of roots of $f^{(n)}$ equipped with the $\O_K$-module structure from $F_f$.

\begin{lemma}\label{pi^n torsion cyclic of}
    Let $M$ an $\O_K$-module, $M_n = M[\varpi^n]$. If\begin{itemize}
        \item $M_1$ has $q = [\O_K : \varpi]$ elements, and
        \item $\varpi : M \to M$ is surjective,
    \end{itemize}
    then $M_n\simeq \O_K/\varpi^n$.
\end{lemma}
\begin{proof}
    Do induction on $n$. The structure theorem of f.g.\! modules over a PID shows that $M_1$ having $q$ elements implies that $M_1\simeq A/\varpi$.
    Now assume it true for $n-1$.
    Look at the sequence \[0\to M_1\to M_n\stackrel{\varpi}{\to} M_{n-1}\to 0.\] Surjectivity of $\varpi$ implies the exactness of this sequence, and thus $M_n$ has $q^n$ elements. In addition, $M_n$ must be cyclic, otherwise $M_1 = M_n[\varpi^n]$ is not cyclic.
\end{proof}

\begin{proposition}
    The $\O_K$-module $\Lambda_{f, n}$ is isomorphic to $\O_K/\varpi^n$, and hence $\enom(\Lambda_{f, n})\simeq \O_K/\varpi^n$.
\end{proposition}
\begin{proof}
    It suffices to show for a chosen $f$, so let's take $f = \varpi T + \dots + T^q$, an Eisenstein polynomial.
    We use the above \cref{pi^n torsion cyclic of} by the following observations.\begin{itemize}
        \item All roots of an Eisenstein polynomial have valuation $>0$.
        \item If $|\alpha| < 1$, then the Newton polygon of $f(T) - \alpha$ shows that its roots have valuation $>0$, and thus $[\varpi] = f(T)$ is surjective on $\Lambda_f$.\qedhere
    \end{itemize}
\end{proof}

\begin{lemma}\label{galois commutes power series}
    Let $L$ be a finite Galois extension of $K$. Then for every $F\in\O_K\fring{X_1, \dots, X_n}$, $\alpha_1,\dots, \alpha_n\in\mathfrak{m}_L$ and $\tau\in\gal(L/K)$,
    \[\tau F(\alpha_1, \dots, \alpha_n) = F(\tau\alpha_1, \dots, \alpha_n).\]
\end{lemma}
\begin{proof}
    Note that $\tau$ acts continuously on $L$, becaunse the extension of valuation for local fields is unique.
    Therefore writing $F = \lim_{m\to\infty} F_m$ gives the desired result.
\end{proof}

\begin{theorem}\label{construction of K_{varpi, n}}
    Let $K_{\varpi, n} := K(\Lambda_{f, n})\subset K^\alg$.
    These fields are independent to the choice of $f$.\begin{enumerate}
        \item [(a)] $K_{\varpi, n}/K$ is totally ramified of degree $q^{n-1}(q-1)$.
        \item [(b)] The action of $\O_K$ on $\Lambda_{f, n}$ defines an isomorphism \begin{equation}
            \left( \O_K/\mathfrak{m}_K^n \right)^\times \simeq \gal(K_{\varpi, n}/K).
        \end{equation}
        \item [(c)] For all $n$, $\varpi$ is a norm from $K_{\varpi, n}$, i.e., $\exists\alpha_n\in K_{\varpi, n}$ with $N_{K_{\varpi, n}/K}(\alpha_n) = \varpi$.
    \end{enumerate}
\end{theorem}
\begin{proof}
    Let $f$ be a polynomial $T^q + \dots + \varpi T$.

    Choose a nonzero root $\varpi_1$ of $f(T)$ and, inductively, a root $\varpi_n$ of $f(T) - \varpi_{n-1}$.
    So $\varpi_n\in\Lambda_{f, n}$, and we obtain a tower of extensions \[K_{\varpi, n}\supset K(\varpi_n)\stackrel{q}{\supset} K(\varpi_{n-1}) \stackrel{q}{\supset}\dots\stackrel{q}{\supset} K(\varpi_1)\stackrel{q-1}{\supset} K.\]
    All the extensions with indicated degrees are given by Eisenstein polynomials, and thus Galois and totally ramified.

    The field $K_{\varpi, n} = K(\Lambda_{f, n})$ is the splitting field of $f^{(n)}$ over $K$, hence $\gal(K_{\varpi, n}/K)$ embeds into the permutation group of the set $\Lambda_{f, n}$. By \cref{galois commutes power series}, the action of $\gal(K_{\varpi, n}/K)$ on $\Lambda_n$ preserves its $\O_K$-action, so
    \[\gal(K_{\varpi_n}/K)\hookrightarrow \aut(\Lambda_{f, n})\simeq (\O_K/\varpi^n)^\times.\]
    So $[K_{\varpi, n} : K]\le (q - 1)q^{n-1}$. Comparing the degree gives $K_{\varpi, n} = K(\varpi_n) $.

    Now we prove (c).
    Let $f^{[n]} := (f/T)\circ f\circ\dots\circ f$. Then $f^{[n]}$ is monic with degree $q^{n-1}(q-1)$ and $f^{[n]}(\varpi_n) = 0$, and thus $f^{[n]}$ is the minimal polynomial of $\varpi_n$ over $K$. So we have \[N_{K_{\varpi, n}/K}(\varpi_n) = (-1)^{q^{n-1}(q-1)}\]
    by the following \cref{compute norm and trace from minimal polynomial}.
\end{proof}

\begin{lemma}\label{compute norm and trace from minimal polynomial}
    Let $L/K$ be a finite extension in an algebraic closure $K^\alg$, and $\alpha\in L$ has minimal polynomial $f$ over $K$ of degree $d$. Suppose \[f(X) = (X-\alpha_1)\cdots(X-\alpha_d)\in K^\alg[X],\] and let $e = [L : K(\alpha)]$
    then \[N_{L/K}(\alpha) = \left(\prod_{i = 1}^d \alpha_i\right)^e,\qquad \tr_{L/K}(\alpha) = e\sum_{i = 1}^d \alpha_i.\]
    Moreover, if \[f(X) = a_dX^d + a_{d-1}X^{d-1} + \dots + a_0,\]then \[N_{L/K}(\alpha) = (-1)^{de}a_0^e,\qquad \tr_{L/K}(\alpha) = -ea_{d-1}.\]
\end{lemma}
\begin{remark}
    This can be deduced from $N_{L/K} = N_{L/K(\alpha)}\circ N_{K(\alpha)/K}$ and $\tr_{L/K} = \tr_{L/K(\alpha)}\circ \tr_{K(\alpha)/K}$.
\end{remark}


Define \[K_\varpi := \bigcup_{n} K_{\varpi, n}.\]
The isomorphisms in \cref{construction of K_{varpi, n}} (b) are \[(\O_K/\varpi^n)^\times\to \gal(K_{\varpi, n}/K)\quad \bar{u}\mapsto (\Lambda_{f, n}\ni \alpha\mapsto [u]_f(\alpha)),\] and clearly lift to an isomorphism \[A^\times\simeq \gal(K_\varpi/K).\]

\subsubsection*{The local Artin map}
The \textbf{local Artin map} is a homomorphism \[\phi_\varpi : K^\times\to \gal(K_\varpi K^\nr/K) = \gal(K^\nr/K)\times \gal(K_\varpi/K)\] defined as follows.
Let $a = u\varpi^m\in K^\times$, then 
\begin{itemize}
    \item $\phi_\varpi(a)|_{K^\nr} := \frob^m$;
    \item $\phi_\varpi(a)(\lambda) := [u^{-1}]_f(\lambda)$, $\forall \lambda\in\bigcup_n\Lambda_n$.
\end{itemize}

\begin{theorem}
    Both $K_{\varpi}$ and $K^\nr$ are independent of the choice of $\varpi$.
\end{theorem}


\subsection{The Local Kronecker-Weber theorem}

\subsection{The Case of \texorpdfstring{$\Q_p$}{}}
Let $K = \Q_p$ and $\varpi = p$. Then $f(T) := (1 + T)^p - 1\in\mathcal{F}_p$.
Note that $f$ is an endomorphism of \[\Gm(X, Y) = X + Y + XY,\] so $F_f = \Gm{}_{/\Z_p}$. Under the isomorphism
\[(\mathfrak{m}, +_{\Gm})\simeq (1 + \mathfrak{m},\ \cdot\ ),\]
the endomorphism $f : a\mapsto (1 + a)^p - 1$ is converted to the Frobenius map $a\mapsto a^p$.

\subsubsection*{The field $(\Q_p)_p$}

For each $r\ge 1$, the $p^r$-torsion part of $\Lambda_f$ is
\[\Lambda_{f, r} = \left\{\alpha\in\Q_p^\alg\left|(1 + \alpha)^{p^r} = 1\right.\right\}\simeq
\left\{\zeta\in(\Q_p^\alg)^{\times}
\left|\zeta^{p^r} = 1\right.\right\} = \mu_{p^r}.\]
The isomorphism is for $\O_K$-modules.
So choose primitive $p^r$-th roots of unity $\zeta_{p^r}$ s.t. $\zeta_{p^r}^p = \zeta_{p^{r-1}}$,
then $\varpi_r := \zeta_{p^r} - 1$ forms a sequence of compatible generators of $\Lambda_{f, r}$.
Therefore \[(\Q_p)_{p, r} = \Q_p(\varpi_r) = \Q_p(\mu_{p^r}),\]
and the ``maximal totally ramified abelian extension''\footnote{Not sure if this terminology is correct ...?} of $\Q_p$ is $(\Q_p)_p = \Q_p(\mu_{p^\infty})$.

\subsubsection*{The local Artin map \texorpdfstring{$\phi_p : \Q_p^\times\to \gal(\Q_p^\ab/\Q_p)$}{}}

It suffices to look at every \[\phi_p : \Q_p^\times\to \gal(\Q_p(\mu_n)/\Q_p).\]
\begin{itemize}
    \item If $n$ is prime to $p$, then $\Q_p(\mu_n)/\Q_p$ is unramified of degree $f$, where $f$ is the minimum natural number s.t. $m\mid p^f - 1$.
    The map $\phi_p$ sends $up^t$ to the $t$-th power of Frobenius-$p^f$ on $\Q_p(\mu_n) = \Q_p(\mu_{p^f - 1})$, and $\ker\phi_p = (p^{f})^{\Z}\times\Z_p^\times$.
    \item If $n = p^r$, then $\Q_p(\mu_{p^r})/\Q_p$ is totally ramified. The map $\phi_p$ sends $up^t$ to the element sending a root of unity $\zeta$ to $\zeta^{\bar u^{-1}}$, where $\bar u\in\Z$ has the same residue modulo $p^r$ as $u$.
    The kernel is $p^\Z\times (1 + p^r\Z_p)$.
\end{itemize}




\input{cycl_ext_Qp.tex}

\end{document}