\documentclass{article}
\usepackage{amsmath, amssymb, amsthm, amsbsy, mathrsfs, stmaryrd}
\usepackage{enumitem}
\usepackage[colorlinks,
linkcolor=cyan,
anchorcolor=blue,
citecolor=blue,
]{hyperref}
\usepackage[capitalize]{cleveref}
\usepackage[margin = 1in, headheight = 12pt]{geometry}
\usepackage{bbm}
\usepackage{tikz-cd}

\newtheorem{theorem}{Theorem}

\theoremstyle{definition}
\newtheorem{definition}{Definition}
\newtheorem{exercise}{Exercise}[section]
\newtheorem{problem}{Problem}
\newtheorem{example}{Example}
\newtheorem{proposition}{Proposition}[section]
\newtheorem{lemma}{Lemma}
\newtheorem{corollary}{Corollary}[section]

\theoremstyle{remark}
\newtheorem*{remark}{Remark}

\renewcommand{\Re}{\mathop{\mathrm{Re}}}
\renewcommand{\Im}{\mathop{\mathrm{Im}}}

% 新命令
% 数学对象
    \newcommand{\R}{\mathbb{R}}
    \newcommand{\C}{\mathbb{C}}
    \newcommand{\Q}{\mathbb{Q}}
    \newcommand{\Z}{\mathbb{Z}}
    \DeclareMathOperator{\GL}{GL}
    \DeclareMathOperator{\SL}{SL}
    \newcommand{\p}{\mathfrak{p}}
    \renewcommand{\P}{\mathbb{P}}
    \newcommand{\A}{\mathbb{A}}
    \newcommand{\F}{\mathbb{F}}
% 集合
    \newcommand{\sminus}{\smallsetminus} %(集合)差
% 范畴
    \newcommand{\op}[1]{{#1}^{\mathrm{op}}} %反范畴
    \DeclareMathOperator{\enom}{End} %自态射
    \DeclareMathOperator{\isom}{Isom} %同构
    \DeclareMathOperator{\aut}{Aut} %自同构
    \DeclareMathOperator{\im}{im} %像
    \newcommand{\Set}{\mathbf{Set}} %集合范畴
    \newcommand{\Abel}{\mathbf{Ab}} %群范畴
    \newcommand{\Ring}{\mathbf{Ring}}
    \newcommand{\Cring}{\mathbf{CRing}}
    \newcommand{\Alg}{\mathbf{Alg}}
    \newcommand{\Mod}{\mathbf{Mod}}
    \DeclareMathOperator{\Id}{id}
%向量空间, 矩阵
    \DeclareMathOperator{\rank}{rank} %秩
    \DeclareMathOperator{\tr}{Tr} %迹
    \newcommand{\tran}[1]{{#1}^{\mathrm{T}}} %转置
    \newcommand{\ctran}[1]{{#1}^{\dagger}} %共轭转置
    \newcommand{\itran}[1]{{#1}^{-\mathrm{T}}} %逆转置
    \newcommand{\ictran}[1]{{#1}^{-\dagger}} %逆共轭转置
    \DeclareMathOperator{\codim}{codim} %余维数
    \DeclareMathOperator{\diag}{diag} %对角阵
    \newcommand{\norm}[1]{\left\| #1\right\|} %范数
    \DeclareMathOperator{\lspan}{span} %张成
    \DeclareMathOperator{\sym}{\mathfrak{Y}}
% 群
    \DeclareMathOperator{\inn}{Inn} %(群)内自同构
    \newcommand{\nsg}{\vartriangleleft} %正规子群
    \newcommand{\gsn}{\vartriangleright} %正规子群
    \DeclareMathOperator{\ord}{ord} %元素的阶
    \DeclareMathOperator{\stab}{Stab} %稳定化子
    \DeclareMathOperator{\sgn}{sgn} %符号函数
% 环, 域
    \DeclareMathOperator{\cha}{char} %特征
    \DeclareMathOperator{\spec}{Spec} %素谱
    \DeclareMathOperator{\maxspec}{MaxSpec} %极大谱
    \DeclareMathOperator{\gal}{Gal}
% 微积分
    % \newcommand*{\dif}{\mathop{}\!\mathrm{d}} %(外)微分算子
% 流形
    \DeclareMathOperator{\lie}{Lie}
%代数几何
    \DeclareMathOperator{\proj}{Proj}
%多项式
    \DeclareMathOperator{\disc}{disc} %判别式
    \DeclareMathOperator{\res}{res} %结式

% 结构简写
    \newcommand{\pdfrac}[2]{\dfrac{\partial #1}{\partial #2}} %偏微分式
    \newcommand{\isomto}{\stackrel{\sim}{\rightarrow}} %有向同构
    \newcommand{\gene}[1]{\left\langle #1 \right\rangle} %生成对象
% 文字缩写
    \newcommand{\opin}{\;\mathrm{open\;in}\;}
    \newcommand{\st}{\;\mathrm{s.t.}\;}
    \newcommand{\ie}{\;\mathrm{i.e.,}\;}

% 重定义命令
\renewcommand{\hom}{\mathop{Hom}}
\renewcommand{\vec}{\boldsymbol}
\renewcommand{\and}{\;\text{and}\;}
\renewcommand{\div}{\mathop{\mathrm{div}}}
\DeclareMathOperator{\Div}{Div}
\newcommand{\tor}{\mathrm{tor}}

% 编号
\newcommand{\cnum}[1]{$#1^\circ$} %右上角带圆圈的编号
\newcommand{\rmnum}[1]{\romannumeral #1}
\newcommand{\m}{\mathfrak{m}}

\newcommand{\myit}{$\diamond$}

\title{Elliptic Curves, n$^\circ$ 2}
\author{Lei Bichang}
% \date{Sep 26}

\begin{document}
\maketitle


\subsection*{Exercise 1}

\begin{enumerate}
    \item [(a)] Let $\phi_{q, i}$ be the $q^{\mathrm{th}}$-Frobenius on $E_i$, $i = 1, 2$.
    Then $\# E_i(\F_q) = \deg (1 - \phi_{q, i})$.
    Since $\psi$ is an isogeny defined over $\F_q$, it is invariant under $\gal(\bar{ \F}_q|\F_q)$,
    so for every $P\in E_1$,\[\psi(\phi_{q,1}(P)) = \psi(P^{\sigma}) = \psi^\sigma(P^\sigma) = \psi(P)^\sigma = \phi_{q, 2}(\psi(P)),\]
    where $\sigma \in \gal(\bar{ \F}_q|\F_q)$ denotes the $q^{\mathrm{th}}$-Frobenius.
    Hence \[\psi\circ (1 - \phi_{q, 1}) = \psi - \psi\circ\phi_{q, 1} = \psi - \phi_{q, 2}\circ\psi = (1 - \phi_{q, 2})\circ\psi.\]
    As $\psi$ is nonzero, $\deg\psi\ne 0$.
    So taking degree on the above equation yields\[\# E_1(\F_q) = \# E_2(\F_q).\]
    
    \item [(b)]
    No. Let $q = 5$, \[E_1 : y^2 = x^3 + x,\quad E_2  : y^2 = x^3 + 2x.\]
    Then \[(x, y)\mapsto (u^2x, u^3y),\quad u = \sqrt[4]{3}\]
    is an isomorphism over $\bar{\F}_5$,
    but \[\# E_1(\F_5) = 4,\ \# E_2(\F_5) = 2.\]



\end{enumerate}






\subsection*{Exercise 2}
\begin{enumerate}
    \item [(a)] $E : y^2 = x^3+1$.

    $\Delta = -16\cdot 27 = -2^4\cdot 3^3$,
    so the equation is minimal for every prime $p$,
    and the possible rational points $(x, y)$ with finite order satisfy $x\in\Z$ and $y = 0, 1, 2, 3, 4, 6$.
    These points are \[(-1, 0),\ (0, \pm 1),\ (2, \pm 3).\]
    We compute the following points.
    \begin{itemize}
        \item $2\cdot (0, \pm 1)$.
        Since $\lambda := \dfrac{3\cdot 0^2}{2\cdot \pm 1} = 0$, we have $x(2\cdot (0,\pm 1)) = 0$, so $2\cdot (0, \pm 1) = (0, \mp 1) = -(0, \pm 1)$.
        Therefore $(0, \pm 1)$ have order $3$.
        \item $2\cdot (2, 3)$.
        Since $\lambda := \dfrac{3\cdot 2^2}{2\cdot 3} = 2$, we have $x(2\cdot (2, 3)) = \lambda^2 - 2\cdot 2 = 0$, so $2\cdot (2, 3) \in \{(0, \pm 1)\}$ have order $3$, and thus $(2, 3)$ have order $3$.
    \end{itemize}
    Hence all the five points are of fintie order, and $E(\Q)_{\tor}\simeq \Z/6\Z$.

    \item [(b)] $E : y^2 = x(x-1)(x+2)$.

    $\Delta = 16(0 - 1)^2(0 + 2)^2(1+2)^2 = 2^63^2$, so the equation is minimal for every prime.
    \begin{itemize}
        \item $E(\Q)[2] = \{O, (0, 0), (1, 0), (-2, 0)\}\simeq\Z/2\Z\times\Z/2\Z$.
        \item $E$ has good reduction at $5$,
        and \[\tilde{E}(\F_5) = \{O, (0, 0), (1, 0), (-2, 0)\}\simeq\Z/2\Z\times\Z/2\Z,\]
        If $5\nmid m$, then $E(\Q)[m]\hookrightarrow E(\F_5)\simeq\Z/2\Z\times\Z/2\Z$.
        So $\# E(\Q)[m]\mid 4$. Hence $\# E(\Q)[p^n] = 0$ for every prime $p\ne 2, 5$.
        \item $E$ has good reduction at $7$,
        and \[\tilde{E}(\F_7) = \{O, (0, 0), (1, 0), (-1, \pm 3), (2, \pm1), (-2, 0), (3, \pm 3), (-3, \pm 3)\}.\]
        This group has order $8$ and four points of order $2$, so $\tilde{E}(\F_7)\simeq\Z/2\Z\times\Z/4\Z$.
        If $7\nmid m$ and $E(\Q)[m]\ne 0$, then $\# E(\Q)[m]\mid 8$.
        Therefore $\# E(\Q)[p^n] = 0$ for every prime $p\ne 2$
        \item Now suppose $m$ is not a power of $2$ and $P\in E[m]$.
        Then $\frac{m}{p^n}P\in E[p^n] = 0$ for every prime $p\ne 2$ and prime power $p^n\mid m$.
        Therefore $P\in E[2^n]$ for some $n\ge 0$,
        and $E(\Q)_\tor = E[2^\infty]$.
        \item Since $\# E(\Q)[2] = \# \tilde{E}(\F_5)$ and $E(\Q)[2]\subset E(\Q)[2^n]\hookrightarrow \tilde{E}(\F_5)$ for all $n\ge 1$, we see that \[E(\Q)_\tor = E(\Q)[2]\simeq \Z/2\Z\times\Z/2\Z.\]
    \end{itemize}

    \item $E : y^2 = x^3 - 43x + 166$.

    $\Delta = -16(4\cdot (-43)^3 + 27\cdot 166^2) = -2^{19}\cdot 13$, so it is minimal for all primes $p\ne 2$.
    \begin{itemize}
        \item $E$ has good reduction at $3$ and $7$,
        and \[\tilde{E}(\F_3) = \{O, (0, \pm 1), (1, \pm 1), (-1, \pm 1)\}\simeq\Z/7\Z,\]
        \[\tilde{E}(\F_5) = \{O, (0, \pm 1), (1, \pm 2), (-2, \pm 2)\}\simeq\Z/7\Z.\]
        So $E(\Q)[p^n] = 0$ for all $p\ne 7$.
        \item Using a calculator, I found \[(3, 8)\in E(\Q).\]
        Then using Sage, I found that the order of $(3, 8)$ is $7$. As $E(\Q)[7]\hookrightarrow \tilde{E}(\F_3)\simeq\Z/7\Z$, we see that $E(\Q)[7]\simeq\Z/7\Z$.
        By a similar argument as before,
        \[E(\Q)_\tor = E(\Q)[7] \simeq\Z/7\Z\] and it is generated by $(3, 8)$.
    \end{itemize}
\end{enumerate}











\end{document}