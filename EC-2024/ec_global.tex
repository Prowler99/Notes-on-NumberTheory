\section{Elliptic Curves over Global Fields}

\subsection{Mordell-Weil Theorem}
\begin{theorem}
    Ler $K$ be a number field, $E$ an elliptic curve over $K$.
    Then $E(K)$ is of finite type.
\end{theorem}
Consequently, \[E(K) = \Z^r\oplus E(K)_\tor.\]
We call $r\in\Z_{\ge 0}$ the \textbf{rank} of $E/K$.
\begin{remark}
    In contrast, if $K$ is a local field (here we mean $K = \R, \C$ or a finite extension of $\Q_p$) or algebraically closed,
    then $E(K)$ is NOT finitely generated. See DM5.
\end{remark}
The proof consists two steps.
\begin{enumerate}
    \item Weak Mordell-Weil theorem: $E(K)/2E(K)$ is fintie.
    \item Descent. Let $\{P_1, \dots, P_r\}$ represents $E(K)/2E(K)$.
    Any $P\in E(K)$ is of the form
    $P = P_i +2Q$ for some $P_i$.
    The idea is that $Q\mapsto P_i + 2Q$ ``increases the complexity (height)'' of the point $Q$, and $E(K)$ is generated by $P_i$'s with points of bounded height.
\end{enumerate}

\subsection{Step 1 - A Special Case: \texorpdfstring{$E/\Q$, $E(\Q)[2]\ne 0$}{E/Q, E(Q)[2] ne 0}}
Take an elliptic curve $E : y^2 = A(x)$ with some rational $2$-torsion point on it,
namely $A(x_0) = 0$ for an $x_0\in\Q$.
Translating \& scaling if necessary, we assume $A(X)\in\Z[X]$ and $x_0 = 0$,
so that $(0, 0)\in E(\Q)[2]$, and then write
\[E : y^2 = x(x^2 + ax + b),\quad a, b\in\Z,\ b\ne 0\footnotemark, a^2 - 4b\ne 0\]
\footnotetext{Because $A(X)$ can't have multiple roots.}
\subsubsection*{Determine \texorpdfstring{$E' = E/\gene{(0, 0)}$}{E' = E/<(0, 0)>}}

Define \[t : E\to E\quad P\mapsto P + (0, 0).\]
This is an involution over $\Q$ of algebraic curves, thus induces an involution
\[t^* : \Q(E)\to \Q(E).\]
The quotient map $E\to E/\gene{(0, 0)}$ is induced by $t$,
so $K(E/\gene{(0, 0)})$ is isomorphic to the subfield of $K(E)$ fixed by $t^*$.

Let $P = (x, y)\in E$, $P_1 = (x_1, y_1) = t(P)$.
What is $P_1$? The line through $P$ and $(0, 0)$ takes the form
\[L : \begin{cases}
    X = \lambda x,\\ Y = \lambda y,
\end{cases},\quad \lambda\in\bar\Q,\]
and intersects $E$ also at $-P_1 = (x_1, -y_1)$\footnote{
    $E$ is not in short Weierstrass equation.
    For a long Weierstrass equation,
    \[-P = \left( x, -y-a_1x-a_3 \right).\]
}.
Hence \[-\frac{y_1}{x_1} = \frac{y}{x},\implies \left( \frac{y_1}{x_1} \right)^2 = \left( \frac{y }{x } \right)^2.\]
One checks that \[f := \left( \frac{y }{x } \right)^2\in \Q(E)\]is invariant under $t^*$.

(Why do) We need another function invariant under $t^*$.
Observe that \[g := y + y_1\in\Q(E)^{t^* = \Id},\]
because $t$ has order $2$.
Now $f, g$ must have algebraic relation.
Begin by compute $P_1$.
Subsituting $L$ into $E$,
we get \[L\cap E : \lambda(x^3\lambda^2 + (ax^2 - y^2)\lambda + bx) = 0.\]
Here $\lambda = 0$ corresponds to $(0, 0)$,
and $\lambda = 1$ corresponds to $P$.
Denote by $\lambda = \lambda_1$ the parameter of $-P_1$,
then \[1\cdot \lambda_1 = \dfrac{b }{x^2},
\quad\implies\begin{cases}
    x_1 = \dfrac{b }{x},\\ y_1 = -\dfrac{by}{x^2},
\end{cases}\]
\begin{align*}
    \implies g^2 = \left( y - \frac{by}{x^2} \right)^2
    = \frac{y^2}{x^2}\left( x^2 - 2b + \frac{b^2}{x^2} \right).
\end{align*}
As \[f = \left( \frac{y}{x} \right)^2 = \frac{x^2+b}{x} + a,\]
we have \begin{align*}
    g^2 = f\left( \left( \frac{x^2+b}{x} \right)^2 - 4b \right) = f\left( (f-a)^2 - 4b \right).
\end{align*}
Therefore, we get a rational map\[\phi := (f, g) = \left( 
    x + a + \frac{b}{x},\ y - \frac{by}{x^2}
 \right) : E\to E',\]
where \[E' : y'^2 = x'(x'^2 + a_1x' + b_1),\quad\begin{cases}
    a_1 = -2a \\ b_1 = a^2 - 4b
\end{cases}\in\Z.\]
Because $\ord_O(x) = -2$ and $\ord_O(y) = -3$, we have $\ord_O(f) = -2$ and $\ord_O(g) = -3$.
So $\phi$ sends $O$ to \begin{align*}
    \phi(O) = \left[ f(O) : g(O) : 1 \right] = [0 : 1 : 0] = O'.
\end{align*}
Moreover, \[\phi(P) = O\implies f(P) = \infty\iff P = O\text{ or } P = (0, 0),\]
hence $\phi$ is a non-constant isogeny.
The field $\Q$ has characteristic $0$, so $\phi$
is separable, and $\deg \phi = \#\ker\phi = 2$.
Comparing degree shows that \[\Q(E')\simeq\Q(f, g) = \Q(E)^{t^* = \Id}.\]

\subsection*{Study $E\to E'\to E$}
The curve $E'$ takes a much similar form to $E$,
so we can do the same thing, and get an isogeny \[\psi = \left( 
    x' + a_1 + \frac{b_1}{x'},\ y' - \frac{b_1y'}{x'^2}
 \right) : E'\to E''\]
of degree $2$ with $\ker\psi = \{O, (0, 0)\}$, and \begin{align*}
    E'' : y''^2 = x''(x''^2  + 4ax'' + 16b ).
\end{align*}
This time $E''$ is isomorphic to $E$ via \(\begin{cases}
    x = x''/4, \\ y = y''/8.
\end{cases}\)
We can write the map still by $\psi$, that is
\[\psi = \left( \frac{1}{4}\left( x' + a_1 + \frac{b_1}{x'} \right),
\frac{1}{8}\left( y' - \frac{b_1y'}{x'^2} \right) \right) : E'\to E.\]
So we obtain an endomorphism $E\to E$ given by $\psi\circ\phi$, whose kernel is $\ker (\psi\circ\phi) = E[2]$,
because \[\phi^{-1}((0, 0)) = \{(x, y)\in E = E(\bar\Q)\mid x^2 + ax + b = 0\} = \text{ the other 2-torsion points on }E.\]
Moreover, $\aut(E) = \{[\pm 1]\}$;
one verifies this by noting that: an automorphism is just a change of variable $(u; r, s, t)$ with $u$ invertible that fixes the equation.
Hence $\psi\circ\phi = [\pm 2]$,
and $\psi(\phi(E)) = 2E$.

\subsection*{Study \texorpdfstring{$\phi(E)\subset E'$}{phi(E)}}
% Let $P = (x, y)\in E$ and $\phi(P) = (u, v)\in E'$.
% We express $(x, y)$ in terms of $(u, v)$.
% By definition,\[\begin{cases}
%     u = \dfrac{y^2}{x^2}, \\ v = x + \dfrac{b}{x} + a.
% \end{cases}\]
\begin{lemma}
    If $(u, v)\in E'(\Q)$,
    then \[(u, v)\in\phi(E(\Q))\iff u\in(\Q^\times)^2\text{ or }\begin{cases}
        u = 0, \\ a^2 - 4b\in (\Q^\times)^2.\qedhere
    \end{cases}\]
\end{lemma}

This suggests us to consider \[q : E'(\Q)\to\Q^\times/(\Q^\times)^2\quad \begin{aligned}
    (u, v) &\mapsto\begin{cases}
        [u], & u\ne 0,\\ [a^2 - 4b], & u = 0
    \end{cases}\\
    O&\mapsto [1].
\end{aligned}\]

\begin{lemma}
    $q$ is a group homomorphism.
\end{lemma}
\begin{proof}
    We need to prove: if $P_1, P_2, P_3\in E'(\Q)$ and $P_1 + P_2  + P_3 = O$,
    then \[q(P_1)q(P_2)q(P_3) \in \left( \Q^\times \right)^2.\]
    Not hard.
\end{proof}

So we can write $\phi(E(\Q)) = \ker q$.
\begin{lemma}\label{E'(Q) to Q/Q2 has finite image}
    The image of $q$ is finite.
\end{lemma}
\begin{proof}
    Any element in $\Q^\times/(\Q^\times)^2$ can be written \textit{uniquely} as $[r]$,
    where $r\in\Z$ and $r$ is square-free.
    We show that: \[[r]\in\im q\implies r\mid b_1.\]
    
    Assume that $(u, v)\in E'(\Q)$, and $q((u, v)) = [r]$,
    where $r\in\Z$ is square free.
    No matter $u = 0$ or not,\[\exists s, t\in\Q,\quad \begin{cases}
        u = rt^2,\\u^2 + a_1u + b_1 = rs^2.
    \end{cases}\]
    Write \[t = \dfrac{l}{m}, \quad l, m\in\Z,\  (l, m) = 1.\]
    Elliminating $u$ in the above equation, we get\begin{equation}\label{eq: a condition on phi(E(Q))}
        r^2l^4 + a_1rl^2m^2 + b_1m^4 = rn^2,\quad\text{where }n := m^2s.
    \end{equation}
    The LHS is in $\Z$ and it is not square-free,
    yet $r$ is square-free, so $n = m^2s\in\Z$.
    Suppose that there is a prime $p\mid r$ but $p\nmid b_1$.
    From \cref{eq: a condition on phi(E(Q))},
    we have \begin{align}
        p\mid b_1m^4\stackrel{p\nmid b_1}{\implies}
        p\mid m\implies
        p^2\mid rn^2\stackrel{r\text{ square-free}}{\implies}
        p\mid n\implies
        p^3\mid r^2l^4\implies
        p\mid l,
    \end{align}
    contradicting $(l , m) = 1$.
\end{proof}



Therefore, we deduce that:\begin{theorem}
    $E'(\Q)/\phi(E(\Q))\simeq \im q$ is finite.
\end{theorem}

\begin{corollary}[a special case of weak Mordell-Weil]
    $E(\Q)/2E(\Q)$ is finite.
\end{corollary}

\subsection*{Some remarks}
This proof offer a way to compute $E(\Q)/2 E(\Q)$
given $E : y^2 = x(x^2 + ax+b)$.
\begin{itemize}
\item \textit{Determine $E'(\Q)/\phi(E(\Q))$}.
The coefficients of $E'$ are \[a_1 = -2a,\ b_1 = a^2 - 4b.\]
By the proof of \cref{E'(Q) to Q/Q2 has finite image}, we solve \cref{eq: a condition on phi(E(Q))}
\[r^2l^4 + a_1rl^2m^2 + b_1m^4 = rn^2\]
in $(l, m, n)\in\Z^3\sminus\{(0, 0, 0)\}$
for $r\mid b_1$ square-free,
then use \[(u, v) = \left( \frac{rl^2}{m^2}, \frac{rnl}{m^3} \right)\] to find all $(u, v)\in E'(\Q)/\phi(E(\Q)) \simeq \im \left( E'(\Q)\to \Q^\times/\left( \Q^\times \right)^2 \right)$,
then compute the points \[\psi(u, v) = \left( \frac{1}{4}\left( u + a_1 + \frac{b_1}{u} \right), \frac{1}{8}\left( v - \frac{b_1v}{u^2} \right) \right)\in\psi\left( E'(\Q)/\phi(E(\Q)) \right).\]
\item \textit{Determine $E(\Q)/\psi(E'(\Q))$}.
Solve \footnote{
    By definition, we can solve
    \[r^2l^4 + 4arl^2m^2 + 16bm^4 = rn^2\]
    to get points $ \left( \frac{rl^2}{m^2}, \frac{rnl}{m^3} \right)\in E''(\Q)$,
    then get $\left( \frac{rl^2}{4m^2}, \frac{rnl}{8m^3} \right)\in E(\Q)$.
    Composing together gives the computation we use.
}
\[r^2l^4 + arl^2m^2 + bm^4 = rn^2\]
for $r\mid b$ to get \[(u, v) = \left( \frac{rl^2}{m^2}, \frac{rnl}{m^3} \right)\in E(\Q)/\psi(E'(\Q)).\]
\item Determine $E/2E(\Q)$.
Note that the terms in the exact sequqnce
\[0\to \psi\left( E'(\Q)/\phi(E) \right)\to E(\Q)/\psi(\phi(E(\Q)))\to E(\Q)/\psi(E'(\Q))\to 0\]
of abelian groups are $\Z/2\Z$-modules,
so it splits.

\end{itemize}


There is no algorithm to solve it, so this method is not effective.
But we can modulo $p$ to show it has no solution and sometimes we can determine $E(\Q)/2E(\Q)$.
\begin{example}
    $E : y^2 = x(x^2 - x + 6)$.
\begin{itemize}
\item Determine $E'(\Q)/\phi(E(\Q))$.\par
We have $a = -1$, $b = 6$, $a_1 = -2a = 2$, $b_1 = a^2 - 4b = -23$,
$\implies r\mid 23,\implies$
$r = \pm1,\pm 23$.
The value $r = -23 = b_1 = a^2 - 4b$, by definition of $q$ when $u = 0$,
corresponds to $(0, 0)$, so $[-23]\in\im q$.
Since $\im q$ is a subgroup of $\Q^\times/(\Q^\times)^2$,
it must be $\im q = \{[1], [-23]\}$ or $\im q = \{[1], [-23], [-1], [23]\}$.


\item Determine $E(\Q)/\psi(E'(\Q))$.



\end{itemize}
    
\end{example}







\subsection{Step 1 - General Case}

\subsection{Step 2 - For \texorpdfstring{$E/\Q$}{E/Q}}
