\documentclass{article}
\usepackage{fontspec}
\usepackage{amsmath, amssymb, amsthm, amsbsy, mathrsfs}
\usepackage{stmaryrd}
\usepackage{enumerate}
\usepackage[colorlinks,
linkcolor=cyan,
anchorcolor=blue,
citecolor=blue,
]{hyperref}
\usepackage[capitalize]{cleveref}
\usepackage[margin = 1in, headheight = 12pt]{geometry}
\usepackage{bbm}
\usepackage{tikz-cd}

\linespread{1.2}


\theoremstyle{definition}
\newtheorem{theorem}{Theorem}
\newtheorem{definition}{Definition}
\newtheorem{exercise}{Exercise}[section]
\newtheorem{problem}{Problem}
\newtheorem{example}{Example}
\newtheorem{proposition}{Proposition}[section]
\newtheorem{lemma}{Lemma}[section]
\newtheorem{corollary}{Corollary}[section]

\theoremstyle{remark}
\newtheorem*{remark}{Remark}

\renewcommand{\Re}{\mathop{\mathrm{Re}}}
\renewcommand{\Im}{\mathop{\mathrm{Im}}}
\renewcommand{\bar}{\overline}
\renewcommand{\tilde}{\widetilde}
\renewcommand{\hat}{\widehat}

% 新命令
% 数学对象
    \newcommand{\R}{\mathbb{R}}
    \newcommand{\C}{\mathbb{C}}
    \newcommand{\Q}{\mathbb{Q}}
    \newcommand{\Z}{\mathbb{Z}}
    \DeclareMathOperator{\GL}{GL}
    \DeclareMathOperator{\SL}{SL}
    \newcommand{\p}{\mathfrak{p}}
    \renewcommand{\P}{\mathbb{P}}
    \newcommand{\A}{\mathbb{A}}
% 集合
    \newcommand{\sminus}{\smallsetminus} %(集合)差
% 范畴
    \newcommand{\op}[1]{{#1}^{\mathrm{op}}} %反范畴
    \DeclareMathOperator{\enom}{End} %自态射
    \DeclareMathOperator{\isom}{Isom} %同构
    \DeclareMathOperator{\aut}{Aut} %自同构
    \DeclareMathOperator{\im}{im} %像
    \newcommand{\Set}{\mathbf{Set}} %集合范畴
    \newcommand{\Abel}{\mathbf{Ab}} %群范畴
    \newcommand{\Ring}{\mathbf{Ring}}
    \newcommand{\Cring}{\mathbf{CRing}}
    \newcommand{\Alg}{\mathbf{Alg}}
    \newcommand{\Mod}{\mathbf{Mod}}
    \DeclareMathOperator{\Id}{id}
%向量空间, 矩阵
    \DeclareMathOperator{\rank}{rank} %秩
    \DeclareMathOperator{\tr}{Tr} %迹
    \newcommand{\tran}[1]{{#1}^{\mathrm{T}}} %转置
    \newcommand{\ctran}[1]{{#1}^{\dagger}} %共轭转置
    \newcommand{\itran}[1]{{#1}^{-\mathrm{T}}} %逆转置
    \newcommand{\ictran}[1]{{#1}^{-\dagger}} %逆共轭转置
    \DeclareMathOperator{\codim}{codim} %余维数
    \DeclareMathOperator{\diag}{diag} %对角阵
    \newcommand{\norm}[1]{\left\| #1\right\|} %范数
    \DeclareMathOperator{\lspan}{span} %张成
    \DeclareMathOperator{\sym}{\mathfrak{Y}}
% 群
    \DeclareMathOperator{\inn}{Inn} %(群)内自同构
    \newcommand{\nsg}{\vartriangleleft} %正规子群
    \newcommand{\gsn}{\vartriangleright} %正规子群
    \DeclareMathOperator{\ord}{ord} %元素的阶
    \DeclareMathOperator{\stab}{Stab} %稳定化子
    \DeclareMathOperator{\sgn}{sgn} %符号函数
% 环, 域
    \DeclareMathOperator{\cha}{char} %特征
    \DeclareMathOperator{\spec}{Spec} %素谱
    \DeclareMathOperator{\maxspec}{MaxSpec} %极大谱
    \newcommand{\spm}{\maxspec}
    \DeclareMathOperator{\gal}{Gal}
    \DeclareMathOperator{\Frac}{Frac}
% 同调代数
    \DeclareMathOperator{\ext}{Ext}
% 微积分
    % \newcommand*{\dif}{\mathop{}\!\mathrm{d}} %(外)微分算子
% 流形
    \DeclareMathOperator{\lie}{Lie}
%代数几何
    \DeclareMathOperator{\proj}{Proj}
%多项式
    \DeclareMathOperator{\disc}{disc} %判别式
    \DeclareMathOperator{\res}{res} %结式
% 结构简写
    \newcommand{\pdfrac}[2]{\dfrac{\partial #1}{\partial #2}} %偏微分式
    \newcommand{\isomto}{\stackrel{\sim}{\rightarrow}} %有向同构
    \newcommand{\gene}[1]{\left\langle #1 \right\rangle} %生成对象
% 文字缩写
    \newcommand{\opin}{\;\mathrm{open\;in}\;}
    \newcommand{\st}{\;\mathrm{s.t.}\;}
    \newcommand{\ie}{\;\mathrm{i.e.,}\;}

% 重定义命令
\renewcommand{\hom}{\mathop{\mathrm{Hom}}}
\DeclareMathOperator{\Hom}{Hom}
\renewcommand{\vec}{\boldsymbol}
\renewcommand{\and}{\;\text{and}\;}

% 编号
\newcommand{\cnum}[1]{$#1^\circ$} %右上角带圆圈的编号
\newcommand{\rmnum}[1]{\romannumeral #1}

\newcommand{\fring}[1]{\llbracket #1 \rrbracket}

\newcommand{\myit}{$\diamond$}
\newcommand{\Gm}{\mathbb{G}_{\mathrm{m}}}
\renewcommand{\O}{\mathcal{O}}
\newcommand{\nr}{\mathrm{nr}}
\newcommand{\alg}{\mathrm{alg}}
\newcommand{\ab}{\mathrm{ab}}
\DeclareMathOperator{\frob}{Frob}
\newcommand{\F}{\mathbb{F}}
\DeclareMathOperator{\Ind}{Ind}
\newcommand{\m}{\mathfrak{m}}
\DeclareMathOperator{\wideg}{wideg}
\DeclareMathOperator{\val}{val}

% \tikzcdset{scale cd/.style={every label/.append style={scale=#1},
%     cells={nodes={scale=#1}}}}

\title{Notes on Local Fields}
% \author{}

\begin{document}
\maketitle

\section{Review: Galois theory}
\subsection{Field Extensions}
Let $L/K$ be an algebraic extension. It is called: \begin{enumerate}
    \item [$\diamond$]\textbf{normal}, if every polynomial $f\in K[T]$ with a root in $L$ splits in $L$, $\iff$ $L$ is the splitting field of a bunch of polynomials over $K$;
    \item [$\diamond$]\textbf{separable}, if for every element in $L$, its minimal polynomial over $K$ has no multiple roots in its splitting field, $\iff$ $\gcd(f, f') = 1$;
    \item [$\diamond$]\textbf{Galois}, if it is normal and separable, i.e., $L$ is the splitting field of a bunch of \textit{seperable} polynomials over $K$. We put $\gal(L/K) := \aut_K(L)$.
\end{enumerate}
\begin{remark} {}
\begin{enumerate}
    \item For a finite \textit{normal} extension $L/K$, $|\aut_K(L)| \le [L:K]$, where the equality holds $\iff L/K$ is separable, i.e. Galois. This is because a $K$-automorphism of $L = K[T]/(f)$ just permutes the roots of $f$.
    \item Normality is NOT transitive. As an example, take $\Q\subset\Q(\sqrt{2})\subset\Q(\sqrt[4]{2})$.
\end{enumerate}
\end{remark}

% We introduce a convenient notion here。
% \begin{definition}
%     Let $\Omega/F$ be a field extension and $\mathcal{C}$ a family of subextensions in $\Omega/F$. We say $\mathcal{C}$ is \textbf{distinguished}, if it satisfies:\begin{enumerate}
%         \item [\textbf{D1}] $\forall L/E/F$, \[L/F\in\mathcal{C}\iff L/E\in\mathcal{C}\ \&\ E/F\in\mathcal{C};\]
%         \item [\textbf{D2}] $\forall L, M$, \[L/F\in\mathcal{C}\implies LM/M\in\mathcal{C}.\]
%     \end{enumerate}    
% \end{definition}
% \begin{remark}
%     Let $\mathcal{C}$ be a distinguished family of subextensions.
% \begin{enumerate}
%     \item The conditions implies that $\mathcal{C}$ is closed under \textit{finite} composition.
%     \item The \textit{union} of all fields in $\mathcal{C}$ is a field, and thus equal to the composition of all fields in $\mathcal{C}$.
%     \item Finite, algebraic, seperable and purely inseperable extensions are distinguished.
% \end{enumerate}
% \end{remark}


\subsection{Galois theory}
Now let $L/K$ be a Galois extension. Equip $\gal(L/K)$ with the following \textbf{Krull topology}: $\forall\sigma\in\gal(L/K)$, a basis of nbhd around $\sigma$ is given by\[\sigma\gal(L/F),\quad\text{where } L/F/K,\; F/K < \infty\text{ \& Galois}.\]
\begin{itemize}
    \item Two elements $\sigma, \tau\in\gal(L/K)$ are ``close'' to each other, if $\sigma|_F = \tau|_F$ for sufficiently large finite Galois subextensions $F/K$.
    \item Both multiplication and inverse on $\gal(L/K)$ are continuous for Krull topology.
    \item The Krull topology is profinite for $L/K$ infinite, whence \[\gal(L/K) \simeq \lim_{\stackrel{\longleftarrow}{F/K < \infty\text{ \& Galois}}}\gal(F/K). \]
    When $L/K < \infty$, this is the discrete topology.
    \item If there is a tower \[K\subset L_1\subset L_2\subset\dots\subset L,\] where all $L_n/K$'s are Galois, and \[L = \bigcup_{n} L_n,\]
    then \[\gal(L/K) = \varprojlim_n\gal(L_n/K).\]

\end{itemize}

Galois theory says that the intermediate fields of $L/K$ corresponds to the closed subgroups of $\gal(L/K)$ bijectively and $\gal(L/K)$-equivariantly.
\begin{enumerate}
    \item [$\rightarrow$:] For an intermediate field $F$, it gives $\gal(L/F)\subset \gal(L/K)$. Note that $L/F$ is Glaois, but $F/K$ is NOT always Galois.
    The Galois group acts on $\{\text{intermediate field of } L/K\}$ via $(\sigma, F) \mapsto \sigma F = \sigma(F)$.
    \item [$\leftarrow$:] For a closed subgroup $H < G$, it fixes a subfield $L^H \subset L$. The Galois group acts on $\{H : H < \gal(L/K)\}$ by conjugation, i.e., $(\sigma, H) \mapsto \sigma H\sigma^{-1}$.
\end{enumerate}
In particular,\begin{enumerate}
    \item [$\diamond$] \textit{Galois} extensions correspond to \textit{normal closed} subgroups, and
    \item [$\diamond$] \textit{finite} extensions correspond to \textit{open} subgroups.
\end{enumerate}

\subsubsection*{Base change}
\begin{proposition}\label{field extension base change}
    Let $L/K$ be Galois. If $M/K$ is any extension, and both $L$ and $M$ are subextensions of $\Omega/K$, then $LM/M$ is Galois, and
    \begin{align*}
        \gal(LM/M) &\stackrel{\sim}{\longrightarrow}\gal(L/L\cap M)\\
        \sigma&\longmapsto \sigma|_L.
    \end{align*}
\end{proposition}
As a corollary, if $L, L'$ are Galois subextensions of $\Omega/K$, then $LL'/K$ is also Galois, and \begin{align*}
    \gal(LL'/K)&\hookrightarrow \gal(L/K)\times \gal(L'/K)\\
    \sigma &\mapsto (\sigma|_L, \sigma|_{L'});
\end{align*}
this embedding is an isomorphism if $L\cap L' = K$.





\section{Extensions of Local Fields}

\subsection{Simple Extensions of DVRs}
Let $A$ be a local ring with ($\mathfrak{m}$, $k$), $f\in A[X]$ a monic polynomial of deg $n$.
We consider the extension \[A \to B_f := A[X]/f.\]

Let $\bar{f}$ be the image of $f$ in $k[X] \simeq A[X]/\mathfrak{m}$ with decomposition \[\bar{f} = \prod_{i}\bar{g}_i^{e_i},\ g_i\in A[X],\ \bar{g}_i\in k[X]\text{ irreducible.}\]
and \[\bar{B}_f := B_f/\mathfrak{m}B_f \simeq A[X]/(\mathfrak{m}, f) \simeq k[X]/(\bar{f}).\]
\begin{lemma}
    $\mathfrak{m}_i := (\mathfrak{m},\ g_i\bmod f)\subset B_f$ are all the distinct maximal ideals of $B_f$.
\end{lemma}
\begin{proof}
    Denote $\pi : B_f\to\bar{B}_f$. We have $B_f/\mathfrak{m}_i \simeq \bar{B}_f/(\bar{g}_i)$, so $\mathfrak{m}_i$'s are maximal.
    Note that $\mathfrak{m}_i = \pi^{-1}(\bar{g}_i)$.

    Take $\mathfrak{n}\in\spm B_f$.
    If $\mathfrak{n}\supset\mathfrak{m}$, then $\mathfrak{n} = \pi^{-1}\pi\mathfrak{n}$,
    and goes to a maximal ideal in $\bar{B}_f$ (because $\bar{B}_f/\pi\mathfrak{n} \simeq B_f/\mathfrak{n}$),
    so $\mathfrak{n} = \pi^{-1}(\bar{g}_i) = \mathfrak{m}_i$.

    So assume that $\mathfrak{m}\not\subset\mathfrak{n}$, then $\mathfrak{n} + \mathfrak{m}B_f = B_f$.\footnote{In this case $\mathfrak{n}/(\mathfrak{n\cap m})\simeq \bar{B}_f$ as $B_f$-module, and thus $\pi^{-1}\pi\mathfrak{n} = B_f$.}
    Therefore \[\frac{B_f}{\mathfrak{n}} = \frac{\mathfrak{n}+\mathfrak{m}B_f}{\mathfrak{n}} \simeq \frac{\mathfrak{m}B_f}{\mathfrak{n}}.\]
    Since $A$ is local and $B_f$ is a f.g. $A$-mod, by Nakayama's lemma, we see $\mathfrak{n} = B_f$. Contradiction.


\end{proof}

Now take $A$ to be a DVR with $\mathfrak{m} = (\varpi)$ and $K = \Frac A$. Put $L := K[X]/(f)$.
We give two cases where $B_f$ is a DVR.

\subsubsection*{Unramified case}
Let $\bar{f}\in k[X]$ be irreducible. Then $B_f$ is a DVR with maximal ideal $\mathfrak{m}B_f$.
\begin{corollary}\label{simple ext of dvr - unramified - is field}
    $f\in A[X]$ is also irreducible, so $L$ is a field.
    Moreover, $B_f$ is the integral closure of $A$ in $L$, and $L/K$ is unramified if $\bar{f}$ is separable.
\end{corollary}
\begin{proof}
    $L = K[X]/f \simeq \left( A[X]/f \right)\otimes_{A} K = B_f\otimes_A K$.
    As $B_f$ is a domain, $L$ is a field and $L = \Frac B_f$.
    Since $A$ is integrally closed, $B_f$ is also integrally closed, so $B_f$ is the integral closure of $A$ in $L$.
\end{proof}
\subsubsection*{Totally ramified case}
Let $f\in A[X]$ be an \textbf{Eisenstein polynomial}, i.e., \[f = X^n + a_{n-1}X^{n-1} + \cdots  + a_0,\ a_i\in\mathfrak{m},\ a_0\notin\mathfrak{m}^2.\]
\begin{proposition}
    $B_f$ is a DVR, with maximal ideal generated by the image of $X$ and residue field $k$.
\end{proposition}
\begin{proof}
    Let $x$ be the image of $X$ in $B_f$.
    We have $\bar{f} = X^n$, so $B_f$ is a local ring with maximal ideal $(\mathfrak{m}, x)$.
    Because $a_0\in\mathfrak{m\setminus m^2}$, $a_0$ must uniformise $\mathfrak{m}\subset A$, and \[-a_0\bmod f = x^n + \cdots + (a_1\bmod f)\,x,\] Therefore $(\mathfrak{m}, x) = (x)$.
\end{proof}
Similar to \cref{simple ext of dvr - unramified - is field}, $f$ is irreducible and $L$ is a field with $B_f$ the integral closure of $A$ in $L$.

\subsection{Hensel's Lemma}
Let $K$ be a local field, or CDVF
\footnote{We define a \textbf{local field} to be a complete discretely valued field, without the assumption of residue field being finite.}.

There are many versions of Hensel's lemma.
A relatively complicated one is: the decomposition of a polynomial modulo $\m_K$ into \textit{coprime} factors can be lifted to $K$.
\begin{theorem}
    [Hensel's lemma]\label{hensel lemma - lift coprime factors}
    Let $f\in\O_K[X]$, $\gamma, \eta\in k[X]$
    s.t. \[\begin{cases}
        \bar{f} = \gamma\eta,\\
        (\gamma, \eta) = 1
    \end{cases}\text{  in }k[X].\]
    Then there exists $g, h\in\O_K[X]$ s.t.\[\begin{cases}
        f = gh, &\text{ in } \O_K[X],\\
        \bar{g} = \gamma, \bar{h} = \eta &\text{ in } k[X].
    \end{cases}\]
\end{theorem}
Also the most famous ones about lifting roots in residue fields.
\begin{theorem}
    \label{hensel lemma - lift simple root}
    Let $f\in \O_K[X]$, $\pi\in\m_K$, $\alpha_0\in \O_K$ s.t. \[\begin{cases}
        P(\alpha_0)\in\pi O_K,\\ 
        P'(\alpha_0)\in\O_L^\times.
    \end{cases}\]
    Then $\exists !\ \alpha\in \alpha_0 + \pi\O_K$ s.t. \[P(\alpha) = 0.\]
\end{theorem}
\begin{theorem}
    \label{hensel lemma - variant}
    Let $f\in\O_K[X]$, $0\le \lambda < 1$,
    $\alpha_0\in\O_K$ s.t. \[|P(\alpha_0)|\le \lambda |P'(\alpha)|^2.\]
    Then $\exists!\ \alpha\in\O_K$ s.t. \[\begin{cases}
        P(\alpha) = 0,\\ 
        |\alpha - \alpha_0| \le\lambda |P'(\alpha_0)|.
    \end{cases}\]
\end{theorem}
Note that in both cases, the lift is \textit{unique}.

\subsubsection*{Proof of Hensel's lemma}
We propose two kind of proofs for them. Full proof is only given to \cref{hensel lemma - lift coprime factors}.

The first one is the traditional $\pi$-adic approximation.
\begin{lemma}\label{bezout with deg condition}
    If $k$ is a field, $P, Q\in k[X]$ are coprime and $R\in k[X]$,
    then \[\exists A, B\in k[X],\quad R = AP + BQ\ \st \deg A\le \deg Q - 1.\]
\end{lemma}
\begin{proof}
    Let $R = A_0P + B_0Q$, then $R = (A_0 - uQ)P + (B_0 + uP)Q$ are all the possibilities.
    By Euclidean division, dividing $A_0$ by $Q$ gives us $u\in k[X]$ with $\deg (A_0 - uQ)\le \deg Q - 1$. 
\end{proof}

\begin{proof}[Proof of \cref{hensel lemma - lift coprime factors}]
    % Write $\deg f = d$, $\deg \gamma = m$, $\implies\deg \eta\le d - m$.
    Let $\pi$ be a uniformiser.
    Take a lift $g_1$ of $\gamma$ with $\deg g_1 = \deg \gamma$, and a lift $h_1$ of $\eta$ with $\deg h_1 = \deg\eta$.
    We seek for : $\{g_n\}_n, \{h_n\}_n\subset\O_K[X]$ s.t. \[f \equiv g_nh_n\bmod \pi^n,\quad g_{n + 1} = g_n + \pi^ny_n,\ h_{n+1} = h_n + \pi^nz_n .\]
    In order $\lim_n g_n, \lim_n h_n\in\O_K[X]$,
    we require $\deg y_n \le \deg \gamma$, $\deg z_n\le \deg \eta$.

    Assume we have found $g_nh_n\equiv f\bmod \pi^n$,
    then we need\begin{align*}
        &{} f\equiv (gn+\pi^ny_n)(h_n + \pi^nz_n)\equiv g_nh_n + \pi^n(g_nz_n + h_ny_n) &\bmod \pi^{n+1}\\
        \implies & \O_K[X]\ni \frac{f - g_nh_n}{\pi^n}\equiv g_nz_n+h_ny_n \equiv \gamma z_n + \eta y_n&\bmod\pi.
    \end{align*}
    Via \cref{bezout with deg condition},
    we find $z_n, y_n\in\O_K[X]$ with \[\deg y_n\le\deg\gamma -1,\implies\deg z_n \le \deg f - \deg\eta.\qedhere\]
\end{proof}


Another proof uses the \textit{fixed point theorem}
\begin{lemma}
    [Fixed point theorem]
    Let $C$ be a complete metric space, $f : C\to C$ a \textbf{contracting map}, i.e, \[\exists\alpha, 0\le \alpha\ {\color{red}< 1} \st\ |f(x) - f(y)|\footnotemark <\alpha |x-y|,\ \forall x, y\in C. \]
    \footnotetext{Not a right notation, but anyway.}
    Then $f$ has a \textit{unique} fixed point in $C$.
\end{lemma}

Recall that the $K[X]$ is equipped with the \textbf{Gauss nrom}: for $f = \sum_{i=0}^n a_iX^i$,
\[|f| := \max\{a_0, \dots, a_n\}.\]

(T.B.C.)
% \begin{proposition}
%     The Gauss norm is an ultrametric norm on the $K$-algebra $K[X]$\footnote{
%         The Gauss norm is also complete on $K\gene{X_1, \dots, X_n} = \varprojlim_m K[X_1, \dots, X_n]/\pi^m$}. 
% \end{proposition}
% \begin{proof}
    
% \end{proof}

% \begin{proof}
%     [Proof of \cref{hensel lemma - lift coprime factors}]
%     Recall that:
    
% \end{proof}

\subsection{Extending the norm}
Let $K$ be a complete normed field\footnote{By a \textbf{complete normed field} $K$,
we always require an \textit{ultrametric / nonarchimedean} norm $|\cdot|_K$. The norm corresponds to a valuation $\val : K\to\R\cup\{\infty\}$ by $\val(x) = -\log_a|x|$ for any chosen $a\in \R_{\ge 1}$, which is not necessarily discrete.
Then \begin{center}
    $K$ is a local field $\iff \m_K$ is a principal ideal $\iff \val(K^\times)$ is a discrete subgroup of $\R$.
\end{center}
}.
Consider an algebraic extension $L/K$, we wonder if the norm extend to $L$.
% For this, we first show that: all equivalent complete norm on $K$ are actually \textit{equal}.

Recall: two norms $|\cdot|_1$ and $|\cdot|_2$
on a $K$-vector space $V$ are \textbf{equivalent} \begin{center}
    $:=$ they give the same topology
\end{center}\[\iff (|x_n|_1\to 0\iff |x_n|_2\to 0).\]
\begin{proposition}
    If $|\cdot|_1$ and $|\cdot|_2$ are two equivalent norms on $K$, then \[\exists\alpha > 0,\quad |\cdot|_1 = |\cdot|_2^\alpha\]
\end{proposition}
\begin{proof}
    ($\impliedby$) Assume $|\cdot|_1\sim|\cdot|_2$.
    \begin{itemize}
\item Let $y\in K$. $|y^n|_i\to 0\iff |y|_i < 1$,
\[\implies \left( |y|_1 < 1\iff |y|_2 < 1 \right).\]
    \end{itemize}
Fix $y\in K^\times$ with $|y|_1\ne 1$. Then $|y|_2\ne 1$.
\begin{itemize}
    \item Let $x\in K$.
    By previous computation, 
    \begin{align*}
        &{} |x^my^{-n}|_1 < 1\iff |x^my^{-n}|_2 < 1, &\forall m, n\in\Z,\\
    \implies &{} |x|_1 < |y|_1^{r}\iff |x|_2 < |y|_2^{r}, &\forall r\in\Q,\\
    \implies &{} |x|_1 < |y|_1^{s}\iff |x|_2 < |y|_2^{s}, &\forall s\in\R\\
    \implies &{} |x|_2 = |x|_1^{\alpha}. &
    \end{align*}
    where $\alpha > 0$ is determined by $|y_2| = |y_1|^\alpha$.\qedhere
\end{itemize}
\end{proof}
\begin{theorem}[Artin]
    Let $K$ be complete normed field, $V$ a f.d.$K$-vector space.
    Then all norms on $V$ are equivalent, and $V$ is complete for them.
\end{theorem}
Note that we don't require $K$ to be locally compact; as a price, the norm on $V$ need to be ultrametric too (which is our convention).
\begin{proof}
    Let $e_1, \dots, e_d$ be a $K$-basis of $V$, $\norm{\cdot}_{\infty}$ the corresponding sup-norm. The sup-norm is complete.
    Then we do induction on $d$ to show $\norm{\cdot}\sim \norm{\cdot}_{\infty}$ for any norm $\norm{\cdot}$. Omitted.
\end{proof}

\begin{corollary}
    Let $K$ is a complete normed field, $L/K < \infty$. If the norm on $K$ extends to a norm on $L$, then their is at most one way to do so, and $L$ will be complete.
\end{corollary}
\begin{proof}
    All such norm will be $|\cdot|^\alpha$ for a fixed norm $|\cdot|$.
    These norms coincide on $K$, so $\alpha = 1$.
\end{proof}

In case of complete \textit{discretely valued} fields, there is indeed such an extension.
\begin{theorem}\label{unique extension of norm for fintie extension local field}
    Let $K$ is a local field, $L/K < \infty$.
    Then there the norm on $K$ extends uniquely to $L$, making $L$ also a local field.
    The norm is given by \[|x|_L = \left|N_{L/K}(x)\right|_K^{1/[L : K]},\]
    and $\O_L = $ integral closure of $\O_K$ in $L$. 
\end{theorem}
We give two proofs.
\begin{proof}
    [Proof (algebraic)]Recall that:
\begin{lemma}\label{extension of Dedekind: integral closure of Dedekind domain in finite extension of its fraction field is Dedekind domain}
    If $A$ is a Dedekind, $L/\Frac(A) < \infty$, $B$ is the integral closure of $A$ in $L$, then: $B$ is a Dedekind domain.
\end{lemma}
Apply this to $A = \O_K$,
we see that $B :=$ integral closure of $\O_K$ in $L$ is a Dedekind domain.
Let \[\m_KB = \mathfrak{P}_1^{e_1}\cdots\mathfrak{P}_r^{e_r}\] be the decomposition of $\m_K$ in $B$. Define $v_i(x) := $ exponent of $\mathfrak{P}_i$ in $xB$.
One verifies that $v(\cdot)/e_i$ extends the valuation $v_K$ on $K$ with value group $\Z$.
The uniqueness forces $r = 1$, and $\O_L = \{x\in L\mid v_i(x) > 0\} = B$.
\end{proof}
Another proof gives the explicit formula for the norm. We need a result on integrality.
\begin{proposition}
    Let $K$ be a local field, $P(X) = a_dX^d + a_{d-1}X^{d-1} + \dots + a_0\in K[X]$ an irreducible polynomial with $a_0a_d\ne 0$.
    Then the Gauss norm of $f$ is
    \[|f| = \max\{|a_0|, |a_d|\}.\]
    In particular, if $f$ is monic and its constant term $a_0\in\O_K$, then $P(X)\in\O_K[X]$.
\end{proposition}
\begin{proof}\label{monic polynomial integral iff const coeff}
    Let $n\in\Z$ s.t. $\pi^nP\in \O_K[X]$ and $\overline{\pi^nP}\ne 0\in k[X] $.
    Let $r$ be the Weierstrass degree of $\pi^nP$,so that \[\pi^nP(X)\bmod\pi = \pi^nX^r(a_r + a_{r+1}X + \dots + a_dX^{d-r}).\] 
    If $0 < r < d$,
    then the decomposition lift to a nontrivial decomposition of $\pi^nP$ in $K[X]$ via \cref{hensel lemma - lift coprime factors}.
    Therefore $r = 0$ or $r = d$.
    Now nate that $|f| = |a_r|$.
\end{proof}
\begin{proof}
    [Proof of \cref{unique extension of norm for fintie extension local field} (analytic)]
    Let $d := [L : K]$.
    We show that $|\cdot|_L := |N_{L/K}(\cdot)|_K^{1/d}$ is indeed a norm on $L$ (it obviously extends $|\cdot|_K$).
    The only nontrivial step is to check the strong triangle inequality, which is equivalent to \[|z|_L < 1\implies |1 + z|_L < 1.\]
    Let $P(X)$ be the minimal polynomial of $z$ over $K$.
    Since
    $N_{L/K}(z) = (-1)^{d} P(0)^{[L : K(z)]}$\footnote{Simple fact, see \cref{compute norm and trace from minimal polynomial}.},
    so by \cref{monic polynomial integral iff const coeff}, \[|z| \le 1\iff P(0)\in\O_K[X]\implies \text{minimal polynomial of }z+1\in \O_K[X]\implies |1 + z| \le 1.\qedhere\]
\end{proof}
\begin{corollary}
    Let $K$ be a local field.
    \begin{enumerate}
        \item [(1)]    The norm on $K$ extends uniquely to its algebraic closure $K^\alg$\footnote{Note that $K^\alg$ is not a local field and not complete. We'll see this later.}.
        \item [(2)] If $L$ and $L'$ are two algebraic extension of $K$,
        then any $K$-embedding $\sigma\in \Hom_K(L, L')$ preserves the norm; i.e., $|\sigma(x)|_{L'} = |x|_L$.
    \end{enumerate}
\end{corollary}

\subsection{Unramified Extensions of Local Fields}
Let $K$ be a local field (i.e., CDVF).
We assume further that both $K$ and its residue field $k = \mathcal{O}_K/\mathfrak{m}$ are perfect.

The slogan is that unramified extensions are just extensions of residue fields.
Using Hensel's lemma, an extension $k(a)/k$ can be lifted to a unique extension $K(\alpha)/K$ over $K$ with \[\gal(K(\alpha)/K)\simeq \gal(k(a)/k).\] Moreover, given an extension $L/K$, there is a maximal unramified subextension $K_0$ in $L$ containing every unramified extensions.

Now we assume $k$ to be finite. Then adjoining roots of unities with order coprime to $p = \cha k$ gives all finite unramified extensions of $K$.

\begin{example}
    Let $K/\Q_p < \infty$ and $k = \mathbb{F}_q$.
    Then the unique extension of $k$ of degree $n$ is the splitting field of $X^{q^n} - X$ over $k$, which equals $k(\mu_{q^n - 1})$ once we fix an algebraic closure of $k$.
    So the unramified extension $K_n/K$ of degree $n$ is the splitting field of $X^{q^n} - X$ over $K$, i.e., \[K_n = K(\mu_{q^{n} - 1}).\] The Galois group $\gal(K_n/K)$ is generated by $\frob_K$, which is determined by \[\frob_K\beta \equiv \beta^q \mod\varpi,\ \forall\beta\in\mathcal{O}_{K_n}\]for any uniformiser $\varpi$ (simultaneously of $K$ and $K_n$).

    What if we adjoin $\zeta_{m}$ to $K$ where $m$ is an arbitary integer prime to $p$?
    The answer is that $K(\mu_m)$ is unramified of degree the smallest positive integer $f$ s.t. $m \mid p^f - 1$, by the following \cref{cyclotomic extension of finite fields} on finite fields.
\end{example}

\begin{lemma}\label{cyclotomic extension of finite fields}
    Let $\zeta_n$ be a primitive $n$-th root of unity over $\F_q$ with $q, n$ coprime. Then $[\F_q(\zeta_n) : \F_q]$ is the smallest integer $f > 0$ s.t. $n \mid q^f - 1$.
\end{lemma}
\begin{proof}
    Because $\cha\F_q\nmid n$, the primitive root $\zeta_n$ exists and $\F_q(\zeta_n)$ is the splitting field of $X^n - 1$ over $\F_q$.
    The degree $f = [\F_q(\zeta_n) : \F_q]$ is the order of $\frob_q$ on $\F_q(\zeta_n)$, i.e., $f$ is the smallest integer s.t. \[\frob_q^f(\zeta_n) = \zeta_n^{q^f} = \zeta_n.\] The definition of primitive root of unity says that \[\zeta_n^{q^f -1} = 1\iff n \mid q^f - 1.\qedhere\]
\end{proof}

\subsection{Newton Polygon}
Let $K$ be a local field with valuation $\val$ extended to $K^\alg$.

For $P = a_0 + a_1 X + \dots + a_dX^d\in K[X]$,
the \textbf{Newton polygon} of $P := \mathrm{NP}(P) := $ convex hull of points \[(0, \val(a_0)), (1, \val(a_1)),\dots, (d, \val(a_d)).\]
\begin{itemize}
    \item $\mathrm{NP}(P)$ is a union of linked segments with increasing slopes.
    \item \textbf{length of a segment} $:=$ its length along $x$-axis.
\end{itemize}

\begin{theorem}
    The number of roots of $P$ in $K^\alg$ with valuation $\lambda$ = the length of $\mathrm{NP}(P)$ with slope $-\lambda$.
\end{theorem}

\subsection{Ramification Groups}

Let $K$ be a CDVF with perfect residue field $k$, $L/K<\infty$ Galois. We will study the Galois group \[G := \gal(L/K)\] by giving filtrations on it.



\subsection{\texorpdfstring{Krasner's lemma and the noncompleteness of $\bar\Q_p$}{Krasner's lemma and the noncompleteness of bar Qp}}
Fix an algebraic closure $\bar\Q_p = \Q_p^\alg$ of $\Q_p$.
Krasner's lemma states that if $\beta\in\bar\Q_p$ is closer to $\alpha\in\bar\Q_p$ than any other conjugate of $\alpha$ over $F$,
then $\alpha\in F(\beta)$. Therefore, if two polynomials are ``close enough'', they will give the same extension.
\begin{theorem}
    [Krasner's lemma]\label{Krasner lemma}
    Let $F/\Q_p < \infty$, $\alpha, \beta\in \bar{\Q}_p$. If \[|\alpha - \beta| < |\alpha-\alpha_i|,\quad i = 2, \dots, n,\]
    where $\alpha_1 = \alpha, \alpha_2, \dots, \alpha_n$ are all the conjugates of $\alpha$ over $F$,
    then \[F(\alpha)\subset F(\beta).\]
\end{theorem}
\begin{proof}
    Let $K/F$ be finite Galois with $\alpha, \beta\in K$. Then $g\alpha, g\in\gal(K/F)$ are all the conjugates of $\alpha$ over $F$.
    Now if $g\in \gal(K/F(\beta))$, then \begin{align*}
        |g\alpha - \alpha|
        &= |(g\alpha - g\beta) + (\beta - \alpha)|\\ &
        \le\min\{|g\alpha - g\beta| , |\alpha - \beta|\} =\footnotemark |\alpha - \beta|
    \end{align*}
    \footnotetext{Because embeddings of finite extensions of $\Q_p$ are isometries (the uniqueness of norm extension).}
    So by the assumption,
    we have $\alpha = g\alpha$,
    i.e., $\alpha\in K^{\gal(K/F(\beta))} = F(\beta)$.
\end{proof}

\begin{theorem}
    For every $d\ge 1$, $\Q_p$ has only finitely many extensions of degree $d$.
\end{theorem}
\begin{proof}
    Every finite extension has a unique maximal unramified extension, so it suffices to show that: there is only finitely many unramified extensions of each $F/\Q_p < \infty$ of given degree $e$.

    For $e\ge 1$, the set of Eisenstein polynomials over $F$ is in bijection with \[\Pi := (\m_F\setminus\m_F^2)\times \underbrace{\m_F\times\dots\times\m_F}_{e-1},\]
    which is compact.
    So we just need to show that for each Eisenstein polynomial $P$, its corresponding point in $\Pi$ has a neighbourhood, in which all polynomials give the same extension.

    (T.B.C.)
\end{proof}

\begin{corollary}
    $\bar{\Q}_p$ is not complete.
\end{corollary}
\begin{proof}
    Now we know $\bar{\Q}_p$ is a countable union of finite dimensional $\Q_p$-vector spaces.
    Recall what Baire's theorem says:
    \begin{theorem}
        [Baire category theorem]\label{Baire thm}
        A complete metric space is a Baire space;
        i.e, a countable intersection of open dense sets is dense.

        As a corollary, a complete metric space is not a countable union of nowhere dense\footnote{Being \textbf{nowhere dense} means its closure has empty interior.} sets.
    \end{theorem}
    A finite dimensional $\Q_p$-vector space is closed and nowhere dense, so the union is not complete.
\end{proof}

Let $\C_p := \widehat{\overline{\Q_p}}$ be the completion of $\bar{\Q}_p$.
Note that neither reidue field nor value group are not extended from $\bar\Q_p$ to $\C_p$:
\begin{itemize}
    \item $v_p(\C_p) = v_p(\bar\Q_p) = \Q$\footnote{
        Consider a Cauchy sequence $\{a_n\}_n$ in $\bar{\Q}_p$.
        The difference $a_m-a_{m + d}$ will eventually have valuation $>v_p(a_m)$,
        making $v_p(\lim_{n}a_n) = v_p(a_m)$.
    }.
    \item $k_{\C_p} = \O_{\C_p}/\m_{\C_p}\simeq\O_{\bar{\Q}_p}/\m_{\bar\Q_p}\simeq\F_p^\alg$.\footnote{
        In a sum $\sum_n a_n\in\C_p$,
        a.e. $a_n\in\m_{\C_p}$. 
    }
\end{itemize}

\begin{theorem}
    $\C_p$ is algebraically closed.
\end{theorem}
\begin{proof}
    The idea is simple: root of lim of polynomial = lim of root of polynomial. Let's make this clear.

    Let $P\in \C_p[X]$ be monic of degree $d$.
    Replacing $P(X)$ by $p^{kd}P(p^{-k} X)$ for $k\gg 0$, we may assume $P\in\O_{\C_p}[X]$.


\end{proof}

\subsection{Ax-Sin-Tate theorem and closed subfields of \texorpdfstring{$\C_p$}{Cp}}
Let $\Q_p\subset K\subset\bar\Q_p$, $G_K := \gal(\bar\Q_p/K)$ the absolute Galois group of $K$.
Galois theory eastablishes a bijection \[
    \{\text{subextension of }\bar\Q_p/\Q_p\}\longleftrightarrow
    \{\text{closed subgroup of }\gal(\bar\Q_p/\Q_p)\}
\] via $K = \bar\Q_p^{G_K}$.
We are going to expand this relation to (certain) subextensions of $\C_p/\Q_p$.

Any $g\in\gal(\bar\Q_p/\Q_p)$ is an isometry, thus extends to an isometry and (continuous) field automorphism of $\C_p$, denoted still by $g$.
So what is $\C_p^{G_K}$?
\begin{theorem}
    [Ax-Sin-Tate]
    $\C_p^{G_K} = \widehat K$.
\end{theorem}

\begin{lemma}
    Let $P(X)\in\bar\Q_p[X]$ be monic of degree $n$,
    s.t. all the roots $\alpha$ of $P$ have bounded valuation bounded from below; i.e., $v_p(\alpha) > c$ for some $c\in\R$.
    Let $n = p^kd$ with $p\nmid d$ or $p = d$.
    Then $P^{(p^k)}$ has a root $\beta$ with \[\begin{cases}
        v_p(\beta)\ge c, &n = p^kd, p\nmid d,\\ 
        v_p(\beta)\ge c - \dfrac{1}{p^k(p-1)},
        &n = p^{k+1}.
    \end{cases}\]
\end{lemma}
\begin{proof}
    Write $P(X) = X^n + a_{n-1}X^n + \dots + a_0$, and $q := p^k$.
    \begin{itemize}
        \item $v_p(a_i)\ge (n-i)c$, because $a_i = \pm$ sum of product of $n - i$ roots.
        \item 
    \end{itemize}
\end{proof}

\section{\texorpdfstring{A Bit of $p$-adic Analysis}{}}
In this section, we consider some basic properties concerning powerseries over a closed subfield $K$ of $\C_p$ as functions.

Let $f(X) = \sum_{i\ge 0} a_iX^i\in K\llbracket X\rrbracket $. We can evaluate $f$ at $z\in \C_p$ iff $a_iz^i\to\infty$, so the \textbf{radius of convergence}
is \[\rho(f) := \sup\{\rho\in\R\ |\ a_i\rho^i\to\infty (i\to\infty)\}.\]
\begin{itemize}
    \item If $|z| < \rho(f)$, then $f(z)$ converges in $\C_p$.
    \item If $|z| > \rho(f)$, then $f$ diverges.
    \item $\rho(f(\alpha X)) = \rho(f)\cdot |\alpha|^{-1}$.
\end{itemize}
We are mainly interested in the power series converging on the unit disk, i.e., \begin{align*}
    H_K :={}& \{f\in K\llbracket X\rrbracket\ |\ \rho(f) > 1\}\\
    ={}&  \{f\in K\llbracket X\rrbracket\ |\ a_i\rho^i\to 0, \forall \rho < 1\}\\
    ={}&  \{f\in K\llbracket X\rrbracket\ |\ f\text{ converges on the open unit disk }\m_{\C_p} = B(0, 1)\}.
\end{align*}
\begin{example}
    $K\otimes_{\O_K}\O_K\llbracket X\rrbracket$ = power series over $K$ with bounded coefficients $\subsetneq H_K$.
\end{example}
\begin{example}
    $\log(1 + X) = \log_{\Gm}(X) = X - \displaystyle\frac{X^2}{2} + \frac{X^3}{3} - \dots\in H_K\setminus K\otimes_{\O_K}\O_K\llbracket X\rrbracket$.
\end{example}

\subsection{The Gauss Norm}
\begin{theorem}
    Let $f(X) = \sum_{i\ge 0} a_iX^i\in K\llbracket X\rrbracket $ with $\rho(f) > 0$, a real number $\rho < \rho(f)$ s.t. $\rho\in |\C_p^\times|$.
    Then $\sup_{i\ge 1}{|a_i|\rho^i}$ is a maximum (i.e., $\sup_{i\ge 1}{|a_i|\rho^i} = |a_j|\rho^j$ for some $j$), and \[\sup_{i\ge 1}{|a_i|\rho^i} = \sup_{|z| = \rho} |f(z)| =: |f|_\rho.\]
\end{theorem}
\begin{proof}
\begin{itemize}
    \item     $\rho<\rho(f)\implies |a_i|\rho^i \to 0\implies \sup_{i\ge 1}{|a_i|\rho^i}$ is a maximum.
    \item $|f(z)| = \left| \sum_{i\ge 1} a_iz^i\right| \le\sup_{i\ge 1} |a_i||z|^i$, so $|f|_\rho\le \sup_{i\ge 1}{|a_i|\rho^i}$.
    \item Take $\alpha\in\C_p$ with $|\alpha| = \rho$,
    and $j\in\Z_{\ge 0}$ s.t. $\sup_{i\ge 1}{|a_i|\rho^i} = |a_j|\rho^j$.
    Let $\beta := a_j\alpha^j$.
    Then \[g(X) := \frac{f(\alpha X)}{\beta}\in\O_{\C_p}\llbracket X\rrbracket.\]
\end{itemize}\end{proof}





\section{Lubin-Tate Theory}

\subsection{Formal Groups}
In this section, a formal group means a commutative formal group law of dimension one. If $f\in A\llbracket T\rrbracket$ and $g\in A\fring{X_1, \dots, X_n}$, then \begin{align*}
    f \circ g &:= f(g(X_1, \dots, X_n)),\\
    g\circ f &:= g(f(X_1), \dots, f(X_n)).
\end{align*}

\begin{lemma}\label{power series invertible iff}
    Let $f = \sum_{i\ge 1}a_iT^i\in A\fring{T}$. Then
    \[\exists g\in A\fring{T} \st f\circ g = g\circ f = T\iff a_1\in A^\times.\]
\end{lemma}
\begin{proof}
    Use $A\fring{T} = \varprojlim A[T]/T^n$. For details, see the proof of \cref{fund prop of F_varpi}.
\end{proof}

\subsection{Lubin-Tate formal groups}
From now on, we write $A := \O_K$.

Choose a uniformiser $\varpi$ of $K$. Define
\[\mathcal{F}_\varpi := \left\{f\in\O_K\fring{T}\ \left|\ \begin{aligned}
    &f(T) \equiv \varpi T&\mod {T^2} \\
    &f(T)\equiv T^q&\mod \varpi
\end{aligned}\right.\right\}.\]
For example, $f(T) = T^q + \varpi T\in\mathcal{F}_\varpi$.
The following lemma is a fundamental property of $\mathcal{F}_\varpi$.

\begin{lemma}\label{fund prop of F_varpi}
    Let $f, g\in\mathcal{F}_\varpi$, $\Phi_1$ be a linear form\footnote{A \textbf{linear form} is a homogeneous polynomial of degree 1.} over $\O_K$. Then there is a \textbf{unique} $\Phi\in\O_K\fring{X_1,\dots, X_n}$, s.t.\[\begin{cases}
        \Phi \equiv \Phi_1 \mod (X_1, \dots, X_n)^2,\\
        f(\Phi(X_1, \dots, X_n)) = \Phi(g(X_1), \dots, g(X_n)).
    \end{cases}\]
\end{lemma}
\begin{proof}
    We use a standard method. Finding $\Phi$ is equivalent to finding $\Phi_r\in A[X_1, \dots, X_n]$ s.t. \[\begin{cases}
        \Phi_{r+1} \equiv \Phi_r &\mod (\deg\ge r+1),\\
        f(\Phi_r)\equiv \Phi_r(g(X_1), \dots, g(X_n)) &\mod(\deg\ge r+1). 
    \end{cases}\]
    The second condition is guaranteed because $X\mapsto h(X)$ is $X$-adic continuous for any power series $h$.

    Suppose we have found $\Phi_r$. We look for $\Phi_{r+1}$ of the form $\Phi_{r+1} = \Phi_r + Q$, where $Q$ is homogeneous of degree $r+1$, s.t. \[f(\Phi_{r+1}) \equiv \Phi_{r+1}(g(X_1), \dots, g(X_n)) \mod \deg\ge r+2.\]
    The LHS is
    \[f(\Phi_r) + f(Q)\equiv f(\Phi_r) + \varpi Q\mod\deg\ge r+2,\]
    while the RHS is
    \[\Phi_r\circ g + Q(\varpi X_1, \dots, \varpi X_n)\equiv \Phi_r\circ g + \varpi^{r+1}Q,\]
    so if such a $Q\in A[X_1, \dots]$ exists, it must satisfy 
    \[\varpi(\varpi^r - 1)Q\equiv f\circ \Phi_r - \Phi_r\circ g\mod\deg\ge r+2\]
    and thus being unique. This procedure also shows that all $\Phi_r$'s are unqie if we require $\Phi_{r+1} - \Phi_r$ to be homogeneous.

    Because $\varpi^r - 1\in A^\times$, it suffices to show \[f(\Phi_r) \equiv \Phi_r\circ g\mod \varpi,\] which is clear.
\end{proof}

By \cref{fund prop of F_varpi}, one may define the \textbf{Lubin-Tate formal groups}.
They are exactly the formal group laws admitting an endomorphism\begin{itemize}
    \item that has derivative at the origin equal to a uniformiser of $K$, and
    \item reduces mod m to the Frobenius map $T\mapsto T_q$.
\end{itemize}
Moreover, these formal groups admit $\O_K$-actions and are isomorphic as formal $\O_K$-modules.

\begin{proposition}
    For each $f\in \mathcal{F}_\varpi$, there is a unique formal group $F_f$ over $\O_K$ admitting $f$ as an endomorphism.
\end{proposition}
\begin{proof}
    \cref{fund prop of F_varpi} gives $F_f\in A\fring{X, Y}$ s.t. \[\begin{cases}
        F_f = X + Y + \deg \ge 2,\\
        f(F_f(X+Y)) = F_f(f(X), f(Y)).
    \end{cases}\]
    The associativity is proved by showing that both $G_1 = F_f(X, F_f(Y, Z))$ and $G_2 = F_f(F_f(X, Y), Z)$ satisfies 
    \[\begin{cases}
        G = X+Y+Z + \deg\ge 2,\\
        f(G) = G(f(X), f(Y), f(Z)).
    \end{cases}\]
    This is a direct application of \cref{fund prop of F_varpi} and will be used many times.
\end{proof}

So Lubin-Tate formal groups exist. Now we investigate their homomorphisms.
\begin{proposition}
    For each $f, g\in\mathcal{F}_\varpi$ and $a\in \O_K$, there is a unique $[a]_{g, f}\in \O_K\fring{T}$ s.t. \[\begin{cases}
        [a]_{g, f} = aT + \dots,\\
        g\circ [a]_{g, f} = [a]_{g, f} \circ f,
    \end{cases}\]and $[a]_{g, f}\in\hom(F_f, F_g)$, i.e. \begin{align*} F_g\circ [a]_{g, f} = [a]_{g, f}\circ F_f.\end{align*}
    As a corollary of \cref{power series invertible iff}, each $u\in A^\times$ gives an isomorphism $[u]_{g, f} : F_f\isomto F_g$, and there is a unique isomorphism $F_f\simeq F_g$ of the form $T + \cdots$.
    \qed
\end{proposition}

We write $[a]_{f} := [a]_{f, f}\in\enom F_f$.
Note that \[[\varpi]_f = f.\]

\begin{proposition}
    For any $a, b\in\O_K$, \[[a+b]_{g, f} = [a]_{g, f} + [b]_{g, f},\]and\[[ab]_{h, f} = [a]_{h, g}\circ [b]_{g, f}.\]
    
    In particular, $\O_K\hookrightarrow\enom  F_f$ as a ring by $a\mapsto [a]_f$, making $F_f$ a formal $\O_K$-module. The canonical isomorphism $[1]_{g, f}$ is an isomorphism of $\O_K$-modules.
    \qed
\end{proposition}

\subsection{Construction of \texorpdfstring{$K_\varpi$}{}}
Fix an algebraic closure $K^\alg$ of $K$.
Each $f\in\mathcal{F}_\varpi$ associates to $\mathfrak{m}_{K^\alg}$ an $\O_K$-module structure via \[\alpha +_{F_f}\beta := F_f(\alpha, \beta)\]and \[a\cdot \alpha := [a]_f(\alpha)\footnote{These power serieses converges because they actually falls in a finite extension of $K$.}.\]for $|\alpha| < 1, |\beta| < 1$ and $a\in \O_K$.
We denote this $\O_K$-module by $\Lambda_f$.
If $g\in\mathcal{F}_\pi$, then the canonical isomorphism $[1] : F_f\to F_g$ yields $\Lambda_f\isomto\Lambda_g$.

The $\varpi^n$-torsion part of $\Lambda_f$ is denoted by $\Lambda_{f, n}$, i.e., $\Lambda_{f, n} := \Lambda_f[[\varpi]_f^n]$. Because $[\varpi]_f = f$, $\Lambda_{f, n}$ is the $\O_K$-module consisting of the roots of $f^{(n)} := f\circ\cdots\circ f$.
If one takes $f$ to be an Eisenstein polynomial, then all the roots of $f^{(n)}$ lie in $\mathfrak{m}_{K^\alg}$, so $\Lambda_{f, n}$ is precisely the set of roots of $f^{(n)}$ equipped with the $\O_K$-module structure from $F_f$.

\begin{lemma}\label{pi^n torsion cyclic of}
    Let $M$ an $\O_K$-module, $M_n = M[\varpi^n]$. If\begin{itemize}
        \item $M_1$ has $q = [\O_K : \varpi]$ elements, and
        \item $\varpi : M \to M$ is surjective,
    \end{itemize}
    then $M_n\simeq \O_K/\varpi^n$.
\end{lemma}
\begin{proof}
    Do induction on $n$. The structure theorem of f.g.\! modules over a PID shows that $M_1$ having $q$ elements implies that $M_1\simeq A/\varpi$.
    Now assume it true for $n-1$.
    Look at the sequence \[0\to M_1\to M_n\stackrel{\varpi}{\to} M_{n-1}\to 0.\] Surjectivity of $\varpi$ implies the exactness of this sequence, and thus $M_n$ has $q^n$ elements. In addition, $M_n$ must be cyclic, otherwise $M_1 = M_n[\varpi^n]$ is not cyclic.
\end{proof}

\begin{proposition}
    The $\O_K$-module $\Lambda_{f, n}$ is isomorphic to $\O_K/\varpi^n$, and hence $\enom(\Lambda_{f, n})\simeq \O_K/\varpi^n$.
\end{proposition}
\begin{proof}
    It suffices to show for a chosen $f$, so let's take $f = \varpi T + \dots + T^q$, an Eisenstein polynomial.
    We use the above \cref{pi^n torsion cyclic of} by the following observations.\begin{itemize}
        \item All roots of an Eisenstein polynomial have valuation $>0$.
        \item If $|\alpha| < 1$, then the Newton polygon of $f(T) - \alpha$ shows that its roots have valuation $>0$, and thus $[\varpi] = f(T)$ is surjective on $\Lambda_f$.\qedhere
    \end{itemize}
\end{proof}

\begin{lemma}\label{galois commutes power series}
    Let $L$ be a finite Galois extension of $K$. Then for every $F\in\O_K\fring{X_1, \dots, X_n}$, $\alpha_1,\dots, \alpha_n\in\mathfrak{m}_L$ and $\tau\in\gal(L/K)$,
    \[\tau F(\alpha_1, \dots, \alpha_n) = F(\tau\alpha_1, \dots, \alpha_n).\]
\end{lemma}
\begin{proof}
    Note that $\tau$ acts continuously on $L$, becaunse the extension of valuation for local fields is unique.
    Therefore writing $F = \lim_{m\to\infty} F_m$ gives the desired result.
\end{proof}

\begin{theorem}\label{construction of K_{varpi, n}}
    Let $K_{\varpi, n} := K(\Lambda_{f, n})\subset K^\alg$.
    These fields are independent to the choice of $f$.\begin{enumerate}
        \item [(a)] $K_{\varpi, n}/K$ is totally ramified of degree $q^{n-1}(q-1)$.
        \item [(b)] The action of $\O_K$ on $\Lambda_{f, n}$ defines an isomorphism \begin{equation}
            \left( \O_K/\mathfrak{m}_K^n \right)^\times \simeq \gal(K_{\varpi, n}/K).
        \end{equation}
        \item [(c)] For all $n$, $\varpi$ is a norm from $K_{\varpi, n}$, i.e., $\exists\alpha_n\in K_{\varpi, n}$ with $N_{K_{\varpi, n}/K}(\alpha_n) = \varpi$.
    \end{enumerate}
\end{theorem}
\begin{proof}
    Let $f$ be a polynomial $T^q + \dots + \varpi T$.

    Choose a nonzero root $\varpi_1$ of $f(T)$ and, inductively, a root $\varpi_n$ of $f(T) - \varpi_{n-1}$.
    So $\varpi_n\in\Lambda_{f, n}$, and we obtain a tower of extensions \[K_{\varpi, n}\supset K(\varpi_n)\stackrel{q}{\supset} K(\varpi_{n-1}) \stackrel{q}{\supset}\dots\stackrel{q}{\supset} K(\varpi_1)\stackrel{q-1}{\supset} K.\]
    All the extensions with indicated degrees are given by Eisenstein polynomials, and thus Galois and totally ramified.

    The field $K_{\varpi, n} = K(\Lambda_{f, n})$ is the splitting field of $f^{(n)}$ over $K$, hence $\gal(K_{\varpi, n}/K)$ embeds into the permutation group of the set $\Lambda_{f, n}$. By \cref{galois commutes power series}, the action of $\gal(K_{\varpi, n}/K)$ on $\Lambda_n$ preserves its $\O_K$-action, so
    \[\gal(K_{\varpi_n}/K)\hookrightarrow \aut(\Lambda_{f, n})\simeq (\O_K/\varpi^n)^\times.\]
    So $[K_{\varpi, n} : K]\le (q - 1)q^{n-1}$. Comparing the degree gives $K_{\varpi, n} = K(\varpi_n) $.

    Now we prove (c).
    Let $f^{[n]} := (f/T)\circ f\circ\dots\circ f$. Then $f^{[n]}$ is monic with degree $q^{n-1}(q-1)$ and $f^{[n]}(\varpi_n) = 0$, and thus $f^{[n]}$ is the minimal polynomial of $\varpi_n$ over $K$. So we have \[N_{K_{\varpi, n}/K}(\varpi_n) = (-1)^{q^{n-1}(q-1)}\]
    by the following \cref{compute norm and trace from minimal polynomial}.
\end{proof}

\begin{lemma}\label{compute norm and trace from minimal polynomial}
    Let $L/K$ be a finite extension in an algebraic closure $K^\alg$, and $\alpha\in L$ has minimal polynomial $f$ over $K$ of degree $d$. Suppose \[f(X) = (X-\alpha_1)\cdots(X-\alpha_d)\in K^\alg[X],\] and let $e = [L : K(\alpha)]$
    then \[N_{L/K}(\alpha) = \left(\prod_{i = 1}^d \alpha_i\right)^e,\qquad \tr_{L/K}(\alpha) = e\sum_{i = 1}^d \alpha_i.\]
    Moreover, if \[f(X) = a_dX^d + a_{d-1}X^{d-1} + \dots + a_0,\]then \[N_{L/K}(\alpha) = (-1)^{de}a_0^e,\qquad \tr_{L/K}(\alpha) = -ea_{d-1}.\]
\end{lemma}
\begin{remark}
    This can be deduced from $N_{L/K} = N_{L/K(\alpha)}\circ N_{K(\alpha)/K}$ and $\tr_{L/K} = \tr_{L/K(\alpha)}\circ \tr_{K(\alpha)/K}$.
\end{remark}


Define \[K_\varpi := \bigcup_{n} K_{\varpi, n}.\]
The isomorphisms in \cref{construction of K_{varpi, n}} (b) are \[(\O_K/\varpi^n)^\times\to \gal(K_{\varpi, n}/K)\quad \bar{u}\mapsto (\Lambda_{f, n}\ni \alpha\mapsto [u]_f(\alpha)),\] and clearly lift to an isomorphism \[A^\times\simeq \gal(K_\varpi/K).\]

\subsubsection*{The local Artin map}
The \textbf{local Artin map} is a homomorphism \[\phi_\varpi : K^\times\to \gal(K_\varpi K^\nr/K) = \gal(K^\nr/K)\times \gal(K_\varpi/K)\] defined as follows.
Let $a = u\varpi^m\in K^\times$, then 
\begin{itemize}
    \item $\phi_\varpi(a)|_{K^\nr} := \frob^m$;
    \item $\phi_\varpi(a)(\lambda) := [u^{-1}]_f(\lambda)$, $\forall \lambda\in\bigcup_n\Lambda_n$.
\end{itemize}

\begin{theorem}
    Both $K_{\varpi}$ and $K^\nr$ are independent of the choice of $\varpi$.
\end{theorem}


\subsection{The Local Kronecker-Weber theorem}

\subsection{The Case of \texorpdfstring{$\Q_p$}{}}
Let $K = \Q_p$ and $\varpi = p$. Then $f(T) := (1 + T)^p - 1\in\mathcal{F}_p$.
Note that $f$ is an endomorphism of \[\Gm(X, Y) = X + Y + XY,\] so $F_f = \Gm{}_{/\Z_p}$. Under the isomorphism
\[(\mathfrak{m}, +_{\Gm})\simeq (1 + \mathfrak{m},\ \cdot\ ),\]
the endomorphism $f : a\mapsto (1 + a)^p - 1$ is converted to the Frobenius map $a\mapsto a^p$.

\subsubsection*{The field $(\Q_p)_p$}

For each $r\ge 1$, the $p^r$-torsion part of $\Lambda_f$ is
\[\Lambda_{f, r} = \left\{\alpha\in\Q_p^\alg\left|(1 + \alpha)^{p^r} = 1\right.\right\}\simeq
\left\{\zeta\in(\Q_p^\alg)^{\times}
\left|\zeta^{p^r} = 1\right.\right\} = \mu_{p^r}.\]
The isomorphism is for $\O_K$-modules.
So choose primitive $p^r$-th roots of unity $\zeta_{p^r}$ s.t. $\zeta_{p^r}^p = \zeta_{p^{r-1}}$,
then $\varpi_r := \zeta_{p^r} - 1$ forms a sequence of compatible generators of $\Lambda_{f, r}$.
Therefore \[(\Q_p)_{p, r} = \Q_p(\varpi_r) = \Q_p(\mu_{p^r}),\]
and the ``maximal totally ramified abelian extension''\footnote{Not sure if this terminology is correct ...?} of $\Q_p$ is $(\Q_p)_p = \Q_p(\mu_{p^\infty})$.

\subsubsection*{The local Artin map \texorpdfstring{$\phi_p : \Q_p^\times\to \gal(\Q_p^\ab/\Q_p)$}{}}

It suffices to look at every \[\phi_p : \Q_p^\times\to \gal(\Q_p(\mu_n)/\Q_p).\]
\begin{itemize}
    \item If $n$ is prime to $p$, then $\Q_p(\mu_n)/\Q_p$ is unramified of degree $f$, where $f$ is the minimum natural number s.t. $m\mid p^f - 1$.
    The map $\phi_p$ sends $up^t$ to the $t$-th power of Frobenius-$p^f$ on $\Q_p(\mu_n) = \Q_p(\mu_{p^f - 1})$, and $\ker\phi_p = (p^{f})^{\Z}\times\Z_p^\times$.
    \item If $n = p^r$, then $\Q_p(\mu_{p^r})/\Q_p$ is totally ramified. The map $\phi_p$ sends $up^t$ to the element sending a root of unity $\zeta$ to $\zeta^{\bar u^{-1}}$, where $\bar u\in\Z$ has the same residue modulo $p^r$ as $u$.
    The kernel is $p^\Z\times (1 + p^r\Z_p)$.
    \item In general, let $n = p^r\cdot m$ with $p\nmid m$. Then $\Q_p(\mu_n) = \Q_p(\mu_{p^r})\Q_p(\mu_m)$, and $\Q_p(\mu_{p^r})\cap\Q_p(\mu_m) = \Q_p$.
\end{itemize}




\input{cycl_ext_Qp.tex}
% \section{Group Cohomology}
In this section we fix a commutative ring $\K$.
\subsection{Cohomology}
Let $G$ be a group. A \textbf{$G$-module} with coefficients in $\K$ is a $\K$-module together with a $\K$-linear \textit{left} $G$-action. Hence the category of $G$-modules with coefficients in $\K$ is isomorphic to the category of $\K[G]$-modules.
\begin{remark}
    In particular, a $G$-module with coefficients in $\Z$ is an abelian group with additive left $G$-action. 
\end{remark}

\begin{example}
We list some important constructions of $G$-modules here.
\begin{enumerate}[(a)]
    \item The \textbf{trivial $G$-module} is $\K$ with the trivial $G$-action.
    \item The group ring $\K[G]$ is a $G$-module with $G$ acting by left-multiplication.
    \item Direct sum and product. Both direct sums and products for $G$-modules as $\K$-modules can be lifted to $G$-modules, by giving $G$-action diagonally, i.e, \[g((m_i)_i) := \left( (gm_i)_i \right).\]
    \item Tensor products. For $M, N\in \Mod_G$, define $M\otimes N\in \Mod_G$ to be $M\otimes_\K N$ with the diagonal $G$-action \[g(x\otimes y) := gx\otimes gy,\quad x\in M, y\in N.\]
    \item Hom module. For $M, N\in\Mod_G$, define $\hom(M, N)\in \Mod_G$ to be $\hom_{\K}(M, N)$ with $G$ acting ``by conjugation'': \[(gf)(x) := g f(g^{-1}x),\quad f\in\Hom_\K(M, N), x\in M.\]
    \begin{itemize}
        \item We have \[\Hom_G(M, N) = \Hom(M, N)^G\] as $G$-modules.
        \item The adjoint $L\otimes_\K (-) \dashv \Hom_\K(L, -)$ in $\Mod_\K$ holds in $\Mod_G$, i.e,
\[\begin{tikzcd}
	{\Hom(L\otimes M,N)} & {\Hom(L, \Hom(M, N))} \\
	\varphi & {x\mapsto y\mapsto\varphi(x\otimes y)} \\
	{\left( x\otimes y\mapsto \psi(x)(y) \right)} & \psi
	\arrow["\sim", leftrightarrow, from=1-1, to=1-2]
	\arrow[maps to, from=2-1, to=2-2]
	\arrow[maps to, from=3-2, to=3-1]
\end{tikzcd}\]
    are isomorphisms of $G$-modules.
    \begin{remark}
        The $K$-modules $M\otimes_\K N$ and $\hom_\K(M, N)$ with their $G$-module structures are \textit{NOT} the tensor product or $\Hom$-set in $\K[G]$-module.
    \end{remark}
    \end{itemize}
    \item Induced module. Let $H < G$ be a subgroup, $N$ a $H$-module. Then $\Ind_H^G N$ is the $K$-module of $H$-invariant functions $G\to N$, i.e., \[\Ind_H^G N := \{\varphi : G \to N\mid \varphi(hg) = h\varphi(g),\;\forall h\in H, g\in G\}\simeq\Hom_H(\K[G], N).\]
    The group $G$ acts on $\Ind_H^G N$ from the left by \[(g\varphi)(x) := \varphi(xg).\]
    We obtain a functor $\Ind_H^G : \Mod_H\to\Mod_G$ by sending $\alpha : N\to N'$ to \[\alpha_* : \Ind_H^G N\to\Ind_H^G N' := \varphi\mapsto\alpha\circ\varphi.\]
    \begin{itemize}
        \item $\Ind_H^G$ is \textit{right adjoint to the forgetful functor} $\Mod_G\to\Mod_H$. The isomorphism is given by
        \[\begin{tikzcd}
            {\Hom_G\left(M, \Ind_H^GN\right)} & {\Hom_H(M, N)} \\
            \alpha & {x\mapsto \alpha(x)(1_G)} \\
            {\left[ x\mapsto (g\mapsto \beta(gx) \right]} & \beta
            \arrow["\sim", leftrightarrow, from=1-1, to=1-2]
            \arrow[maps to, from=2-1, to=2-2]
            \arrow[maps to, from=3-2, to=3-1]
        \end{tikzcd}\]
        where $M\in \Mod_G$, $N\in \Mod_H$.
        \item $\Ind_H^G$ is an exact funtor.
        \item For any $\K$-module $M$, we define \[\Ind^G M := \Ind_{\{1\}}^G M =\{\varphi : G\to M\}.\] An \textbf{induced module} is a $G$-module of the form $\Ind^GM$ for some $\K$-module $M$.
        \item Let $M$ be a $G$-module. Define $M_* := \Ind^G M$, then we have an embedding \[M\hookrightarrow M_* := x \mapsto [g\mapsto gx]\]of $G$-modules. The exact sequence \begin{equation}
            0\to M\to M_*\to M_\dagger \to 0
        \end{equation} in $\Mod_G$, where $M_\dagger := M_*/M$, will be used many times in ``dimensional shifting''.
    \end{itemize}
\end{enumerate}
\end{example}

Let $M$ be a $G$-module, $r\ge 0$ a natural number.
We define the \textbf{ $r$-th cohomology groups of $G$ with coefficients in $M$} to be the value of the $r$-th right derived functor of the left-exact functor \[(-)^G\simeq \Hom_G(\K, -) : \Mod_G\to\Mod_K\] at $M$.
But for this definition to make sense, we need to show that:
\begin{lemma}\label{Mod G has enough injectives}
    The category $\Mod_G$ has enough injectives.
\end{lemma}
\begin{proof}
    The category $\Abel$ has enough injectives. Let $M \in\Mod_G$, $I\in\Abel$ injective with $M\hookrightarrow I$. Applying the exact functor $\Ind^G$ gives \[M\hookrightarrow M_* := \Ind^G M \hookrightarrow\Ind^G I. \] So it remains to show that
    \begin{itemize}
        \item the functor $\Ind^G$ preserves injectives,
    \end{itemize}
    which follows from $\Hom_G(-, \Ind^G I)\simeq\Hom_\Z(-, I)$.
\end{proof}


\begin{proposition}[Shapiro's lemma]\label{Shapiro's lemma}
Let $H < G$ be a subgroup.
The isomorphism\[(-)^H \simeq \Hom_H(\Z, -) \simeq \Hom_G\left(\Z, \Ind^G_H(-)\right)\simeq \left( \Ind_H^G(-) \right)^G\]induces a canonical isomorphism \[H^\bullet \left(G, \Ind^G_H (-)\right)\simeq H^\bullet (H, -),\]
which is compatible with the long exact sequence.
\end{proposition}
\begin{proof}
    
\end{proof}

\begin{corollary}\label{induced modules have trivial cohomology}
    If $M$ is an induced $G$-module, then $H^r(G, M) = 0$ for all $r\ge 1$.\qed
\end{corollary}


\subsection{Compute Cohomology via cochains}

Homological algebra tells us that \[H^r(G, M) = R^r\Hom_G(\Z, -)(M) = \ext^r(\Z, M) = R^r\Hom_G(-, M)(\Z),\]
so we can use the projective resolution of $\Z\in\Mod_G$ to compute $H^\bullet(G, M)$.

Denote by $P_r$ the free $\Z$-module with basis $G^{r+1} = G\times\dots\times G$ and endow $P_r$ with the $G$-action
\[g(g_0, g_1,\dots, g_r) := (gg_0, gg_1, \dots, gg_r).\]
Define $d_r : P_r\to P_{r-1}$ by \[d_r(g_0, \dots, g_r) := \sum_{i=0}^r(-1)^i(g_0,\dots, \hat{g}_i, \dots, g_r).\]
Then \[\dots\to P_1\stackrel{d_1}{\to} P_0\stackrel{d_0}{\to} \Z\] is exact, i.e., a projective resolution of $\Z$.

Note that $\varphi\in\Hom_G(P_r, M)$ is equivalent to a function $\varphi : G^{r+1}\to M$ s.t. \[\varphi(gg_0, \dots, gg_r) = g\varphi(g_0, \dots, g_r),\]
which is thus determined by its value on the set $\{(1, g_1, \dots, g_r) : g_i\in G\}$. Therefore we consider the abelian group\footnote{The group structure on $C^r(G, M)$ is point-wise addition.} $C^r(G^r, M) := \{\varphi : G \to M\}$. Note that $G^0 = 1$ and thus $C^0(G, M) = M$.
Define a homomorphism \[d^r : C^r(G, M) \to C^{r+1}(G, M)\] by $(d^r\varphi)(g_1, \dots, g_{r+1})$\begin{align}
    := g_1\varphi(g_2, \dots, g_{r+1}) + \sum_{j=1}^r(-1)^j\varphi(g_1, \dots, \hat{g}_j, \dots, g_r) + (-1)^{r+1}\varphi(g_1, \dots, g_r).
\end{align}
Let \[Z^r(G, M) := \ker d^r,\ B^r(G, M) := \im d^{r-1}.\]
One can prove that $d^r\circ d^{r-1} = 0$, and \[H^r(G, M) = Z^r(G, M)/B^r(G, M).\]


\subsection{The Inflation-Restriction Exact Sequence}



\subsection{Homology}

For $M\in \Mod_G$, define its \textbf{coinvariant} to be the quotient
\[M_G := M\big/ \gene{gm - m\mid g\in G, m\in M} = M/(G - \Id)M\in \Abel.\]

\begin{lemma}
    The assignment $M\mapsto M_G$ defines a right-exact functor \[(-)_G\simeq \Z\otimes_{\Z[G]}(-) : \Mod_{\Z[G]}\to\Abel\]
\end{lemma}
\begin{proof}
Consider the augmentation map $\Z[G]\to\Z$ which is an additive homomorphism sending all $g\in G$ to $1\in\Z$.
Its kernel $I_G$ is called the \textbf{augmentation ideal}.
Note that:\begin{itemize}
    \item $I_G\subset\Z[G]$ is the free abelian subgroup with basis $\{g - 1 \mid g\in G, g\ne 1\}$.
\end{itemize}
Therefore \[M_G = M/I_GM\simeq \Z[G]/I_G\otimes_{\Z[G]} M\simeq \Z\otimes_{\Z[G]} M.\qedhere\]
\end{proof}
We define the \textbf{$r$-th homology groups of $G$ with coefficients in $M\in\Mod_G$} to be the value of the $r$-th left derived functor of the right-exact functor $(-)_G$.

\subsection{The Tate cohomology groups}

In this subsection, let $G$ be a \textit{finite} group.

Recall that the norm $N_G : M \to M$ for a $G$-module $M$ is defined by \[N_G(x) := \sum_{g\in G}gx,\quad x\in G.\]
One observes that \[\im N_G\subset M^G,\quad I_GM\subset \ker N_G.\]
Therefore $N_G$ factors as \[M\twoheadrightarrow M/I_GM = M_G\to M^G\hookrightarrow M,\] and we got an exact sequence \[0\to\ker N_G/I_GM\to M_G\to M^G\to M^G/\im N_G\to 0.\]

The map $H_0(G, M)\to H^0(G, M)$ induced by the norm map on $M$ connects homologies and cohomologies. We define the \textbf{Tate cohomology groups} by \[\hat H^r (G, M) := \begin{cases}
    H^r(G, M), &r\ge 1,\\
    M^G/N_G(M), &r = 0,\\
    \ker (N_G : M\to M)/I_GM, & r = -1,\\
    H_{-r-1} (G, M), & r\le -2.
\end{cases}\]

\begin{proposition}
    If $M$ is induced, then $\hat H^\bullet(G, -) = 0$.
\end{proposition}

(connecting $H^r$ to $H^{r + 2}$.)

\begin{example}\label{eg: H1 to H-1 for finite cyclic group}
    Let $G$ be a finite cyclic group generated by $\sigma$.
    Then \[I_G = \gene{\sigma^n m - m\mid m\in M, n\in\Z } = \gene{\sigma m - m\mid m\in M},\]
    \[\hat H^{-1}(G, M) = \ker(N_G)/(\sigma - 1)M.\]
    In this case, choosing a generator $\sigma$ of $G$ defines an explicit isomorphism
    \begin{align*}
        \hat H^1(G, M) &\to \hat H^{-1}(G, M)\\
        \varphi&\mapsto \varphi(\sigma).
    \end{align*}
    Indeed, crossed homomorphisms $G\to M$ are defined by their value on generators of $G$, and for $\varphi : G\to M$ a crossed homomorphism, \[
    \varphi(\sigma^n) = \sigma^{n-1}\varphi(\sigma) + \sigma^{n-2}\varphi(\sigma) + \cdots + \sigma\varphi(\sigma) + \varphi(\sigma),\ \forall \sigma\in G.\]
    Therefore, if $G\simeq \Z/n\Z$ is generated by $\sigma$ of order $n$,
    then \[\varphi\text{ is a crossed homomorphism }\iff x := \varphi(\sigma) \text{ verifies } N_Gx = \sum_{g\in G} gx = x + \sigma x + \cdots + \sigma^{n-1}x =0.\]
    \[\varphi \text{ is principal }\iff \varphi(\sigma) \in \left( \sigma - 1 \right)M.\]
    As $Z^1(G, M)\to M,\ \varphi\to \varphi(\sigma)$ is a group homomorphism, we get the isomorphism.
\end{example}


\subsection{Cohomology of \texorpdfstring{$L$ and $L^\times$}{PDFstring}}
In this subsection, let $L/K$ be a \textit{finite} Galois extension, $G := \gal(L/K)$. Then both $L$ and $L^\times$ have natural $G$-module structures.

\subsubsection{Hilbert's Theorem 90 and \texorpdfstring{$H^1(G, L^\times)$}{H1(G, L cross)}}

\begin{theorem}[Dedekind-Artin]\label{dedekind theorem on linearly independence of characters}
    Let $\Gamma$ be a monoid, $E$ be a integral domain, and $\Hom_{\times}(\Gamma, E)$ the set of monoid homomorphisms $\Gamma\to E$.
    \footnote{
        The set $\Hom_{\times}(\Gamma, E)$ admits a $E$-module structure defined point-wisely.
        The elements in $\Hom_{\times}(\Gamma, E)$ are sometimes called characters.
    }
    Then $\Hom_{\times}(\Gamma, E)$ is a linearly independent set over $E$; i.e, for $a_\chi\in E$,
    \[\sum_{\chi\in\Hom_{\times}(\Gamma, E)} a_\chi\chi(\cdot) = 0\text{ on }E
    \implies a_\chi = 0,\forall\chi.\]
\end{theorem}
\begin{proof}
    Suppose that $J := \{\chi\in\Hom_{\times}(\Gamma, E)\mid a_\chi\ne 0\}\ne\varnothing$.
    The idea is to {\color{blue} take $(a_\chi)_\chi$ s.t.
    $J = J((a_\chi)_\chi)$ is nonempty but minimal}.

    Since $\chi(1) = 1\ne 0\in E$, we have $\# J > 1$.
    Let $\xi, \eta$ be two different characters $\Gamma\to E$. Then $\exists g\in\Gamma$ s.t. $\xi(g)\ne \eta(g)$.
    Note that \[\sum_{\chi\in J} a_\chi \chi(g)\chi(\cdot) = \sum_{\chi\in J} a_\chi\chi(g\,\cdot) = 0,\]
    \[\sum_{\chi\in J}a_\chi\xi(g)\chi(\cdot) = \xi(g)\sum_{\chi\in J}a_\chi\chi(\cdot) = 0,\]
    and subtracting these two identities yields
    \[\sum_{\chi\in J\sminus\{\xi\}} a_\chi(\chi(g) - \xi(g))\chi(\cdot) = 0.\]
    This new identity is nontrivial sicne $\eta(g) - \chi(g)\ne 0$, but concerns strictly lesser characters than $J$. Contradiction.
\end{proof}

\begin{proposition}\label{Hilbert 90 - multiplicative - cohomology}
    $H^1(\gal(L/K), L^\times) = 0$.\par
    In other words, if $\varphi : G\to L^\times$ is a crossed homomorphism, i.e., \[\varphi(gh) = g\varphi(h)\varphi(g),\ \forall g, h\in G,\]
    then $\exists b_\varphi\in L^\times$ s.t. \[\varphi(g) = \frac{g b_\varphi}{b_\varphi},\ \forall g\in G.\]
\end{proposition}
\begin{proof}
    Take $a\in L^\times$ and define \[b := \sum_{g\in G}\varphi(g)\cdot ga\in L.\]
    Then \begin{align*}
        hb = \sum_{g\in G} h\varphi(g)\cdot hga
        = \sum_{g\in G}\frac{\varphi(hg)}{\varphi(h)} hga = \frac{b}{\varphi(h)}.
    \end{align*}
    Hence if $b\ne 0$, we would have $\varphi(g) = b/gb = g(b^{-1})/b^{-1}$.
    By \cref{dedekind theorem on linearly independence of characters},
    $\gal(L/K)\subset \Hom_\times(L, L)$ is linearly independent over $L$,
    so $\sum_{g\in G}\varphi(g)g(\cdot) : L\to L$ is a non-zero function, and thus can we find $a\in L$ with $b\ne 0$.
\end{proof}

\begin{corollary}[Hilbert 90]\label{Hilbert 90 - multiplicative}
    Let $L/K$ be a finite cyclic extension, $\sigma$ a generator of $G = \gal(L/K)$, and $a\in L$.
    If $N_{L/K}a = 1$, then $\exists b\in L^\times$ s.t. $a = \sigma b/b$.
\end{corollary}
\begin{proof}
    For the $G$-module $L^\times$, the norm map \[N_G = N_{L/K} : x\mapsto\prod_{g\in G}  gx.\]
    So \[\dfrac{\ker(N_{L/K})}{(\sigma(\cdot) / \Id(\cdot))L^\times} = \hat H^{-1}(G, L^\times) \simeq H^1(G, L^\times) = 0.\qedhere\]
\end{proof}

\subsubsection{Normal Basis and \texorpdfstring{$H^r(G, L)$}{Hr(G, L)}}

\begin{theorem}[Normal basis theorem]\label{normal basis theorem for fintie galois}
    Any finite Galois extension $L/K$ admits a normal basis; i.e, $\exists x\in L$ s.t. $\{\sigma x\mid \sigma\in\gal(L/K)\}$ forms a $K$-basis of $L$.
\end{theorem}
We prove this in two cases: 1) $K$ is infinite
and 2) $L/K$ is finite cyclic.
\begin{proof}[Proof in case $K$ infinite]
    (T.B.C.)
\end{proof}

\begin{proof}[Proof in case $G$ cyclic]
    (T.B.C.)
\end{proof}

\begin{proposition}
    $L$ is an induced $G = \gal(L/K)$-module, hence
    $H^r(G, L) = 0$ for all $r \ge 1$.
\end{proposition}
\begin{proof}
    By \cref{normal basis theorem for fintie galois},
    we choose $x\in L$ with $L = \bigoplus_{g\in G} Kgx$, giving an isomorphism \[K[G]\to L,\quad \sum_{g\in G}a_gg\to\sum_{g\in G}a_ggx\]
    as $G$-modules. Hence as a $G$-module,
    $L\simeq K[G]\simeq K\otimes_\Z \Z[G]\simeq\Ind^G(K)$.
\end{proof}

\begin{corollary}\label{Hilbert 90 - additive}
    Let $L/K$ be a finite cyclic extension, $\sigma$ a generator of $G$, and $a\in L$.
    If $\tr_{L/K}a = 0$,
    then $\exists b\in L$ s.t. $a = \sigma b - b$.
\end{corollary}
\begin{proof}
    For the $G$-module $L$, the norm map \[N_G = \tr_{L/K} : x\mapsto \sum_{g\in G}gx.\]
    Now use $H^{1}(G, L) \simeq \hat H^{-1}(G, L)$.
\end{proof}

\subsubsection{Kummer Theory}





\end{document}