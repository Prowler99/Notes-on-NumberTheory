\documentclass{article}
\usepackage{amsmath, amssymb, amsthm, amsbsy, mathrsfs}
\usepackage{enumitem}
\usepackage[capitalize]{cleveref}
\usepackage[margin = 1in, headheight = 12pt]{geometry}
\usepackage{bbm}
\usepackage{tikz-cd}

\newtheorem{theorem}{Theorem}

\theoremstyle{definition}
\newtheorem{definition}{Definition}
\newtheorem{exercise}{Exercise}[section]
\newtheorem{problem}{Problem}
\newtheorem{example}{Example}
\newtheorem{proposition}{Proposition}[section]
\newtheorem{lemma}{Lemma}[section]
\newtheorem{corollary}{Corollary}[section]

\theoremstyle{remark}
\newtheorem*{remark}{Remark}

\DeclareMathOperator{\gal}{Gal}

\renewcommand{\Re}{\mathop{\mathrm{Re}}}
\renewcommand{\Im}{\mathop{\mathrm{Im}}}
\renewcommand{\bar}{\overline}
\renewcommand{\tilde}{\widetilde}

% 新命令
% 数学对象
    % 基础空间
    \newcommand{\R}{\mathbb{R}}
    \newcommand{\C}{\mathbb{C}}
    \newcommand{\Q}{\mathbb{Q}}
    \newcommand{\Z}{\mathbb{Z}}
    % 常量
    % \newcommand{\e}{\mathrm{e}} %自然底数
    % 群
    \DeclareMathOperator{\GL}{GL}
    \DeclareMathOperator{\SL}{SL}
% 算符&映射&函子
    % 集合
    \newcommand{\sminus}{\smallsetminus} %(集合)差
    % 范畴
    \newcommand{\op}[1]{{#1}^{\mathrm{op}}} %反范畴
    \DeclareMathOperator{\enom}{End} %自态射
    \DeclareMathOperator{\isom}{Isom} %同构
    \DeclareMathOperator{\aut}{Aut} %自同构
    \DeclareMathOperator{\im}{im} %像
    %向量空间, 矩阵
    \DeclareMathOperator{\rank}{rank} %秩
    \DeclareMathOperator{\tr}{tr} %迹
    \newcommand{\tran}[1]{{#1}^{\mathrm{T}}} %转置
    \newcommand{\ctran}[1]{{#1}^{\dagger}} %共轭转置
    \newcommand{\itran}[1]{{#1}^{-\mathrm{T}}} %逆转置
    \newcommand{\ictran}[1]{{#1}^{-\dagger}} %逆共轭转置
    \DeclareMathOperator{\codim}{codim} %余维数
    \DeclareMathOperator{\diag}{diag} %对角阵
    \newcommand{\norm}[1]{\left\| #1\right\|} %范数
    \DeclareMathOperator{\spec}{Spec} %谱
    \DeclareMathOperator{\lspan}{span} %张成
    \DeclareMathOperator{\sym}{\mathfrak{Y}}
    % 群
    \DeclareMathOperator{\inn}{Inn} %(群)内自同构
    \newcommand{\nsg}{\vartriangleleft} %正规子群
    \newcommand{\gsn}{\vartriangleright} %正规子群
    \DeclareMathOperator{\ord}{ord} %元素的阶
    \DeclareMathOperator{\stab}{Stab} %稳定化子
    \DeclareMathOperator{\sgn}{sgn} %符号函数
    % 环, 域
    \DeclareMathOperator{\cha}{char} %特征
    % \DeclareMathOperator{\spec}{Spec} %素谱
    \DeclareMathOperator{\maxspec}{MaxSpec} %极大谱
    % 微积分
    % \newcommand*{\dif}{\mathop{}\!\mathrm{d}} %(外)微分算子
    % 流形
    \DeclareMathOperator{\lie}{Lie}
% 结构简写
    \newcommand{\pdfrac}[2]{\dfrac{\partial #1}{\partial #2}} %偏微分式
    \newcommand{\isomto}{\stackrel{\sim}{\rightarrow}} %有向同构
    \newcommand{\gene}[1]{\left\langle #1 \right\rangle} %生成对象
% 文字缩写
    \newcommand{\opin}{\;\mathrm{open\;in}\;}
    \newcommand{\st}{\;\mathrm{s.t.}\;}
    \newcommand{\ie}{\;\mathrm{i.e.,}\;}

% 重定义命令
\renewcommand{\hom}{\mathop{Hom}}
\renewcommand{\vec}{\boldsymbol}
\renewcommand{\and}{\;\text{and}\;}

% 编号
\newcommand{\cnum}[1]{$#1^\circ$} %右上角带圆圈的编号
\newcommand{\rmnum}[1]{\romannumeral #1}

\title{Cyclotomic Extensions of $\Q_p$}
\author{}
\date{}

\begin{document}
\maketitle

Let $p$ be a prime number.

\section{Galois theory}
Let $L/K$ be an algebraic extension. It is called: \begin{enumerate}
    \item [$\diamond$]\textbf{normal}, if every polynomial $f\in K[T]$ with a root in $L$ split in $L$ $\iff$ $L$ is the splitting field of a bunch of polynomials over $K$;
    \item [$\diamond$]\textbf{separable}, if for every element in $L$, its minimal polynomial over $K$ has no multiple roots in its splitting field;
    \item [$\diamond$]\textbf{Galois}, if it is normal and separable, whence we put $\gal(L/K) := \aut_K(L)$.
\end{enumerate}
For a finite normal extension $L/K$, $|\aut_K(L)| \le [L:K]$, where the equality holds $\iff L/K$ is separable, i.e. Galois. This is because a $K$-automorphism of $L = K[T]/(f)$ just maps a root of $f$ to another.

Let $L/K$ be a finite Galois extension. The basic results of Galois theory says that the intermediate fields of $L/K$ corresponds to the subgroups of $\gal(L/K)$ bijectively and $\gal(L/K)$-equivariantly, where Galois extensions corresponds to normal subgroups.
They are descibed as below.
\begin{enumerate}
    \item For an intermediate field $F$, it gives $\gal(L|F)\subset \gal(L/K)$. Note that $L/F$ is Glaois, but $F/K$ is NOT always Galois.
    The Galois group acts on $\{\text{intermediate field of } L/K\}$ by $(\sigma, F) \mapsto \sigma F = \sigma(F)$.
    \item For a subgroup $H < G$, it fix a subfield $L^H \subset L$. The Galois group act on $\{H : H < \gal(L/K)\}$ be conjugation, i.e., $(\sigma, H) \mapsto \sigma H\sigma^{-1}$.
\end{enumerate}




\end{document}