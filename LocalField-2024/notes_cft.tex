\documentclass{article}
\usepackage{fontspec}
\usepackage{amsmath, amssymb, amsthm, amsbsy, mathrsfs}
\usepackage{stmaryrd}
\usepackage{enumerate}
\usepackage[colorlinks,
linkcolor=cyan,
anchorcolor=blue,
citecolor=blue,
]{hyperref}
\usepackage[capitalize]{cleveref}
\usepackage[margin = 1in, headheight = 12pt]{geometry}
\usepackage{bbm}
\usepackage{tikz-cd}

% \setmainfont{Arial}

\linespread{1.2}


\theoremstyle{definition}
\newtheorem{theorem}{Theorem}
\newtheorem{definition}{Definition}
\newtheorem{exercise}{Exercise}[section]
\newtheorem{problem}{Problem}
\newtheorem{example}{Example}
\newtheorem{proposition}{Proposition}[section]
\newtheorem{lemma}{Lemma}[section]
\newtheorem{corollary}{Corollary}[section]

\theoremstyle{remark}
\newtheorem*{remark}{Remark}

\renewcommand{\Re}{\mathop{\mathrm{Re}}}
\renewcommand{\Im}{\mathop{\mathrm{Im}}}
\renewcommand{\bar}{\overline}
\renewcommand{\tilde}{\widetilde}
\renewcommand{\hat}{\widehat}

% 新命令
% 数学对象
    \newcommand{\R}{\mathbb{R}}
    \newcommand{\C}{\mathbb{C}}
    \newcommand{\Q}{\mathbb{Q}}
    \newcommand{\Z}{\mathbb{Z}}
    \DeclareMathOperator{\GL}{GL}
    \DeclareMathOperator{\SL}{SL}
    \newcommand{\p}{\mathfrak{p}}
    \renewcommand{\P}{\mathbb{P}}
    \newcommand{\A}{\mathbb{A}}
    \newcommand{\Ga}{{\mathbb{G}_{\mathrm{a}}}}
    \newcommand{\Gm}{{\mathbb{G}_{\mathrm{m}}}}
% 集合
    \newcommand{\sminus}{\smallsetminus} %(集合)差
% 范畴
    \newcommand{\op}[1]{{#1}^{\mathrm{op}}} %反范畴
    \DeclareMathOperator{\enom}{End} %自态射
    \DeclareMathOperator{\End}{End} %自态射
    \DeclareMathOperator{\isom}{Isom} %同构
    \DeclareMathOperator{\aut}{Aut} %自同构
    \DeclareMathOperator{\im}{im} %像
    \newcommand{\Set}{\mathbf{Set}} %集合范畴
    \newcommand{\Abel}{\mathbf{Ab}} %群范畴
    \newcommand{\Ring}{\mathbf{Ring}}
    \newcommand{\Cring}{\mathbf{CRing}}
    \newcommand{\Alg}{\mathbf{Alg}}
    \newcommand{\Mod}{\mathbf{Mod}}
    \DeclareMathOperator{\Id}{id}
%向量空间, 矩阵
    \DeclareMathOperator{\rank}{rank} %秩
    \DeclareMathOperator{\tr}{Tr} %迹
    \newcommand{\tran}[1]{{#1}^{\mathrm{T}}} %转置
    \newcommand{\ctran}[1]{{#1}^{\dagger}} %共轭转置
    \newcommand{\itran}[1]{{#1}^{-\mathrm{T}}} %逆转置
    \newcommand{\ictran}[1]{{#1}^{-\dagger}} %逆共轭转置
    \DeclareMathOperator{\codim}{codim} %余维数
    \DeclareMathOperator{\diag}{diag} %对角阵
    \newcommand{\norm}[1]{\left\| #1\right\|} %范数
    \DeclareMathOperator{\lspan}{span} %张成
    \DeclareMathOperator{\sym}{\mathfrak{Y}}
% 群
    \DeclareMathOperator{\inn}{Inn} %(群)内自同构
    \newcommand{\nsg}{\vartriangleleft} %正规子群
    \newcommand{\gsn}{\vartriangleright} %正规子群
    \DeclareMathOperator{\ord}{ord} %元素的阶
    \DeclareMathOperator{\stab}{Stab} %稳定化子
    \DeclareMathOperator{\sgn}{sgn} %符号函数
% 环, 域
    \DeclareMathOperator{\cha}{char} %特征
    \DeclareMathOperator{\spec}{Spec} %素谱
    \DeclareMathOperator{\maxspec}{MaxSpec} %极大谱
    \newcommand{\spm}{\maxspec}
    \DeclareMathOperator{\gal}{Gal}
    \DeclareMathOperator{\Frac}{Frac}
% 同调代数
    \DeclareMathOperator{\ext}{Ext}
% 微积分
    % \newcommand*{\dif}{\mathop{}\!\mathrm{d}} %(外)微分算子
% 流形
    \DeclareMathOperator{\lie}{Lie}
%代数几何
    \DeclareMathOperator{\proj}{Proj}
%多项式
    \DeclareMathOperator{\disc}{disc} %判别式
    \DeclareMathOperator{\res}{res} %结式
% 结构简写
    \newcommand{\pdfrac}[2]{\dfrac{\partial #1}{\partial #2}} %偏微分式
    \newcommand{\isomto}{\stackrel{\sim}{\rightarrow}} %有向同构
    \newcommand{\gene}[1]{\left\langle #1 \right\rangle} %生成对象
% 文字缩写
    \newcommand{\opin}{\;\mathrm{open\;in}\;}
    \newcommand{\st}{\;\mathrm{s.t.}\;}
    \newcommand{\ie}{\;\mathrm{i.e.,}\;}

% 重定义命令
\renewcommand{\hom}{\mathop{\mathrm{Hom}}}
\DeclareMathOperator{\Hom}{Hom}
\renewcommand{\vec}{\boldsymbol}
\renewcommand{\and}{\;\text{and}\;}

% 编号
\newcommand{\cnum}[1]{$#1^\circ$} %右上角带圆圈的编号
\newcommand{\rmnum}[1]{\romannumeral #1}

\newcommand{\fring}[1]{\llbracket #1 \rrbracket}

\newcommand{\myit}{$\diamond$}
\renewcommand{\O}{\mathcal{O}}
\newcommand{\nr}{\mathrm{nr}}
\newcommand{\alg}{\mathrm{alg}}
\newcommand{\ab}{\mathrm{ab}}
\DeclareMathOperator{\frob}{Frob}
\newcommand{\F}{\mathbb{F}}
\DeclareMathOperator{\Ind}{Ind}
\newcommand{\m}{\mathfrak{m}}
\DeclareMathOperator{\wideg}{wideg}
\DeclareMathOperator{\val}{val}
\DeclareMathOperator{\height}{ht}
\newcommand{\cyc}{\mathrm{cycl}}
\DeclareMathOperator{\art}{Art}
% \tikzcdset{scale cd/.style={every label/.append style={scale=#1},
%     cells={nodes={scale=#1}}}}

\title{Notes on Local Fields}
\author{L}

\begin{document}
\maketitle

\section{Review: Galois theory}
\subsection{Field Extensions}
Let $L/K$ be an algebraic extension. It is called: \begin{enumerate}
    \item [$\diamond$]\textbf{normal}, if every polynomial $f\in K[T]$ with a root in $L$ splits in $L$, $\iff$ $L$ is the splitting field of a bunch of polynomials over $K$;
    \item [$\diamond$]\textbf{separable}, if for every element in $L$, its minimal polynomial over $K$ has no multiple roots in its splitting field, $\iff$ $\gcd(f, f') = 1$;
    \item [$\diamond$]\textbf{Galois}, if it is normal and separable, i.e., $L$ is the splitting field of a bunch of \textit{seperable} polynomials over $K$. We put $\gal(L/K) := \aut_K(L)$.
\end{enumerate}
\begin{remark} {}
\begin{enumerate}
    \item For a finite \textit{normal} extension $L/K$, $|\aut_K(L)| \le [L:K]$, where the equality holds $\iff L/K$ is separable, i.e. Galois. This is because a $K$-automorphism of $L = K[T]/(f)$ just permutes the roots of $f$.
    \item Normality is NOT transitive. As an example, take $\Q\subset\Q(\sqrt{2})\subset\Q(\sqrt[4]{2})$.
\end{enumerate} 
\end{remark}

% We introduce a convenient notion here.
% \begin{definition}
%     Let $\Omega/F$ be a field extension and $\mathcal{C}$ a family of subextensions in $\Omega/F$. We say $\mathcal{C}$ is \textbf{distinguished}, if it satisfies:\begin{enumerate}
%         \item [\textbf{D1}] $\forall L/E/F$, \[L/F\in\mathcal{C}\iff L/E\in\mathcal{C}\ \&\ E/F\in\mathcal{C};\]
%         \item [\textbf{D2}] $\forall L, M$, \[L/F\in\mathcal{C}\implies LM/M\in\mathcal{C}.\]
%     \end{enumerate}    
% \end{definition}
% \begin{remark}
%     Let $\mathcal{C}$ be a distinguished family of subextensions.
% \begin{enumerate}
%     \item The conditions implies that $\mathcal{C}$ is closed under \textit{finite} composition.
%     \item The \textit{union} of all fields in $\mathcal{C}$ is a field, and thus equal to the composition of all fields in $\mathcal{C}$.
%     \item Finite, algebraic, seperable and purely inseperable extensions are distinguished.
% \end{enumerate}
% \end{remark}


\subsection{Galois theory}
Now let $L/K$ be a Galois extension. Equip $\gal(L/K)$ with the following \textbf{Krull topology}: $\forall\sigma\in\gal(L/K)$, a basis of nbhd around $\sigma$ is given by\[\sigma\gal(L/F),\quad\text{where } L/F/K,\; F/K < \infty\text{ \& Galois}.\]
\begin{itemize}
    \item Two elements $\sigma, \tau\in\gal(L/K)$ are ``close'' to each other, if $\sigma|_F = \tau|_F$ for sufficiently large finite Galois subextensions $F/K$.
    \item Both multiplication and inverse on $\gal(L/K)$ are continuous for Krull topology.
    \item The Krull topology is profinite for $L/K$ infinite, whence \[\gal(L/K) \simeq \lim_{\stackrel{\longleftarrow}{F/K < \infty\text{ \& Galois}}}\gal(F/K). \]
    When $L/K < \infty$, this is the discrete topology.
    \item If there is a tower \[K\subset L_1\subset L_2\subset\dots\subset L,\] where all $L_n/K$'s are Galois, and \[L = \bigcup_{n} L_n,\]
    then \[\gal(L/K) = \varprojlim_n\gal(L_n/K).\]

\end{itemize}

Galois theory says that the intermediate fields of $L/K$ corresponds to the closed subgroups of $\gal(L/K)$ bijectively and $\gal(L/K)$-equivariantly.
\begin{enumerate}
    \item [$\rightarrow$:] For an intermediate field $F$, it gives $\gal(L/F)\subset \gal(L/K)$. Note that $L/F$ is Glaois, but $F/K$ is NOT always Galois.
    The Galois group acts on $\{\text{intermediate field of } L/K\}$ via $(\sigma, F) \mapsto \sigma F = \sigma(F)$.
    \item [$\leftarrow$:] For a closed subgroup $H < G$, it fixes a subfield $L^H \subset L$. The Galois group acts on $\{H : H < \gal(L/K)\}$ by conjugation, i.e., $(\sigma, H) \mapsto \sigma H\sigma^{-1}$.
\end{enumerate}
In particular,\begin{enumerate}
    \item [$\diamond$] \textit{Galois} extensions correspond to \textit{normal closed} subgroups, and
    \item [$\diamond$] \textit{finite} extensions correspond to \textit{open} subgroups.
\end{enumerate}

\subsubsection*{Base change}
\begin{proposition}\label{field extension base change}
\[\begin{tikzcd}[sep = tiny]
	&& LM \\
	&&& M \\
	L \\
	& K
	\arrow[no head, from=2-4, to=1-3]
	\arrow[no head, from=3-1, to=1-3]
	\arrow["{\text{Galois}}", no head, from=4-2, to=3-1]
	\arrow[no head, from=4-2, to=2-4]
\end{tikzcd}\]    Let $L/K$ be Galois. If $M/K$ is any extension, and both $L$ and $M$ are subextensions of $\Omega/K$, then $LM/M$ is Galois, and
    \begin{align*}
        \gal(LM/M) &\stackrel{\sim}{\longrightarrow}\gal(L/L\cap M)\\
        \sigma&\longmapsto \sigma|_L.
    \end{align*}
\end{proposition}
As a corollary, if $L, L'$ are Galois subextensions of $\Omega/K$, then $LL'/K$ is also Galois, and \begin{align*}
    \gal(LL'/K)&\hookrightarrow \gal(L/K)\times \gal(L'/K)\\
    \sigma &\mapsto (\sigma|_L, \sigma|_{L'}).
\end{align*}
This embedding is an isomorphism if $L\cap L' = K$.





\section{Extensions of Local Fields}

\subsection{Simple Extensions of DVRs}
Let $A$ be a local ring with ($\mathfrak{m}$, $k$), $f\in A[X]$ a monic polynomial of deg $n$.
We consider the extension \[A \to B_f := A[X]/f.\]

Let $\bar{f}$ be the image of $f$ in $k[X] \simeq A[X]/\mathfrak{m}$ with decomposition \[\bar{f} = \prod_{i}\bar{g}_i^{e_i},\ g_i\in A[X],\ \bar{g}_i\in k[X]\text{ irreducible.}\]
and \[\bar{B}_f := B_f/\mathfrak{m}B_f \simeq A[X]/(\mathfrak{m}, f) \simeq k[X]/(\bar{f}).\]
\begin{lemma}
    $\mathfrak{m}_i := (\mathfrak{m},\ g_i\bmod f)\subset B_f$ are all the distinct maximal ideals of $B_f$.
\end{lemma}
\begin{proof}
    Denote $\pi : B_f\to\bar{B}_f$. We have $B_f/\mathfrak{m}_i \simeq \bar{B}_f/(\bar{g}_i)$, so $\mathfrak{m}_i$'s are maximal.
    Note that $\mathfrak{m}_i = \pi^{-1}(\bar{g}_i)$.

    Take $\mathfrak{n}\in\spm B_f$.
    If $\mathfrak{n}\supset\mathfrak{m}$, then $\mathfrak{n} = \pi^{-1}\pi\mathfrak{n}$,
    and goes to a maximal ideal in $\bar{B}_f$ (because $\bar{B}_f/\pi\mathfrak{n} \simeq B_f/\mathfrak{n}$),
    so $\mathfrak{n} = \pi^{-1}(\bar{g}_i) = \mathfrak{m}_i$.

    So assume that $\mathfrak{m}\not\subset\mathfrak{n}$, then $\mathfrak{n} + \mathfrak{m}B_f = B_f$.\footnote{In this case $\mathfrak{n}/(\mathfrak{n\cap m})\simeq \bar{B}_f$ as $B_f$-module, and thus $\pi^{-1}\pi\mathfrak{n} = B_f$.}
    Therefore \[\frac{B_f}{\mathfrak{n}} = \frac{\mathfrak{n}+\mathfrak{m}B_f}{\mathfrak{n}} \simeq \frac{\mathfrak{m}B_f}{\mathfrak{n}}.\]
    Since $A$ is local and $B_f$ is a f.g. $A$-mod, by Nakayama's lemma, we see $\mathfrak{n} = B_f$. Contradiction.


\end{proof}

Now take $A$ to be a DVR with $\mathfrak{m} = (\varpi)$ and $K = \Frac A$. Put $L := K[X]/(f)$.
We give two cases where $B_f$ is a DVR.

\subsubsection*{Unramified case}
Let $\bar{f}\in k[X]$ be irreducible. Then $B_f$ is a DVR with maximal ideal $\mathfrak{m}B_f$.
\begin{corollary}\label{simple ext of dvr - unramified - is field}
    $f\in A[X]$ is also irreducible, so $L$ is a field.
    Moreover, $B_f$ is the integral closure of $A$ in $L$, and $L/K$ is unramified if $\bar{f}$ is separable.
\end{corollary}
\begin{proof}
    $L = K[X]/f \simeq \left( A[X]/f \right)\otimes_{A} K = B_f\otimes_A K$.
    As $B_f$ is a domain, $L$ is a field and $L = \Frac B_f$.
    Since $A$ is integrally closed, $B_f$ is also integrally closed, so $B_f$ is the integral closure of $A$ in $L$.
\end{proof}
\subsubsection*{Totally ramified case}
Let $f\in A[X]$ be an \textbf{Eisenstein polynomial}, i.e., \[f = X^n + a_{n-1}X^{n-1} + \cdots  + a_0,\ a_i\in\mathfrak{m},\ a_0\notin\mathfrak{m}^2.\]
\begin{proposition}
    $B_f$ is a DVR, with maximal ideal generated by the image of $X$ and residue field $k$.
\end{proposition}
\begin{proof}
    Let $x$ be the image of $X$ in $B_f$.
    We have $\bar{f} = X^n$, so $B_f$ is a local ring with maximal ideal $(\mathfrak{m}, x)$.
    Because $a_0\in\mathfrak{m\setminus m^2}$, $a_0$ must uniformise $\mathfrak{m}\subset A$, and \[-a_0\bmod f = x^n + \cdots + (a_1\bmod f)\,x,\] Therefore $(\mathfrak{m}, x) = (x)$.
\end{proof}
Similar to \cref{simple ext of dvr - unramified - is field}, $f$ is irreducible and $L$ is a field with $B_f$ the integral closure of $A$ in $L$.

\subsection{Unramified Extensions of Local Fields}
Let $K$ be a CDVF\footnote{CDVF stands for complete discrete valuation field.}. We assume further that both $K$ and its residue field $k = \mathcal{O}_K/\mathfrak{m}$ are perfect.

The slogan is that unramified extensions are just extensions of residue fields.
Using Hensel's lemma, an extension $k(a)/k$ can be lifted to a unique extension $K(\alpha)/K$ over $K$ with \[\gal(K(\alpha)/K)\simeq \gal(k(a)/k).\] Moreover, given an extension $L/K$, there is a maximal unramified subextension $K_0$ in $L$ containing every unramified extensions.

Now we assume $k$ to be finite. Then adjoining roots of unities with order coprime to $p = \cha k$ gives all finite unramified extensions of $K$.

\begin{example}
    Let $K/\Q_p < \infty$ and $k = \mathbb{F}_q$.
    Then the unique extension of $k$ of degree $n$ is the splitting field of $X^{q^n} - X$ over $k$, which equals $k(\mu_{q^n - 1})$ once we fix an algebraic closure of $k$.
    So the unramified extension $K_n/K$ of degree $n$ is the splitting field of $X^{q^n} - X$ over $K$, i.e., \[K_n = K(\mu_{q^{n} - 1}).\] The Galois group $\gal(K_n/K)$ is generated by $\frob_K$, which is determined by \[\frob_K\beta \equiv \beta^q \mod\varpi,\ \forall\beta\in\mathcal{O}_{K_n}\]for any uniformiser $\varpi$ (simultaneously of $K$ and $K_n$).

    What if we adjoin $\zeta_{m}$ to $K$ where $m$ is an arbitary integer prime to $p$?
    The answer is that $K(\mu_m)$ is unramified of degree the smallest positive integer $f$ s.t. $m \mid p^f - 1$, by the following \cref{cyclotomic extension of finite fields} on finite fields.
\end{example}

\begin{lemma}\label{cyclotomic extension of finite fields}
    Let $\zeta_n$ be a primitive $n$-th root of unity over $\F_q$ with $q, n$ coprime. Then $[\F_q(\zeta_n) : \F_q]$ is the smallest integer $f > 0$ s.t. $n \mid q^f - 1$.
\end{lemma}
\begin{proof}
    Because $\cha\F_q\nmid n$, the primitive root $\zeta_n$ exists and $\F_q(\zeta_n)$ is the splitting field of $X^n - 1$ over $\F_q$.
    The degree $f = [\F_q(\zeta_n) : \F_q]$ is the order of $\frob_q$ on $\F_q(\zeta_n)$, i.e., $f$ is the smallest integer s.t. \[\frob_q^f(\zeta_n) = \zeta_n^{q^f} = \zeta_n.\] The definition of primitive root of unity says that \[\zeta_n^{q^f -1} = 1\iff n \mid q^f - 1.\qedhere\]
\end{proof}

\subsection{Ramification Groups}

Let $K$ be a CDVF with perfect residue field $k$, $L/K<\infty$ Galois. We will study the Galois group \[G := \gal(L/K)\] by giving filtrations on it.





\section{\texorpdfstring{A Bit of $p$-adic Analysis}{}}
In this section, we consider some basic properties concerning powerseries over a closed subfield $K$ of $\C_p$ as functions.

Let $f(X) = \sum_{i\ge 0} a_iX^i\in K\llbracket X\rrbracket $. We can evaluate $f$ at $z\in \C_p$ iff $a_iz^i\to\infty$, so the \textbf{radius of convergence}
is \[\rho(f) := \sup\{\rho\in\R\ |\ a_i\rho^i\to\infty (i\to\infty)\}.\]
\begin{itemize}
    \item If $|z| < \rho(f)$, then $f(z)$ converges in $\C_p$.
    \item If $|z| > \rho(f)$, then $f$ diverges.
    \item $\rho(f(\alpha X)) = \rho(f)\cdot |\alpha|^{-1}$.
\end{itemize}
We are mainly interested in the power series converging on the unit disk, i.e., \begin{align*}
    H_K :={}& \{f\in K\llbracket X\rrbracket\ |\ \rho(f) > 1\}\\
    ={}&  \{f\in K\llbracket X\rrbracket\ |\ a_i\rho^i\to 0, \forall \rho < 1\}\\
    ={}&  \{f\in K\llbracket X\rrbracket\ |\ f\text{ converges on the open unit disk }\m_{\C_p} = B(0, 1)\}.
\end{align*}
\begin{example}
    $K\otimes_{\O_K}\O_K\llbracket X\rrbracket$ = power series over $K$ with bounded coefficients $\subsetneq H_K$.
\end{example}
\begin{example}
    $\log(1 + X) = \log_{\Gm}(X) = X - \displaystyle\frac{X^2}{2} + \frac{X^3}{3} - \dots\in H_K\setminus K\otimes_{\O_K}\O_K\llbracket X\rrbracket$.
\end{example}

\subsection{The Gauss Norm}
\begin{theorem}
    Let $f(X) = \sum_{i\ge 0} a_iX^i\in K\llbracket X\rrbracket $ with $\rho(f) > 0$, a real number $\rho < \rho(f)$ s.t. $\rho\in |\C_p^\times|$.
    Then $\sup_{i\ge 1}{|a_i|\rho^i}$ is a maximum (i.e., $\sup_{i\ge 1}{|a_i|\rho^i} = |a_j|\rho^j$ for some $j$), and \[\sup_{i\ge 1}{|a_i|\rho^i} = \sup_{|z| = \rho} |f(z)| =: |f|_\rho.\]
\end{theorem}
\begin{proof}
\begin{itemize}
    \item     $\rho<\rho(f)\implies |a_i|\rho^i \to 0\implies \sup_{i\ge 0}{|a_i|\rho^i}$ is a maximum.
    \item $|f(z)| = \left| \sum_{i\ge 0} a_iz^i\right| \le\sup_{i\ge 1} |a_i||z|^i$, so $|f|_\rho\le \sup_{i\ge 1}{|a_i|\rho^i}$.
    \item Take $\alpha\in\C_p$ with $|\alpha| = \rho$,
    and $j\in\Z_{\ge 0}$ s.t. $\sup_{i\ge 1}{|a_i|\rho^i} = |a_j|\rho^j$.
    Let $\beta := a_j\alpha^j$.
    We aim to find $|z| = \rho$ s.t. $|f(z)| = |\beta|$.
    Consider \[g(X) = \sum_{i\ge 0}g_iX^i := \frac{f(\alpha X)}{\beta}\in\O_{\C_p}\llbracket X\rrbracket.\]
    Moreover, the coefficients $g_i = \dfrac{a_i\alpha^i}{\beta}\to 0$ as $i\to\infty$,
    because $|g_i| = \beta^{-1}|a_i|\rho^i$.
    So $\bar{g}(X)\in k_{\C_p}\llbracket X  \rrbracket$ is actually a polynomial, and it is nonzero since $|g_j| = 1$.
    Take $\bar{w}\in\bar{k}^\times$ s.t. $\bar g(\bar w)\ne 0$. Then a lift $w\in\O_{\C_p}^\times$ verifies $|g(w)| = 1$.
    Hence $|f(\alpha w)| = |\beta|$ and $|\alpha w| = |\alpha| = \rho$.\qedhere
\end{itemize}\end{proof}

Thus, the expression $|f|_\rho\in\R\cup\{+\infty\}$ is defined on $\rho\in\R$.
In addition,
\begin{itemize}
    \item $\rho\to |f|_\rho$ is continuous,
    \item $|f|_\sigma \le |f|_\rho$ if $\sigma\le\rho < \rho(f)$.
\end{itemize}
$\implies$ the \textbf{maximum modulus principle} holds: $|f|_\rho = \sup_{|z|\le \rho} |f(z)| = \max_{|z|\le\rho} |f(z)|$ for $\rho < \rho(f)$.

\begin{itemize}
    \item $|\cdot|_\rho$ is multiplicative: $|fg|_{\rho} = |f|_\rho|g|_\rho$.
\end{itemize}

\begin{example}
    If $f\in H_K$, then \textit{as a function}:\begin{itemize}
        \item $f$ is bounded on $\m_{C_p}\iff f\in K\otimes_{\O_K}\O_K\llbracket X \rrbracket$,
        \item $f$ is bounded by $1$ on $\m_{\C_p}\iff f\in\O_K\llbracket X \rrbracket$.
    \end{itemize}
\end{example}

\subsection{Weierstrass Preparation Theorem}
For $f(X) = \sum_{i\ge 0}a_iX^i\in\O_K\llbracket X \rrbracket$, we define its \textbf{Weierstrass degree} $:= \wideg(f) :=$ smallest $i\in\Z_{\ge 0}$ s.t. $a_i\in\O_K^\times$.
\begin{itemize}
    \item $\wideg$ is multiplicative.
    \item $\wideg(f) = \infty\iff f\in\m_{K}\llbracket X \rrbracket$.
    \item $\wideg(f) = 0\iff a_0\in\O_K\times\iff f\in\left( \O_K\llbracket X \rrbracket \right)^\times$.
    \item If $K/\Q_p < \infty$, then for $f\in K\otimes_{\O_K}\O_K\llbracket X \rrbracket$,
    $\exists ! n\in\Z$ s.t. $\pi^n f$ has finite Weierstrass degree, which is the smallest degree of the term in $f$ with minimum valuation.
\end{itemize}
\begin{remark}
    The last statement fails if $K$ is not finite over $\Q_p$, i.e., if there is no uniformiser. For example, $f(X) = \sum_{i\ge 1}\frac{1}{p^i}X^i$.
\end{remark}
From now on, assume $K/\Q_p < \infty$ with uniformiser $\pi$.
\begin{proposition}
    [Euclidean Division]\label{Euclidean division for power series}
    Let $f\in\O_K\llbracket X \rrbracket$ with $\wideg(f) < \infty$.
    Then: $\forall g\in \O_K\llbracket X \rrbracket$, $\exists ! q\in \O_K\llbracket X \rrbracket$ \& $r\in \O_K[X]$\footnote{The residue $r(X)$ is a polynomial!} s.t. \[g = q\cdot f + r,\ \deg(r)\le \wideg(f) - 1.\]
\end{proposition}
\begin{proof}
    Idea is, again, $\pi$-adic approximation.
    
    First we do ``Euclidean division" in $k\llbracket X \rrbracket $.
    Write $\bar f(X) = X^nf_0(X)$ with $f_0(X)\in k\llbracket X \rrbracket^\times$.
    For $ h = \sum_{i\ge 0}h_iX^i\in k\llbracket X \rrbracket$,
    it decomposes as\[h = X^ns + r,\text{  with } r = h_0 + \dots + h_{n-1}X^{n-1}\]
    \[\implies h = q\cdot f + r, \text{  where } q = s\cdot f_0^{-1}.\]

    Therefore,\begin{align*}
        g &= q_0f + r_0 + \pi g_1 &\text{with } \deg r_0 \le n-1,\\
        &= (q_0 + \pi q_1)f + (r_0 + \pi r_1) + \pi^2 g_2 &\text{ with }\deg r_1\le n - 1\\
        &=\cdots & \\
    \implies g &= qf + r, &\text{with } q = \sum_{i\ge 0}\pi^iq_i, r = \sum_{i\ge 1}\pi^ir_i.
    \end{align*}

    \textit{Unicity.}
    If $qf + r = 0$, then $\underbrace{\bar{q}\bar{f}}_{\text{divided by }X^n} + \underbrace{\bar{r}}_{\deg\le n - 1} = 0$, so $\bar{q}\bar{f} = \bar{r} = 0$.
    Deduce inductively $\bmod \pi^n$.
\end{proof}

For a polynomial $P(X)\in \O_K[X]$, we say $P(X)$ is \textbf{distinguished}, if it is monic with other coefficients in $\m_K$, i.e, \[P(X) = X^n + a_{n-1}X^{n-1} + \dots + a_0,\quad a_{n-1},\dots, a_0\in\m_{K}.\]
\begin{itemize}
    \item The Newton polygon of a distinguished polynomial $P$ will be above $x$-axis with only the end point on $x$-axis, and all slopes are $ < 0$. So every root of $P$ lies in $\m_{\Q_p^\alg}$.
\end{itemize}

\begin{theorem}[Weierstrass Preparation Theorem]\label{Weierstrass preparation}
    Let $f\in\O_K \llbracket X \rrbracket$ with $\wideg f < \infty$.\par
    Then $\exists !$ distinguished polynomial $P\in\O_K \llbracket X \rrbracket$ with $\deg P = \wideg f$, s.t. \[f(X) = P(X)\cdot u(X),\quad u\in\left( \O_K\llbracket X \rrbracket \right)^\times.\]
\end{theorem}

So, power series over $K$ with bounded coefficients would have finitely many zeros in the unit disk.
\begin{corollary}\label{zero of power series with bounded coefficients}
    Let $f(X)\in K\otimes_{\O_K}\O_K\llbracket X \rrbracket$.\begin{enumerate}
        \item $f(X) = \pi^\mu P(X) u(X)$ uniquely, where $\mu\in \Z$, $P$ a distinguished polynomial, $u\in\left( \O_K\llbracket X \rrbracket \right)^\times$.
        \item $f$ has finitely many zeros in $\m_{\C_p}$, and they are actually in $\m_{\Q_p^\alg}$. The number of zeros is $\wideg (\pi^{-\mu} f) = \deg P$\footnote{I want to call this ``the Weierstrass degree of $f$".}.\qed
    \end{enumerate}
\end{corollary}
\begin{corollary}
    $K\otimes_{\O_K}\O_K\llbracket X \rrbracket$ is a PID.
\end{corollary}
\begin{proof}
    For $I = (\{f_i\}_i)$, write $f_i = \pi^{\mu_i}P_iu_i$, then $I = \left(\gcd_i (P_i)\right)$.
\end{proof}

\begin{theorem}
    Let $f\in H_K$, $\rho < 1$. Then $f$ has finitely many zeros in $B(0, \rho)$, all of which are in $\m_{\Q_p^\alg}$.
\end{theorem}
\begin{remark}
    $f\in H_K$ \textit{could} have infinitely many zeros in $\m_{\C_p} = B(0, 1)$.
    For example, we see in the homework that the zeros of $\log_F$ in $\m_{\C_p}$ are $F[p^\infty]$, which is infinite in many cases, such as $F = \Gm$.
\end{remark}
\begin{proof}
    We may assume $\rho\in|\C_p|$.

    Take $L/\Q_p < \infty$ and $\alpha\in\m_L$ with $|\alpha| = \rho$.
    Then $f(\alpha X)\in L\otimes_{\O_L}\O_L\llbracket X \rrbracket$, because $|a_i|\rho^i\to 0$ for $f = \sum a_iX^i\in H_K$.
    Hence $f(\alpha X)$ has finitely many zeros in $\m_{\C_p} = B(0, 1)$ and they are algebraic over $\Q_p$.
    These zeros are in bijection with zeros of $f(X)$ in $B(0, \rho)$. 
\end{proof}

Now we can prove the converse of \cref{zero of power series with bounded coefficients}.
\begin{theorem}
    If $f\in H_K$, then\[f\in K\otimes_{\O_K}\O_K\llbracket X \rrbracket\iff f\text{ has finitely many zeros in }\m_{\C_p}.\]
\end{theorem}
\begin{proof}
    ($\impliedby$)
    First, take $\rho\in \m_{\C_p}$ and $\alpha\in \m_{\Q_p}$ with $|\alpha| = \rho$.

\end{proof}

\subsection{\texorpdfstring{$p$-adic Banach Spaces}{}}
Let $K/\Q_p < \infty$ with uniformiser $\pi$, $k := \O_K/\pi$.

\section{Lubin-Tate Theory}

\subsection{Formal Groups}
In this section, a formal group means a commutative formal group law of dimension one. If $f\in A\llbracket T\rrbracket$ and $g\in A\fring{X_1, \dots, X_n}$, then \begin{align*}
    f \circ g &:= f(g(X_1, \dots, X_n)),\\
    g\circ f &:= g(f(X_1), \dots, f(X_n)).
\end{align*}

\begin{lemma}\label{power series invertible iff}
    Let $f = \sum_{i\ge 1}a_iT^i\in A\fring{T}$. Then
    \[\exists g\in A\fring{T} \st f\circ g = g\circ f = T\iff a_1\in A^\times.\]
\end{lemma}
\begin{proof}
    Use $A\fring{T} = \varprojlim A[T]/T^n$. For details, see the proof of \cref{fund prop of F_varpi}.
\end{proof}

\subsection{Lubin-Tate formal groups}
From now on, we write $A := \O_K$.

Choose a uniformiser $\varpi$ of $K$. Define
\[\mathcal{F}_\varpi := \left\{f\in\O_K\fring{T}\ \left|\ \begin{aligned}
    &f(T) \equiv \varpi T&\mod {T^2} \\
    &f(T)\equiv T^q&\mod \varpi
\end{aligned}\right.\right\}.\]
For example, $f(T) = T^q + \varpi T\in\mathcal{F}_\varpi$.
The following lemma is a fundamental property of $\mathcal{F}_\varpi$.

\begin{lemma}\label{fund prop of F_varpi}
    Let $f, g\in\mathcal{F}_\varpi$, $\Phi_1$ be a linear form\footnote{A \textbf{linear form} is a homogeneous polynomial of degree 1.} over $\O_K$. Then there is a \textbf{unique} $\Phi\in\O_K\fring{X_1,\dots, X_n}$, s.t.\[\begin{cases}
        \Phi \equiv \Phi_1 \mod (X_1, \dots, X_n)^2,\\
        f(\Phi(X_1, \dots, X_n)) = \Phi(g(X_1), \dots, g(X_n)).
    \end{cases}\]
\end{lemma}
\begin{proof}
    We use a standard method. Finding $\Phi$ is equivalent to finding $\Phi_r\in A[X_1, \dots, X_n]$ s.t. \[\begin{cases}
        \Phi_{r+1} \equiv \Phi_r &\mod (\deg\ge r+1),\\
        f(\Phi_r)\equiv \Phi_r(g(X_1), \dots, g(X_n)) &\mod(\deg\ge r+1). 
    \end{cases}\]
    The second condition is guaranteed because $X\mapsto h(X)$ is $X$-adic continuous for any power series $h$.

    Suppose we have found $\Phi_r$. We look for $\Phi_{r+1}$ of the form $\Phi_{r+1} = \Phi_r + Q$, where $Q$ is homogeneous of degree $r+1$, s.t. \[f(\Phi_{r+1}) \equiv \Phi_{r+1}(g(X_1), \dots, g(X_n)) \mod \deg\ge r+2.\]
    The LHS is
    \[f(\Phi_r) + f(Q)\equiv f(\Phi_r) + \varpi Q\mod\deg\ge r+2,\]
    while the RHS is
    \[\Phi_r\circ g + Q(\varpi X_1, \dots, \varpi X_n)\equiv \Phi_r\circ g + \varpi^{r+1}Q,\]
    so if such a $Q\in A[X_1, \dots]$ exists, it must satisfy 
    \[\varpi(\varpi^r - 1)Q\equiv f\circ \Phi_r - \Phi_r\circ g\mod\deg\ge r+2\]
    and thus being unique. This procedure also shows that all $\Phi_r$'s are unqie if we require $\Phi_{r+1} - \Phi_r$ to be homogeneous.

    Because $\varpi^r - 1\in A^\times$, it suffices to show \[f(\Phi_r) \equiv \Phi_r\circ g\mod \varpi,\] which is clear.
\end{proof}

By \cref{fund prop of F_varpi}, one may define the \textbf{Lubin-Tate formal groups}.
They are exactly the formal group laws admitting an endomorphism\begin{itemize}
    \item that has derivative at the origin equal to a uniformiser of $K$, and
    \item reduces mod m to the Frobenius map $T\mapsto T_q$.
\end{itemize}
Moreover, these formal groups admit $\O_K$-actions and are isomorphic as formal $\O_K$-modules.

\begin{proposition}
    For each $f\in \mathcal{F}_\varpi$, there is a unique formal group $F_f$ over $\O_K$ admitting $f$ as an endomorphism.
\end{proposition}
\begin{proof}
    \cref{fund prop of F_varpi} gives $F_f\in A\fring{X, Y}$ s.t. \[\begin{cases}
        F_f = X + Y + \deg \ge 2,\\
        f(F_f(X+Y)) = F_f(f(X), f(Y)).
    \end{cases}\]
    The associativity is proved by showing that both $G_1 = F_f(X, F_f(Y, Z))$ and $G_2 = F_f(F_f(X, Y), Z)$ satisfies 
    \[\begin{cases}
        G = X+Y+Z + \deg\ge 2,\\
        f(G) = G(f(X), f(Y), f(Z)).
    \end{cases}\]
    This is a direct application of \cref{fund prop of F_varpi} and will be used many times.
\end{proof}

So Lubin-Tate formal groups exist. Now we investigate their homomorphisms.
\begin{proposition}
    For each $f, g\in\mathcal{F}_\varpi$ and $a\in \O_K$, there is a unique $[a]_{g, f}\in \O_K\fring{T}$ s.t. \[\begin{cases}
        [a]_{g, f} = aT + \dots,\\
        g\circ [a]_{g, f} = [a]_{g, f} \circ f,
    \end{cases}\]and $[a]_{g, f}\in\hom(F_f, F_g)$, i.e. \begin{align*} F_g\circ [a]_{g, f} = [a]_{g, f}\circ F_f.\end{align*}
    As a corollary of \cref{power series invertible iff}, each $u\in A^\times$ gives an isomorphism $[u]_{g, f} : F_f\isomto F_g$, and there is a unique isomorphism $F_f\simeq F_g$ of the form $T + \cdots$.
    \qed
\end{proposition}

We write $[a]_{f} := [a]_{f, f}\in\enom F_f$.
Note that \[[\varpi]_f = f.\]

\begin{proposition}
    For any $a, b\in\O_K$, \[[a+b]_{g, f} = [a]_{g, f} + [b]_{g, f},\]and\[[ab]_{h, f} = [a]_{h, g}\circ [b]_{g, f}.\]
    
    In particular, $\O_K\hookrightarrow\enom  F_f$ as a ring by $a\mapsto [a]_f$, making $F_f$ a formal $\O_K$-module. The canonical isomorphism $[1]_{g, f}$ is an isomorphism of $\O_K$-modules.
    \qed
\end{proposition}

\subsection{Construction of \texorpdfstring{$K_\varpi$}{}}
Fix an algebraic closure $K^\alg$ of $K$.
Each $f\in\mathcal{F}_\varpi$ associates to $\mathfrak{m}_{K^\alg}$ an $\O_K$-module structure via \[\alpha +_{F_f}\beta := F_f(\alpha, \beta)\]and \[a\cdot \alpha := [a]_f(\alpha)\footnote{These power serieses converges because they actually falls in a finite extension of $K$.}.\]for $|\alpha| < 1, |\beta| < 1$ and $a\in \O_K$.
We denote this $\O_K$-module by $\Lambda_f$.
If $g\in\mathcal{F}_\pi$, then the canonical isomorphism $[1] : F_f\to F_g$ yields $\Lambda_f\isomto\Lambda_g$.

The $\varpi^n$-torsion part of $\Lambda_f$ is denoted by $\Lambda_{f, n}$, i.e., $\Lambda_{f, n} := \Lambda_f[[\varpi]_f^n]$. Because $[\varpi]_f = f$, $\Lambda_{f, n}$ is the $\O_K$-module consisting of the roots of $f^{(n)} := f\circ\cdots\circ f$.
If one takes $f$ to be an Eisenstein polynomial, then all the roots of $f^{(n)}$ lie in $\mathfrak{m}_{K^\alg}$, so $\Lambda_{f, n}$ is precisely the set of roots of $f^{(n)}$ equipped with the $\O_K$-module structure from $F_f$.

\begin{lemma}\label{pi^n torsion cyclic of}
    Let $M$ an $\O_K$-module, $M_n = M[\varpi^n]$. If\begin{itemize}
        \item $M_1$ has $q = [\O_K : \varpi]$ elements, and
        \item $\varpi : M \to M$ is surjective,
    \end{itemize}
    then $M_n\simeq \O_K/\varpi^n$.
\end{lemma}
\begin{proof}
    Do induction on $n$. The structure theorem of f.g.\! modules over a PID shows that $M_1$ having $q$ elements implies that $M_1\simeq A/\varpi$.
    Now assume it true for $n-1$.
    Look at the sequence \[0\to M_1\to M_n\stackrel{\varpi}{\to} M_{n-1}\to 0.\] Surjectivity of $\varpi$ implies the exactness of this sequence, and thus $M_n$ has $q^n$ elements. In addition, $M_n$ must be cyclic, otherwise $M_1 = M_n[\varpi^n]$ is not cyclic.
\end{proof}

\begin{proposition}
    The $\O_K$-module $\Lambda_{f, n}$ is isomorphic to $\O_K/\varpi^n$, and hence $\enom(\Lambda_{f, n})\simeq \O_K/\varpi^n$.
\end{proposition}
\begin{proof}
    It suffices to show for a chosen $f$, so let's take $f = \varpi T + \dots + T^q$, an Eisenstein polynomial.
    We use the above \cref{pi^n torsion cyclic of} by the following observations.\begin{itemize}
        \item All roots of an Eisenstein polynomial have valuation $>0$.
        \item If $|\alpha| < 1$, then the Newton polygon of $f(T) - \alpha$ shows that its roots have valuation $>0$, and thus $[\varpi] = f(T)$ is surjective on $\Lambda_f$.\qedhere
    \end{itemize}
\end{proof}

\begin{lemma}\label{galois commutes power series}
    Let $L$ be a finite Galois extension of $K$. Then for every $F\in\O_K\fring{X_1, \dots, X_n}$, $\alpha_1,\dots, \alpha_n\in\mathfrak{m}_L$ and $\tau\in\gal(L/K)$,
    \[\tau F(\alpha_1, \dots, \alpha_n) = F(\tau\alpha_1, \dots, \alpha_n).\]
\end{lemma}
\begin{proof}
    Note that $\tau$ acts continuously on $L$, becaunse the extension of valuation for local fields is unique.
    Therefore writing $F = \lim_{m\to\infty} F_m$ gives the desired result.
\end{proof}

\begin{theorem}\label{construction of K_{varpi, n}}
    Let $K_{\varpi, n} := K(\Lambda_{f, n})\subset K^\alg$.
    These fields are independent to the choice of $f$.\begin{enumerate}
        \item [(a)] $K_{\varpi, n}/K$ is totally ramified of degree $q^{n-1}(q-1)$.
        \item [(b)] The action of $\O_K$ on $\Lambda_{f, n}$ defines an isomorphism \begin{equation}
            \left( \O_K/\mathfrak{m}_K^n \right)^\times \simeq \gal(K_{\varpi, n}/K).
        \end{equation}
        \item [(c)] For all $n$, $\varpi$ is a norm from $K_{\varpi, n}$, i.e., $\exists\alpha_n\in K_{\varpi, n}$ with $N_{K_{\varpi, n}/K}(\alpha_n) = \varpi$.
    \end{enumerate}
\end{theorem}
\begin{proof}
    Let $f$ be a polynomial $T^q + \dots + \varpi T$.

    Choose a nonzero root $\varpi_1$ of $f(T)$ and, inductively, a root $\varpi_n$ of $f(T) - \varpi_{n-1}$.
    So $\varpi_n\in\Lambda_{f, n}$, and we obtain a tower of extensions \[K_{\varpi, n}\supset K(\varpi_n)\stackrel{q}{\supset} K(\varpi_{n-1}) \stackrel{q}{\supset}\dots\stackrel{q}{\supset} K(\varpi_1)\stackrel{q-1}{\supset} K.\]
    All the extensions with indicated degrees are given by Eisenstein polynomials, and thus Galois and totally ramified.

    The field $K_{\varpi, n} = K(\Lambda_{f, n})$ is the splitting field of $f^{(n)}$ over $K$, hence $\gal(K_{\varpi, n}/K)$ embeds into the permutation group of the set $\Lambda_{f, n}$. By \cref{galois commutes power series}, the action of $\gal(K_{\varpi, n}/K)$ on $\Lambda_n$ preserves its $\O_K$-action, so
    \[\gal(K_{\varpi_n}/K)\hookrightarrow \aut(\Lambda_{f, n})\simeq (\O_K/\varpi^n)^\times.\]
    So $[K_{\varpi, n} : K]\le (q - 1)q^{n-1}$. Comparing the degree gives $K_{\varpi, n} = K(\varpi_n) $.

    Now we prove (c).
    Let $f^{[n]} := (f/T)\circ f\circ\dots\circ f$. Then $f^{[n]}$ is monic with degree $q^{n-1}(q-1)$ and $f^{[n]}(\varpi_n) = 0$, and thus $f^{[n]}$ is the minimal polynomial of $\varpi_n$ over $K$. So we have \[N_{K_{\varpi, n}/K}(\varpi_n) = (-1)^{q^{n-1}(q-1)}\]
    by the following \cref{compute norm and trace from minimal polynomial}.
\end{proof}

\begin{lemma}\label{compute norm and trace from minimal polynomial}
    Let $L/K$ be a finite extension in an algebraic closure $K^\alg$, and $\alpha\in L$ has minimal polynomial $f$ over $K$ of degree $d$. Suppose \[f(X) = (X-\alpha_1)\cdots(X-\alpha_d)\in K^\alg[X],\] and let $e = [L : K(\alpha)]$
    then \[N_{L/K}(\alpha) = \left(\prod_{i = 1}^d \alpha_i\right)^e,\qquad \tr_{L/K}(\alpha) = e\sum_{i = 1}^d \alpha_i.\]
    Moreover, if \[f(X) = a_dX^d + a_{d-1}X^{d-1} + \dots + a_0,\]then \[N_{L/K}(\alpha) = (-1)^{de}a_0^e,\qquad \tr_{L/K}(\alpha) = -ea_{d-1}.\]
\end{lemma}
\begin{remark}
    This can be deduced from $N_{L/K} = N_{L/K(\alpha)}\circ N_{K(\alpha)/K}$ and $\tr_{L/K} = \tr_{L/K(\alpha)}\circ \tr_{K(\alpha)/K}$.
\end{remark}


Define \[K_\varpi := \bigcup_{n} K_{\varpi, n}.\]
The isomorphisms in \cref{construction of K_{varpi, n}} (b) are \[(\O_K/\varpi^n)^\times\to \gal(K_{\varpi, n}/K)\quad \bar{u}\mapsto (\Lambda_{f, n}\ni \alpha\mapsto [u]_f(\alpha)),\] and clearly lift to an isomorphism \[A^\times\simeq \gal(K_\varpi/K).\]

\subsubsection*{The local Artin map}
The \textbf{local Artin map} is a homomorphism \[\phi_\varpi : K^\times\to \gal(K_\varpi K^\nr/K) = \gal(K^\nr/K)\times \gal(K_\varpi/K)\] defined as follows.
Let $a = u\varpi^m\in K^\times$, then 
\begin{itemize}
    \item $\phi_\varpi(a)|_{K^\nr} := \frob^m$;
    \item $\phi_\varpi(a)(\lambda) := [u^{-1}]_f(\lambda)$, $\forall \lambda\in\bigcup_n\Lambda_n$.
\end{itemize}

\begin{theorem}
    Both $K_{\varpi}$ and $K^\nr$ are independent of the choice of $\varpi$.
\end{theorem}


\subsection{The Local Kronecker-Weber theorem}

\subsection{The Case of \texorpdfstring{$\Q_p$}{}}
Let $K = \Q_p$ and $\varpi = p$. Then $f(T) := (1 + T)^p - 1\in\mathcal{F}_p$.
Note that $f$ is an endomorphism of \[\Gm(X, Y) = X + Y + XY,\] so $F_f = \Gm{}_{/\Z_p}$. Under the isomorphism
\[(\mathfrak{m}, +_{\Gm})\simeq (1 + \mathfrak{m},\ \cdot\ ),\]
the endomorphism $f : a\mapsto (1 + a)^p - 1$ is converted to the Frobenius map $a\mapsto a^p$.

\subsubsection*{The field $(\Q_p)_p$}

For each $r\ge 1$, the $p^r$-torsion part of $\Lambda_f$ is
\[\Lambda_{f, r} = \left\{\alpha\in\Q_p^\alg\left|(1 + \alpha)^{p^r} = 1\right.\right\}\simeq
\left\{\zeta\in(\Q_p^\alg)^{\times}
\left|\zeta^{p^r} = 1\right.\right\} = \mu_{p^r}.\]
The isomorphism is for $\O_K$-modules.
So choose primitive $p^r$-th roots of unity $\zeta_{p^r}$ s.t. $\zeta_{p^r}^p = \zeta_{p^{r-1}}$,
then $\varpi_r := \zeta_{p^r} - 1$ forms a sequence of compatible generators of $\Lambda_{f, r}$.
Therefore \[(\Q_p)_{p, r} = \Q_p(\varpi_r) = \Q_p(\mu_{p^r}),\]
and the ``maximal totally ramified abelian extension''\footnote{Not sure if this terminology is correct ...?} of $\Q_p$ is $(\Q_p)_p = \Q_p(\mu_{p^\infty})$.

\subsubsection*{The local Artin map \texorpdfstring{$\phi_p : \Q_p^\times\to \gal(\Q_p^\ab/\Q_p)$}{}}

It suffices to look at every \[\phi_p : \Q_p^\times\to \gal(\Q_p(\mu_n)/\Q_p).\]
\begin{itemize}
    \item If $n$ is prime to $p$, then $\Q_p(\mu_n)/\Q_p$ is unramified of degree $f$, where $f$ is the minimum natural number s.t. $m\mid p^f - 1$.
    The map $\phi_p$ sends $up^t$ to the $t$-th power of Frobenius-$p^f$ on $\Q_p(\mu_n) = \Q_p(\mu_{p^f - 1})$, and $\ker\phi_p = (p^{f})^{\Z}\times\Z_p^\times$.
    \item If $n = p^r$, then $\Q_p(\mu_{p^r})/\Q_p$ is totally ramified. The map $\phi_p$ sends $up^t$ to the element sending a root of unity $\zeta$ to $\zeta^{\bar u^{-1}}$, where $\bar u\in\Z$ has the same residue modulo $p^r$ as $u$.
    The kernel is $p^\Z\times (1 + p^r\Z_p)$.
\end{itemize}




\input{cycl_ext_Qp.tex}
\section{Periods of Characters}
% \section{Group Cohomology}
In this section we fix a commutative ring $\K$.
\subsection{Cohomology}
Let $G$ be a group. A \textbf{$G$-module} with coefficients in $\K$ is a $\K$-module together with a $\K$-linear \textit{left} $G$-action. Hence the category of $G$-modules with coefficients in $\K$ is isomorphic to the category of $\K[G]$-modules.
\begin{remark}
    In particular, a $G$-module with coefficients in $\Z$ is an abelian group with additive left $G$-action. 
\end{remark}

\begin{example}
We list some important constructions of $G$-modules here.
\begin{enumerate}[(a)]
    \item The \textbf{trivial $G$-module} is $\K$ with the trivial $G$-action.
    \item The group ring $\K[G]$ is a $G$-module with $G$ acting by left-multiplication.
    \item Direct sum and product. Both direct sums and products for $G$-modules as $\K$-modules can be lifted to $G$-modules, by giving $G$-action diagonally, i.e, \[g((m_i)_i) := \left( (gm_i)_i \right).\]
    \item Tensor products. For $M, N\in \Mod_G$, define $M\otimes N\in \Mod_G$ to be $M\otimes_\K N$ with the diagonal $G$-action \[g(x\otimes y) := gx\otimes gy,\quad x\in M, y\in N.\]
    \item Hom module. For $M, N\in\Mod_G$, define $\hom(M, N)\in \Mod_G$ to be $\hom_{\K}(M, N)$ with $G$ acting ``by conjugation'': \[(gf)(x) := g f(g^{-1}x),\quad f\in\Hom_\K(M, N), x\in M.\]
    \begin{itemize}
        \item We have \[\Hom_G(M, N) = \Hom(M, N)^G\] as $G$-modules.
        \item The adjoint $L\otimes_\K (-) \dashv \Hom_\K(L, -)$ in $\Mod_\K$ holds in $\Mod_G$, i.e,
\[\begin{tikzcd}
	{\Hom(L\otimes M,N)} & {\Hom(L, \Hom(M, N))} \\
	\varphi & {x\mapsto y\mapsto\varphi(x\otimes y)} \\
	{\left( x\otimes y\mapsto \psi(x)(y) \right)} & \psi
	\arrow["\sim", leftrightarrow, from=1-1, to=1-2]
	\arrow[maps to, from=2-1, to=2-2]
	\arrow[maps to, from=3-2, to=3-1]
\end{tikzcd}\]
    are isomorphisms of $G$-modules.
    \begin{remark}
        The $K$-modules $M\otimes_\K N$ and $\hom_\K(M, N)$ with their $G$-module structures are \textit{NOT} the tensor product or $\Hom$-set in $\K[G]$-module.
    \end{remark}
    \end{itemize}
    \item Induced module. Let $H < G$ be a subgroup, $N$ a $H$-module. Then $\Ind_H^G N$ is the $K$-module of $H$-invariant functions $G\to N$, i.e., \[\Ind_H^G N := \{\varphi : G \to N\mid \varphi(hg) = h\varphi(g),\;\forall h\in H, g\in G\}\simeq\Hom_H(\K[G], N).\]
    The group $G$ acts on $\Ind_H^G N$ from the left by \[(g\varphi)(x) := \varphi(xg).\]
    We obtain a functor $\Ind_H^G : \Mod_H\to\Mod_G$ by sending $\alpha : N\to N'$ to \[\alpha_* : \Ind_H^G N\to\Ind_H^G N' := \varphi\mapsto\alpha\circ\varphi.\]
    \begin{itemize}
        \item $\Ind_H^G$ is \textit{right adjoint to the forgetful functor} $\Mod_G\to\Mod_H$. The isomorphism is given by
        \[\begin{tikzcd}
            {\Hom_G\left(M, \Ind_H^GN\right)} & {\Hom_H(M, N)} \\
            \alpha & {x\mapsto \alpha(x)(1_G)} \\
            {\left[ x\mapsto (g\mapsto \beta(gx) \right]} & \beta
            \arrow["\sim", leftrightarrow, from=1-1, to=1-2]
            \arrow[maps to, from=2-1, to=2-2]
            \arrow[maps to, from=3-2, to=3-1]
        \end{tikzcd}\]
        where $M\in \Mod_G$, $N\in \Mod_H$.
        \item $\Ind_H^G$ is an exact funtor.
        \item For any $\K$-module $M$, we define \[\Ind^G M := \Ind_{\{1\}}^G M =\{\varphi : G\to M\}.\] An \textbf{induced module} is a $G$-module of the form $\Ind^GM$ for some $\K$-module $M$.
        \item Let $M$ be a $G$-module. Define $M_* := \Ind^G M$, then we have an embedding \[M\hookrightarrow M_* := x \mapsto [g\mapsto gx]\]of $G$-modules. The exact sequence \begin{equation}
            0\to M\to M_*\to M_\dagger \to 0
        \end{equation} in $\Mod_G$, where $M_\dagger := M_*/M$, will be used many times in ``dimensional shifting''.
    \end{itemize}
\end{enumerate}
\end{example}

Let $M$ be a $G$-module, $r\ge 0$ a natural number.
We define the \textbf{ $r$-th cohomology groups of $G$ with coefficients in $M$} to be the value of the $r$-th right derived functor of the left-exact functor \[(-)^G\simeq \Hom_G(\K, -) : \Mod_G\to\Mod_K\] at $M$.
But for this definition to make sense, we need to show that:
\begin{lemma}\label{Mod G has enough injectives}
    The category $\Mod_G$ has enough injectives.
\end{lemma}
\begin{proof}
    The category $\Abel$ has enough injectives. Let $M \in\Mod_G$, $I\in\Abel$ injective with $M\hookrightarrow I$. Applying the exact functor $\Ind^G$ gives \[M\hookrightarrow M_* := \Ind^G M \hookrightarrow\Ind^G I. \] So it remains to show that
    \begin{itemize}
        \item the functor $\Ind^G$ preserves injectives,
    \end{itemize}
    which follows from $\Hom_G(-, \Ind^G I)\simeq\Hom_\Z(-, I)$.
\end{proof}


\begin{proposition}[Shapiro's lemma]\label{Shapiro's lemma}
Let $H < G$ be a subgroup.
The isomorphism\[(-)^H \simeq \Hom_H(\K, -) \simeq \Hom_G\left(\K, \Ind^G_H(-)\right)\simeq \left( \Ind_H^G(-) \right)^G\]induces a canonical isomorphism \[H^\bullet \left(G, \Ind^G_H (-)\right)\simeq H^\bullet (H, -),\]
which is compatible with the long exact sequence.
\end{proposition}
\begin{proof}
    
\end{proof}

\begin{corollary}\label{induced modules have trivial cohomology}
    If $M$ is an induced $G$-module, then $H^r(G, M) = 0$ for all $r\ge 1$.\qed
\end{corollary}


\subsection{Compute Cohomology via cochains}

Homological algebra tells us that \[H^r(G, M) = R^r\Hom_G(\Z, -)(M) = \ext^r(\Z, M) = R^r\Hom_G(-, M)(\Z),\]
so we can use the projective resolution of $\Z\in\Mod_G$ to compute $H^\bullet(G, M)$.

Denote by $P_r$ the free $\Z$-module with basis $G^{r+1} = G\times\dots\times G$ and endow $P_r$ with the $G$-action
\[g(g_0, g_1,\dots, g_r) := (gg_0, gg_1, \dots, gg_r).\]
Define $d_r : P_r\to P_{r-1}$ by \[d_r(g_0, \dots, g_r) := \sum_{i=0}^r(-1)^i(g_0,\dots, \hat{g}_i, \dots, g_r).\]
Then \[\dots\to P_1\stackrel{d_1}{\to} P_0\stackrel{d_0}{\to} \Z\] is exact, i.e., a projective resolution of $\Z$.

Note that $\varphi\in\Hom_G(P_r, M)$ is equivalent to a function $\varphi : G^{r+1}\to M$ s.t. \[\varphi(gg_0, \dots, gg_r) = g\varphi(g_0, \dots, g_r),\]
which is thus determined by its value on the set $\{(1, g_1, \dots, g_r) : g_i\in G\}$. Therefore we consider the abelian group\footnote{The group structure on $C^r(G, M)$ is point-wise addition.} $C^r(G^r, M) := \{\varphi : G \to M\}$. Note that $G^0 = 1$ and thus $C^0(G, M) = M$.
Define a homomorphism \[d^r : C^r(G, M) \to C^{r+1}(G, M)\] by $(d^r\varphi)(g_1, \dots, g_{r+1})$\begin{align}
    := g_1\varphi(g_2, \dots, g_{r+1}) + \sum_{j=1}^r(-1)^j\varphi(g_1, \dots, \hat{g}_j, \dots, g_r) + (-1)^{r+1}\varphi(g_1, \dots, g_r).
\end{align}
Let \[Z^r(G, M) := \ker d^r,\ B^r(G, M) := \im d^{r-1}.\]
One can prove that $d^r\circ d^{r-1} = 0$, and \[H^r(G, M) = Z^r(G, M)/B^r(G, M).\]

\begin{example}[$H^1$]
    An 1-cocycle $c : G\to M$ is called a \textbf{crossed homomorphism}.
    We have \[
        H^1(G, M) = \frac{Z^1(G, M )}{B^1(G, M )}
        = \frac{\{c  :G\to M\mid c(gh) = c(g) + gc(h)\}}{\{g\mapsto gm - m\mid m\in M \}}.\]

    Now fix a $G$-module $M$ and let $E$ be an \textbf{extension of $\K$ by $M$},
    meaning that $E$ is a $G$-module with an exact sequence\[0\to M\to E\stackrel{\pi}{\to} \K\to 0.\]
    Take $e\in E$ with $\pi(e) = 1$.
    Then $ge - e\in\ker\pi = M$ for $g\in G$,
    and the map \[G\to M,\quad g\mapsto ge - e\]
    is a cocycle. Moreover, different choices of the lift $e$ are cohomologous.
    Hence, the extension $E$ of $\K$ by $M$ defines $[E]\in H^1(G, M)$,
    and $[E] = 1\iff E\simeq M\oplus \K$.
\end{example}

\begin{example}
    If $G$ acts trivially on $M$,
    then a crossed homomorphism is a homomorphism,
    and $H^1(G, M) = \Hom_{\text{Grp}}(G, M)$.
\end{example}

\begin{example}[$H^1$ for finite cyclic groups]\label{eg: H1 to H-1 for finite cyclic group}
    Let $G$ be a finite cyclic group generated by $\sigma$.
    Then \[I_G = \gene{\sigma^n m - m\mid m\in M, n\in\Z } = \gene{\sigma m - m\mid m\in M},\]
    \[\hat H^{-1}(G, M) = \ker(N_G)/(\sigma - 1)M.\]
    In this case, choosing a generator $\sigma$ of $G$ defines an explicit isomorphism
    \begin{align*}
        \hat H^1(G, M) &\to \hat H^{-1}(G, M)\\
        \varphi&\mapsto \varphi(\sigma).
    \end{align*}
    Indeed, crossed homomorphisms $G\to M$ are defined by their value on generators of $G$, and for $\varphi : G\to M$ a crossed homomorphism, \[
    \varphi(\sigma^n) = \sigma^{n-1}\varphi(\sigma) + \sigma^{n-2}\varphi(\sigma) + \cdots + \sigma\varphi(\sigma) + \varphi(\sigma),\ \forall \sigma\in G.\]
    Therefore, if $G\simeq \Z/n\Z$ is generated by $\sigma$ of order $n$,
    then \[\varphi\text{ is a crossed homomorphism }\iff x := \varphi(\sigma) \text{ verifies } N_Gx = \sum_{g\in G} gx = x + \sigma x + \cdots + \sigma^{n-1}x =0.\]
    \[\varphi \text{ is principal }\iff \varphi(\sigma) \in \left( \sigma - 1 \right)M.\]
    As $Z^1(G, M)\to M,\ \varphi\to \varphi(\sigma)$ is a group homomorphism, we get the isomorphism.
\end{example}

\begin{example}[$H^1$ for infinite cyclic groups with value in finite $G$-modules]
    Let $G$ be infinite and topologically generated by $\sigma$,
    and $M$ be a \textit{finite} $G$-module.
    Then \[H^1(G, M)\simeq M/(\sigma - 1)M.\]
    via $\varphi \leftrightarrow \varphi(\sigma)$.
\begin{proof}
It suffices to show that for every $m\in M$,
the assignment $\varphi(\sigma^n) := \sum_{i=0}^{n-1}\sigma^i\varphi(\sigma)$ defines a cocyle on $G$.

Since $M$ is finite, there exists $n,k\in\Z$ s.t.
\[\sigma^n m = m,\quad km = 0.\]
Therefore, if $i\equiv j\bmod kn$ and $i > j$,
then $
    \varphi(\sigma^i) - \varphi(\sigma^j)
    = \sigma^jm + \dots + \sigma^{i-1}m
$ is a multiple of \[k(1 + g + \dots + g^{n-1})m = 0.\]
So $\varphi : \gene{\sigma}\simeq \Z\to M$ factors through a cocycle $\Z/kn\Z\to M$.
{\color{blue}(I am confused.)}
\end{proof}
\end{example}


% \subsection{Homology}

% For $M\in \Mod_G$, define its \textbf{coinvariant} to be the quotient
% \[M_G := M\big/ \gene{gm - m\mid g\in G, m\in M} = M/(G - \Id)M\in \Abel.\]

% \begin{lemma}
%     The assignment $M\mapsto M_G$ defines a right-exact functor \[(-)_G\simeq \Z\otimes_{\Z[G]}(-) : \Mod_{\Z[G]}\to\Abel\]
% \end{lemma}
% \begin{proof}
% Consider the augmentation map $\Z[G]\to\Z$ which is an additive homomorphism sending all $g\in G$ to $1\in\Z$.
% Its kernel $I_G$ is called the \textbf{augmentation ideal}.
% Note that:\begin{itemize}
%     \item $I_G\subset\Z[G]$ is the free abelian subgroup with basis $\{g - 1 \mid g\in G, g\ne 1\}$.
% \end{itemize}
% Therefore \[M_G = M/I_GM\simeq \Z[G]/I_G\otimes_{\Z[G]} M\simeq \Z\otimes_{\Z[G]} M.\qedhere\]
% \end{proof}
% We define the \textbf{$r$-th homology groups of $G$ with coefficients in $M\in\Mod_G$} to be the value of the $r$-th left derived functor of the right-exact functor $(-)_G$.

% \subsection{The Tate cohomology groups}

% In this subsection, let $G$ be a \textit{finite} group.

% Recall that the norm $N_G : M \to M$ for a $G$-module $M$ is defined by \[N_G(x) := \sum_{g\in G}gx,\quad x\in G.\]
% One observes that \[\im N_G\subset M^G,\quad I_GM\subset \ker N_G.\]
% Therefore $N_G$ factors as \[M\twoheadrightarrow M/I_GM = M_G\to M^G\hookrightarrow M,\] and we got an exact sequence \[0\to\ker N_G/I_GM\to M_G\to M^G\to M^G/\im N_G\to 0.\]

% The map $H_0(G, M)\to H^0(G, M)$ induced by the norm map on $M$ connects homologies and cohomologies. We define the \textbf{Tate cohomology groups} by \[\hat H^r (G, M) := \begin{cases}
%     H^r(G, M), &r\ge 1,\\
%     M^G/N_G(M), &r = 0,\\
%     \ker (N_G : M\to M)/I_GM, & r = -1,\\
%     H_{-r-1} (G, M), & r\le -2.
% \end{cases}\]

% \begin{proposition}
%     If $M$ is induced, then $\hat H^\bullet(G, -) = 0$.
% \end{proposition}

% (connecting $H^r$ to $H^{r + 2}$.)




% \subsection{Cohomology of Topological Groups}


\subsection{Non-commutative Cohomology}
Let $G$ be a topological group, and $M$ be a topological (not necessarily commutative) group with a \textit{continuous} left $G$-action compatible with the group structure on $M$, namely a continuous map\[G\times M\to M,\quad (g, m)\mapsto gm,\]
s.t. \(
    (g_1g_2)m = g_1(g_2m),\ 1m = m;\quad
    g(m_1m_2) = gm_1\cdot gm_2,\ g1 = 1.
\)
% \begin{remark}
%     Assume that $G$ is profinite, and $M$ is discrete. Then TFAE:\begin{itemize}
%     \item $G\times M\to M$ is continuous.
%     \item $M = \bigcup_{K} M^K$, where $K$ goes through all \textit{normal open} subgroups of $G$.
%     \item For all $m\in M$, its stabiliser $\stab_G(m)$ in $G$ is open.
%     \end{itemize}
% \end{remark}

We define only $H^0$ and $H^1$ without additional structure.
Define \[H^0(G, M) := M^G = \{m\in M\mid gm = m,\forall g\in G\},\]
which is a group.

A (1-)cocycle on $G$ is a continuous crossed homomorphism, namely a continuous map $c : G\to M$ s.t. \[c(gh) = c(g)\cdot gc(h).\]
\begin{itemize}
\item $c : G\to M$ is a cocycle $\implies c(1) = 1$.
\item $m\in M\leadsto g\mapsto m^{-1}gm$ is a cocycle.
\end{itemize}
If $c\in Z^1(G,M)$ and $m\in M$,
then $g\mapsto m^{-1}c(g)gm$ is a cocycle.
This defines a right $M$-action on $Z^1(G, M)$,
and thereby defines an equivalence relation $\sim$, called \textbf{cohomologous}, allowing us to define \[H^1(G, M) := Z^1(G, M)/\!\sim.\]
Note that $H^1(G, M)$ is only a \textbf{pointed set},
in which the special point is \[1 = [g\mapsto 1] = [g\mapsto m^{-1}gm].\]

Let $1\to X\stackrel{u }{\to} E\stackrel{v}{\to} Y\to 1$ be a short exact sequence of (continuous) $G$-groups.
Taking $H^*(G, -)$ gives a long exact sequence (up to $H^1$)
\[1\to X^G\to E^G\to Y^G\stackrel{\delta}{\to} H^1(G, X)\to H^1(G, E)\to H^1(G, Y),\]
where the connecting homomorphism $\delta : H^0(G, Y)\to H^1(G, X)$ is defined as follows:
if $y\in Y^G$ is the image of some $e\in E$,
then $\delta(y)\in H^1(G, X)$ is represented by the cocycle
\[g\mapsto \delta(y)(g) = e^{-1}\cdot ge\in\ker(E\to Y) = \im(X\to E) \simeq X.\]

\begin{example}[Classify semi-linear representations]
    Let $R$ be a \textit{commutative} topological ring with a continuous $G$-action compatible with the ring structure on $R$,
    $X$ be a free $R$-module of rank $d$ with a semi-linear $G$-action.
By choosing a basis $e = \{e_1, \dots, e_d\}$ of $X$,
we write for each $g\in G$ the matrix $M_e(g)$ in the basis $e$, and thus define a cocyle \[G\to\GL_d(R),\quad g\mapsto M_e(g).\]
\begin{itemize}
\item[-]Indeed, $G$ acts on $\GL_d(R)$ ``element-wisely''\footnote{
    Note that if $g\in G$ and $A\in\GL_d(R)$, $gA = g\circ A\circ g^{-1}$ as functions $R^d\to R^d$
}, i.e, \[gA = g (a_{ij})_{i, j} := (ga_{ij})_{i, j}.\]
Write \(\vec{e} = (e_1\ \cdots\ e_d) \).
Recall that the $i$-th column $(g_{1i}\ \cdots\ g_{di})^{\mathrm t}$ of $M_e(g)$ is defined by
\[ge_i = g_{1i}e_1 + \cdots + g_{di}e_d = \vec e
\cdot \begin{pmatrix}
    g_{1i}\\ \vdots \\ g_{di}
\end{pmatrix}.\]
Or $g\vec{e} = e\cdot M_e(g)$.
If \[x = \vec{e}\begin{pmatrix}
    x_1\\ \vdots\\ x_d
\end{pmatrix},\quad g\in G,\]
then \[gx =\vec{e} \cdot M_e(g)\cdot \begin{pmatrix}
    gx_1\\ \vdots \\ gx_d
\end{pmatrix}.\]
Hence \[ghx = \vec{e}\cdot M_e(g)\cdot gM_e(h)\cdot\begin{pmatrix}
    ghx_1\\ \vdots\\ ghx_d
\end{pmatrix},\]
i.e., $M_e(gh) = M_e(g)\cdot gM_e(h)$.

\end{itemize}
Let $M$ be a $R$-module.


If $f = \{f_1, \dots, f_d\}$ is another basis of $X$,
and $P$ is the matrix of $f$ in $e$, i.e., \[f_i = \vec{e}\cdot i\text{-th column of }P.
\]
Then \[M_f(g) = P^{-1}\cdot M_e(g)\cdot gP.\]
\begin{itemize}
\item[-]
Write $\vec{f} = \vec{e}\cdot P$,
then $\vec{e}PM_f(g) =\vec{f}M_f(g) = g\vec{f} = g(\vec{e}P) = g\vec{e}\cdot gP = \vec{e}M_e(g)g(P).$
\end{itemize}
Therefore, we assign to each $R$-semi-linear $G$-representation $X$ a class $[X]\in H^1(G, \GL_d(R))$.
\end{example}

\subsection{The Inflation-Restriction Exact Sequence}
Let $G$ be a topological group and $M$ a smooth $G$-group.
% If $H$ is also a topological group, and $\varphi G\to H$ is a continuous homomorphism, then $\varphi$ induces a smooth $H$-group structure on $M$ in the obvious way $hM := \varphi(h)M$.
For a \textit{closed} normal subgroup $H\triangleleft G$, it induces a \textbf{restriction} map \[\res : H^1(G, M)\to H^1(H, M), \quad \res(c)(h) = c(h)\]
and an \textbf{inflation} map
\[\inf : H^1(G/H, M^H)\to H^1(G, M),\quad \inf(c)(g) := c(\bar g).\]

The group $G$ acts on $H^1(H, M)$ by \[(gc)(h) := g(c(g^{-1}hg)).\]
This action restricted to $H$ is trivial\footnote{
    See the proof of (1) in \cref{inflation-restriction sequence}} on $H^1(H, M)$, hence $G/H$ acts on $H^1(H, M)$.
\begin{proposition}[The inflation-restriction sequence]\label{inflation-restriction sequence}
    This sequence is exact:\[
    0\to H^1(G/H, M^H)\stackrel{\inf}{\to} H^1(G, M)\stackrel{\res}{\to} H^1(H, M)^{G/H}.\]
\end{proposition}
\begin{proof}
    This sequence says three things:
\begin{enumerate}
\item [(1)] $\res(H^1(G, M))\subset H^1(H, M)^{G/H}$.\par
For $c\in Z^1(G, M)$, \begin{align*}
    (g\res(c))(h) = gc(g^{-1}hg)
    = gc(g^{-1})\cdot c(hg) = c(g)^{-1}\cdot c(h)\cdot hc(g).
\end{align*}
So $g\res(c)$ is cohomologous to $\res(c)$ for all $g\in G$. 
\item [(2)] $\res(c) = 1\iff c\in\inf(H^1(G/H, M^H))$.\par
For $c\in H^1(G/H, M^H)$, \[\res(\inf(c))(h) = c(\bar h) = c(1) = 1.\]
that is $\res\circ\inf = 1$. Conversely,
if $\res(c) = 1$, then the map $c|_H$ is cohomologous to $1$, which implies that $c(g)$ is determined by $\bar g\in G/H$, meaning that $c$ is inflated.
\item [(3)] $\inf(c) = 1\iff c = 1$.\par
If $\inf(c) = 1$,
then $\exists m\in M$ s.t. $c(\bar g) = \inf(c)(g) = m^{-1}gm$.
In particular, $m^{-1}hm = c(\bar h) = c(\bar{1}) = 1$, so $m\in M^H$ and $c\in Z^1(G/H, M^H)$ is cohomologous to $1$.\qedhere
\end{enumerate}
\end{proof}


\subsection{Some Applications in Galois Cohomology}
In this subsection, let $L/K$ be a Galois extension, $G := \gal(L/K)$. Then both $L$ and $L^\times$ have natural $G$-module structures.

\subsubsection{Hilbert's Theorem 90 and \texorpdfstring{$H^1(G, \GL_d(L))$}{H1(G, L cross)}}

\begin{theorem}[Dedekind-Artin]\label{dedekind theorem on linearly independence of characters}
    Let $\Gamma$ be a monoid, $E$ be a integral domain, and $\Hom_{\times}(\Gamma, E)$ the set of monoid homomorphisms $\Gamma\to E$.
    \footnote{
        The set $\Hom_{\times}(\Gamma, E)$ admits a $E$-module structure defined point-wisely.
        The elements in $\Hom_{\times}(\Gamma, E)$ are sometimes called characters.
    }
    Then $\Hom_{\times}(\Gamma, E)$ is a linearly independent set over $E$; i.e, for $a_\chi\in E$,
    \[\sum_{\chi\in\Hom_{\times}(\Gamma, E)} a_\chi\chi(\cdot) = 0\text{ on }E
    \implies a_\chi = 0,\forall\chi.\]
\end{theorem}
\begin{proof}
    Suppose that $J := \{\chi\in\Hom_{\times}(\Gamma, E)\mid a_\chi\ne 0\}\ne\varnothing$.
    The idea is to {\color{blue} take $(a_\chi)_\chi$ s.t.
    $J = J((a_\chi)_\chi)$ is nonempty but minimal}.

    Since $\chi(1) = 1\ne 0\in E$, we have $\# J > 1$.
    Let $\xi, \eta$ be two different characters $\Gamma\to E$. Then $\exists g\in\Gamma$ s.t. $\xi(g)\ne \eta(g)$.
    Note that \[\sum_{\chi\in J} a_\chi \chi(g)\chi(\cdot) = \sum_{\chi\in J} a_\chi\chi(g\,\cdot) = 0,\]
    \[\sum_{\chi\in J}a_\chi\xi(g)\chi(\cdot) = \xi(g)\sum_{\chi\in J}a_\chi\chi(\cdot) = 0,\]
    and subtracting these two identities yields
    \[\sum_{\chi\in J\sminus\{\xi\}} a_\chi(\chi(g) - \xi(g))\chi(\cdot) = 0.\]
    This new identity is nontrivial sicne $\eta(g) - \chi(g)\ne 0$, but concerns strictly lesser characters than $J$. Contradiction.
\end{proof}

\begin{proposition}\label{Hilbert 90 - multiplicative - cohomology}
    $H^1(\gal(L/K), L^\times) = 0$.\par
    In other words, if $\varphi : G\to L^\times$ is a crossed homomorphism, i.e., \[\varphi(gh) = g\varphi(h)\varphi(g),\ \forall g, h\in G,\]
    then $\exists b_\varphi\in L^\times$ s.t. \[\varphi(g) = \frac{g b_\varphi}{b_\varphi},\ \forall g\in G.\]
\end{proposition}
\begin{proof}
    Take $a\in L^\times$ and define \[b := \sum_{g\in G}\varphi(g)\cdot ga\in L.\]
    Then \begin{align*}
        hb = \sum_{g\in G} h\varphi(g)\cdot hga
        = \sum_{g\in G}\frac{\varphi(hg)}{\varphi(h)} hga = \frac{b}{\varphi(h)}.
    \end{align*}
    Hence if $b\ne 0$, we would have $\varphi(g) = b/gb = g(b^{-1})/b^{-1}$.
    By \cref{dedekind theorem on linearly independence of characters},
    $\gal(L/K)\subset \Hom_\times(L, L)$ is linearly independent over $L$,
    so $\sum_{g\in G}\varphi(g)g(\cdot) : L\to L$ is a non-zero function, and thus can we find $a\in L$ with $b\ne 0$.
\end{proof}

\begin{corollary}
    Let $L/K$ be a finite cyclic extension, $\sigma$ a generator of $G = \gal(L/K)$, and $a\in L$.
    If $N_{L/K}a = 1$, then $\exists b\in L^\times$ s.t. $a = \sigma b/b$.
\end{corollary}
\begin{proof}
    For the $G$-module $L^\times$, the norm map \[N_G = N_{L/K} : x\mapsto\prod_{g\in G}  gx.\]
    So \[\dfrac{\ker(N_{L/K})}{(\sigma(\cdot) / \Id(\cdot))L^\times} = \hat H^{-1}(G, L^\times) \simeq H^1(G, L^\times) = 0.\qedhere\]
\end{proof}

Note that $L^\times = \GL_1(L)$.
The result above extends to higher $\GL_d(L)$.
\begin{theorem}[Artin]\label{artin theorem on algebraically independence of characters}
    If $L$ is an infinite field,
    $G$ is a finite subgroup of field automorphisms $\aut(L)$ of $L$,
    then the elements of $G$ are algebraically independent over $L$.
\end{theorem}

\begin{theorem}[Hilbert 90]\label{Hilbert 90 - H1(Gal GL) = 0}
    If $L/K$ is finite Galois, then $H^1(\gal(L/K), \GL_d(L)) = 0$ for all $d\in\Z_{\ge 1}$.
\end{theorem}
% We prove this generalized version in two cases: 1) $K$ is infinite
% and 2) $L/K$ is finite cyclic\footnote{
%     Note that $\GL_d(L)$ is not abelian, so
%         $H^1(G, \GL_d(L)) \neq {\ker N_G\over(\sigma/\Id)\GL_d(L)}$ a priori.
% }.
\begin{proof}
Let $\varphi : G = \gal(L/K)\to \GL_d(L)$ be a cocycle.
Similarly, take $a\in L^\times$ and consider \[P(a) := \sum_{g\in G} ga\cdot \varphi(g)\in \mathrm{M}_d(L).\]
Then \[hP(a) = \sum_{g\in G} hga\cdot h\varphi(g)
= \sum_{g\in G} hga\cdot \varphi(h)^{-1}\varphi(hg) = \varphi(h)^{-1}P(a),\]
so once $P(a)\in \GL_d(L)$,
we would have $\varphi(g) = P(a)\left( hP(a ) \right)^{-1} = \left( P(a)^{-1} \right)^{-1}h(P(a)^{-1})$.
Let $\vec{X} = \{X_g\}_{g\in G}$ be a set of variables. Consider
\[    Q(\vec{X}) := \det\left( \sum_{g\in G}X_g\varphi(g) \right)\in L[\vec{X}].\]
Note that $Q(\{g(\cdot )\}_{g\in G}) : L\to L$ is a polynomial in automorphisms of $L$, and $Q(\{ga\}_{g\in G}) = \det P(a)$.
The polynomial $Q\ne 0$ because, for instance,
$Q$ evaluated at $(X_1, \dots) = (1, 0, \dots, 0)$ is
$\det \varphi(1) = 1$.

\begin{itemize}
\item \textit{$K$ infinite.}
By Artin's \cref{artin theorem on algebraically independence of characters},
$Q(\{g(\cdot )\}_{g\in G})\ne 0$, hence $\exists a\in L$ s.t. $\det P(a)\ne 0$.

\item \textit{$K$ finite.}
% Note that when $K$ is finite,
% point-wise multiplication of $\sigma, \tau\in\Gal(L/K)$
% is still in $\gal(L/K)$:
In this case,
the point-wise multiplication of finitely many $g\in\gal(L/K)$
takes the form $x\mapsto x^{n}$ for some $n\in\Z$,
which is still a multiplicative map $L\to L$.
Hence $Q(\{g(\cdot )\}_{g\in G})$ is a linear combination
of characters, and we can apply Dirichlet's \cref{dedekind theorem on linearly independence of characters}.
\qedhere
\end{itemize}


\end{proof}

\subsubsection{Normal Basis and \texorpdfstring{$H^r(G, L)$}{Hr(G, L)}}

\begin{theorem}[Normal basis theorem]\label{normal basis theorem for finite galois}
    Any finite Galois extension $L/K$ admits a normal basis; i.e, $\exists x\in L$ s.t. $\{\sigma x\mid \sigma\in\gal(L/K)\}$ forms a $K$-basis of $L$.
\end{theorem}
% Again, we prove this in two cases: 1) $K$ is infinite
% and 2) $L/K$ is finite cyclic.
% \begin{proof}[Proof in case $K$ infinite]
%     (T.B.C.)
% \end{proof}

% \begin{proof}[Proof in case $G$ cyclic]
%     (T.B.C.)
% \end{proof}

\begin{proposition}
    $L$ is an induced $G = \gal(L/K)$-module, hence
    $H^r(G, L) = 0$ for all $r \ge 1$.
\end{proposition}
\begin{proof}
    By \cref{normal basis theorem for finite galois},
    we choose $x\in L$ with $L = \bigoplus_{g\in G} Kgx$, giving an isomorphism \[K[G]\to L,\quad \sum_{g\in G}a_gg\to\sum_{g\in G}a_ggx\]
    as $G$-modules. Hence as a $G$-module,
    $L\simeq K[G]\simeq K\otimes_\Z \Z[G]\simeq\Ind^G(K)$.
\end{proof}
\begin{remark}
    We can use $H^1(G, \GL_2(L)) = 0$ to deduce that $H^1(G, L) = 0$ via the following trick:
    a cocycle $c : G\to L$ defines a cocycle \[\begin{pmatrix}
        1 & c \\ & 1
    \end{pmatrix} : G\to\GL_2(L).\]
    Hence, 
\end{remark}

\begin{corollary}\label{Hilbert 90 - additive}
    Let $L/K$ be a finite cyclic extension, $\sigma$ a generator of $G$, and $a\in L$.
    If $\tr_{L/K}a = 0$,
    then $\exists b\in L$ s.t. $a = \sigma b - b$.
\end{corollary}
\begin{proof}
    For the $G$-module $L$, the norm map \[N_G = \tr_{L/K} : x\mapsto \sum_{g\in G}gx.\]
    Now use $H^{1}(G, L) \simeq \hat H^{-1}(G, L)$.
\end{proof}

\subsubsection{Kummer Theory}


\end{document}