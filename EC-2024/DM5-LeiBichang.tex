\documentclass{article}
\usepackage{amsmath, amssymb, amsthm, amsbsy, mathrsfs, stmaryrd}
\usepackage{enumitem}
\usepackage[colorlinks,
linkcolor=cyan,
anchorcolor=blue,
citecolor=blue,
]{hyperref}
\usepackage[capitalize]{cleveref}
\usepackage[margin = 1in, headheight = 12pt]{geometry}
\usepackage{bbm}
\usepackage{tikz-cd}

\linespread{1.2}

\newtheorem{theorem}{Theorem}

\theoremstyle{definition}
\newtheorem{definition}{Definition}
\newtheorem{exercise}{Exercise}[section]
\newtheorem{problem}{Problem}
\newtheorem{example}{Example}
\newtheorem{proposition}{Proposition}[section]
\newtheorem{lemma}{Lemma}
\newtheorem{corollary}{Corollary}[section]

\theoremstyle{remark}
\newtheorem*{remark}{Remark}

\renewcommand{\Re}{\mathop{\mathrm{Re}}}
\renewcommand{\Im}{\mathop{\mathrm{Im}}}

% 新命令
% 数学对象
    \newcommand{\R}{\mathbb{R}}
    \newcommand{\C}{\mathbb{C}}
    \newcommand{\Q}{\mathbb{Q}}
    \newcommand{\Z}{\mathbb{Z}}
    \DeclareMathOperator{\GL}{GL}
    \DeclareMathOperator{\SL}{SL}
    \newcommand{\p}{\mathfrak{p}}
    \renewcommand{\P}{\mathbb{P}}
    \newcommand{\A}{\mathbb{A}}
    \newcommand{\F}{\mathbb{F}}
% 集合
    \newcommand{\sminus}{\smallsetminus} %(集合)差
% 范畴
    \newcommand{\op}[1]{{#1}^{\mathrm{op}}} %反范畴
    \DeclareMathOperator{\enom}{End} %自态射
    \DeclareMathOperator{\isom}{Isom} %同构
    \DeclareMathOperator{\aut}{Aut} %自同构
    \DeclareMathOperator{\im}{im} %像
    \newcommand{\Set}{\mathbf{Set}} %集合范畴
    \newcommand{\Abel}{\mathbf{Ab}} %群范畴
    \newcommand{\Ring}{\mathbf{Ring}}
    \newcommand{\Cring}{\mathbf{CRing}}
    \newcommand{\Alg}{\mathbf{Alg}}
    \newcommand{\Mod}{\mathbf{Mod}}
    \DeclareMathOperator{\Id}{id}
%向量空间, 矩阵
    \DeclareMathOperator{\rank}{rank} %秩
    \DeclareMathOperator{\tr}{Tr} %迹
    \newcommand{\tran}[1]{{#1}^{\mathrm{T}}} %转置
    \newcommand{\ctran}[1]{{#1}^{\dagger}} %共轭转置
    \newcommand{\itran}[1]{{#1}^{-\mathrm{T}}} %逆转置
    \newcommand{\ictran}[1]{{#1}^{-\dagger}} %逆共轭转置
    \DeclareMathOperator{\codim}{codim} %余维数
    \DeclareMathOperator{\diag}{diag} %对角阵
    \newcommand{\norm}[1]{\left\| #1\right\|} %范数
    \DeclareMathOperator{\lspan}{span} %张成
    \DeclareMathOperator{\sym}{\mathfrak{Y}}
% 群
    \DeclareMathOperator{\inn}{Inn} %(群)内自同构
    \newcommand{\nsg}{\vartriangleleft} %正规子群
    \newcommand{\gsn}{\vartriangleright} %正规子群
    \DeclareMathOperator{\ord}{ord} %元素的阶
    \DeclareMathOperator{\stab}{Stab} %稳定化子
    \DeclareMathOperator{\sgn}{sgn} %符号函数
% 环, 域
    \DeclareMathOperator{\cha}{char} %特征
    \DeclareMathOperator{\spec}{Spec} %素谱
    \DeclareMathOperator{\maxspec}{MaxSpec} %极大谱
    \DeclareMathOperator{\gal}{Gal}
% 微积分
    % \newcommand*{\dif}{\mathop{}\!\mathrm{d}} %(外)微分算子
% 流形
    \DeclareMathOperator{\lie}{Lie}
%代数几何
    \DeclareMathOperator{\proj}{Proj}
%多项式
    \DeclareMathOperator{\disc}{disc} %判别式
    \DeclareMathOperator{\res}{res} %结式

% 结构简写
    \newcommand{\pdfrac}[2]{\dfrac{\partial #1}{\partial #2}} %偏微分式
    \newcommand{\isomto}{\stackrel{\sim}{\rightarrow}} %有向同构
    \newcommand{\gene}[1]{\left\langle #1 \right\rangle} %生成对象
% 文字缩写
    \newcommand{\opin}{\;\mathrm{open\;in}\;}
    \newcommand{\st}{\;\mathrm{s.t.}\;}
    \newcommand{\ie}{\;\mathrm{i.e.,}\;}

% 重定义命令
\renewcommand{\hom}{\mathop{Hom}}
\renewcommand{\vec}{\boldsymbol}
\renewcommand{\and}{\;\text{and}\;}
\renewcommand{\div}{\mathop{\mathrm{div}}}
\DeclareMathOperator{\Div}{Div}
\newcommand{\tor}{\mathrm{tor}}

% 编号
\newcommand{\cnum}[1]{$#1^\circ$} %右上角带圆圈的编号
\newcommand{\rmnum}[1]{\romannumeral #1}
\newcommand{\m}{\mathfrak{m}}

\newcommand{\myit}{$\diamond$}
\newcommand{\ns}{\mathrm{ns}}

\title{Elliptic Curves}
\author{Lei Bichang}
% \date{Sep 26}

\begin{document}
\maketitle

\subsection*{Exercise 1}
\begin{enumerate}
\item [(a)] 
For a finitely generated abelian group $G$,
denote by $\rank G$ the rank of $G$.

Let $\phi : E_1\to E_2$ be a non-constant isogeny over $K$.
Then $\phi$ induces a map
\[\phi_K : E_1(K)\to E_2(K),\]
which is clearly a group homomorphism.
This gives an injection \[E_1(K)/\ker\phi_K\hookrightarrow E_2(K)\] of abelian groups of finite type.
So $\rank (E_1(K)/\ker\phi_K)\le \rank E_2(K)$.
Since $\ker\phi_K \subset \ker\phi$ is finite,
we have \[\rank E_1(K) = \rank (E_1(K)/\ker\phi_K).\]
Hence $\rank E_1(K)\le \rank E_2(K)$.
Doing the same thing to a non-constant isogeny $E_2\to E_1$ over $K$,
say $\hat{\phi}$\footnote{
I don't recall if we have shown in class that: if $\phi$ is defined over $K$,
then $\hat{\phi}$ is defined over $K$.
This can be proved by checking directly that: all the three maps in
\[E_2\to\Div_0(E_2)\stackrel{\phi^*}{\to}\Div_0(E_1)\to E_1\]
are $G_K$-invariant.},
we get $\rank E_2(K)\le \rank E_1(K)$.
So the ranks of $E_1$ and $E_2$ are equal.
\item [(b)]
No.
I checked on LMFDB that $E_1 : y^2 = x^3 + x$ has rank $0$, and $E_2 : y^2 = x^3 + 3x$ has rank $1$.
But $E_1$ and $E_2$ are isogenous via \[x\mapsto u^{2}x,\ y\mapsto u^3y,\quad u = \sqrt[4]{3}\]
over $\Q(u)$.

\end{enumerate}

\subsection*{Exercise 2}
\begin{enumerate}
\item [(a)] $E : y^2 = x(x^2 + 3x + 5)$.\par
$a = 3,\ b = 5,\ a_1 = -2a = -6,\ b_1 = a^2-4b = -11$.
\begin{itemize}
\item \textbf{Determine $\psi(E'(\Q)/\phi(E(\Q)))$}.\par
The integers $r\mid b_1$ are \[r = \pm 1, \pm 11.\]
Write \[\begin{cases}
    u = rt^2,\\  u^2 + a_1u + b_1 = \dfrac{v^2}{u} = rs^2,
\end{cases}\quad t = \frac{l}{m},\; (l, m) = 1,\quad s = \frac{n }{m^2}. \]
which gives the equation\begin{equation}\label{eq: 2-1 phi}
    r^2l^4 + a_1rl^2m^2 + b_1m^4 = rn^2,
\end{equation}
i.e, $r^2l^4 -6rl^2m^2 -11m^4 = rn^2$.
The value $r = -11 = b_1 = a^2 - 4b$ corresponds to $(0, 0)$.
Since $\im q$ is a group, it must be $\{[1], [-11]\}$ or $\{[1], [-11], [-1], [11]\}$.

Substitute $r = -1$ in \cref{eq: 2-1 phi} gives
\begin{equation}\label{eq: 2-1 phi r=-1}
    l^4 + 6l^2m^2 - 11m^4 = -n^2,
\end{equation}
which has a solution $(l, m, n) = (1, 1, 2)$,
corresponding to \[(u, v) = \left( \frac{rl^2}{m^2}, \frac{rnl}{m^3} \right) = (-1, -2)\in E'(\Q).\]
The image of $(-1, -2)$ in $E''(\Q)$ is
\[\psi(u, v) = \left( u + a_1 + \frac{b_1}{u}, v - \frac{b_1v}{u^2} \right) = (4, -24).\]
The isomorphism $E''\to E$ is \[x = x''/4,\quad y = y''/8,\]
so the corresponding point in $E(\Q)$ is $(1, -3)$.

\item \textbf{Determine $E(\Q)/\psi(E'(\Q))$}.\par
Next, solve \begin{equation}\label{eq: 2-1 psi}
    r^2l^4 + 3rl^2m^2 + 5m^4 = rn^2
\end{equation}
for $r\mid 5$.
The value $r = b = 5$ corresponds to $(0, 0)$.
Because $a^2 - 4b < 0$, we have $rs^2 = u^2 + au + b > 0$, so $r > 0$. Hence $[-1], [-5]\notin \im q'$,
and thus $(0, 0)$ generates $E(\Q)/\psi(E'(\Q))$.

\end{itemize}

Finally, $E(\Q)/2E(\Q) = \gene{(0, 0), (1,-3)}\simeq \left( \Z/2\Z \right)^2$.
\begin{itemize}
\item [\textbf{Rank}.]
The rank of $E$ is $1$.
Since $(0, 0)$ has order $2$,
the rank of $E$ is $0$ or $1$, depending on whether $(1, -3)$ has finite order or not.
I don't know how to do this by hand without spending too much time and ink, but using Sage I can tell that $(1, -3)$ has infinite order by computing $iP$ for $2\le i\le 12$ or by letting the program tell me its order directly.

\end{itemize}


% $\implies \{0, 1\}\ni l^4  = -m^4\bmod 3$
\item [(b)] $E:y^2=x(x^2-2x+9)$.\par
$a = -2,\ b = 9,\ a_1 = -2a = 4,\ b_1 = a^2 - 4b = -32$.
\begin{itemize}
\item Solve \begin{equation}
    rl^4 + 4l^2m^2  - \frac{32}{r} m^4 = n^2
\end{equation}
for $r\mid 32$ square-free, that is $r = \pm1, \pm2$. $[r] = [-2] = [-32]$ corresponds to $(0, 0)$,
so $\im q = \{[1], [-2]\}$ or $\{[1], [-2], [-1], [2]\}$.
Let $r = 2$, so that \begin{equation}\label{eq: 2-2 phi r=2}
    2l^4 + 4l^2m^2 - 16m^4 = n^2.
\end{equation}
Completing the square then modulo $3$
\[\implies\{0, 2\}\ni 2(l^2 + m^2)^2 = n^2 \in \{0, 1\}\bmod 3,\]
\[\implies l^2\equiv m^2\equiv n^2\equiv 0\bmod 3,\]
$\implies 3\mid l$ and $3\mid m$, contradicting $(l, m) = 1$.
Hence \cref{eq: 2-2 phi r=2} has no nontrivial solution in $\Z^3$.
\item Solve\begin{equation}
    rl^4 - 2l^2m^2 + \frac{9}{r}m^4 = n^2
\end{equation}
for $r\mid 9$ square free, i.e., $r = \pm1, \pm 3$.
$[r] = [1] = [9]$ corresponds to $(0, 0)$.
$b_1 = a^2 - 4b < 0$,
so $rs^2 = u^2 + au + b > 0$.
Thus it remains to check $r = 3$:
\begin{equation}
    3l^4 - 2l^2m^2 + 3m^4 = n^2.
\end{equation}
This equation has solution $(l,m,n) = (1,1,2)$,
corresponding to \[(u, v) = \left( \frac{rl^2}{m^2},\frac{rln}{m^3} \right) = (3, 6)\in E(\Q).\]
Since $(0, 0)$ corresponds to the identity $[1]$, we have $E(\Q)/\psi(E'(\Q)) = \gene{(3, 6)}$.
\end{itemize}
So $E(\Q)/2E(\Q) = \gene{(3, 6)}\simeq\Z/2\Z$.
\begin{itemize}
\item [\textbf{Rank}.]
The rank of $E$ is $0$, because $(0, 0)$ has order $2$.
\end{itemize}

\item [(c)] $E:y^2=x(x^2+2x+9)$.\par
$a = 2,\ b = 9,\ a_1 = -2a = -4,\ b_1 = a^2-4b = -32$.
\begin{itemize}
\item Solve \begin{equation}
    rl^4 - 4l^2m^2  - \frac{32}{r} m^4 = n^2
\end{equation}
for $r = \pm1, \pm2$.
$[r] = [-1] = [-32]$ corresponds to $(0, 0)$.
Let $r = 2$, then
\begin{equation}\label{eq: 2-3 phi r=2}
    2l^4 - 4l^2m^2 - 16m^4 = n^2.
\end{equation}
has a solution $(2, 1, 0)$,
corresponding to \[(u, v) = \left( 8,0 \right)\in\psi^{-1}((0, 0))\subset E'(\Q).\]

\item Solve\begin{equation}
    rl^4 + 2l^2m^2 + \frac{9}{r}m^4 = n^2
\end{equation}
for $r = \pm 1, \pm 3$.
Since $b_1 < 0$, we have $r = 1, 3$.
Let $r = 3$:
\begin{equation}\label{eq: 2-3 psi r=3}
    3l^4 + 2l^2m^2 + 3m^4 = n^2.
\end{equation}
Modulo $3$, we get \[\{0, 2\}\ni 2(lm)^2 = n^2\in\{0, 1\},\]
\[\implies (lm)^2 = n^2 = 0\mod 3,\]
$\implies 3\mid lm$ and $3\mid n$.
If $3\mid l$, then \cref{eq: 2-3 psi r=3} shows that $3^2\mid 3m^4$, so $3\mid m$, which is a contradiction. Similarly, $3\mid m\implies 3\mid l$ and leads to contradiction.
Therefore, \cref{eq: 2-3 psi r=3} has no nontrivial integer solution.
\end{itemize}
So $E(\Q)/2E(\Q) = \gene{(0, 0)}\simeq \Z/2\Z$.
\begin{itemize}
\item [\textbf{Rank}.]
The rank of $E$ is $0$, because $(0, 0)$ is a point of order $2$.
\end{itemize}

\end{enumerate}

\subsection*{Exercise 3}
\begin{enumerate}
\item [(a)] A finitely generated abelian group has finitely many torsion elements.
If $K$ is algebraically closed,
then $E(K)[n] = E[n] \simeq \Z/n\Z\times \Z/n\Z$
for all integers $n$ that are not divided by $\cha K$.
Therefore $E(K)_\tor = \bigcup_{n\ge 1}E(K)[n]$ cannot be finite, thus $E(K)$ is not of finite type.
\item [(b)]
For a set $S$, denote by $|S|$ the cardinality of a set $S$.

A finitely generated abelian group is finite or countable. So to prove that $E(\R)$ is not of finite type, it suffices to show that $E(\R)$ is uncountable.

As $\cha \R = 0$, we may assume that $E$ is defined by $y^2 = f(x)$, where $f(X) = X^3 + aX +b\in \R[X]$.
Then $f(\R) = \R$. So for every $y\in\R$, there exists $x\in\R$ s.t. $(x, y)\in E$.
This means that the map \[E(\R)\setminus \{O\}\to \R,\ (x, y)\mapsto y\] is surjective, and thus $|E(\R)|\ge |\R| > \aleph_0$.

\item [(c)]
Similar to (b), we show that $E(\Q_p)$ is uncountable using Hensel's lemma.

Assume that $E$ is given by a minimal Weierstrass equation $F(x, y) = 0$,
where $F(X, Y)\in \Z_p[X, Y]$, so that the curve $\tilde{E}$ is given by $\tilde{F}(x, y) = 0$.
Let $\pi : E_0(\Q_p)\to\tilde{E}_\ns(\F_p)$ be the reduction map.
Take $P_0 = (x_0, y_0)\in \tilde{E}_\ns(\F_p)\setminus\{O\}\ne\varnothing$.
By the definition of singularity,
\[\pdfrac{\tilde{F}}{X}(x_0, y_0) \ne 0
\quad\text{   or   }\quad
\pdfrac{\tilde{F}}{Y}(x_0, y_0) \ne 0.\]
\begin{itemize}
\item Assume first that $\pdfrac{\tilde{F}}{X}(x_0, y_0) \ne 0$.
Denote by $a\mapsto\bar{a}$ the quotient map $\Z_p\to\F_p$.

Let $y\in\Z_p$ be any lift of $y_0\in\F_p$,
and let \[f_y(X) := F(X, y)\in \Z_p[X].\]
Then modulo $p$, we have $\overline{f_y(x_0)} = 0$ in $\F_p$,
and \[\overline{f_y'(x_0)} = \overline{\pdfrac{F(X, Y)}{X}}(x_0, y_0) = \pdfrac{\tilde{F}}{X}(x_0, y_0)\ne 0.\]
So by Hensel's lemma, there is a unique $x\in\Z_p$ s.t. $F(x, y) = f_y(x) = 0$.

The set \[y + p\Z_p = \{z\in\Z_p\mid \bar{z} = y_0\}\] has cardinality equal to $\Z_p$,
which is an uncountable set.
The above construction gives an injection $y + p\Z_p\hookrightarrow E_0(\Q_p)$.
Therefore, there are uncountably many points in $E_0(\Q_p)\subset E(\Q_p)$.

\item If $\pdfrac{\tilde{F}}{Y}(x_0, y_0) \ne 0$, we can argue in a much similar way that $E(\Q_p)$ is uncountable.
\end{itemize}

\end{enumerate}











\end{document}