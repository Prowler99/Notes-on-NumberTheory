\documentclass{article}

% \documentclass{article}
\usepackage{fontspec}
\usepackage{amsmath, amssymb, amsthm, amsbsy, mathrsfs}
\usepackage{stmaryrd}
\usepackage{enumerate}

\usepackage{xr-hyper} % 支持跨文件的超链接
\usepackage[colorlinks,
linkcolor=cyan,
anchorcolor=blue,
citecolor=blue,
]{hyperref}
\usepackage[capitalize]{cleveref}

\externaldocument[UND-]{../Undergrad/something}
\externaldocument[ANT-]{../ANT-2022-summer/note_summer_ANT}
\externaldocument[EC-]{../EC-2024/notes_elliptic_curves}
\externaldocument[LF-]{../LocalField-2024/notes_cft}
\externaldocument[MF-]{../ModularForm-2024-fall/note_modularForm}

\usepackage[margin = 1in, headheight = 12pt]{geometry}
\usepackage{bbm}
\usepackage{tikz-cd}

% \setmainfont{Arial}

\linespread{1.2}

\theoremstyle{definition}

\newtheorem{theorem}{Theorem}

% This environment is terrible
\newenvironment{theoremprime}[1]
  {\renewcommand{\thetheorem}{\ref{#1}$'$}%
   \addtocounter{theorem}{-1}%
   \begin{theorem}}
  {\end{theorem}}

\newtheorem{definition}{Definition}
\newtheorem{propdef}[definition]{Prop-Def}
\newtheorem{exercise}{Exercise}[section]
\newtheorem{problem}{Problem}[section]
\newtheorem{example}{Example}[section]
\newtheorem{proposition}{Proposition}[section]
\newtheorem{lemma}{Lemma}[section]
\newtheorem{corollary}{Corollary}[section]

\theoremstyle{remark}
\newtheorem*{remark}{Remark}

\renewcommand{\Re}{\mathop{\mathrm{Re}}}
\renewcommand{\Im}{\mathop{\mathrm{Im}}}
\renewcommand{\bar}{\overline}
\renewcommand{\tilde}{\widetilde}
\renewcommand{\hat}{\widehat}

% 新命令
% 数学对象
    \newcommand{\R}{\mathbb{R}}
    \newcommand{\C}{\mathbb{C}}
    \newcommand{\Q}{\mathbb{Q}}
    \newcommand{\Z}{\mathbb{Z}}
    \DeclareMathOperator{\GL}{GL}
    \DeclareMathOperator{\SL}{SL}
    \newcommand{\p}{\mathfrak{p}}
    \renewcommand{\P}{\mathbb{P}}
    \newcommand{\A}{\mathbb{A}}
    \newcommand{\Ga}{{\mathbb{G}_{\mathrm{a}}}}
    \newcommand{\Gm}{{\mathbb{G}_{\mathrm{m}}}}
% 集合
    \newcommand{\sminus}{\smallsetminus} % (集合)差
    \newcommand{\inject}{\hookrightarrow} % 单射
    \newcommand{\surject}{\twoheadrightarrow}
% 范畴
    \newcommand{\op}[1]{{#1}^{\mathrm{op}}} % 反范畴
    \DeclareMathOperator{\Hom}{Hom}
    \DeclareMathOperator{\enom}{End} % 自态射
    \DeclareMathOperator{\End}{End} % 自态射
    \DeclareMathOperator{\isom}{Isom} % 同构
    \DeclareMathOperator{\aut}{Aut} % 自同构
    \DeclareMathOperator{\im}{im} % 像
    \newcommand{\Set}{\mathbf{Set}} % 集合范畴
    \newcommand{\Abel}{\mathbf{Ab}} % 群范畴
    \newcommand{\Ring}{\mathbf{Ring}}
    \newcommand{\Cring}{\mathbf{CRing}}
    \newcommand{\Alg}{\mathbf{Alg}}
    \newcommand{\Mod}{\mathbf{Mod}}
    \DeclareMathOperator{\Id}{id}
%向量空间, 矩阵
    \DeclareMathOperator{\rank}{rank} % 秩
    \DeclareMathOperator{\tr}{Tr} % 迹
    \newcommand{\tran}[1]{{#1}^{\mathrm{T}}} % 转置
    \newcommand{\ctran}[1]{{#1}^{\dagger}} % 共轭转置
    \newcommand{\itran}[1]{{#1}^{-\mathrm{T}}} % 逆转置
    \newcommand{\ictran}[1]{{#1}^{-\dagger}} % 逆共轭转置
    \DeclareMathOperator{\codim}{codim} % 余维数
    \DeclareMathOperator{\diag}{diag} % 对角阵
    \newcommand{\norm}[1]{\left\| #1\right\|} % 范数
    \DeclareMathOperator{\lspan}{span} % 张成
% 群
    \DeclareMathOperator{\inn}{Inn} % (群)内自同构
    \newcommand{\nsg}{\vartriangleleft} % 正规子群
    \newcommand{\gsn}{\vartriangleright} % 正规子群
    \DeclareMathOperator{\ord}{ord} % 元素的阶
    \DeclareMathOperator{\stab}{Stab} % 稳定化子
    \DeclareMathOperator{\sgn}{sgn} % 符号函数
% 环, 域
    \DeclareMathOperator{\cha}{char} % 特征
    \DeclareMathOperator{\spec}{Spec} % 素谱
    \DeclareMathOperator{\maxspec}{MaxSpec} % 极大谱
    \newcommand{\spm}{\maxspec} %极大谱
    \DeclareMathOperator{\gal}{Gal} % Galois群
    \DeclareMathOperator{\Gal}{Gal} % Galois群
    \DeclareMathOperator{\Frac}{Frac} % 分式环
% 同调代数
    \DeclareMathOperator{\ext}{Ext} 
    \DeclareMathOperator{\Tor}{Tor}
% 微积分
    \newcommand*{\dif}{\mathop{}\!\mathrm{d}} %(外)微分算子
% 流形
    \DeclareMathOperator{\lie}{Lie}
%代数几何
    \DeclareMathOperator{\proj}{Proj} % 射影谱
%多项式
    \DeclareMathOperator{\disc}{disc} % 判别式
    \DeclareMathOperator{\res}{res} % 结式
% 结构简写
    \newcommand{\pdfrac}[2]{\dfrac{\partial #1}{\partial #2}} % 偏微分式
    \newcommand{\isomto}{\stackrel{\sim}{\rightarrow}} % 有向同构
    \newcommand{\gene}[1]{\left\langle #1 \right\rangle} % 生成对象
% 文字缩写
    \newcommand{\opin}{\;\mathrm{open\;in}\;}
    \newcommand{\st}{\;\mathrm{s.t.}\;}
    \newcommand{\ie}{\;\mathrm{i.e.,}\;}

% 重定义命令
\renewcommand{\hom}{\mathop{\mathrm{Hom}}}
\renewcommand{\vec}{\boldsymbol}
\renewcommand{\and}{\;\text{and}\;}

% 编号
\newcommand{\cnum}[1]{$#1^\circ$} %右上角带圆圈的编号
\newcommand{\rmnum}[1]{\romannumeral #1}
\newcommand{\myit}{$\diamond$}

% \tikzcdset{scale cd/.style={every label/.append style={scale=#1},
%     cells={nodes={scale=#1}}}}


\newcommand{\F}{\mathbb{F}}
\newcommand{\perf}{\mathrm{perf}}
\newcommand{\sep}{\mathrm{sep}}
\renewcommand{\O}{\mathcal{O}}
\newcommand{\m}{\mathfrak{m}}
\DeclareMathOperator{\pic}{Pic}
\DeclareMathOperator{\frob}{Frob}
\newcommand{\llpar}{(\!(}
\newcommand{\rrpar}{)\!)}
% todo: define (())
\newcommand{\ab}{\mathrm{ab}}

\title{Notes on Drinfeld Modules and Explicit CFT for Function Fields}

\begin{document}
\maketitle

Pre-date: March 10! It is close!

1) Give a 30min (strict limit !!!) talk. Ideally more like 25min + 5 min for questions.  The talks will be in March. I will try to reserve a room, and will give a more precise time/date when possible.

2) Write an ``extended summary" (meaning around ~5 pages NOT!!! >=10) of you article. It should summarise the article and its main ideas and be accessible to advanced Master students (i.e., the other students in this group).

\section{Review on CFT}
Let $F$ be a global field, $C_F = \A_F^\times/F^\times$ be its idele class group, and $F^\ab$ be its maximal abelian extension inside a separable closure in a fixed algebraic closure $\bar F$.
The class field theory asserts that the Artin map
\[\theta_F : C_F\to\gal(F^\ab/F)\]
is a continuous group homomorphism with dense image,
establishing a bijection
\[\{\text{finite abelian extensions of }F\}\longleftrightarrow \{\text{finite index open subgroups of }C_F\}.\]
The direction ``$\to$'' is computable: for a finite abelian $L/F$,
the composition $C_F\stackrel{\theta_F}{\to}\gal(F^\ab/F)\to\gal(L/F)$ is surjective,
and its kernel $U = N_{L/F}(C_L)$ is the corresponding open subgroup of $C_F$, where $N_{L/F} : C_L\to C_F$ is the norm map\footnote{
    The norm for a idele is just the multiplication of the norm at every places.}.
But the other direction ``$\gets$'' is not known in general:
given a finite index open subgroup of $C_F$,
the Artin map $\theta_F$ doesn't produce the generators of the corresponding extension $L/F$.

The goal of explictit class field theory is to find this inverse.
% \subsection{Example: explicit class field theory for \texorpdfstring{$\Q$}{Q}}



\section{Function Fields and Drinfeld Modules}
Let $F$ be a global function field with a fixed place $\infty$, and with field of constants $k = \F_q$.
If $\lambda$ is a place of $F$,
we denote by $F_\lambda$ the completion at $\lambda$,  by $\O_\lambda\subset F_\lambda$ the valuation ring,
by $\F_\lambda := \O_\lambda/\m_\lambda$ the residue field at $\lambda$.
% and by $N(\lambda) := \#\F_\lambda$ its cardinality.
Since we are working with function fields, the Teichm\"uller lifting $\F_\lambda\hookrightarrow \O_\lambda$ is a field homomorphism;
we regard $\F_\lambda\subset\O_\lambda\subset F_\lambda$ as a subfield via this embedding.
% any uniformizer $t\in\m_\lambda$ yields an isomorphism
% \[\O_\lambda\simeq \F_\lambda[t],\]
% giving us an embedding $\F_\infty\hookrightarrow \O_\lambda$.


For any extension $L$ of $k$, we fixed an algebraic closure $\bar{L}$.

\subsection{Function fields}

\subsubsection{holomorphy ring}
Let $S$ be a non-empty set of (not all the) places of $F$.
Define \[\O^S := \bigcap_{\lambda\notin S} \O_\lambda = \{x\in F\mid \ord_\lambda(x)\ge 0,\ \forall\lambda\notin S\}\]
to be the subring of $F$ consisting of elements regular away from $S$.
A \textbf{holomorphy ring} is a ring of this form.
For example, our $A = \O^{\{\infty\}}$ is a holomorphy ring.

\begin{proposition}
Consider a holomorphy ring $\O^S$.
\begin{enumerate}[(1)]
    \item $\Frac(\O^S) = F$.
    \item $\O^S$ is a Dedekind domain.
    \item There is a bijection \[\{\text{place of }F\text{ not in }S\}\longleftrightarrow \spm \O^S\] giving by $\lambda\mapsto \m_\lambda\cap\O^S$, which induces isomorphisms \[\F_\lambda = \O_\lambda/\m_\lambda\simeq \O^S/(\m_\lambda\cap\O^S)\]
\end{enumerate}
\end{proposition}
So we can regard $\lambda$ as a maximal ideal of $A$.

\subsubsection{The Weil group}
Let $L$ be an extension of $k$.
The field $k$ is perfect,
so the algebraic closure $\bar k$ of $k$ in $\bar F$ is contained in $L^\sep$,
and the absolute Galois group $\gal_L = \gal(F^\sep/F)$ stablizes $\bar k$.
Hence we have
an exact sequence of topological groups
\[1\longrightarrow\gal(L^\sep/L\bar{k})\longrightarrow \gal_L\stackrel{\deg}{\longrightarrow}\hat\Z\to 0,\]
where $\deg : \gal_L\to \gal_k\simeq\hat\Z$ is defined by
\[\sigma(x) = \frob_q^{\deg(\sigma)}(x),\quad \sigma\in \gal_L,\ x\in\bar{k}.\]

The \textbf{Weil group} is the subgroup $W_L$ of $\gal_L$ of elements that acts on $\bar{k}$ by an integral power of the Frobenius-$q$, i.e. \[\sigma(x) = x^{q^{\deg(\sigma)}},\quad \sigma\in W_L,\ x\in\bar{k}.\]
The kernel of the map $\deg : W_L\to\Z$ is still $\gal(L^\sep/L\bar{k})$.
We endow $W_L$ with the weakest topology for which
\[1\longrightarrow \gal(L^\sep/L\bar{k})\longrightarrow W_L\stackrel{\deg}{\longrightarrow} \Z\longrightarrow0\]
is an exact sequence of topological groups,
where \begin{itemize}
    \item $\gal(L^\sep/L\bar{k})$ has its usual profinite topology,
    \item $\Z$ has discrete topology\footnote{This is not the topology induced from $\Z\subset \hat\Z$.}.
\end{itemize}
With respect to this topology,
the inclusion $W_L\hookrightarrow \gal_L$ is continuous with dense image. (?)


\subsection{Definition of Drinfeld modules}
\subsubsection{Endomorphisms of the additive group}
Consider the additive group $\Ga_{/L}$ over $L$,
which is not only a group scheme, but also a $k$-vector space scheme,
and we consider the ring $\End_{k}(\Ga_{/L})$ of all $k$-linear endomorphism.
\begin{proposition}
    $\End_{k}(\Ga_{/L}) = L[\tau]$, where $\tau$ is the Frobenius-$q$ endomorphism.
\end{proposition}
We explain the notation in the proof.
\begin{proof}
    An endomorphism $\Ga\to\Ga$ of schemes over $L$ is given by an $L$-algebra homomorphism $\Phi: L[X]\to L[X]$,
    hence it is determined by the image $\varphi(X) = \Phi(X)$\footnote{
        Note that if $\varphi(X) = a_nX^n + \cdots + a_0$,
        then \[\varphi(f(X)) = a_nf(X)^n  + \cdots + a_0\] and
        \[\Phi(f(X)) = f(\Phi(X)) = f(\varphi(X))\] are \textit{\textbf{different}} in general.
    } of $X$.
    It respects the group-scheme structure if it commutes with the co-multiplication map (also an $L$-algebra homomorphism) \[\Delta : F[X]\to F[X]\otimes_L F[X],\quad X\mapsto X\otimes 1 + 1\otimes X.\]
    which amounts to\[(\Phi\otimes\Phi)(\Delta(X)) = (\Phi\otimes\Phi)(X\otimes 1 + 1\otimes X) =  \Phi(X)\otimes 1 + 1\otimes\Phi(X) = 
    \varphi(X)\otimes 1 + 1\otimes\varphi(X)\]
    equals \[\Delta(\Phi(X)) = \Delta(\varphi(X)) = \varphi(\Delta(X)) = \varphi(X\otimes 1 + 1\otimes X).\]
    % Since the tensor is over $L$ and $\varphi$ has coefficients in $L$,
    This is to say that\footnote{
        Recall that the multiplicative structure on $B\otimes_A C$ is given by \[(b\otimes b')\cdot(c\otimes c') = bb'\otimes cc'.\]
    } $\varphi$ is additive, i.e. $\varphi(X+Y) = \varphi(X) + \varphi(Y)$.

    We require furthur that $\Phi$ respects the ``co-$k$-scalar multiplication'', which I don't have the formula right now.
    So let's use the functor point of view.
    Take $c\in k$.
    Yoneda tells us that
    \[\Hom_{[k\text{-}\cat{Alg}^\opp, \cat{Grp}]}(\Ga, \Ga) \simeq \Ga(L[X]), \quad \phi\mapsto \phi(\Id_{L[X]}),\]
    so the co-$c$-multiplication is given by $X\mapsto cX$.
    Therefore $\Phi$ respects this map if $\varphi(cX) = c\varphi(X)$.

    In conclusion,
    \begin{align*}
        \enom_{k}(\Ga_{/L}) &= \{k\text{-linear polynomials in } L[X] \}\\
        &= \left\{\left.\sum_{i}a_i X^{p^i}\right|a_i\in L,\;\sum a_icX^{p^i} = \sum a_ic^{p^i}X^{p^i},\forall c\in k = \F_q\right\}\\
        &= \left\{ \left.\sum_{i}a_i X^{q^i}\right| a_i\in L \right\}
        = \left\{ \left.\left( \sum_{i}a_i \tau^{i} \right)(X)\right| a_i\in L \right\},
    \end{align*}
    where $\tau(X) := X^q$.

    Note that $\tau : L[X]\to L[X]$ is additive,
    but doesn't commutes with elements in $L$: \[\tau a = a^q\tau,\quad\forall a\in L.\]
    Therefore $L[\tau]$ is a \textit{non-commutative} subring of $\End(L[X])$, where multiplication is composition;
    it is a ring of \textbf{twisted polynomials}. And we have $\enom_{k}(\Ga_{/L})\simeq L[\tau]$.
\end{proof}
\begin{remark}
    $\tau$ corresponds to the Frobenius-$q$ endomorphism of $\Ga_{/L}$. (What is this? $\Ga_{/L}$ is NOT over $\F_q = k$.)
\end{remark}

\subsubsection{Drinfeld modules and isogenies}
Let $A$ be a $k$-algebra.
A \textbf{Drinfeld $A$-module\footnote{
    There is more general definition, but this one suffices.
} over $L$} is a homomorphism
\[\phi : A\to L[\tau]\quad x\mapsto \phi(x) =: \phi_x\]
of $k$-algebras such that $\phi(A)$ is \textit{not contained} in $L\subset L[\tau]$.

Let $\phi$ and $\phi'$ be two Drinfeld modules $A\to L[\tau]$.
An \textbf{isogeny} over $L$ from $\phi$ to $\phi'$
is an $f\in L[\tau]\sminus\{0\}$ such that \[f\phi_a = \phi_a'f,\quad\forall a\in A.\]
An \textbf{isomorphism} over $L$ from $\phi$ to $\phi'$ is an invertible isogeny, namely an isogeny $f\in L[\tau]^\times$.
If $M/L$ is an extension,
then a Drinfeld module over $L$ induces natrually a Drinfeld module over $M$,
and we can talk about isogenies over $M$ for Drinfeld modules over $L$.



% \begin{remark}[Another interpretation]
% Let $A$ be an $k$-algebra together with a $k$-algebra homomorphism $\iota : A\to L$.
% A \textbf{Drinfeld $A$-module over $L$} is an $A$-module structure on $\Ga_{/L}$,
% such that the induced action on Lie algebra is given by $A\stackrel{\iota}{\to} L$.

% \end{remark}

Let \[\partial : L[\tau]\to L\quad \sum_i a_i\tau^i\mapsto a_0\]
be the homomorphism of taking the constant term.
We say that a Drinfeld module $\phi : A\to L[\tau]$ \textbf{has generic characteristic},
if \[\partial\circ \phi : A\to L[\tau]\surject L\]
is \textit{injective}. This implies that $\phi$ is injective.

% \subsubsection{Good reduction}
% Consider the valuation ring $\O_L$ of $L$.
% We say that an $A$-Drinfeld module $\phi$ over $L$ has \textbf{good reduction},
% if it is isomorphic over $L$ to a Drinfeld module $\psi$, such that \[\psi_x\in\O_L[\tau],\quad \mu_\psi(x)\in \O_L^\times,\quad\forall x\in A,\]
% where $\mu_\psi(x) = \text{ leading coefficient of } \psi_x$. 


\subsection{The Drinfeld modules we need}

In what follows, we take $A := \O^{\{\infty\}}\subset F$ to be the subring of $F$ consisting of functions that are regular away from $\infty$,
and we assume that every Drinfeld modules $\phi : A\to L[\tau]$ is of generic characteristic,
so that $\partial\circ \phi : A\inject L$ is injective and it extends to an embedding \[F\hookrightarrow L.\]
Through the latter, we view $F$ as a subfield of $L$.

% \subsubsection{The rank}

Let $L^\perf$ be the purely inseparable closure of $L$ in $\bar L$,
then $L^\perf((\tau^{-1}))$ is a well-defined skew-field\footnote{
    We need to have all $p$-th root,
    so that $\tau^{-1}a = a^{1/q}\tau$ is always valid.
}, containing $L[\tau]$ as a subring.


Under our assumption,
$\phi : A\hookrightarrow L[\tau]$ is injective, so it extends to a unique embedding
\[\phi : F\hookrightarrow L^{\perf}((\tau^{-1})).\]
The function \[v_\phi : F\to \Z\cup\{\infty\}\quad x\mapsto \ord_{\tau^{-1}}(\phi_x) \]
is a nontrivial\footnote{
    Because $\phi(A)\not\subset L$.
} valuation, and $v_\phi(x)\le 0$ for all $x\in A\sminus\{0\}$.
Therefore $v_\phi$ is equivalent to the valuation $\ord_\infty$ attached to the place $\infty$.
We define the \textbf{rank of $\phi$} to be the rational number $r\in\Q$ such that
\begin{equation}\label{eq: def of rank}
    \ord_{\tau^{-1}}(\phi_x) = rd_\infty\ord_\infty(x),
\end{equation}
for $x\in F$, where $d_\infty = [\F_\infty : k]$ is the inertia degree of $F$ at $\infty$.
The tank $r$ is always an integer (by a proposition we may encounter later).
Since $L^\perf((\tau^{-1}))$ is complete under $\ord_{\tau^{-1}}$, the homomorphism $\phi : F\to L^\perf((\tau^{-1}))$ gives rise to a unique homomorphism
\[\phi : F_\infty\to L^\perf((\tau^{-1}))\]
such that $\ord_{\tau^{-1}}(\phi_x) = rd_\infty\ord_\infty(x)$ for all $x\in F_\infty$.

Now the map $\phi$ restricts to a homomorphism
\[\phi : \F_\infty\subset\O_\infty\to L^\perf\llbracket \tau^{-1} \rrbracket.\]
Composing with $\partial : L^{\perf}\llbracket \tau^{-1} \rrbracket\to L^\perf$ of taking constant term,
we obtain an embedding \[\partial\circ\phi|_{\F_\infty} : \F_\infty\hookrightarrow L^\perf,\]
whose image lies in $L$ ({\color{red} why?}).


\subsection{\texorpdfstring{$\varepsilon$}{epsilon}-normalized Drinfeld modules}

Let $\phi : A\to L[\tau]$ be a Drinfeld module of rank $r$, extending to an embedding $\phi : F\to L^\perf((\tau^{-1}))$.
For $x\in F_\infty^\times$,
we define \[\mu_\phi(x) := \text{first non-zero coefficient of }\phi_x\text{ as a Laurent series in }\tau^{-1},\]
so that $\mu_\phi(x)\in (L^\perf)^\times$,
and the first term, i.e.\ the term with \textit{highest} $\tau$-order, of $\phi_x$ is
\[\mu_\phi(x)\tau^{-rd_\infty\ord_\infty(x)}.\]
In particular, if $x\in A$, $\mu_\phi(x)$ is the leading coefficient of $\phi_x\in L[\tau]$,
which is what we used before to define reduction type.


By definition, for $x, y\in F_\infty^\times$,
\[\mu_\phi(xy)=\mu_\phi(x)\mu_\phi(y)^{1/q^{rd_\infty\mathrm{~ord}_\infty(x)}}.\]
Recall that $\phi$ gives us an embedding
\[\partial\circ\phi|_{\F_\infty} : \F_\infty\hookrightarrow L\]
With respect to this embedding, {\color{red} why?}
\[\mu_\phi(x) = x, \quad\forall x\in\F_\infty\]

A \textbf{sign function for $F_\infty$} is a
group homomorphism $F_\infty^\times\to\F_\infty^\times$ such that $\varepsilon|_{\F_\infty^\times} = \Id_{\F_\infty^\times}$.
These functions can be described completely.
A uniformizer $\pi$ of $F_\infty$,
yields a decomposition \[F_\infty^\times\simeq \F_\infty\times (1 + \m_\infty)\times \pi^{\Z}.\]
If $p^r$ = cardinality of $\F_\infty$,
then $1 + \m_\infty$ is a pro-$p$ group,
but $\F_\infty^\times$ has order $p^r - 1$,
so $\varepsilon$ must be trivial on $\m_\infty$.
Therefore $\varepsilon$ is determined by its value $\varepsilon(\pi)$.

% \begin{example}
%     For a Drinfeld module $\phi$, the function $\mu_\phi : $
% \end{example}

Let $\varepsilon : F_\infty\to\F_\infty$ be a sign function for $F_\infty$.
We say that $\phi$ is \begin{itemize}
    \item \textbf{normalized}, if \[\mu_\phi(x)\in \F_\infty,\quad \forall x\in F_\infty,\]
    \item $\varepsilon$\textbf{-normalized},
    if \[\exists\sigma\in\aut_k(\F_\infty),\quad \phi = \sigma\circ\varepsilon.\]
\end{itemize}

\begin{lemma}
    Let $\varepsilon$ be a sign function for $F_\infty$.
    Any Drinfeld module over $L$ is isomorphic over $\bar L$ to some $\varepsilon$-normalized Drinfeld module.
\end{lemma}


\subsection{The action of an ideal on a Drinfeld module}
Let $\phi : A\to L[\tau]$ be a Drinfeld module.
For an ideal $\mathfrak{a}$ of $A$, Define 
\[I_{\mathfrak{a}, \phi} := \text{ ideal of } L[\tau] \text{ generated by } \{\phi_a\mid a\in\mathfrak{a}\}.\]
Every \textit{left}-ideal of $L[\tau]$ is principal,\footnote{
    By an argument similar to $L[X]$, probably.
}
so \[I_{\mathfrak{a}, \phi} = L[\tau]\phi_{\mathfrak{a}}\]
for a \textit{unique monic} $\phi_\mathfrak{a}\in L[\tau]$.
It is a plain to verify that for every $x\in A$,
$I_{\mathfrak{a}, \phi}$ absorb $\phi_x$ also from the right,
i.e. $I_{\mathfrak{a}, \phi}\phi_x\subset I_{\mathfrak{a}, \phi}$,
and therefore gives us a \textit{unique} Drinfeld module \[\mathfrak{a}*\phi : A\to L[\tau]\quad x\mapsto (\mathfrak{a}*\phi)_x,\]
which is characterized by\[\phi_\mathfrak{a}\cdot \phi_x = (\mathfrak{a}*\phi)_x\cdot\phi_\mathfrak{a},\]
namely that $\phi_\mathfrak{a}$ is an isogeny from $\phi$ to $\mathfrak{a}*\phi$.

\begin{lemma}\label{action on a Drinfeld module by an ideal}
    Let $\mathfrak{a}$ and $\mathfrak{b}$ be non-zero ideals of $A$,
    then\[
    \phi_{\mathfrak{ab}} = (\mathfrak{b}*\phi)_\mathfrak{a}\cdot \phi_{\mathfrak{b}},\]
    \[\mathfrak{ab}*\phi = \mathfrak{a}*(\mathfrak{b}*\phi).\]
\end{lemma}
\begin{lemma}\label{action on a Drinfeld module by a principal ideal}
    Let $\mathfrak{a} = (w)\ne 0$ be a principal ideal of $A$,
    then \[
    \phi_{(w)} = \mu_\phi(w)^{-1}\cdot\phi_w,\]
    \[
    ((w)*\phi)_x = \mu_\phi(w)^{-1}\cdot\phi_x\cdot\mu_\phi(w),\ \forall x\in A.\]
    In particular, $\phi\simeq (w) * \phi$ (not given by $\phi_{(w)}$).
\end{lemma}
\begin{lemma}\label{action on a Drinfeld module by an ideal commutes with field extension}
    Let $\sigma : L\hookrightarrow M$ be a field extension, inducing a Drinfeld module \[\sigma(\phi) : A\to M[\tau],\ x\mapsto \sigma(\phi)_x = \sigma(\phi_x).\]
    Then \[\sigma(\mathfrak{a}*\phi) = \mathfrak{a} * \sigma(\phi),\]
    \[\sigma(\phi_\mathfrak{a}) = \sigma(\phi)_\mathfrak{a}.\]
\end{lemma}

% \begin{example}
%     The trivial ideal $A = (1)$ fixes $\phi$ and $\phi_A = \phi_1 = 1$.
% \end{example}

Now we can extend the action of ideals to
\begin{itemize}
    \item   $\mathcal{I}_A$, the group of fractional ideals of $A$
\end{itemize}
More precisely,
for $w\in A\sminus \{0\}$,
\cref{action on a Drinfeld module by a principal ideal} suggests us to define
\[\left( (w^{-1})*\phi \right)_x := \mu_\phi(w)\cdot\phi_x\cdot \mu_{\phi}(w)^{-1}.\]
For a general fractional ideal $w^{-1}\mathfrak{a}$
where $\mathfrak{a}$ is an integral ideal of $A$,
we set \[(w^{-1}\mathfrak{a}) * \phi := w^{-1} * (\mathfrak{a} * \phi) : x\mapsto \mu_\phi(w)\cdot (\mathfrak{a} * \phi)_x\cdot \mu_\phi(w)^{-1}.\]
\cref{action on a Drinfeld module by an ideal} shows that these formulae define an action of $\mathcal{I}_A$
on the set of Drinfeld modules $A\to L[\tau]$.


Given a sign function $\varepsilon : F_\infty\to\F_\infty$ for $F_\infty$,
we can consider \begin{itemize}

    \item $\mathcal{P}^+_A$, a subgroup of the group $\mathcal{P}$ of principal fractional ideals of $A$, which is generated by $x\in F^\times$ with $\varepsilon(x) = 1$, and
    \item the \textbf{narrow class group} $\pic^+(A) := \mathcal{I}_A/\mathcal{P}^+_A$.
\end{itemize}
If, in addition, $\phi$ is $\varepsilon$-normalized,
then $\mathcal{P}^+$ fixes $\phi$ by \cref{action on a Drinfeld module by a principal ideal},
giving an action of $\pic^+(A)$.

\subsection{Torsion submodule}
A Drinfeld module $\phi : A\to L[\tau]$
defines an $A$-module structure on $\bar L$ by\[x\cdot b := \phi_x(b),\quad\forall x\in A, b\in\bar L.\footnotemark\]
\footnotetext{Note that if $\phi_x = \sum_{a_i\tau^i}$,
then \[\phi_x(b) = \sum_{i}\tau^i(b) = \sum_{i}b^{q^i}.\]
{\color{red} At least I think so!}}
All $\phi_x$ has coefficient in $L$, so $\phi$, in particular, gives an $A$-module structure on $L^\sep$.



For an ideal $\mathfrak{a}$ of $A$,
we define \[\phi[\mathfrak{a}] := \left\{b\in\bar L\mid \phi_\mathfrak{a}(b) = 0\right\} = \left\{ b\in\bar L\mid \phi_x(b) = 0,\forall x\in\mathfrak{a} \right\},\]
an $A/\mathfrak{a}$-module and an $A$-submodule of $\bar L$ with $A$-module structure induced by $\phi$.

\begin{proposition}\label{torsion submodule of an ideal}
    Let $\phi$ be a Drinfeld module of rank $r$,
    $\mathfrak{a}$ an ideal of $A$.
    Then $\phi[\mathfrak{a}]$ is a free $A/\mathfrak{a}$-module of rank $r$,
    and it is contained in $F^\sep$.
\end{proposition}
\begin{proof}
    Every $\phi_x$ acts by a polynomial of the form\[\phi_x(T) = a_0T + a_1T^q + \cdots + a_nT^{q^n}.\]
    This polynomial is separable, because $x\mapsto \phi_x\mapsto a_0$ is injective, which implies that $\phi_x'(T) = a_0\ne 0$ if $\phi_x \ne 0$.

    For the other claim, we use the structure of modules over Dedekind domains.
\end{proof}



\subsection{Hayes modules}

Let $\C_\infty$ be a completion of an algebraic closure of $F_\infty$.
It is $\infty$-adically complete and algebraically closed.

Fix a sign function $\varepsilon : F_\infty\to\F_\infty$ for $F_\infty$.
% \begin{definition}
A \textbf{Hayes module for $\varepsilon$}
is a Drinfeld module $\phi : A\to \C_\infty[\tau]$ over $\C_\infty$,
such that\begin{itemize}
\item it is of rank $1$,
\item it is $\varepsilon$-normalized,
\item $\partial\circ\phi : A\hookrightarrow\C_\infty$ is the inclusion $A\subset F\subset F_\infty\subset \C_\infty$.
\end{itemize}
Let $X_\varepsilon$ be the set of Hayes modules for $\varepsilon$.


If $\mathfrak{a}$ is an ideal of $A$, and $\phi\in X_\varepsilon$
then $\mathfrak{a}*\phi\in X_\varepsilon$.
By some discussion before,
this defines an action of $\pic^+(A) = \mathcal{I}_A/\mathcal{P}^+_A$
on $X_\varepsilon$.
\begin{proposition}\label{action of Pic+ on the set of Hayes modules}
    The set $X_\varepsilon$ is a principal homogeneous space for $\pic^+(A)$, i.e. $\pic^+(A)$ acts \textit{freely} and \textit{transitively} on $X_\varepsilon$.
\end{proposition}
% \end{definition}
\subsubsection{Galois action on \texorpdfstring{$X_\varepsilon$}{Xepsilon}}
We define the \textbf{narrow Hilbert class field} ot the \textbf{normalizing field for $(F, \infty, \varepsilon)$} to be the extension \[H_A^+ := F\left( \text{coefficient of }\phi_x\mid \phi\in X_\varepsilon, x\in A \right)\]
of $F$ in $\C_\infty$.

\begin{theorem}\label{property of HA+}
\begin{enumerate}[(1)]
    \item For any $\phi\in X_\varepsilon$ and $x\in A$,
    \[H_A^+ = F\left( \text{coefficient of }\phi_x \right)\]
    \item Let $B$ be the integral closure of $A$ in $H_A^+$.
    For any $\phi\in X_\varepsilon$ and $x\in A$, $\phi_x\in H_A^+[\tau]$ has integral coefficient, i.e. $\phi_x$ has coefficient in $B$.
    \item The extension $H_A^+/F$ is finite abelian, and it is unramified away from $\infty$.
\end{enumerate}
By \cref{action on a Drinfeld module by an ideal commutes with field extension},
there is a natrual action of $\gal(H_A^+/F)$ on $X_\varepsilon$.
For a fixed $\phi\in X_\varepsilon$,
$\phi$ induces an injective group homomorphism
\[\Psi : \gal(H_A^+/F)\inject \pic^+(A),\]
such that \[\sigma(\phi) = \Psi(\sigma) * \phi,\quad \forall\sigma\in\gal_F.\]
\begin{enumerate}[(1)]
    \item[(4)] For each non-zero prime $\mathfrak{p}$ of $A$,
    the class of $\Psi(\frob_\mathfrak{p})$ in $\pic^+(A)$ equals the class of $\mathfrak{p}$.
    \item[(5)] $\Psi : \gal(H_A^+/F)\to \pic^+(A)$ is an isomorphism.
\end{enumerate}

\end{theorem}

% \subsubsection{Reduction of Hayes modules}

% \begin{corollary}\label{Hayes modules have good reduction away from infty}
%     Every Hayes module $\phi$ has \textbf{good reduction} over $H_A^+$ at every finite place $\mathfrak{P}$ \textit{not} over $\infty$, i.e. the composition of reduction modulo $\mathfrak{P}$ with $\phi$ is a Drinfeld module of rank $1$ over $B/\mathfrak{P}$.
% \end{corollary}
% \begin{proof}
%     \color{red} after finishing construciton of Artin$^{-1}$.
% \end{proof}



\section{Construction of the Inverse to the Artin Map}
We fix the tuple $(F, \infty, \varepsilon)$
and a Hayes module $\phi\in X_\varepsilon$.

\subsection{\texorpdfstring{$\lambda$}{lambda}-adic representation}
Let $\lambda$ be a place of $F$ different from $\infty$, and we denote the corresponding maximal ideal of $A$ still by $\lambda$.

Take $e\ge 1$ and consider $\phi[\lambda^e]$.
By \cref{torsion submodule of an ideal},
$\phi[\lambda^e]$ is an $A/\lambda^e$-module of rank $1$.
Define the \textbf{$\lambda$-adic Tate module} to be\[T_\lambda(\phi) := \Hom_A(F_\lambda/\O_\lambda,\ \phi[\lambda^\infty]).\]
\begin{proposition}
    $T_\lambda(\phi)$ is a free $\O_\lambda$-module of rank $1$.
\end{proposition}
\begin{proof}
    The ring $\O_\lambda$ is a DVR, so
    \[
        \Hom_A(F_\lambda/\O_\lambda,\ \phi[\lambda^\infty])
        = \varprojlim_e \Hom_A(\O_\lambda/\m_\lambda^e, \phi[\lambda^\infty])
        = \varprojlim_e \Hom_A(A/\lambda^e, \phi[\lambda^\infty])
        = \varprojlim_e \Hom_A(A/\lambda^e, \phi[\lambda^e]).\qedhere
    \]
\end{proof}

Hence \[V_\lambda(\phi) := T_\lambda(\phi)\otimes_{\O_\lambda} F_\lambda\] is a $1$-dimensional $F_\lambda$-vector space.

% Of course the next step is to try to find a Galois action.

Using the isomophism $\Psi : \gal(H_A^+/F)\simeq \pic^+(A)$ from \cref{property of HA+},
any ideal $\mathfrak{a}\in\Psi(\sigma)$ of $A$ satisfies that $\sigma(\phi) = \mathfrak{a}*\phi$,
and thus we have two isogenies between $\sigma(\phi)$ and $\phi$, such that
\begin{itemize}
    \item $\sigma$ induces an isomorphism $V_\lambda(\sigma) : V_\lambda(\phi)\simeq V_\lambda(\sigma(\phi))$,
    \item $\phi_\mathfrak{a}$ induces an isomorphism\footnote{
        Since $\phi$ has rank $1$, it is equivalent to that $V_\lambda(\phi_\mathfrak{a})$ is non-zero.
        This is true, because, parallel to elliptic curves, taking Tate module is a faithful functor, i.e. for any two Drinfeld modules $\phi$ and $\phi'$ over $L$, the map
        \[\Hom_{L}(\phi, \phi')\hookrightarrow \Hom_{\O_\lambda}(T_\lambda(\phi), T_\lambda(\phi'))\]
        is injective.
    } $V_\lambda(\phi_\mathfrak{a}) : V_\lambda(\phi)\simeq V_\lambda(\mathfrak{a}*\phi)$.
\end{itemize}
So we obtain an element \[V_\lambda(\phi_\mathfrak{a})^{-1}\circ V_\lambda(\sigma)\in\GL_{F_\lambda}(V_\lambda(\sigma)) = F_\lambda^\times\cdot\Id,\]
corresponding to an element $\rho_\lambda^\mathfrak{a}(\sigma)\in F_\lambda^\times$.

\begin{lemma}\label{lem: to define non-infty-adic representation}
    Let $\sigma, \gamma\in\gal_F$ and $\mathfrak{a}, \mathfrak{b}$ be ideals of $A$.
    \begin{enumerate}[(i)]
        \item If $\sigma(\phi) = \mathfrak{a}*\phi$ and $\gamma(\phi) = \mathfrak{b}*\phi$,
        then $(\sigma\gamma)(\phi) = (\mathfrak{a}\mathfrak{b}) * \phi$, and $\rho_{\lambda}^{\mathfrak{ab}}(\sigma\gamma) = \rho_\lambda^\mathfrak{a}(\sigma)\rho_\lambda^\mathfrak{b}(\gamma)$.
        \item If $\sigma(\phi) = \mathfrak{a}*\phi = \mathfrak{b}*\phi$, then $\mathfrak{b}^{-1}\mathfrak{a}$ is generated by a \textit{unique} $w\in F_\infty^+\cap F^\times$, and $\rho_\lambda^\mathfrak{a}(\sigma)\rho_\lambda^\mathfrak{b}(\sigma)^{-1} = w$.
        \item If $\sigma(\phi) = \mathfrak{a} * \phi$,
        then $\ord_\lambda(\rho_\lambda^\mathfrak{a}(\sigma)) = -\ord_\lambda(\mathfrak{a})\footnotemark$.
    \footnotetext{
        Recall that we identify $\lambda$ with a prime ideal of $A$.
        The number $\ord_{\lambda}(\mathfrak{a})$ is the largest power of $\lambda$ dividing $\mathfrak{a}$.}
    \end{enumerate}
\end{lemma}

If $\sigma\in\gal_{H_A^+}$,
then $\sigma(\phi) = \phi = A * \phi$.
By \cref{lem: to define non-infty-adic representation} (i),
we obtain a homomorphism
\[\rho_\lambda : \gal_{H_A^+}\to \O_\lambda^\times\quad \sigma\mapsto \rho_\lambda^A(\sigma).\]

\begin{lemma}\label{lem: non-infty-adic representation at narrow Hilbert class field is almost unramified}
    $\rho_\lambda : \gal_{H_A^+}\to\O_\lambda^\times$ is continuous and unramified at all places of $H_A^+$ not over $\lambda$ or $\infty$.
\end{lemma}


\subsection{\texorpdfstring{$\infty$}{infty}-adic representation}
(Merge this and the last section in pre.)
Let $F_\infty^+ := \{x\in F_\infty^\times\mid \varepsilon(x) = 1\} = \ker(\varepsilon : F_\infty\to \F_\infty^\times)$.
Recall that the Hayes module $\phi : A\to H_A^+[\tau]$ extends to an injective homomorphism $\phi : F_\infty\to \left( H_A^+ \right)^\perf\llpar\tau^{-1}\rrpar$.

\begin{lemma}\label{lem: to define infty-adic representation}
    Let $\sigma, \gamma\in W_F$ and $\mathfrak{a}, \mathfrak{b}$ be ideals of $A$.

    There exists some series $u\in F^\sep\llbracket\tau^{-1}\rrbracket^\times$, such that \[u^{-1}\phi(F_\infty)u\subset \bar{k}\llpar\tau^{-1}\rrpar.\]
    For such a series $u$, if $\sigma(\phi) = \mathfrak{a} * \phi$,
    % the series \[\phi_\mathfrak{a}^{-1}\cdot\sigma(u)\cdot\tau^{\deg(\sigma)}\cdot u^{-1}\in \bar{F}\llpar\tau^{-1}\rrpar\]
    % is in $\phi(F_\infty^+)$, and it is independent to the choice of $u$.
    % Since $\phi : F_\infty\to \left( H_A^+ \right)^\perf\llpar\tau^{-1}\rrpar$ is injective,
    then there is a unique element $\rho_\infty^\mathfrak{a}(\sigma)\in F_\infty^+$,
    such that \[\phi_\mathfrak{a}^{-1}\cdot\sigma(u)\cdot\tau^{\deg(\sigma)}\cdot u^{-1} = \phi(\rho_\infty^\mathfrak{a}(\sigma)).\]
    This element satisfies the following properties:
    \begin{enumerate}[(i)]
    \item If $\sigma(\phi) = \mathfrak{a}*\phi$ and $\gamma(\phi) = \mathfrak{b}*\phi$,
    then $(\sigma\gamma)(\phi) = (\mathfrak{a}\mathfrak{b}) * \phi$, and $\rho_{\lambda}^{\mathfrak{ab}}(\sigma\gamma) = \rho_\lambda^\mathfrak{a}(\sigma)\rho_\lambda^\mathfrak{b}(\gamma)$.
    \item If $\sigma(\phi) = \mathfrak{a}*\phi = \mathfrak{b}*\phi$, then $\mathfrak{b}^{-1}\mathfrak{a}$ is generated by a \textit{unique} $w\in F_\infty^+\cap F^\times$, and $\rho_\lambda^\mathfrak{a}(\sigma)\rho_\lambda^\mathfrak{b}(\sigma)^{-1} = w$.
    \end{enumerate}
\end{lemma}

Similar to the $\lambda$-adic case, restricting to the Weil group over $H_A^+$ gives us a homomorphism
\[\rho_\infty : W_{H_A^+}\to F_\infty^+,\quad \sigma\mapsto \rho_\infty^A(\sigma).\]


\begin{lemma}\label{lem: infty-adic representation at Weil group of narrow Hilbert class field is almost unramified}
    $\rho_\infty : W_{H_A^+}\to F_\infty^+$ is continuous and unramified at all places of $H_A^+$ not over $\infty$.
\end{lemma}

\subsection{The inverse of the Artin map}
For each $\sigma\in W_F$,
fix an ideal $\mathfrak{a}_\sigma$ of $A$,
such that \[\sigma(\phi) = \mathfrak{a}_\sigma * \phi.\]
By \cref{lem: to define non-infty-adic representation} (iii),
$\rho_\lambda^{\mathfrak{a}_\sigma}(\sigma)\in \O_\lambda^\times$ for almost all places $\lambda$.
Hence $\left( \rho_\lambda^{\mathfrak{a}_\sigma}(\sigma) \right)_\lambda$ is an idele of $F$;
we define $\rho(\sigma)$ to be its class in $C_F$.
By \cref{lem: to define non-infty-adic representation} (ii) and \cref{lem: to define infty-adic representation} (ii),
for different choices of $\mathfrak{a}_\sigma$,
$\rho_\lambda^{\mathfrak{a}_\sigma}(\sigma)$ will differ by an element in $F^\times$.
Therefore $\rho(\sigma)$ is independent to the choice of $\mathfrak{a}_\sigma$,
and the map \[\rho : W_F\to C_F,\quad \sigma\mapsto\rho(\sigma)\]
is a group homomorphism by \cref{lem: to define non-infty-adic representation} (i) and \cref{lem: to define infty-adic representation} (i).

The restriction of $\rho : W_F\to C_F$ to $W_{H_A^+}$ is
\[W_{H_A^+}\stackrel{\prod_{\lambda}\rho_\lambda}{\longrightarrow} F_\infty^+\times \prod_{\lambda\ne\infty} \O_\lambda^\times\hookrightarrow\A_F^\times\surject C_F.\]
This homomorphism is continuous since all $\rho_\lambda$ are continuous.
The group $W_{H_A^+}$ has finite index in $W_F$,
so $\rho$ is continuous on $W_F$.
The group $C_F$ is abelian,
so $\rho$ factors through the maximal abelian quotient $W_F^\ab$,
% and we denote this map by $\rho : W_F^\ab\to C_F$ again.
and taking profinite completion yields a continuous homomorphism
\[\hat\rho : \gal_F\to \hat C_F\]
that factors through the maximal abelian quotient $\gal_F^\ab = \gal(F^\ab/F)$.

Recall that the Artin map $\theta_F : C_F\to \gal(F^\ab/F)$ extends to a topological isomophism \[\hat\theta_F : \hat C_F\to \gal(F^\ab/F).\]
\begin{theorem}
    The map $\hat\rho : \gal(F^\ab/F)\to \hat{C}_F$ is a topological isomophism independent to the choice of $\infty$, $\varepsilon$ and $\phi$,
    and the map \[\gal(F^\ab/F)\to \hat{C}_F\quad \sigma\mapsto \hat\rho(\sigma)^{-1}\]
    is the inverse of the Artin map $\hat\theta_F : \hat C_F\to \gal(F^\ab/F)$.
\end{theorem}
\begin{proof}

First, we need an arithmetic input.
\begin{lemma}\label{value of local representation at Frobenius different from itself and infty}
    Let $\lambda$ be a place of $F$, $\mathfrak{p}$ be another place of $F$ that is not $\lambda$ or $\infty$.
    Then $\rho_\lambda^\mathfrak{p}(\frob_\mathfrak{p}) = 1$.
\end{lemma}
\begin{remark}[Explaination to the notation $\rho_\lambda^\mathfrak{p}(\frob_\mathfrak{p})$]
    Let $\lambda$ and $\mathfrak{p}$ be places of $F$ with $\mathfrak{p}\ne \infty$.
    By \cref{property of HA+},
    the extension $H_A^+/F$ is unramified at all places $\ne\infty$,
    and the unique $\frob_\mathfrak{p}\in\gal(H_A^+/F)$
    satisfies $\frob_\mathfrak{p}(\phi) = \mathfrak{p} * \phi$.
    Also by \cref{property of HA+}, that $\gal_F$-action on $X_\varepsilon$ factors through $\gal(H_A^+/F)$,
    hence any (non-unique) $\frob_\p\in \gal_F$ satisfies $\frob_\p(\phi) = \mathfrak{p} * \phi$.
\end{remark}

Now we begin the proof. We will denote the class of $\alpha\in \A_F^\times$ in $C_F$ by $[\alpha]$.

Let $U < C_F$ be an open subgroup of finite index.
The subgroup $\rho^{-1}(U) < W_F^\ab$ is open.
Consider the finite abelian extension $L_U := (F^\ab)^{\rho^{-1}(U)}$ of $F$ fixed by this subgroup,
so that $\gal_{L_U}^\ab = \text{the closure of }\rho^{-1}(U)$ in $\gal_F^\ab$.
Hence we have an injective continuous homomorphism
\footnote{I hope these are true..? i.e. if $H$ is a dense subgroup of $G$ and $U$ is open in $H$, then $H/U\simeq G/\bar{U}$.}
\[\rho_U : \gal(L_U/F) \simeq\gal_F^\ab/\gal_{L_U}^\ab \simeq W_F^\ab/\rho^{-1}(U)\hookrightarrow C_F/U.\]
Let $S_U$ be the set of places consists of\begin{itemize}
    \item $\infty$, and
    \item $\mathfrak{p}$ for which there exists some idele $\alpha\in\O_\mathfrak{p}^\times\hookrightarrow\footnotemark \A_F^\times$ whose class in $C_F$ is not in $U$.
    \footnotetext{$\alpha = (\cdots, 1, \alpha_\mathfrak{p}, 1, \cdots)$ for some $\alpha_\mathfrak{p}\in\O_\mathfrak{p}^\times$.}
\end{itemize}
Since $U$ is open in $C_F$, the set $S_U$ is finite.

For a place $\mathfrak{p}\notin S_U$,
choose a uniformizer $\pi_\mathfrak{p}$ of $F_\mathfrak{p}$
and consider the idele $\pi_\mathfrak{p} = (\cdots, 1, \pi_\p, 1, \cdots)\in \A_F^\times$.
\begin{lemma}\label{lem: generators of finite quotient of idele}
    $C_F/U$ is generated by $\{\pi_\mathfrak{p}\}_{\mathfrak{p}\notin S_U}$.
\end{lemma}
\begin{proof}[Proof of \cref{lem: generators of finite quotient of idele}]
    Let $V$ be the preimage of $U$ in $\A_F^\times$, $W$ be the subgroup of $\A_F^\times$ generated by $V$ and $\{\pi_\mathfrak{p}\}_{\mathfrak{p}\notin S_U}$.
    We need to show that $W = \A_F^\times$.

    Take an arbitary $\alpha\in\A_F^\times$.
    By definition of $S_U$, $\prod_{\p\notin S_U}\O_\mathfrak{p}^\times\subset V$,
    so there is some integer $e\in\Z$ such that \[\prod_{\p\in S_U} (1 + \m_p^e)\times\prod_{\p\notin S_U} \O_\p^\times\subset V.\]
    By weak approximation theorem, there is some $x\in F^\times$,
    such that $\ord_{\p}(\alpha_\p - x) > \max\{e, \ord_\p(\alpha_p)\}$ for all $\p\in S_U$.
    This implies that $x^{-1}\alpha_\p\in 1 + \m_\p^e$,
    and thus \[x^{-1}\alpha\in \prod_{\p\notin S_U} F_\p^\times\times\prod_{\p\in S_U} (1 + \m_p^e)
    =\prod_{\p\notin S_U} \O_\p^\times\pi_\p^\Z\times\prod_{\p\in S_U} (1 + \m_\p^e)
    \subset W.\]
    As $x\in F^\times\subset V\subset W$, we have proved $\alpha\in W$.
\end{proof}

Now consider the idele \[\beta := \left( \rho_\lambda^\mathfrak{p}(\frob_\mathfrak{p}) \right)_\lambda\cdot \pi_\mathfrak{p}\in \A_F^\times\]
for some $\frob_\p\in W_F$.
By \cref{value of local representation at Frobenius different from itself and infty}, $\beta_\lambda = 1$ for all $\lambda\ne \mathfrak{p}$.
By \cref{lem: to define non-infty-adic representation} (iii),
\[\ord_\mathfrak{p}(\beta_\mathfrak{p}) = -\ord_{\mathfrak{p}}(\p)\cdot 1 = 0.\]
Hence the image of $\beta$ in $C_F$ is in $U$,
namely $\rho_U(\frob_\p) = [\rho(\frob_\p)]\cdot U = [\pi_\p^{-1}]\cdot U\in C_F/U$.
Consequently,\begin{itemize}
    \item $L_U/F$ is unramified at $\p\notin S_U$, since there is a unique $\frob_\p\in\gal(L_U/F)$ by the injectivity of $\rho_U$;
    \item $\rho_U : \gal(L_U/F)\to C_F/U$ is surjective and thus an isomophism.
\end{itemize}


Next, we show that these $L_U$ are all the finite abelian extensions of $F$.
For each open $U < C_F$ of finite index,
the continuous isomophism\[C_F/U\to \gal(L_U/F)\quad \alpha\mapsto (\rho_U^{-1}(\alpha))^{-1}\]
maps $\pi_\p$ to $\frob_\p$. This is Artin map\footnote{
    See \href{https://math.stackexchange.com/questions/1361278/which-properties-determine-the-uniqueness-of-the-global-artin-map?rq=1}{this post on MSE}, for instance.
}.
So \[\gal(L_U/F)\to C_F/U\quad \sigma\mapsto \rho_U(\sigma)^{-1}\]
is the inverse to \[\theta_U : C_F/U\to \gal(L_U/F)\quad \alpha\mapsto\theta_F(\alpha)|_{L_U},\]
the Artin map at this finite level.
If $L$ is a finite abelian extension of $F$,
then the corresponding open subgroup $U_L$ of $C_F$
according to class field theory is the kernel of
\[C_F\to \gal(L/F)\quad \alpha\mapsto\theta_F(\alpha)|_{L}.\]
Therefore $L = L_{U_L}$,
and $F^\ab = \bigcup_{U}L_U$.

Now we can pass to the limit of the compatible isomophisms ${\rho_U}_U$ and go back to see that $\hat\rho : \gal_F^\ab\to C_F$ is an isomophism,
whose inverse is the ``multiplicative inverse'' of the Artin map $\hat\theta_F$.
\end{proof}

\begin{corollary}
    The homomorphism $\rho : W_F^\ab\to C_F$ is a topological isomophism, and the map \[W_F^\ab\to C_F\quad \sigma\mapsto \rho(\sigma)^{-1}\]
    is the inverse of the Artin map $\theta_F : C_F\to W_F^\ab$.
\end{corollary}

\section{Example: the Rational Function Field}
Let $F = k(t)$.
We consider the usual place $\infty$ and $A = k[t]$,
so that $F_\infty = k\llpar t\rrpar$,
$\F_\infty = k$,
$\m_\infty = t^{-1}k\llbracket t^{-1}\rrbracket$,
$\ord_{\infty}(t^{-1}) = 1$.
Let $\varepsilon : F_\infty^\times\to k^\times$ be the unique sign function such that $\varepsilon(t^{-1}) = 1$,
so that $F_\infty^+ = t^{\Z}\cdot(1 + \m_\infty)$.

% \subsection{The Carlitz module}

The \textbf{Carlitz module} $\phi$ is defined by \[\phi : A = k[t]\to F[\tau]\quad t\mapsto \phi_t := t + \tau.\]
It is a Hayes module for $\varepsilon$, and the normalizing field for $(F, \infty, \varepsilon)$ is $H_A^+ = F$,
so $\phi$ is the only Hayes module for $\varepsilon$.

We have defined the representations \[\rho_\lambda : W_F^\ab\to F_\lambda^\times\quad \sigma\mapsto \rho_\lambda^A(\sigma)\]
for every place $\lambda$ of $F$.
% By the uniqueness of Artin map, $\rho_\lambda$ is determined by the relation \[\rho_\lambda(\frob_\mathfrak{p}) = \mathfrak{p}\]
% for every irreducible polynomial $\mathfrak{p}\in A = k[t]$ that is not over $\lambda$.
For $\lambda\ne \infty$, the representation $\rho_\lambda$ comes from a continuous Galois representation $\rho_\lambda :\gal_F\to \O_\lambda^\times$.
For $\infty$, $\rho_\infty$ takes value in $F_\infty^+$.
So the isomophism between the (abelianized) Weil group and the idele class group factors as
\begin{equation}\label{formula: the isomophism from Weil group of rational function field to idele class group}
    W_F^\ab \stackrel{\prod_{\lambda}\rho_\lambda}{\longrightarrow} F_\infty^+\times\prod_{\lambda\ne\infty}\O_\lambda^\times\to C_F.
\end{equation}
Similar to $\Q$, we have an isomophism\[\A_F^\times \simeq F^\times \times F_\infty^+\times \prod_{\lambda\ne \infty} \O_\lambda^\times\]
for $F = k(t)$ as follows.
Every place $\lambda\ne\infty$ has a ``canonical'' uniformizer $\p\in k[t]$, namely the unique monic irreducible polynomial,
and we write $x_\p = u_\p\p^{n_\p}$ with $u_\p\in \O_\p^\times$.
Put \[f := a_\infty\prod_{\p}\p^{n_\p}\in k(t)^\times.\]
At the place $\infty$, we write $f^{-1}x_\infty = a_\infty t^{n} + {}$ terms with lower degree in $t$, where $a_\infty\in k$.
Then $(a_\infty f)^{-1}x\in F_\infty^+\times\prod_{\lambda\ne\infty}\O_\lambda^\times$.
This gives the decomposition above,
which implies that the second arrow in (\ref{formula: the isomophism from Weil group of rational function field to idele class group}) is an isomophism, and thus so is the first arrow
\[W_F^\ab\stackrel{\prod_{\lambda}\rho_\lambda}{\longrightarrow} \prod_{\lambda\ne\infty}\O_\lambda^\times \times t^{\Z}\times (1 + \m_\infty).\]
Taking profinite completion, we got a decomposition
\[\gal(F^\ab/F)
% \simeq \prod_{\lambda\ne\infty}\O_\lambda^\times \times \hat F_\infty^+
\simeq \prod_{\lambda\ne\infty}\O_\lambda^\times \times t^{\hat\Z}\times (1 + \m_\infty)\]
of $\gal_F^\ab$, which gives three disjoint abelian extension of $F$ whose compositum is $F^\ab$.

\subsection*{Description of \texorpdfstring{$F^\ab$}{F^ab}}
Recall that if $L/K$ is an extension of function fields with fields of constants $k_L$ and $k_K$ respectively, we say that:
\begin{itemize}
    \item $L/K$ is a \textbf{constant field extension}, if $L = Kk_L$;
    \item $L/K$ is a \textbf{geometric extension}, if $k_L = k_K$.
\end{itemize}

\subsubsection*{The ``cyclotomic'' extension \texorpdfstring{$K_\infty$}{K_infty}}
For $\lambda\ne\infty$, the representation $\rho_\lambda : \gal_F\to \O_\lambda^\times$ is precisely the Galois representation on $T_\lambda(\phi)$, where $\phi$ is the Carlitz module.
The representation \[\chi := \prod_{\lambda\ne\infty}\rho_\lambda : \gal_F\to \prod_{\lambda\ne\infty}\O_\lambda = \hat A^\times\]
is the inverse limit of \[\chi_m : \gal_F\to (A/(m))^\times\] from $\gal_F$-action on $\phi[m]$ for all monic irreducible $m\in A = k[t]$, ordered by divisibility.
Hence the field fixed by $\ker(\chi)$ is \[K_\infty = \bigcup_{m} F(\phi[m]).\]
The extension $K_\infty/F$ is a geometric extension, tamely ramified at $\infty$.

\subsubsection*{The extension of constants \texorpdfstring{$\bar{k}(t)$}{bark(t)}}
For each $\sigma\in W_F$, the factor in $t^{\Z}\simeq \Z$ is
$\ord_{t}(\rho_\infty(\sigma)) = - \ord_{\infty}(\rho_\infty(\sigma))$, which equals $-\ord_{\tau^{-1}}(\phi(\rho_\infty(\sigma)))$ by (\ref{eq: def of rank}).
By \cref{lem: to define infty-adic representation}, $\phi(\rho_\infty(\sigma)) = \sigma(u)\tau^{\deg(\sigma)}u^{-1}$,
so $-\ord_{\tau^{-1}}(\phi(\rho_\infty(\sigma))) = \deg(\sigma)$.
This shows that the projection $W_F\to \Z$ is precisely the map $\deg$.
The field fixed by (the closure of) $\ker(\deg)$ is $\bar{k}(t)$,
and the extension $\bar{k}(t)/k(t)$ is the maximal constant field extension.


\subsubsection*{The wildly ramified extension \texorpdfstring{$L_\infty$}{L_\infty}}
By discussion above, the projection onto $1 + \m_\infty$ is
% (the multiplicative inver of)
\[W_F\to 1 + \m_\infty\quad \sigma\mapsto \rho_\infty(\sigma)/\ord_t(\rho_\infty(\sigma)) = \rho_\infty(\sigma)/\deg(\sigma).\]
Taking profinite completion, we get a Galois representation
$\beta : \gal_F\to 1 + \m_\infty$.
Denote by $L_\infty$ the fixed field of $\ker(\beta)$.
The extension $L_\infty/F$ is unramified away from $\infty$ and wildly ramified at $\infty$.



\section{Comparision with Elliptic Curves}




\section{Proof of (some) lemmas}






\end{document}