\section{\texorpdfstring{A Bit of $p$-adic Analysis}{A bit of p-adic analysis}}
In this section, we consider some basic properties concerning powerseries over a closed subfield $K$ of $\C_p$ as functions.

Let $f(X) = \sum_{i\ge 0} a_iX^i\in K\llbracket X\rrbracket $. We can evaluate $f$ at $z\in \C_p$ iff $a_iz^i\to\infty$, so the \textbf{radius of convergence}
is \[\rho(f) := \sup\{\rho\in\R\ |\ a_i\rho^i\to\infty (i\to\infty)\}.\]
\begin{itemize}
    \item If $|z| < \rho(f)$, then $f(z)$ converges in $\C_p$.
    \item If $|z| > \rho(f)$, then $f$ diverges.
    \item $\rho(f(\alpha X)) = \rho(f)\cdot |\alpha|^{-1}$.
\end{itemize}
We are mainly interested in the power series converging on the unit disk, i.e., \begin{align*}
    H_K :={}& \{f\in K\llbracket X\rrbracket\ |\ \rho(f) > 1\}\\
    ={}&  \{f\in K\llbracket X\rrbracket\ |\ a_i\rho^i\to 0, \forall \rho < 1\}\\
    ={}&  \{f\in K\llbracket X\rrbracket\ |\ f\text{ converges on the open unit disk }\m_{\C_p} = B(0, 1)\}.
\end{align*}
\begin{example}
    $K\otimes_{\O_K}\O_K\llbracket X\rrbracket$ = power series over $K$ with bounded coefficients $\subsetneq H_K$.
\end{example}
\begin{example}
    $\log(1 + X) = \log_{\Gm}(X) = X - \displaystyle\frac{X^2}{2} + \frac{X^3}{3} - \dots\in H_K\setminus K\otimes_{\O_K}\O_K\llbracket X\rrbracket$.
\end{example}

\subsection{The Gauss Norm}
\begin{theorem}
    Let $f(X) = \sum_{i\ge 0} a_iX^i\in K\llbracket X\rrbracket $ with $\rho(f) > 0$, a real number $\rho < \rho(f)$ s.t. $\rho\in |\C_p^\times|$.
    Then $\sup_{i\ge 1}{|a_i|\rho^i}$ is a maximum (i.e., $\sup_{i\ge 1}{|a_i|\rho^i} = |a_j|\rho^j$ for some $j$), and \[\sup_{i\ge 1}{|a_i|\rho^i} = \sup_{|z| = \rho} |f(z)| =: |f|_\rho.\]
\end{theorem}
\begin{proof}
\begin{itemize}
    \item     $\rho<\rho(f)\implies |a_i|\rho^i \to 0\implies \sup_{i\ge 0}{|a_i|\rho^i}$ is a maximum.
    \item $|f(z)| = \left| \sum_{i\ge 0} a_iz^i\right| \le\sup_{i\ge 1} |a_i||z|^i$, so $|f|_\rho\le \sup_{i\ge 1}{|a_i|\rho^i}$.
    \item Take $\alpha\in\C_p$ with $|\alpha| = \rho$,
    and $j\in\Z_{\ge 0}$ s.t. $\sup_{i\ge 1}{|a_i|\rho^i} = |a_j|\rho^j$.
    Let $\beta := a_j\alpha^j$.
    We aim to find $|z| = \rho$ s.t. $|f(z)| = |\beta|$.
    Consider \[g(X) = \sum_{i\ge 0}g_iX^i := \frac{f(\alpha X)}{\beta}\in\O_{\C_p}\llbracket X\rrbracket.\]
    Moreover, the coefficients $g_i = \dfrac{a_i\alpha^i}{\beta}\to 0$ as $i\to\infty$,
    because $|g_i| = \beta^{-1}|a_i|\rho^i$.
    So $\bar{g}(X)\in k_{\C_p}\llbracket X  \rrbracket$ is actually a polynomial, and it is nonzero since $|g_j| = 1$.
    Take $\bar{w}\in\bar{k}^\times$ s.t. $\bar g(\bar w)\ne 0$. Then a lift $w\in\O_{\C_p}^\times$ verifies $|g(w)| = 1$.
    Hence $|f(\alpha w)| = |\beta|$ and $|\alpha w| = |\alpha| = \rho$.\qedhere
\end{itemize}\end{proof}

Thus, the expression $|f|_\rho\in\R\cup\{+\infty\}$ is defined on $\rho\in\R$.
In addition,
\begin{itemize}
    \item $\rho\to |f|_\rho$ is continuous,
    \item $|f|_\sigma \le |f|_\rho$ if $\sigma\le\rho < \rho(f)$.
\end{itemize}
$\implies$ the \textbf{maximum modulus principle} holds: $|f|_\rho = \sup_{|z|\le \rho} |f(z)| = \max_{|z|\le\rho} |f(z)|$ for $\rho < \rho(f)$.

\begin{itemize}
    \item $|\cdot|_\rho$ is multiplicative: $|fg|_{\rho} = |f|_\rho|g|_\rho$.
\end{itemize}

\begin{example}
    If $f\in H_K$, then \textit{as a function}:\begin{itemize}
        \item $f$ is bounded on $\m_{C_p}\iff f\in K\otimes_{\O_K}\O_K\llbracket X \rrbracket$,
        \item $f$ is bounded by $1$ on $\m_{\C_p}\iff f\in\O_K\llbracket X \rrbracket$.
    \end{itemize}
\end{example}

\subsection{Weierstrass Preparation Theorem}
For $f(X) = \sum_{i\ge 0}a_iX^i\in\O_K\llbracket X \rrbracket$, we define its \textbf{Weierstrass degree} $:= \wideg(f) :=$ smallest $i\in\Z_{\ge 0}$ s.t. $a_i\in\O_K^\times$.
\begin{itemize}
    \item $\wideg$ is multiplicative.
    \item $\wideg(f) = \infty\iff f\in\m_{K}\llbracket X \rrbracket$.
    \item $\wideg(f) = 0\iff a_0\in\O_K^\times\iff f\in\left( \O_K\llbracket X \rrbracket \right)^\times$.
    \item If $K/\Q_p < \infty$, then for $f\in K\otimes_{\O_K}\O_K\llbracket X \rrbracket$,
    $\exists ! n\in\Z$ s.t. $\pi^n f$ has finite Weierstrass degree, which is the smallest degree of the term in $f$ with minimum valuation (maximum norm).
\end{itemize}
\begin{remark}
    The last statement fails if $K$ is not finite over $\Q_p$, i.e., if there is no uniformiser. For example, $f(X) = \sum_{i\ge 1}\frac{1}{p^i}X^i$.
\end{remark}
From now on, assume $K/\Q_p < \infty$ with uniformiser $\pi$.
\begin{proposition}
    [Euclidean Division]\label{Euclidean division for power series}
    Let $f\in\O_K\llbracket X \rrbracket$ with $\wideg(f) < \infty$.
    Then: $\forall g\in \O_K\llbracket X \rrbracket$, $\exists ! q\in \O_K\llbracket X \rrbracket$ \& $r\in \O_K[X]$\footnote{The residue $r(X)$ is a polynomial!} s.t. \[g = q\cdot f + r,\ \deg(r)\le \wideg(f) - 1.\]
\end{proposition}
\begin{proof}
    Idea is, again, $\pi$-adic approximation.
    
    First we do ``Euclidean division" in $k\llbracket X \rrbracket $.
    Write $\bar f(X) = X^nf_0(X)$ with $f_0(X)\in k\llbracket X \rrbracket^\times$.
    For $ h = \sum_{i\ge 0}h_iX^i\in k\llbracket X \rrbracket$,
    it decomposes as\[h = X^ns + r,\text{  with } r = h_0 + \dots + h_{n-1}X^{n-1}\]
    \[\implies h = q\cdot f + r, \text{  where } q = s\cdot f_0^{-1}.\]

    Therefore,\begin{align*}
        g &= q_0f + r_0 + \pi g_1 &\text{with } \deg r_0 \le n-1,\\
        &= (q_0 + \pi q_1)f + (r_0 + \pi r_1) + \pi^2 g_2 &\text{ with }\deg r_1\le n - 1\\
        &=\cdots & \\
    \implies g &= qf + r, &\text{with } q = \sum_{i\ge 0}\pi^iq_i, r = \sum_{i\ge 1}\pi^ir_i.
    \end{align*}

    \textit{Unicity.}
    If $qf + r = 0$, then $\underbrace{\bar{q}\bar{f}}_{\text{divided by }X^n} + \underbrace{\bar{r}}_{\deg\le n - 1} = 0$, so $\bar{q}\bar{f} = \bar{r} = 0$.
    Deduce inductively $\bmod \pi^n$.
\end{proof}
\begin{remark}
    Jiang Jiedong provided a proof for this theorem when $K$ is not finite over $\Q_p$.
\end{remark}

For a polynomial $P(X)\in \O_K[X]$, we say $P(X)$ is \textbf{distinguished}, if it is monic with other coefficients in $\m_K$, i.e, \[P(X) = X^n + a_{n-1}X^{n-1} + \dots + a_0,\quad a_{n-1},\dots, a_0\in\m_{K}.\]
\begin{itemize}
    \item The Newton polygon of a distinguished polynomial $P$ will be above $x$-axis with only the end point on $x$-axis, and all slopes are $ < 0$. So every root of $P$ lies in $\m_{\Q_p^\alg}$.
\end{itemize}

\begin{theorem}[Weierstrass Preparation Theorem]\label{Weierstrass preparation}
    Let $f\in\O_K \llbracket X \rrbracket$ with $\wideg f < \infty$.\par
    Then $\exists !$ distinguished polynomial $P\in\O_K \llbracket X \rrbracket$ with $\deg P = \wideg f$, s.t. \[f(X) = P(X)\cdot u(X),\quad u\in\left( \O_K\llbracket X \rrbracket \right)^\times.\]
\end{theorem}

So, power series over $K$ with bounded coefficients would have finitely many zeros in the unit disk.
\begin{corollary}\label{zero of power series with bounded coefficients}
    Let $f(X)\in K\otimes_{\O_K}\O_K\llbracket X \rrbracket$.\begin{enumerate}
        \item $f(X) = \pi^\mu P(X) u(X)$ uniquely, where $\mu\in \Z$, $P$ a distinguished polynomial, $u\in\left( \O_K\llbracket X \rrbracket \right)^\times$.
        \item $f$ has finitely many zeros in $\m_{\C_p}$, and they are actually in $\m_{\Q_p^\alg}$. The number of zeros is $\wideg (\pi^{-\mu} f) = \deg P$\footnote{I want to call this ``the Weierstrass degree of $f$".}.\qed
    \end{enumerate}
\end{corollary}
\begin{corollary}
    $K\otimes_{\O_K}\O_K\llbracket X \rrbracket$ is a PID.
\end{corollary}
\begin{proof}
    For $I = (\{f_i\}_i)$, write $f_i = \pi^{\mu_i}P_iu_i$, then $I = \left(\gcd_i (P_i)\right)$.
\end{proof}

\begin{theorem}
    Let $f\in H_K$, $\rho < 1$. Then $f$ has finitely many zeros in $B(0, \rho)$, all of which are in $\m_{\Q_p^\alg}$.
\end{theorem}
\begin{remark}
    $f\in H_K$ \textit{could} have infinitely many zeros in $\m_{\C_p} = B(0, 1)$.
    For example, we saw in the homework that the zeros of $\log_F$ in $\m_{\C_p}$ are $F[p^\infty]$, which is infinite in many cases, such as $F = \Gm$.
\end{remark}
\begin{proof}
    We may assume $\rho\in|\C_p|$.

    Take $L/\Q_p < \infty$ and $\alpha\in\m_L$ with $|\alpha| = \rho$.
    Then $f(\alpha X)\in L\otimes_{\O_L}\O_L\llbracket X \rrbracket$, because $|a_i|\rho^i\to 0$ for $f = \sum a_iX^i\in H_K$.
    Hence $f(\alpha X)$ has finitely many zeros in $\m_{\C_p} = B(0, 1)$ and they are algebraic over $\Q_p$.
    These zeros are in bijection with zeros of $f(X)$ in $B(0, \rho)$. 
\end{proof}

Now we can prove the converse of \cref{zero of power series with bounded coefficients}.
\begin{theorem}\label{power series bounded iff has finite zeros in open unit disk}
    If $f\in H_K$, then\[f\in K\otimes_{\O_K}\O_K\llbracket X \rrbracket\iff f\text{ has finitely many zeros in }\m_{\C_p}.\]
\end{theorem}
\begin{proof}
    ($\impliedby$)
    Assume that $f = \sum_{i\ge 0}f_iX^i$ has $n$ zeros in $\m_{\C_p}$.

    Take $\rho\in \m_{\C_p}$ and $\alpha\in \m_{\Q_p}$ with $|\alpha| = \rho$.
    By previous results, \begin{align*}
        \# \{\text{zero of } f\text{ in }B(0, \rho)\} &
        = \text{``Weierstrass degree" of } f(\alpha X)\\ &
        = \min\left\{j\in\Z_{\ge 0}\left| \rho^j|f_j| = \max_{i\in\Z_{\ge 0}}\rho^i|f_i|\right.\right\}.
    \end{align*}
    Hence \[\min\left\{j\in\Z_{\ge 0}\left| \rho^j|f_j| = \max_{i\in\Z_{\ge 0}}\rho^i|f_i|\right.\right\}\le n,\]
    \[\iff \rho^i|f_i|\le \max \left\{|f_0|, \rho|f_1|,\dots, \rho^n|f|_n\right\},\; \forall i\ge 0.\]
    Letting $i\to\infty$ tells us that the coefficients of $f$ are bounded.
\end{proof}

\subsection{\texorpdfstring{$p$-adic Banach Spaces}{}}
Let $K/\Q_p < \infty$ with uniformiser $\pi$, $k := \O_K/\pi$.
