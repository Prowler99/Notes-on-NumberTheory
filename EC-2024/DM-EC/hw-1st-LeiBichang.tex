\documentclass{article}
\usepackage{amsmath, amssymb, amsthm, amsbsy, mathrsfs, stmaryrd}
\usepackage{enumitem}
\usepackage[colorlinks,
linkcolor=cyan,
anchorcolor=blue,
citecolor=blue,
]{hyperref}
\usepackage[capitalize]{cleveref}
\usepackage[margin = 1in, headheight = 12pt]{geometry}
\usepackage{bbm}
\usepackage{tikz-cd}

\newtheorem{theorem}{Theorem}

\theoremstyle{definition}
\newtheorem{definition}{Definition}
\newtheorem{exercise}{Exercise}[section]
\newtheorem{problem}{Problem}
\newtheorem{example}{Example}
\newtheorem{proposition}{Proposition}[section]
\newtheorem{lemma}{Lemma}[section]
\newtheorem{corollary}{Corollary}[section]

\theoremstyle{remark}
\newtheorem*{remark}{Remark}

\renewcommand{\Re}{\mathop{\mathrm{Re}}}
\renewcommand{\Im}{\mathop{\mathrm{Im}}}

% 新命令
% 数学对象
    \newcommand{\R}{\mathbb{R}}
    \newcommand{\C}{\mathbb{C}}
    \newcommand{\Q}{\mathbb{Q}}
    \newcommand{\Z}{\mathbb{Z}}
    \DeclareMathOperator{\GL}{GL}
    \DeclareMathOperator{\SL}{SL}
    \newcommand{\p}{\mathfrak{p}}
    \renewcommand{\P}{\mathbb{P}}
    \newcommand{\A}{\mathbb{A}}
    \newcommand{\F}{\mathbb{F}}
% 集合
    \newcommand{\sminus}{\smallsetminus} %(集合)差
% 范畴
    \newcommand{\op}[1]{{#1}^{\mathrm{op}}} %反范畴
    \DeclareMathOperator{\enom}{End} %自态射
    \DeclareMathOperator{\isom}{Isom} %同构
    \DeclareMathOperator{\aut}{Aut} %自同构
    \DeclareMathOperator{\im}{im} %像
    \newcommand{\Set}{\mathbf{Set}} %集合范畴
    \newcommand{\Abel}{\mathbf{Ab}} %群范畴
    \newcommand{\Ring}{\mathbf{Ring}}
    \newcommand{\Cring}{\mathbf{CRing}}
    \newcommand{\Alg}{\mathbf{Alg}}
    \newcommand{\Mod}{\mathbf{Mod}}
    \DeclareMathOperator{\Id}{id}
%向量空间, 矩阵
    \DeclareMathOperator{\rank}{rank} %秩
    \DeclareMathOperator{\tr}{Tr} %迹
    \newcommand{\tran}[1]{{#1}^{\mathrm{T}}} %转置
    \newcommand{\ctran}[1]{{#1}^{\dagger}} %共轭转置
    \newcommand{\itran}[1]{{#1}^{-\mathrm{T}}} %逆转置
    \newcommand{\ictran}[1]{{#1}^{-\dagger}} %逆共轭转置
    \DeclareMathOperator{\codim}{codim} %余维数
    \DeclareMathOperator{\diag}{diag} %对角阵
    \newcommand{\norm}[1]{\left\| #1\right\|} %范数
    \DeclareMathOperator{\lspan}{span} %张成
    \DeclareMathOperator{\sym}{\mathfrak{Y}}
% 群
    \DeclareMathOperator{\inn}{Inn} %(群)内自同构
    \newcommand{\nsg}{\vartriangleleft} %正规子群
    \newcommand{\gsn}{\vartriangleright} %正规子群
    \DeclareMathOperator{\ord}{ord} %元素的阶
    \DeclareMathOperator{\stab}{Stab} %稳定化子
    \DeclareMathOperator{\sgn}{sgn} %符号函数
% 环, 域
    \DeclareMathOperator{\cha}{char} %特征
    \DeclareMathOperator{\spec}{Spec} %素谱
    \DeclareMathOperator{\maxspec}{MaxSpec} %极大谱
    \DeclareMathOperator{\gal}{Gal}
% 微积分
    % \newcommand*{\dif}{\mathop{}\!\mathrm{d}} %(外)微分算子
% 流形
    \DeclareMathOperator{\lie}{Lie}
%代数几何
    \DeclareMathOperator{\proj}{Proj}
%多项式
    \DeclareMathOperator{\disc}{disc} %判别式
    \DeclareMathOperator{\res}{res} %结式

% 结构简写
    \newcommand{\pdfrac}[2]{\dfrac{\partial #1}{\partial #2}} %偏微分式
    \newcommand{\isomto}{\stackrel{\sim}{\rightarrow}} %有向同构
    \newcommand{\gene}[1]{\left\langle #1 \right\rangle} %生成对象
% 文字缩写
    \newcommand{\opin}{\;\mathrm{open\;in}\;}
    \newcommand{\st}{\;\mathrm{s.t.}\;}
    \newcommand{\ie}{\;\mathrm{i.e.,}\;}

% 重定义命令
\renewcommand{\hom}{\mathop{Hom}}
\renewcommand{\vec}{\boldsymbol}
\renewcommand{\and}{\;\text{and}\;}

% 编号
\newcommand{\cnum}[1]{$#1^\circ$} %右上角带圆圈的编号
\newcommand{\rmnum}[1]{\romannumeral #1}


\newcommand{\myit}{$\diamond$}

\title{Elliptic Curves}
\author{Lei Bichang}
\date{Sep 19}

\begin{document}
\maketitle


\subsection*{Exercise 1}
\begin{enumerate}
\item A point $(x, y)\in C_P$ is singular iff \begin{equation}\label{ex1 eq1}
    \begin{cases}
        y^2 = P(x),\\
        2y = P'(x) = 0.
    \end{cases}
\end{equation}
If $\cha k\ne 2$, then the condition is equivalent to $y = 0$ and $x$ being a multiple root of $P$.
Hence $C_P$ is smooth iff $P$ has no multiple roots in $k$.

If $\cha k = 2$, then since $k$ is algebraically closed, \cref{ex1 eq1} can be satisfied whenever $P'$ has a root in $k$.
So $C_P$ is smooth iff $P'$ is a nonzero constant, which means that $P(X)$ has degree $1$.

\item The projective closure $\bar{C}_P$ of $C_P$ in $\P^2(k)$ is defined by \begin{equation*}
    \frac{Y^2}{Z^2} = P\left( \frac{X}{Z} \right).
\end{equation*}
If $\cha k = 2$ and $\bar{C}_P$ is smooth, then $P(X) = aX + b$ with $a\in k^\times$ and $b\in k$, so $\bar{C}_P$ is defined by \[Y^2 = aXZ + bZ^2.\]
Let $F := aXZ + bZ^2 - Y^2$.
and $A = [x : y : 0]\in\bar{C}_P$. If $x = 0$, then $y^2 = 0$, which implies $y = 0$. Therefore $x\ne 0$, and thus \[\pdfrac{F}{Z}(A) = ax \ne 0.\] Hence $\bar{C}_P$ is smooth.

Now suppose $\cha k\ne 2$ and $P$ has no multiple roots. Then $\bar{C}_P$ is smooth iff all the points of $\bar{C}_P$ on the chart $X \ne 0$ are smooth.

Wrong answer:
\begin{enumerate}
    \item [] Let $C'$ be the intersection of $\bar{C}_P$ with the chart $X\ne 0$, $m := \deg P$ and $2n \ge m$ be an even integer.
    % then $C'$ is an affine curve defined by \[(Z^{n-1}Y)^2 = Z^{2n}P\left(\frac{X}{Z}\right).\]
    Let $z:=Z/X$ and $w:= Z^{n-1}Y/X^n$,
    then $z^{2n}P(1/z)\in k[z]$, and $C'$ is the affine plane curve\[w^2= z^{2n}P\left( \frac{1}{z} \right).\]
    {\color{red}This is not $C'$. It cannot even embed into $\P^2$.}
    By assumption, write \[P(X) = \prod_{i=1}^m (X-a_i)\]with $a_i\in k$ distinct, then \[z^{2n}P\left( \frac{1}{z} \right) = z^{2n-m}\prod_{i=1}^m(1-a_iz).\]
    Therefore, the polynomial $z^{2n}P\left( \frac{1}{z} \right)$ would have no multiple roots, if we choose $n$ s.t. $2n$ is the smallest integer greater than or equal to $m$. Such $n$ always exists, and taking such $n$ tells us that $C'$ is smooth. Hence, $\bar{C}_P$ is smooth.
    
    In conclusion, the projective closure $\bar{C}_P$ is smooth iff $P$ is smooth.
    
\end{enumerate}


\item If $\cha k\mid d$, then $C_P$ is smooth iff $P$ has degree $1$, and $\bar{C}_P$ is smooth in this case.

If $\cha k \nmid d$, $C_P$ is smooth iff $P$ has no multiple roots.
The projective closure $\bar{C}_P$ is now defined by \[\frac{Y^d}{Z^d} = P\left( \frac{X}{Z} \right).\]
Look at the affine curve $C' := C\cap\{[X : Y]\in\P^1 | X \ne 0\}$ again, and let $n$ be the smallest integer s.t. $dn\ge \deg P$, $z := Z/X$, $w := Z^{n-1}Y/X$, then $C'$ is the affine plane curve defined by \[w^d = z^{dn}P\left( \frac{1}{z} \right).\]
This curve is smooth iff $dn = \deg P$ or $dn = \deg P + 1$. Therefore, $\bar{C}_P$ is smooth iff $P$ has no multiple roots, and $d$ divides $\deg P$ or $\deg P+1$.

\end{enumerate}

\subsection*{Exercise 2}
\begin{enumerate}
    \item A point $[x, y, z]\in C$ is singular iff \begin{equation}\label{ex2 eq1}
        \begin{cases}
            x^3 + y^3 + z^3 + dxyz = 0,\\
            3x^2 = - dyz,\\
            3y^2 =- dxz,\\
            3z^2 =- dxy.
        \end{cases}
    \end{equation}
    Note that if one of $x, y, z$ is zero, the other two are also zero. Hence $xyz \ne 0$.
    Multiply the last three equations and divide the result by $(xyz)^2$, we get $d^3 = -27$.

    Conversely, suppose $d^3 = -27$,
    then $d = -3\omega$ with $\omega\in\mu_3\subset \bar{k}$. 
    Note that $[1 : \omega : \omega]$ is a singular point on $C$, so $C$ is not smooth.

    \item Since $O = [1 : -1 : 0]\in C(k)$, we can deduce that $C$ is an elliptic curve once we know that the genus of $C$ is $g_C = 1$.

    Let \[\pi : C\to\P^1,\quad [x : y : z] \mapsto [x : y].\]
    This rational map is nonconstant and have degree $3$.

    Consider $P = [x : 1 : z]\in C$ in the chart $Y\ne 0$. The corresponding affine curve $C_0$ is \begin{equation}\label{ex2 y=1}
        z^3 + dxz + x^3 + 1 = 0
    \end{equation} and the map is \[\pi(x, z) = x\]
    If $\pi$ ramifies at $P$, then the equation, regarded as a polynomial in $z$, would have discriminant \begin{equation}\label{ex2 disc}
        -(4\cdot (dx)^3 + 27\cdot (x^3+1)^2) = 0.
    \end{equation}
    So $x^3$ is a solution to a quadratic equation with no multiple roots, and thus gives us $6$ values of $x$ s.t. $P$ could possibly be a ramification point.
    \begin{itemize}
        \item {\color{red}If $d \ne 0$}, then clearly $e_\pi(P) \ne 3$.   
        Therefore, $\pi$ has six ramification points of index $2$ in $C_0$.
        \item {\color{red} If $d = 0$, then the solution to \cref{ex2 disc} are $x = -\omega$ with $\omega^3 = 1$, and \cref{ex2 y=1} becomes \[z^3 = 1.\]
        Hence, $\pi$ has three ramification points $[-\omega : 1 : 0]$ of degree $3$.}
    \end{itemize}

    On $C\setminus C_0$, $X^3 + Z^3 = 0$, so \[C\setminus C_0 = \{[1 : 0 : -1], [1 : 0 : -\omega], [1 : 0 : -\omega^2]\},\]
    where $\omega\in\mu_3$ and $\omega\ne 1$.
    Working on the chart $X\ne 0$, the corresponding affine curve is \[z^3 + dyz + y^3 + 1\] and the map is \[\pi(y, z) = y.\]
    So $\pi$ does not ramify at the points in $C\setminus C_0$.

    By Riemann-Hurwitz formula,
    \[2g_C-2 = 3\cdot (-2)  + 6\cdot 1\] when $d\ne 0$, or {\color{red}\[2g_C-2 = 3\cdot (-2)  + 3\cdot 2\] when $d = 0$.}
    Therefore, $g_C = 1$.



\end{enumerate}

\subsection*{Exercise 3}
The curve $E$ is given by a Weierstrass equation with $4\cdot 1^3 + 27\cdot 0 = 4\ne 0$, so $E$ together with $O := [0 : 1 : 0]\in E(\F_5)$ defines an elliptic curve over $\F_5$.

Suppose that $[x : y : 1]\in E(\F_5)$. Then $x(x^2+1) = x^3 + x = y^2$ is a square in $\F_5$, which is $0$, $1$ or $4 = -1$.
Direct computation shows that
\begin{align*}
    E(\F_5) = \left\{[0 : 0 : 1], [2 : 0 : 1], [-2 : 0 : 1], [0 : 1 : 0]\right\}.
\end{align*}


\subsection*{Exercise 4}
\begin{enumerate}
\item A point $(x, y)\in C_0$ is singular iff \begin{equation}\label{ex4 eq1}
    \begin{cases}
        (1+x)^2(1+y)^2 = xy,\\
        2(1+x)(1+y)^2 = y,\\
        2(1+x)^2(1+y) = x.
    \end{cases}
\end{equation}
If $y = 0$, then \cref{ex4 eq1} has no solution.

If $y\ne 0$, then $\cha k \ne 2$, and divide the 1st equation by the 2nd tells us that $x = 1$. By symmetry or a similar argument, we know that if $x\ne 0$ then $y = 1$.
So singularity could only appear at $(1, 1)$, which indeed satisfies \cref{ex4 eq1} only when $\cha k = 3$ {\color{red} or $\cha k = 5$}.
Therefore, $C_0$ is smooth iff $\cha k \ne 3$ or $5$.

\item The curve $\bar{C}_0$ is defined by \[(Z+X)^2(Z+Y)^2 = XYZ^2.\]
The point $O := [1 : 0 : 0]\in\bar{C}_0$ is singular, because $O$ lies in $X = 1$, while both $\pdfrac{f}{y}$ and $\pdfrac{f}{z}$ vanish at $O$, where $y = Y/X$, $z = Z/X$, and $f = (z + 1)^2(z+y)^2 + yz^2$.

\item The curve $C$ is defined by \begin{equation}\label{ex4 eq3}
    (Z+X)^2(Z' + Y)^2 = XYZZ'.
\end{equation}
The set $C\setminus C_0$ consists of the points on $C$ with $Z = 0$ or $Z' = 0$.

Let $Z = 0$ in \cref{ex4 eq3}, we get \[X^2(Z' + Y)^2 = 0,\] so $X = 0$ or $Z' + Y = 0$. But $(X : Z)$ is a homogeneous coordinate of $\P^1$, so $X$ and $Z$ cannot be zero simultaneously. Hence we have $Z' + Y = 0$, giving one point $([1 : 0], [1 : -1])$. The case of $Z' = 0$ is similar, and the result is \[C\setminus C_0 = \{O, O'\},\]
where \[O := ([1 : 0], [1 : -1]),\ O' := ([1, -1], [1, 0]).\]

\item Let $C_0$ be smooth. It suffices to check that $O$ and $O'$ are smooth.
These points lie in the chart $X = Y = 1$. Let $z = Z/X$, $z' = Z'/Y$, then the defining equation becomes \begin{equation}\label{ex4 eq4}
    (z + 1)^2(z' + 1)^2 = zz'.
\end{equation}
The affine curve defined by this equation is isomorphic to $C_0$, so it is smooth. Therefore $O$ and $O'$ are smooth, and thus $C$ is smooth.

\item Both $O$ and $O'$ are $k$-rational, so we just need to calculate the genus $g_C$ of $C$.
\textit{I assume $\cha k\ne 2$ from now on, because I don't know how to deal with $\cha k = 2$...}

Let $f : E\to \P^1$ be the composition of the embedding $E\hookrightarrow \P^1\times\P^1$ and the projection $\P^1\times\P^1\twoheadrightarrow\P^1$ to the first factor; i.e.,
\[f([X : Z], [Y, Z']) = [X : Z].\]
This map is nonconstant of degree $2$.
A point $(x, y)\in C_0$ is a ramification point for $f$ iff $y$ is a double root of the polynomial \[(1+x)^2(1+Y)^2-xY\] in $Y$, i.e., the discriminant \[(2(1+x)^2 - x)^2 - 4(1+x)^4 = -x(4x^2 + 7x + 4) = 0.\] So $f$ ramifies at three points in $C_0$. In particular, $f$ ramifies at $(x, y) = (0, -1)$ and does not ramify at $(x, y) = (-1, 0)$. Then by looking at \cref{ex4 eq4}, we deduce immediately that $f$ ramifies at $O$ and does not ramify at $O'$.

Now we can apply the Riemann-Hurwitz formula to $f$, and obtain \[2g_C - 2 = 2\cdot (2\cdot 0 - 2) + 4\cdot (2 - 1),\]
given $\cha k \ne 2$. Hence $g_C = 1$.

\textit{Weierstrass equation.}
The equation of $C_0$ is \[(1+x)^2y^2 + (2(1+x)^2 - x)y + (1+x)^2 = (1+x)^2y^2 + (2x^2+3x+2) y + (1+x)^2.\]
Under the birational transformation
\[\A^2\dashrightarrow \A^2,\ (x, y) \mapsto (x, 2(1+x)^2y + 2x^2 + 3x+2),\]
the equation becomes \[y^2 - (2x^2+3x+2)^2 + 4(1 + x)^4 = 0,\]
which reduces to \[y^2 = -4x^3 - 7x^2 - 4x.\]



\end{enumerate}

\end{document}