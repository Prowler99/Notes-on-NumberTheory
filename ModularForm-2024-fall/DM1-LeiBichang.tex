\documentclass{article}
\usepackage{amsmath, amssymb, amsthm, amsbsy, mathrsfs, stmaryrd}
\usepackage{enumitem}
\usepackage[colorlinks,
linkcolor=cyan,
anchorcolor=blue,
citecolor=blue,
]{hyperref}
\usepackage[capitalize]{cleveref}
\usepackage[margin = 1in, headheight = 12pt]{geometry}
\usepackage{bbm}
\usepackage{tikz-cd}

\newtheorem{theorem}{Theorem}

\theoremstyle{definition}
\newtheorem{definition}{Definition}
\newtheorem{exercise}{Exercise}[section]
\newtheorem{problem}{Problem}
\newtheorem{example}{Example}
\newtheorem{proposition}{Proposition}[section]
\newtheorem{lemma}{Lemma}[section]
\newtheorem{corollary}{Corollary}[section]

\theoremstyle{remark}
\newtheorem*{remark}{Remark}

\renewcommand{\Re}{\mathop{\mathrm{Re}}}
\renewcommand{\Im}{\mathop{\mathrm{Im}}}

% 新命令
% 数学对象
    \newcommand{\R}{\mathbb{R}}
    \newcommand{\C}{\mathbb{C}}
    \newcommand{\Q}{\mathbb{Q}}
    \newcommand{\Z}{\mathbb{Z}}
    \DeclareMathOperator{\GL}{GL}
    \DeclareMathOperator{\SL}{SL}
    \newcommand{\p}{\mathfrak{p}}
    \renewcommand{\P}{\mathbb{P}}
    \newcommand{\A}{\mathbb{A}}
    \newcommand{\F}{\mathbb{F}}
% 集合
    \newcommand{\sminus}{\smallsetminus} %(集合)差
% 范畴
    \newcommand{\op}[1]{{#1}^{\mathrm{op}}} %反范畴
    \DeclareMathOperator{\enom}{End} %自态射
    \DeclareMathOperator{\isom}{Isom} %同构
    \DeclareMathOperator{\aut}{Aut} %自同构
    \DeclareMathOperator{\im}{im} %像
    \newcommand{\Set}{\mathbf{Set}} %集合范畴
    \newcommand{\Abel}{\mathbf{Ab}} %群范畴
    \newcommand{\Ring}{\mathbf{Ring}}
    \newcommand{\Cring}{\mathbf{CRing}}
    \newcommand{\Alg}{\mathbf{Alg}}
    \newcommand{\Mod}{\mathbf{Mod}}
    \DeclareMathOperator{\Id}{id}
%向量空间, 矩阵
    \DeclareMathOperator{\rank}{rank} %秩
    \DeclareMathOperator{\tr}{Tr} %迹
    \newcommand{\tran}[1]{{#1}^{\mathrm{T}}} %转置
    \newcommand{\ctran}[1]{{#1}^{\dagger}} %共轭转置
    \newcommand{\itran}[1]{{#1}^{-\mathrm{T}}} %逆转置
    \newcommand{\ictran}[1]{{#1}^{-\dagger}} %逆共轭转置
    \DeclareMathOperator{\codim}{codim} %余维数
    \DeclareMathOperator{\diag}{diag} %对角阵
    \newcommand{\norm}[1]{\left\| #1\right\|} %范数
    \DeclareMathOperator{\lspan}{span} %张成
    \DeclareMathOperator{\sym}{\mathfrak{Y}}
% 群
    \DeclareMathOperator{\inn}{Inn} %(群)内自同构
    \newcommand{\nsg}{\vartriangleleft} %正规子群
    \newcommand{\gsn}{\vartriangleright} %正规子群
    \DeclareMathOperator{\ord}{ord} %元素的阶
    \DeclareMathOperator{\stab}{Stab} %稳定化子
    \DeclareMathOperator{\sgn}{sgn} %符号函数
% 环, 域
    \DeclareMathOperator{\cha}{char} %特征
    \DeclareMathOperator{\spec}{Spec} %素谱
    \DeclareMathOperator{\maxspec}{MaxSpec} %极大谱
    \DeclareMathOperator{\gal}{Gal}
% 微积分
    % \newcommand*{\dif}{\mathop{}\!\mathrm{d}} %(外)微分算子
% 流形
    \DeclareMathOperator{\lie}{Lie}
%代数几何
    \DeclareMathOperator{\proj}{Proj}
%多项式
    \DeclareMathOperator{\disc}{disc} %判别式
    \DeclareMathOperator{\res}{res} %结式

% 结构简写
    \newcommand{\pdfrac}[2]{\dfrac{\partial #1}{\partial #2}} %偏微分式
    \newcommand{\isomto}{\stackrel{\sim}{\rightarrow}} %有向同构
    \newcommand{\gene}[1]{\left\langle #1 \right\rangle} %生成对象
% 文字缩写
    \newcommand{\opin}{\;\mathrm{open\;in}\;}
    \newcommand{\st}{\;\mathrm{s.t.}\;}
    \newcommand{\ie}{\;\mathrm{i.e.,}\;}

% 重定义命令
\renewcommand{\hom}{\mathop{Hom}}
\renewcommand{\vec}{\boldsymbol}
\renewcommand{\and}{\;\text{and}\;}

% 编号
\newcommand{\cnum}[1]{$#1^\circ$} %右上角带圆圈的编号
\newcommand{\rmnum}[1]{\romannumeral #1}


\newcommand{\myit}{$\diamond$}

\title{Modular Forms}
\author{Lei Bichang}
\date{}

\begin{document}
\maketitle


\subsection*{Exercise 1}
\begin{itemize}
    \item We know $\#\GL_2(\F_p) = (p^2-1)(p^2-p)$  and \[\Gamma(p) = \ker \left[ \det : \GL_2(\F_p)\twoheadrightarrow\F_p^\times \right],\]
    therefore\[[\SL_2(\Z) : \Gamma(p)] = \#\SL_2(\F_p) = \frac{(p^2-1)(p^2-p)}{p-1} = p^3-p.\]
    
    \item Note that \[\Gamma_1(p)\to \F_p,\quad \begin{pmatrix}
        a & b \\ c & d
    \end{pmatrix}\mapsto b\bmod p\] is a surjective group homomorphism with kernel $\Gamma(p)$, so \[[\SL_2(\Z) : \Gamma_1(p)] = \frac{[\SL_2(\Z) : \Gamma(p)]}{[\Gamma_1(p) : \Gamma(p)]} = \frac{p^3-p}{p} = p^2-1.\]

    \item Note that \[\Gamma_0(p)\to\F_p^\times,\quad 
    \begin{pmatrix}
        a & b \\ c & d
    \end{pmatrix}\mapsto a\bmod p\] is a surjective group homomorphism with kernel $\Gamma_1(p)$,
    so \[[\SL_2(\Z) : \Gamma_0(p)] = \frac{[\SL_2(\Z) : \Gamma_1(p)]}{[\Gamma_0(p) : \Gamma_1(p)]} = \frac{p^2-1}{p-1} = p + 1.\]
\end{itemize}

\subsection*{Exercise 2}
\begin{itemize}
\item \textit{Existence.}
From Exercise 1, we know that \[[\SL_2(\Z) : \Gamma(2)] = 6.\]
Hence, finding such a $\Gamma$ is equivalent to finding a level $2$ congruence subgroup of $\SL_2(\Z)$ in which the index of $\Gamma(2)$ is $3$.

Let \[A := \begin{pmatrix}
    0 & -1 \\ 1 & 1
\end{pmatrix}\] and $\Gamma$ the subgroup of $\SL_2(\Z)$ generated by $\Gamma(2)$ and $A$.
Note that \[A\notin \Gamma(2),\ A^2 = \begin{pmatrix}
    -1 & -1 \\ 1 & 0
\end{pmatrix}\notin \Gamma(2),\ A^3 = -I\in\Gamma(2),\]
and $\Gamma(2)\triangleleft\Gamma$ since $\Gamma(2)\triangleleft \SL_2(\Z)$. Therefore, $\Gamma/\Gamma(2)$ is generated by $A$ and has cardinality $3$.
So $\Gamma$ is a congruence subgroup of level $2$ and index $2$ in $\SL_2(\Z)$.

\item \textit{Uniqueness.}
Let $\Gamma'$ be such a group.
Since $[\Gamma' : \Gamma(2)] = 3$, the image of $\Gamma$ under the map $\bmod\,2 : \SL_2(\Z)\to\SL_2(\F_2)$ must be a subgroup of $\SL_2(\F_2)$ of order $3$.

The group $\SL_2(\F_2)$ is a non-abelian group of order $6$, so it is isomorphic to the symmtry group $S_3$, and it has a unique subgroup of order $3$. This subgroup is generated by $B := \begin{pmatrix}
    0&1\\1&1
\end{pmatrix}$ and must be the image of $\Gamma'$.
Since $B$ is the image of $A$ under the map $\bmod\, 2$, we have $A\in\Gamma'$, and thus $\Gamma' \supset \left<\Gamma(2), A\right> = \Gamma$. Comparing indexes shows that $\Gamma' = \Gamma$.

\item As $-I\in \Gamma(2) \subset \Gamma$, we see $\bar{\Gamma} = \Gamma$.

\item \textit{Cusp.}
The $\left<A\right>$-orbit of $\infty$ is $\{\infty, 0, -1\}$.
Every rational number can be written as $r = a/b$ with $a, b\in \Z$ satisfying one of the following three cases:
\begin{itemize}
    \item $2\nmid a$ and $2\mid b$, whence \[r\in \Gamma(2)\cdot \infty = \left\{\left.\frac{a}{c}\right|a\in 1 + 2\Z,\; c\in 2\Z\right\}\;\]
    \item $2\mid a$ and $2\nmid b$, whence \[r\in \Gamma(2)\cdot 0 = \left\{\left.\frac{b}{d}\right|b\in 2\Z,\; d\in 1+2\Z\right\};\]
    \item $2\nmid a$ and $2\nmid b$, whence \[r\in \Gamma(2)\cdot (-1) = \left\{\left.\frac{b-a}{d-c}\right|a, c\in 1 + 2\Z,\; b, d\in 2\Z\right\}.\]
\end{itemize}
Therefore, $\Gamma\cdot \infty = \Q\cup\{\infty\}$, and $\Gamma$ has only one cusp.

Since $\begin{pmatrix}
    1&2\\0&1
\end{pmatrix}\in\Gamma(2)$, the width is at most $2$.
If $T = \begin{pmatrix}
    1&1\\0&1
\end{pmatrix}\in\Gamma$, then $S = \begin{pmatrix}
    0&-1\\1&0
\end{pmatrix}= AT^{-1}\in\Gamma$. But $\SL_2(\Z)$ is generated by $S$ and $T$, so $S$ and $T$ cannot be in $\Gamma$ together. Therefore, $T\notin \Gamma$, and the width of the only cusp for $\Gamma$ is $2$.

\end{itemize}

\end{document}