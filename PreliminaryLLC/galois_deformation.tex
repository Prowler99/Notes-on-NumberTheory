\documentclass{article}

\documentclass{article}
\usepackage{amsmath, amssymb, amsthm, amsbsy, mathrsfs, stmaryrd}
\usepackage{enumitem}
\usepackage[colorlinks,
linkcolor=cyan,
anchorcolor=blue,
citecolor=blue,
]{hyperref}
\usepackage[capitalize]{cleveref}
\usepackage[margin = 1in, headheight = 12pt]{geometry}
\usepackage{bbm}
\usepackage{tikz-cd}

\newtheorem{theorem}{Theorem}

\theoremstyle{definition}
\newtheorem{definition}{Definition}
\newtheorem{exercise}{Exercise}[section]
\newtheorem{problem}{Problem}
\newtheorem{example}{Example}
\newtheorem{proposition}{Proposition}[section]
\newtheorem{lemma}{Lemma}[section]
\newtheorem{corollary}{Corollary}[section]

\theoremstyle{remark}
\newtheorem*{remark}{Remark}

\renewcommand{\Re}{\mathop{\mathrm{Re}}}
\renewcommand{\Im}{\mathop{\mathrm{Im}}}

% 新命令
% 数学对象
    \newcommand{\R}{\mathbb{R}}
    \newcommand{\C}{\mathbb{C}}
    \newcommand{\Q}{\mathbb{Q}}
    \newcommand{\Z}{\mathbb{Z}}
    \DeclareMathOperator{\GL}{GL}
    \DeclareMathOperator{\SL}{SL}
    \newcommand{\p}{\mathfrak{p}}
    \renewcommand{\P}{\mathbb{P}}
    \newcommand{\A}{\mathbb{A}}
% 集合
    \newcommand{\sminus}{\smallsetminus} %(集合)差
% 范畴
    \newcommand{\op}[1]{{#1}^{\mathrm{op}}} %反范畴
    \DeclareMathOperator{\enom}{End} %自态射
    \DeclareMathOperator{\isom}{Isom} %同构
    \DeclareMathOperator{\aut}{Aut} %自同构
    \DeclareMathOperator{\im}{im} %像
    \newcommand{\Set}{\mathbf{Set}} %集合范畴
    \newcommand{\Abel}{\mathbf{Ab}} %群范畴
    \newcommand{\Ring}{\mathbf{Ring}}
    \newcommand{\Cring}{\mathbf{CRing}}
    \newcommand{\Alg}{\mathbf{Alg}}
    \newcommand{\Mod}{\mathbf{Mod}}
    \DeclareMathOperator{\Id}{id}
%向量空间, 矩阵
    \DeclareMathOperator{\rank}{rank} %秩
    \DeclareMathOperator{\tr}{Tr} %迹
    \newcommand{\tran}[1]{{#1}^{\mathrm{T}}} %转置
    \newcommand{\ctran}[1]{{#1}^{\dagger}} %共轭转置
    \newcommand{\itran}[1]{{#1}^{-\mathrm{T}}} %逆转置
    \newcommand{\ictran}[1]{{#1}^{-\dagger}} %逆共轭转置
    \DeclareMathOperator{\codim}{codim} %余维数
    \DeclareMathOperator{\diag}{diag} %对角阵
    \newcommand{\norm}[1]{\left\| #1\right\|} %范数
    \DeclareMathOperator{\lspan}{span} %张成
    \DeclareMathOperator{\sym}{\mathfrak{Y}}
% 群
    \DeclareMathOperator{\inn}{Inn} %(群)内自同构
    \newcommand{\nsg}{\vartriangleleft} %正规子群
    \newcommand{\gsn}{\vartriangleright} %正规子群
    \DeclareMathOperator{\ord}{ord} %元素的阶
    \DeclareMathOperator{\stab}{Stab} %稳定化子
    \DeclareMathOperator{\sgn}{sgn} %符号函数
% 环, 域
    \DeclareMathOperator{\cha}{char} %特征
    \DeclareMathOperator{\spec}{Spec} %素谱
    \DeclareMathOperator{\maxspec}{MaxSpec} %极大谱
    \DeclareMathOperator{\gal}{Gal}
% 微积分
    % \newcommand*{\dif}{\mathop{}\!\mathrm{d}} %(外)微分算子
% 流形
    \DeclareMathOperator{\lie}{Lie}
%代数几何
    \DeclareMathOperator{\proj}{Proj}
%多项式
    \DeclareMathOperator{\disc}{disc} %判别式
    \DeclareMathOperator{\res}{res} %结式

% 结构简写
    \newcommand{\pdfrac}[2]{\dfrac{\partial #1}{\partial #2}} %偏微分式
    \newcommand{\isomto}{\stackrel{\sim}{\rightarrow}} %有向同构
    \newcommand{\gene}[1]{\left\langle #1 \right\rangle} %生成对象
% 文字缩写
    \newcommand{\opin}{\;\mathrm{open\;in}\;}
    \newcommand{\st}{\;\mathrm{s.t.}\;}
    \newcommand{\ie}{\;\mathrm{i.e.,}\;}

% 重定义命令
\renewcommand{\hom}{\mathop{Hom}}
\renewcommand{\vec}{\boldsymbol}
\renewcommand{\and}{\;\text{and}\;}

% 编号
\newcommand{\cnum}[1]{$#1^\circ$} %右上角带圆圈的编号
\newcommand{\rmnum}[1]{\romannumeral #1}


\newcommand{\myit}{$\diamond$}

\title{}
\author{}
\date{}

\begin{document}
\maketitle

\end{document}

\newcommand{\F}{\mathbb{F}}
\newcommand{\compl}[1]{\widehat{\mathfrak{Ar}_{#1}}}
\renewcommand{\O}{\mathcal{O}}
\newcommand{\m}{\mathfrak{m}}
\DeclareMathOperator{\ad}{ad}
\newcommand{\gp}{\mathrm{gp}}
\DeclareMathOperator{\der}{Der}
\newcommand{\cont}{\mathrm{cont}}

\title{Galois Deformations}
\author{-}

\begin{document}
\maketitle

\section{Review of Category Theory and Homological Algebra}

All the set-theoretic issues are ignored for now.

\subsection{Representability}
Let $\mathfrak{C}$ be a category.
We define the functors\footnote{
    There is also the version for $h_{\mathfrak{C}} : \mathfrak{C}\to [\mathfrak{C}^{\opp}, \Set]$ and $\ev_{\mathfrak{C}} : [\mathfrak{C}^\opp, \Set]\times\mathfrak{C}\to \Set$.
}
\[\begin{array}{rlcrl}
	h^{\mathfrak{C}}: {\mathfrak{C}}^\opp &\hspace{-0.5em}\longrightarrow [\mathfrak{C}, \Set],
    &  & \ev^{\mathfrak{C}}: [\mathfrak{C}, \Set] \times \mathfrak{C} & \hspace{-0.5em}\longrightarrow \Set \\
	S & \hspace{-0.5em}\longmapsto \Hom_{\mathfrak{C}}(S, \cdot)
    &  & (F, S) & \hspace{-0.5em}\longmapsto F(S).
\end{array}\]
\begin{theorem}[Yoneda]\label{Yoneda lemma}
    There is an isomorphism
    \[\Hom_{[\mathfrak{C, \Set}]^\opp}(-, h^{\mathfrak{C}}(-))\simeq \ev^{\mathfrak{C}}\] as functors $[\mathfrak{C}, \Set] \times \mathfrak{C}\to\Set$ given by
    \begin{align*}
    \Hom_{{[\mathfrak{C}, \Set]^\opp}}\left(F, h^{\mathfrak{C}}(S)\right)
    & \longrightarrow F(S)
    \\ \left( 
        F \stackrel{\phi}{\leftarrow}\Hom_{\mathfrak{C}}(S, -)
     \right)
    & \longmapsto \phi_S(\Id_S)
   \end{align*}
   for all $F : \mathfrak{C}\to\Set $ and $S\in\mathfrak{C}$,
   and the functor $h^\mathfrak{C} : \mathfrak{C}^\opp\to [\mathfrak{C}, \Set]$ is fully faithful.
\end{theorem}
% \begin{proof}
    
% \end{proof}

We say that a functor $F : \mathfrak{C}\to\Set$ is \textbf{representable}, if there is $X\in\mathfrak{C}$ along with an isomorphism
\[\phi : \Hom_\mathfrak{C}(X, -)\simeq F\]
as functors.
Note that the functor $\phi$ is determined\footnote{
    This does \textit{not} mean that we can decode $\phi$ from $u$ without knowing $\phi$ a priori?
} by the universal element $u := \phi_X(\Id_X)\in F(X)$,
from which every thing in $F(T)$ is pushed forward, i.e. for any morphism $f : X\to T$ in $\mathfrak{C}$,
the unique corresponding element in $F(T)$ is
$\phi_T(f) = F(f)(\phi_X(\Id_X)) = F(f)(u)$.

\subsection{The Ext Functors}

% \begin{definition}
%     Let $F : \mathfrak{A}_1\times\mathfrak{A}_2\to\mathfrak{B}$
%     be a bi-functor between abelian categories
%     that is \textit{left-exact} on both variables.
%     We say that $F$ is \textbf{balanced},
%     if $F(-, I_2)$ and $F(I_1, -)$ are both \textit{exact} for every injective object $I_i\in\mathfrak{A}_i$.
% \end{definition}
% \begin{theorem}
%     If $F : \mathfrak{A}_1\times\mathfrak{A}_2\to\mathfrak{B}$ is a balanced left-exact functor,
%     and both $\mathfrak{A}_i$ has enough injectives,
%     then 
% \end{theorem}

Let $\mathfrak{A}$ be an abelian category with enough projective and injective objects.
We have \[\ext^i_\mathfrak{A}(X, Y) := \mathrm{R}^i\Hom_\mathfrak{A}(X, -)(Y)\simeq \mathrm{R}^i\Hom_\mathfrak{A}(-, Y)(X)\]
for $X, Y\in \mathfrak{A}$, $i\ge 0$.

We will focus on $\ext^1$.
An \textbf{extension of $A$ by $B$}\footnote{In a category where these operations make sense.} is a short exact sequence \[\xi : 0\to B\to X\to A\to 0.\]
(I may denote $\xi$ by $X$ if there is no confusion.)
An isomorphism of two extensions $X$ and $X'$ of $A$ by $B$ is a commutative diagram
% https://q.uiver.app/#q=WzAsMTAsWzAsMCwiMCJdLFsxLDAsIkIiXSxbMiwwLCJYIl0sWzMsMCwiQSJdLFs0LDAsIjAiXSxbNCwxLCIwIl0sWzAsMSwiMCJdLFsxLDEsIkIiXSxbMiwxLCJYJyJdLFszLDEsIkEiXSxbMCwxXSxbMSwyXSxbMiwzXSxbMyw0XSxbNiw3XSxbNyw4XSxbOCw5XSxbOSw1XSxbMyw5LCIiLDEseyJsZXZlbCI6Miwic3R5bGUiOnsiaGVhZCI6eyJuYW1lIjoibm9uZSJ9fX1dLFsxLDcsIiIsMSx7ImxldmVsIjoyLCJzdHlsZSI6eyJoZWFkIjp7Im5hbWUiOiJub25lIn19fV0sWzIsOCwiXFxzaW1lcSJdXQ==
\[\begin{tikzcd}
    0 & B & X & A & 0 \\
    0 & B & {X'} & A & 0
    \arrow[from=1-1, to=1-2]
    \arrow[from=1-2, to=1-3]
    \arrow[equals, from=1-2, to=2-2]
    \arrow[from=1-3, to=1-4]
    \arrow["\simeq", from=1-3, to=2-3]
    \arrow[from=1-4, to=1-5]
    \arrow[equals, from=1-4, to=2-4]
    \arrow[from=2-1, to=2-2]
    \arrow[from=2-2, to=2-3]
    \arrow[from=2-3, to=2-4]
    \arrow[from=2-4, to=2-5]
\end{tikzcd}\]
An extension of $A$ by $B$ that is isomorphic to \[0\to B \hookrightarrow A\oplus B\to A\to 0\]
is said to be split.

Given an extension $\xi : 0\to B\to X\to A\to 0$ of $A$ by $B$,
the cohomological functors $\ext^*(A, -)$
induces the exact sequence\[
\hom(A, X)\to\hom(A, A)\stackrel{\partial_\xi}{\to}\ext^1(A, B).\]
Let's look at the class $\Theta(\xi) := \partial_\xi(\Id_A)\in \ext^1(A, B) = 0$.
If $\Theta(\xi) = 0$,
then there is a section $f : A\to X$ of $X\to A$ in $\xi$, i.e. $\xi$ is split.
This means that $\Theta(\xi)\in\ext^1(A, B)$
is the \textit{obstruction} for $\xi$ to be split.

\begin{theorem}\label{Ext1 classifies extensions}
    Let $R$ be a (possibly non-commutative) ring.
    For left $R$-modules $A$ and $B$,
    there is a natural bijection
    \[\left\{ \begin{aligned}
        &\text{isomorphism classes of}\\
        & \text{extensions of } A \text{ by } B
    \end{aligned}\right\}\stackrel{1 : 1}{\longleftrightarrow}\ext^1_R(A, B)\]
    given by $\Theta : \xi\mapsto \partial_\xi(\Id_A)$.
\end{theorem}
% \begin{proof}
%     By naturality of the long exact sequence,
%     $\Theta(\xi)$ depends only on the isomorphism class of $\xi$ ?

%     Fix an extension \[0\to M\stackrel{j}{\to} P\to A\to 0\]with $P$ projective.
%     \begin{enumerate}
%     \item[] \textit{Surjectivity}.
%     (how?)Take $x\in\ext^1(A, B)$.

%     \item[] \textit{Injectivity}.
%     (how?)
%     \qedhere
%     \end{enumerate}
% \end{proof}

\begin{example}\label{extension of group module by self = H^1 in adjoint}
Let $k$ be a topological ring (field if necessary),
$G$ be a topological group,
$V$ be a continuous $k[G]$-module that is free of $k$-rank $d$.
Then there is a canonical isomorphism
\[\ext^1_{k[G]}(V, V)\simeq H^1(G, \ad V).\]
(There should be a constructive proof, but I failed...)
% \begin{proof}
%     We construct an isomorphism similar to that in \cref{Ext1 classifies extensions}.
%     An extension \[\xi: 0\to V\stackrel{i }{\to} E\stackrel{p }{\to} V\to 0\]
%     of $k[G]$-modules.
%     Tensoring with $V^\vee$ over $k$,
%     we get
%     % \[0\to\ad V\to E\otimes_k V^\vee\to \ad V\to 0.\]
%     \[0\to\ad V\stackrel{i_*}{\to} E\otimes_k V^\vee\stackrel{p_*}{\to} \ad V\to 0.\]
    
%     % % Trial to construct the isomophism of extensions
%     % Now the natural $k$-linear map $\ev : \ad V\to k$
%     % is $G$-invariant,
%     % hence $L_V := \ker(\ad V\to k)$ is a $k[G]$-module, giving an extension
%     % \[0\to \ad V\stackrel{i_*}{\to} \frac{E\otimes V^\vee}{i_*(L_V)}\stackrel{\ev\circ p_*}{\longrightarrow}k\to 0.\]
%     % % CNNOT TENSOR WITH DUAL! Even dimensions don't match!

%     Then we take the long exact sequence
%     \[(E\otimes_k V^\vee)^G\to (\ad V)^G\stackrel{\partial}{\to} H^1(G, \ad V) \]
%     and consider the class $\Phi(\xi) := \partial(\Id_V)\in H^1(G, \ad V)$ $= \ext^1_G(k, \ad V)$.
%     What next? Do I need the map $Z^1(G, \ad V)\to \ext^1_G(k, \ad V)$?
    

% \end{proof}
We propose another proof in the next subsection.
\end{example}

\subsection{Universal \texorpdfstring{$\delta$}{delta}-Functors}
We concentrate on cohomological things.
% \subsubsection{\texorpdfstring{$\delta$}{delta}-functors}
\begin{definition}
    A (covariant) \textbf{cohomological $\delta$-functor}
    is a collection of additive functors \[\{T^n : \mathfrak{A}\to\mathfrak{B}\}_{n\ge 0}\]
    indexed by non-negative integers,
    which induces  \textit{functorially} a long exact sequences in $\mathfrak{B}$ from a short exact sequence in $\mathfrak{A}$.
    More precisely, for each exact sequence
    \[0\to A\to B\to C\to 0\quad\text{in }\mathfrak{A},\]
    there are fixed morphisms
    \[\delta^n : T^n(C)\to T^{n+1}(A)\quad\text{in }\mathfrak{B},\quad n\ge 0,\]
    s.t.
    \[0\to T^0(A)\to T^0(B)\to T^0(C)\stackrel{\delta^0}{\to} T^1(A)\to\cdots\]
    is exact\footnote{
        In particular, $T^0$ is left-exact.
    };
    moreover, a morphism of short exact sequences in $\mathfrak{A}$ induces a morphism of long exact sequences in $\mathfrak{B}$.
\end{definition}

For instance,
taking cohomology for chain complexes \[H^* : \cat{Ch}_{\ge 0}(\mathfrak{A})\to\mathfrak{A}\]
% or taking group cohomology \[H^*(G, -) : k[G]\text{-}\cat{Mod}\to k\text{-}\cat{Mod}\]
or taking right-derivation of a left-exact functor
are cohomological $\delta$-functors.

\begin{definition}
    The cohomological $\delta$-functors from $\mathfrak{A}$ to $\mathfrak{B}$ form a category,
where morphisms are the natural tranformations commuting with the $\delta^n$'s.
A \textbf{universal cohomological $\delta$-functor}
is a $\delta$-functor $T = (T^n)$,
such that for any $\delta$-functor $S = (S^n)$
and a morphism $f^0 : T\to S$,
there is a unique morphism $f : T\to S$ extending $f^0$.
\end{definition}

So a universal $\delta$-functor is like an initial object among $\delta$-functors but it is ``weaker''.

\begin{theorem}
    If $F : \mathfrak{A}\to\mathfrak{B}$ is a left-exact additive functor,
    then (if $\mathfrak{A}$ has enough injectives) the right derivations $R^*F : \mathfrak{A}\to\mathfrak{B}$ form a universal $\delta$-functor.
\end{theorem}

% \subsubsection{Effaceable functors}

% An \textbf{effaceable} functor is an additive functor
% $F : \mathfrak{A}\to\mathfrak{B}$,
% such that
% \[\forall A\in\mathfrak{A},\ \exists\text{ monomorphism } u : A\hookrightarrow I\text{ in }\mathfrak{A},\st F(u) = 0\text{ in }\mathfrak{B}.\]
% \begin{theorem}
%     If $T$ is a \textit{cohomological} $\delta$-functor s.t.\ for all $n\ge 1$,
%     $T^n$ is \textit{effaceable},
%     then $T$ is a universal cohomological $\delta$-functor.
% \end{theorem}

\begin{proof}[Another proof of \cref{extension of group module by self = H^1 in adjoint}]
Let $k$ be a field.
We show that both $H^*(G, V^\vee\otimes_k(-))$ and $\ext^*_G(V, -)$ are universal $\delta$-functors. Then since they agree at $i = 0$, they must agree everywhere.

The functors $\ext^*_G(V, -)$ are derived from $\Hom_G(V, -)$,
so they are universal.
For $H^*(G, V^\vee\otimes_k(-))$,
since $V^\vee\otimes_k(-)$ is exact (Really? In which category?),
we have\footnote{
    I've never learnt this, but it seems very true and I accept it for now.
} \[H^*(G, V^\vee\otimes_k(-)) = \mathrm{R}^*\Hom_G(k, -)\circ \left( V^\vee\otimes_k(-) \right) = R^*\left( \Hom_G(k, -)\circ \left( V^\vee\otimes_k(-) \right) \right),\]
which is also a derived functor.
\end{proof}

\section{Deformation of Representations of Profinite Groups}

\subsection{The category of complete Noetherian algebras}

Let $\O$ be a Noetherian ring with residue field $k$.
We consider the category $\compl{\O}$ of \textit{complete Noetherian local $\O$-algebras with residue field $k$},
where morphisms are continuous $\O$-homomorphisms.
By ``complete'', we mean that $A\in\compl{\O}$ is a topological ring isomorphic to a projective limit of local (finite?) Artinian $\O$-algebras.
But since $A$ is Noetherian,
$A$ is $\m_A$-adically complete.
\begin{proposition}
    For every $A\in\compl{\O}$,
    the topology on $A$ equals the $\m_A$-adic topology.
    Moreover, every $\O$-algebra homomorphism $A\to B$ where $B$ is a complete local $\O$-algebra\footnote{As before, $B$ is a projective limit of local Artinian $\O$-algebras. But we don't require Noetherianity.} is continuous.
\end{proposition}

In practice(?), we take a finite extension $L/\Q_p$ with residue field $k$ and set $\O := \O_L$, the ring of integers in $L$, which is complete and contains the ring $W(k)$ of Witt vectors of $k$.



% For $A\in\compl{\O}$,
% we call \[t_A^\vee := \m_A/(\m_A^2  + \m_\O A)\]
% the \textbf{Zariski} (or \textbf{relative}) \textbf{cotangent space of $A$ over $\O$},
% and \[t_A := \Hom_{k}(\m_A/(\m_A^2  + \m_\O), k)\]
% the \textbf{relative tangent space of $A$ over $\O$}.



% \begin{proposition}
%     For any homomorphism $\O\to A$ of local rings (no need for completeness) with the same residue field
%     $k$,
%     there is a perfect pairing \[\m_A/(\m_A^2 + \m_\O A)\times \der_\O(A, k)\to k\]
%     of $k$-vector spaces,
%     where\footnote{
%         For $a\in A$, denote by $\bar a\in k$ the residue class.
%     } \[\der_\O(A, k) = \{\delta\in\Hom_{\O}(A, k)\mid \delta(ab) = \bar a\delta(b) + \bar b\delta(a),\ \forall a, b\in A\}\]
%     is the ring of $\O$-linear derivations from $A$ to $k$.
%     In particular, \[\Hom_{k}(\m_A/(\m_A^2  + \m_\O), k)\simeq \der_\O(A,k).\]
% \end{proposition}
% Before proving, note that $A = \im(\O\to A) + \m_A$, and $A$ is generated by $\im(\O\to A)$ and $\m_A$ as an $\O$-module, because $A/\m_A = \O/\m_\O$.
% \begin{proof}
%     Consider the natural pairing $\m_A\times\der_\O(A, k)\tok$ given by evaluation.

%     \noindent\textit{Right kernel.}
%     If $\delta(a) = 0$ for all $a\in\m_A$,
%     then $\delta\in\der_\O(k, k) = 0$.

%     \noindent\textit{Left kernel.}
%     Both $\m_A$ and $\m_\O$ acts by $0$ on $k$,
%     so any $\O$-linear derivation $\delta : A\tok$ (T.B.C.)
% \end{proof}

% \begin{lemma}\label{morphism of complete algebra surjective iff cotangent surjective}
%     A morphism $A\to B$ in $\compl{\O}$ is surjective if and only if the induced map
%     $t_A^*\to t_B^*$ is surjective.
% \end{lemma}
% \begin{proof}
%     For any morphism $A\to B$ in $\compl{\O}$,
%     we have the commutative diagram
% % https://q.uiver.app/#q=WzAsMTAsWzAsMCwiMCJdLFsxLDAsIlxcbV9cXE8gQS8oXFxtX1xcTyBBXFxjYXBcXG1fQV4yKSJdLFsyLDAsIlxcbV9BL1xcbV9BXjIiXSxbMywwLCJcXG1fQS8oXFxtX0FeMitcXG1fXFxPIEEpIl0sWzQsMCwiMCJdLFswLDEsIjAiXSxbMSwxLCJcXG1fXFxPIEIvKFxcbV9cXE8gQlxcY2FwXFxtX0JeMikiXSxbMiwxLCJcXG1fQi9cXG1fQl4yIl0sWzMsMSwiXFxtX0EvKFxcbV9BXjIrXFxtX1xcTyBBKSJdLFs0LDEsIjAiXSxbMCwxXSxbMSwyXSxbMiwzXSxbMyw0XSxbNSw2XSxbNiw3XSxbNyw4XSxbMSw2XSxbMiw3XSxbMyw4XSxbOCw5XV0=
% \[\begin{tikzcd}
% 	0 & {\m_\O A/(\m_\O A\cap\m_A^2)} & {\m_A/\m_A^2} & {\m_A/(\m_A^2+\m_\O A)} & 0 \\
% 	0 & {\m_\O B/(\m_\O B\cap\m_B^2)} & {\m_B/\m_B^2} & {\m_B/(\m_B^2+\m_\O B)} & 0
% 	\arrow[from=1-1, to=1-2]
% 	\arrow[from=1-2, to=1-3]
% 	\arrow[from=1-2, to=2-2]
% 	\arrow[from=1-3, to=1-4]
% 	\arrow[from=1-3, to=2-3]
% 	\arrow[from=1-4, to=1-5]
% 	\arrow[from=1-4, to=2-4]
% 	\arrow[from=2-1, to=2-2]
% 	\arrow[from=2-2, to=2-3]
% 	\arrow[from=2-3, to=2-4]
% 	\arrow[from=2-4, to=2-5]
% \end{tikzcd}\]
% where the rows are exact.
% The left column is surjective, because by the comments above,
% \[\]
% (some commutative algebra...)
% \end{proof}

\subsection{Deformation functors}

Let $G$ be a profinite group, $k$ be a finite field of characteristic $p$,
$V$ an $k[G]$-module of $k$-dimension $d$
with $G$ acting continuously\footnote{
    This means that the map \[G\times V\to V\quad (g, v)\mapsto gv\]
    is continuous; or equivalently, $G\to \GL(V)$ is continuous.
}.
We fix a $k$-basis $\beta_k$ of $V$.

Take $A\in\compl{\O}$.
A \textbf{deformation} of $V$ to $A$
is a pair $(V_A, \iota_A)$, where\begin{itemize}
    \item $V_A$ is an $A[G]$-module that is free of finite rank over $A$, and
    \item $\iota_A : V_A\otimes_{A}k\simeq V$ is an isomorphism of $k[G]$-modules. 
\end{itemize}
A \textbf{lift} or \textbf{framed deformation} of $(V, \beta_k)$ to $A$
is a triple $(V, \iota_A, \beta_A)$,
where \begin{itemize}
    \item $(V, \iota_A)$ is a deformation of $V$ to $A$,
    \item $\beta_A$ is a basis of $V_A$ over $A$ that reduces to $\beta_k$ via $\iota_A$.
\end{itemize}
Define $D_V(A)$ (resp.\ $D^\oblong_V(A)$) to be the set of isomorphism classes of deformations (resp.\ framed deformations) of $V$ to $A$.
\begin{remark}
    We can define deformations in terms of representations. But be careful to the diffrence between ``$=$'' and ``$\simeq$''!

    Let $A$ be a commutative ring.
    An $A[G]$-module $V_A$ free of finite rank $d$ over $A$ is equivalent to a finite free $A$-module $V_A^0$ with a morphism $G\to \GL(V_A^0)$.
    An isomophism $\varphi : V_A\simeq W_A$ of $A[G]$-modules is equivalent to an isomophism $\varphi^0 : V_A^0\simeq W_A^0$ such that % https://q.uiver.app/#q=WzAsMyxbMCwwLCJHIl0sWzEsMCwiXFxHTFxcbGVmdChWX0FeMFxccmlnaHQpIl0sWzEsMSwiXFxHTFxcbGVmdChXX0FeMFxccmlnaHQpIl0sWzAsMV0sWzEsMiwiXFxHTChcXHZhcnBoaV4wKSJdLFswLDJdXQ==
    \[\begin{tikzcd}
        G & {\GL\left(V_A^0\right)} \\
        & {\GL\left(W_A^0\right)}
        \arrow[from=1-1, to=1-2]
        \arrow[from=1-1, to=2-2]
        \arrow["{\GL(\varphi^0)}", from=1-2, to=2-2]
    \end{tikzcd}\]
    Now we pick an $A$-basis $\beta_A$ for $V_A^0$,
    then the datum $(V_A, \beta_A)$ is equivalent to the datum $(V_A^0, \beta_A,( \rho_V : G\to\GL_d(A)))$, and the above diagram extends to% https://q.uiver.app/#q=WzAsNSxbMCwwLCJHIl0sWzEsMCwiXFxHTFxcbGVmdChWX0FeMFxccmlnaHQpIl0sWzEsMSwiXFxHTFxcbGVmdChXX0FeMFxccmlnaHQpIl0sWzIsMCwiXFxHTF9kKEEpIl0sWzIsMSwiXFxHTF9kKEEpIl0sWzAsMV0sWzEsMiwiXFxHTChcXHZhcnBoaV4wKSJdLFswLDJdLFsxLDMsIlxcc2ltZXEiXSxbMiw0LCJcXHNpbWVxIiwyXSxbMyw0LCI/Il1d
    \[\begin{tikzcd}
        G & {\GL\left(V_A^0\right)} & {\GL_d(A)} \\
        & {\GL\left(W_A^0\right)} & {\GL_d(A)}
        \arrow[from=1-1, to=1-2]
        \arrow[from=1-1, to=2-2]
        \arrow["\simeq", from=1-2, to=1-3]
        \arrow["{\GL(\varphi^0)}", from=1-2, to=2-2]
        \arrow["{\simeq}", from=1-3, to=2-3]
        \arrow["\simeq"', from=2-2, to=2-3]
    \end{tikzcd}\]
    The last vertical arrow is induced by 
    changing the $A$-basis of $A^d$,
    so it is a conjugation by some element $C\in \GL_d(A)$,
    i.e. $\rho_W = C\rho_VC^{-1}$.
    Now it is easy to see that \[D_V^\oblong(A) = \Hom_{\gp}^{\cont}(G, \GL_d(A)),\quad D_V(A) = \frac{\Hom_{\gp}^{\cont}(G, \GL_d(A))}{\text{conjugation by an element in }\ker(\GL_d(A)\to\GL_d(k))}.\]
\end{remark}




\subsection{Representability}

A profinite group $G$ satisfies the \textbf{Mazur's finiteness condition} $\Phi_p$,
if for every \textit{open} subgroup $G'\subset G$,
the $\F_p$-vector space $\Hom_{\gp}(G', \F_p)$ of continuous group homomorphisms is finite. 

\begin{theorem}[Mazur]\label{functor of framed deformation and deformation are pro-representable under finiteness condition}
    Assume that $G$ satisfies condition $\Phi_p$.\begin{enumerate}
    \item [(a)] $D_V^\oblong$ is representable by an $R_V^\oblong\in\compl{\O}$.
    \item [(b)] If Schur's lemma $\enom_{k[G]}(V) = k$ is true, then $D_V$ is representable by an $R_V\in\compl{\O}$. 
    \end{enumerate}
    By universality, $R$ and $R_V$ are unique up to a unique isomophism in $\compl{\O}$.
\end{theorem}
\begin{remark}
    We explain how some conditions are applied.
\begin{itemize}
    % \item We may replace $\O = \O_L$ by any Noetherian local ring with residue field $k$, and the theorem holds {\color{red} Really? Need to check!}
    \item We may extend the deformation functors (to the pro-category of $\mathfrak{Ar}_\O$?) by dropping the condition of being Noetherian\footnote{That is, define $D_V^\oblong$ and $D_V$ on the category of complete local $\O$-algebras with residue field $k$.
    As before, ``complete'' means to be a projective limit of Artinian algebras.},
    and they are still representable by $R_V^\oblong$ and $R_V$.
    \item If $V$ is \textit{absolutely irreducible}, then $\enom_{k[G]}(V) = k$.
    In this simpler setting, one can construct $R_V$ as a subring of $R_V^\oblong$ directly.
    \item Without the condition $\Phi_p$, the ring $R_V^\oblong$ and $R_V$ exist (in the pro-category of $\mathfrak{Ar}_\O$) if we don't require them to be Noetherian. They are Noetherian if and only if $\dim_k H^1(G, \ad V) < +\infty$,
    and the latter condition is implied by $G$ satisfying $\Phi_p$.
\end{itemize}

\end{remark}

\subsubsection{Construction of \texorpdfstring{$R_V^\oblong$}{RVbox}}
We are looking for a universal representation $\rho^\oblong : G\to\GL_d\left( R_V^{\oblong} \right)$, in the sense that for any lift $\rho_A : G\to\GL_d(A)$ of $\bar{\rho}$ with $A\in\compl{\O}$,
there is a unique morphism $R^\oblong_V\to A$
s.t. $G\stackrel{\rho^\oblong}{\to}\GL_d\left( R_V^{\oblong} \right)\to\GL_d(A)$ equals $\rho_A$.

Suppose first that $G$ is finite
with presentation given by $s$ generators and $t$ relations: \[G = \gene{g_1, \dots, g_s\mid r_1(g_1, \dots, g_s), \dots, r_t(g_1, \dots, g_s)}.\]
Let \[\mathcal{R} := \O\left[ \left\{X_{ij}^k\right\}_{1\le i, j\le d}^{1\le k\le s} \right] \Big/ \mathcal{I},\]
where $\mathcal{I}$ is the ideal generated by all \textit{entries} of the matrices
\[r_l(X^1, \dots, X^k) - \Id,\quad X^k = \left( X^k_{ij} \right)_{i,j},\; 1\le k\le s,\ 1\le l\le t.\]
One sees immdediately that:
\begin{lemma}\label{lem: bijection between representations and homomorphism to the groud ring}
    The ring $\mathcal{R}$ is Noetherian.
    For every $A\in\compl{\O}$, there is a canonical bijection
    \begin{align*}
        \Hom_{\O\text{-alg}}(\mathcal{R}, A)\longleftrightarrow \Hom_{\text{gp}}(G, \GL_d(A))
    \end{align*}
    given by $\displaystyle\left( X_{ij}^k\mapsto a_{ij}^k \right)\longmapsto \left( g_k\mapsto \left( a^k_{ij} \right)_{1\le i, j\le d} \right)$.\qed
\end{lemma}

The ring $\mathcal{R}$ carries no topology.
Consider the kernel $\mathcal{J}$ of the homomorphism \[\mathcal{R}\to k\quad X^k_{ij}\mapsto \text{the } (i,j)\text{-entry of }\bar{\rho}(g_k),\]
namely the one corresponding to $\bar\rho$.
We define $R_V^\oblong := \varprojlim_{n} \mathcal{R}/\mathcal{J}^n$ to be the $\mathcal{J}$-adic completion of $\mathcal{R}$,
and define $\rho^\oblong : G\to \GL_d(R^\oblong_V)$ by $\rho^\oblong(g_k) := X^k$.
\begin{lemma}
    Let $G$ be a finite group.
    The ring $R_V^\oblong$ constructed above is a complete local $\O$-algebra,
    and $\rho^\oblong$ is a well-defined framed deformation that is universal.
\end{lemma}
\begin{proof}
    We verify that $\rho^\oblong$ is a (continuous) lift of $\bar\rho$ in the following steps.
    \begin{itemize}
        \item Every $\mathcal{R}/\mathcal{J}^n$ is local of dimension $0$,
        because every prime ideal contains its nilradical $\mathcal{J}/\mathcal{J}^n$, which is maximal,
        and thus it can only be $\mathcal{J}/\mathcal{J}^n$.
        In particular, $\mathcal{R}/\mathcal{J}^n$ is Artinian.
        \item $R_V^\oblong$ is local with maximal ideal $\mathfrak{m}_{R_V^\oblong} = \ker(R_V^\oblong\to \mathcal{R}/\mathcal{J})$ and residue field $k$,
        because for any $x\in R_V^\oblong\sminus \mathfrak{m}_{R_V^\oblong}$,
        we can deduce inductively that the images $x_n$ of $x$ under $R_V^\oblong\to \mathcal{R}/\mathcal{J}^n$ is non-nilpotent and, by the previous step, invertible,
        yielding a series $\left( x_n^{-1} \right)_{n}$ whose compatibility is easy to check.
        \item $\rho^\oblong$ is well-defined (i.e.\ all matrices in $\rho^\oblong(G)$ are invertible) and lifts $\bar\rho$,
        because % https://q.uiver.app/#q=WzAsNSxbMCwxLCJHIl0sWzEsMSwiXFxHTF9kKFxcRikiXSxbMiwxLCJcXEYiXSxbMiwwLCJSX1ZeXFxvYmxvbmciXSxbMSwwLCJcXEdMX2RcXGxlZnQoUl9WXlxcb2Jsb25nXFxyaWdodCkiXSxbMCwxLCJcXGJhclxccmhvIl0sWzEsMiwiXFxkZXQiXSxbMywyXSxbMCw0LCJcXHJob15cXG9ibG9uZyJdLFs0LDMsIlxcZGV0Il0sWzQsMV1d
        \[\begin{tikzcd}
            & {\mathrm{Mat}_d\left(R_V^\oblong\right)} & {R_V^\oblong} \\
            G & {\GL_d(k)} & k
            \arrow["\det", from=1-2, to=1-3]
            \arrow[from=1-2, to=2-2]
            \arrow[from=1-3, to=2-3]
            \arrow["{\rho^\oblong}", from=2-1, to=1-2]
            \arrow["{\bar\rho}", from=2-1, to=2-2]
            \arrow["\det", from=2-2, to=2-3]
        \end{tikzcd}\]commutes.
    \end{itemize}
    For universality, take a continuous lift $\rho : G\to\GL_d(A)$ of $\bar\rho$ with \[A \simeq \varprojlim_{\mathfrak{a}} A/\mathfrak{a}\]where $\mathfrak{a}\subset A$ are open ideals and $A/\mathfrak{a}$ are Artinian.
    Let $f : \mathcal{R}\to A$ be the corresponding $\O$-homomorphism obtained from \cref{lem: bijection between representations and homomorphism to the groud ring}.
    Since $\rho$ reduces to $\bar\rho$,
    we have $f(\mathcal{J})\subset\m_A$,
    and $f(\mathcal{J}^n)\subset \m_A^n$ for all $n\ge 1$.
    For any $\mathfrak{a}$,
    since $A/\mathfrak{a}$ is Artinian,
    the chain $\m_A/\mathfrak{a}\supset (\m_A/\mathfrak{a})^2\supset\cdots$ terminates.
    Hence there is some $n\ge 1$ such that $\m_A^n\subset\mathfrak{a}$,
    i.e. the composition $R\stackrel{f}{\to} A\to A/\mathfrak{a}$ is continuous w.r.t.\ the $\mathcal{J}$-adic topology on $\mathcal{R}$.
    Therefore, the map $f$ extends uniquely to a continuous homomorphism
    \[\hat f : R_V^\oblong\to A,\]
    such that $f = \hat f\circ (\mathcal{R}\to R_V^\oblong)$.
    Again by \cref{lem: bijection between representations and homomorphism to the groud ring}, the representation
    $\rho = \GL_d(\hat f)\circ\rho^\oblong : G\to\GL_d(R_V^\oblong)\to \GL_d(A)$.
\end{proof}

In the general case of $G$ being profinite,
we write $G = \varprojlim_i G/H_i$ with $H_i\subset \ker\bar{\rho}$ open and normal in $G$ and consider the universal lifts $(R_i, \rho_i)$ of the representations $G/H_i\to k$ from $\bar\rho$.
For $G/H_i\to G/H_j$, the universality of $\rho_i$ provides the dotted arrow in the commutative diagram % https://q.uiver.app/#q=WzAsNCxbMCwwLCJHL0hfaSJdLFsxLDAsIlxcR0xfZChSX2kpIl0sWzEsMSwiXFxHTF9kKFJfaikiXSxbMCwxLCJHL0hfaiJdLFswLDEsIlxccmhvX2kiXSxbMSwyLCIiLDAseyJzdHlsZSI6eyJib2R5Ijp7Im5hbWUiOiJkYXNoZWQifX19XSxbMCwzXSxbMywyLCJcXHJob19qIl1d
\[\begin{tikzcd}
    {G/H_i} & {\GL_d(R_i)} \\
    {G/H_j} & {\GL_d(R_j)}
    \arrow["{\rho_i}", from=1-1, to=1-2]
    \arrow[from=1-1, to=2-1]
    \arrow[dashed, from=1-2, to=2-2]
    \arrow["{\rho_j}", from=2-1, to=2-2]
\end{tikzcd}\]
Therefore we obtain $(R^\oblong_V, \rho^\oblong) := \varprojlim_i(R_i, \rho_i)$.
\begin{lemma}\label{lem: universal framed deformation exists in pro-artin}
    $R_V^\oblong$ is a complete local $\O$-algebra,
    and $\rho^\oblong$ is the universal framed deformation of $V$.
\end{lemma}
\begin{proof}
    The ring $R_V^\oblong$ is a projective limit of complete local $\O$-algebras, so is it.

    By definition, $D_V^\oblong = \Hom_{\gp}^\cont(G, \GL_d(-))$.
    So for $A = \varprojlim_{i} A_i$ with $A_i$ being Artinian quotients, \begin{align*}
       D_V^\oblong(A) &= \varprojlim_i\Hom_{\gp}^\cont(G, \GL_d(A_i))
       = \varprojlim_i\varinjlim_j \Hom_{\gp}^\cont(G/H_j, \GL_d(A_i))\\
       &= \varprojlim_i\varinjlim_j \Hom_{\O\text{-alg}}^\cont (R_j, A_i)
       = \varprojlim_i\Hom_{\O\text{-alg}}^\cont (R_V^\oblong, A_i)\\
       &= \Hom_{\O\text{-alg}}^\cont(R_V^\oblong, A).\footnotemark
       \qedhere
    \end{align*}
    \footnotetext{Might be some set-theoretic problems here...?}
\end{proof}


\subsubsection{Construction of \texorpdfstring{$R_V$}{RV} for absolutely irreducible \texorpdfstring{$V$}{V}}
Assume that $V$ is absolutely irreducible.
Let $R_V$ be the smallest closed sub-$\O$-algebra of $R_V^\oblong$ that contains $\tr\rho^\oblong(g)$ for all $g\in G$.
We prove that $D_V$ is representable by $R_V$ assuming the following proposition.
\begin{proposition}\label{deformation of absolutely irreducible representation is defined on the subring containing all traces of group action}
    Let $A\in\compl{\O}$,
    $W$ be an $A[G]$-module that is free of finite rank over $A$,
    $A'\subset A$ be a subring such that $A'\in\compl{\O}$ with topology induced from $A$.
    If $A'$ contains the traces of all the $G$-action on $W$, i.e.\[\tr (g|_W)\in A',\quad \forall g\in G,\]
    and $W\otimes_A k$ is absolutely irreducible,
    then there is an $A'[G]$-module that is free of finite rank over $A'$,
    such that $W'\otimes_{A'} A\simeq W$ as $A[G]$-modules.
\end{proposition}

Let $(V^\oblong, \iota^\oblong, \beta^\oblong)$ be a framed deformation given by $\rho^\oblong : G\to \GL_d(R_V^\oblong)$.
By the previous \cref{deformation of absolutely irreducible representation is defined on the subring containing all traces of group action},
there is an $R_V[G]$-module $\tilde{V}$ such with two isomophisms $\tilde{V}\otimes_{R_V} R_V^\oblong\simeq V^\oblong$ and
\[\iota : \tilde{V}\otimes_{R_V} k \simeq \tilde{V}\otimes_{R_V}R_V^\oblong\otimes_{R_V^\oblong} k\simeq V^\oblong\otimes_{R_V^\oblong} k\stackrel{\iota^\oblong}{\simeq} V\]
\begin{lemma}
    $(\tilde{V}, \iota)$ is a universal deformation of the absolutely irreducible $k[G]$-module $V$.
\end{lemma}
\begin{proof}
    Let $(V_A, \iota_A)$ be a deformation of $V$ to $A\in\compl{\O}$,
    and $(V_A, \iota_A, \beta_A)$ be a framed deformation of $V$.
    By the universality of $V^\oblong$ and \cref{deformation of absolutely irreducible representation is defined on the subring containing all traces of group action},
    there is a unique map $R_V\inject R_V^\oblong\to A$,
    % such that
    % $\tilde{V}$ is a deformation of $V_A$ to $R_V$:
    % \[\tilde{V}\otimes_{R_V} A \simeq \tilde{V}\otimes_{R_V} R_V^\oblong\otimes_{R_V^\oblong} A
    % \simeq V^\oblong \otimes_{R_V^\oblong} A \simeq V_A.\]

    % It is left to show that any deformation $W$ of $V_A$ to $R_V$ is isomorphic to $\tilde{V}$.

\end{proof}

\subsubsection{The tangent space}


Let $k[\varepsilon] := k[X]/(X^2)$, which is called the ring of \textbf{dual numbers}.
For a functor $D : \compl{\O}\to \Set$ sending the terminal object $k$ to the terminal object $D(k) = \{\bullet\}$,
we call the set $t_D := D(k[\varepsilon])$\footnote{
    $D(k)$ is a singleton, so $t_D = $ ``$\ker(D(k[\varepsilon])\to D(k))$''.
} the \textbf{Zariski tangent space} of $D$.
If there is a fixed bijection $D(k[\varepsilon]\oplus k[\varepsilon])\simeq D(k[\varepsilon])\times D(k[\varepsilon])$,
we equip $t_D$ with the $k$-linear structure given by this bijection.

\begin{itemize}
\item Assume that $D : \compl{\O}\to \Set$ is representable by $R\in\compl{\O}$.
Then the tangent space \[
t_D\simeq \Hom_{\O}(R, k[\varepsilon])\simeq \Hom_{k}(\m_R/
(\m_R^2 + \m_\O), k[\varepsilon]) = t_R\]
is the Zariski or relative tangent space of $R$ over $\O$.
(what is the last isomorphism (if there is one...)?)
\end{itemize}

Define $\ad V := \enom_k (V)\simeq V^\vee\otimes_k V$ with the standard $G$-module structure $\ad\bar\rho = \bar\rho^\vee\otimes\bar\rho$.

\begin{lemma}
    There are canonical isomorphisms\footnote{
        In $\ext^1$, we consider \textit{continuous} extension classes.
    }
    \[D_V(k[\varepsilon])\simeq \ext^1_{k[G]}(V, V)\simeq H^1(G, \ad V).\]
\end{lemma}
\begin{proof}
    \begin{enumerate}
    \item [(1)]    
    Given an extension \[0\longrightarrow V\stackrel{i}{\longrightarrow} W\stackrel{\pi}{\longrightarrow} V\longrightarrow 0\]
    of $k[G]$-modules,
    we define the $k[G]$-linear action of $\varepsilon$ on $W$ by $\varepsilon|_W := i\circ \pi$,
    which endows $W$ with an $k[\varepsilon][G]$-module structure and an isomorphism \[
    W\otimes_{k[\varepsilon]}k = W/\varepsilon W = W/i(V) \stackrel{\pi}{\simeq} V.\]
    % the module $W$ has a direct summand
    % \[V\stackrel{\iota}{\simeq} W\otimes_{k[\varepsilon]} k\hookrightarrow W\otimes_{k[\varepsilon]}k\oplus W\otimes_{k[\varepsilon]}k\varepsilon = W\] as $k$-modules, which is actually a direct summand as $k[G]$-modules.

    Conversely, for a deformation $(W, \iota)$ of $V$ to $k[\varepsilon]$,
    we get an extension of $V$ by itself
    % https://q.uiver.app/#q=WzAsOCxbMCwwLCIwIl0sWzEsMCwiViJdLFsxLDEsIldcXG90aW1lc197XFxGW1xcdmFyZXBzaWxvbl19XFxGIl0sWzIsMSwiV1xcb3RpbWVzX3tcXEZbXFx2YXJlcHNpbG9uXX1cXEZbXFx2YXJlcHNpbG9uXSJdLFsyLDAsIlciXSxbMywwLCJWIl0sWzQsMCwiMCJdLFszLDEsIlcvXFx2YXJlcHNpbG9uIFdcXHNpbWVxIFdcXG90aW1lc197XFxGW1xcdmFyZXBzaWxvbl19XFxGIl0sWzAsMV0sWzEsMiwiXFxpb3RhIiwyXSxbMiwzLCIiLDIseyJzdHlsZSI6eyJ0YWlsIjp7Im5hbWUiOiJob29rIiwic2lkZSI6InRvcCJ9fX1dLFszLDQsIiIsMix7ImxldmVsIjoyLCJzdHlsZSI6eyJoZWFkIjp7Im5hbWUiOiJub25lIn19fV0sWzEsNF0sWzQsNV0sWzUsNl0sWzQsNywiIiwyLHsic3R5bGUiOnsiaGVhZCI6eyJuYW1lIjoiZXBpIn19fV0sWzcsNSwiXFxpb3RhIiwyXV0=
\[\begin{tikzcd}
	0 & V & W & V & 0 \\
	& {W\otimes_{k[\varepsilon]}k} & {W\otimes_{k[\varepsilon]}k[\varepsilon]} & {W/\varepsilon W\simeq W\otimes_{k[\varepsilon]}k}
	\arrow[from=1-1, to=1-2]
	\arrow[from=1-2, to=1-3]
	\arrow["\iota"', from=1-2, to=2-2]
	\arrow[from=1-3, to=1-4]
	\arrow[two heads, from=1-3, to=2-4]
	\arrow[from=1-4, to=1-5]
	\arrow[hook, from=2-2, to=2-3]
	\arrow[equals, from=2-3, to=1-3]
	\arrow["\iota"', from=2-4, to=1-4]
\end{tikzcd}\]
    as $k[G]$-modules.\footnote{
        The fact $W\simeq V\oplus V$ as $k[G]$-modules doesn't mean that the extension split.}
The first isomophism is thereby established.

\item [(2)] The second isomophism is a general fact that we have extracted as \cref{extension of group module by self = H^1 in adjoint}.\qedhere
\end{enumerate}
\end{proof}

We use the abbreviation $h^i(\cdots) := \dim_k H^i(\cdots)$.
\begin{lemma}
    If $G$ satisfies condition $\Phi_p$,
    then $D_V(k[\varepsilon])$
    is a finite dimensional $k$-vector space,
    and \[\dim_k D_V^\oblong(k[\varepsilon]) = \dim_k D_V(k[\varepsilon]) + d^2 - h^0(G, \ad V)\]
    is also finite.
\end{lemma}
\begin{proof}
    Let $G' := \ker(G\to \GL(V))$.
    Since $G$ acts continuously,
    $G'$ is an open normal subgroup of $G$.
    Consider the inflation-restriction exact sequence
    \[0\to H^1(G/G', \ad V)\to H^1(G, \ad V)\to H^1(G', \ad V)^{G/G'}.\]
    The left term is obviously finite.
    For the right term,
    $G'$ acts trivially,
    so\footnote{
        We have \[\Hom_{\gp}(G, V)\simeq \Hom_{\gp}(G, k)\otimes_k V\]
        for any group $G$ and any \textit{finite} dimensional vector space $V$ over a field $k$.
    } \[H^1(G', \ad V) = \Hom_{\gp}(G', \ad V) \simeq\Hom_{\gp}(G', \F_p)\otimes_{\F_p} \ad V\]
    is finite by condition $\Phi_p$.
    Therefore $\dim_k D_V(k[\varepsilon]) = h^1(G, \ad V) < \infty$.

    (Do the equation later.)
\end{proof}

\begin{lemma}
    Let $A$ be a complete local $\O$-algebra with residue field $k$.
    If $\{\alpha_i\}_{i\in I}\subset\m_A$
    generates the relative cotangent space $t_A^\vee = \m_A/(\m_A^2  +\m_\O)$ of $ A$ over $\O$
    as an $\O$-module,
    then the homomorphism\[\O\llbracket X_i\mid i\in I \rrbracket\to A\quad X_i\mapsto \alpha_i\]is surjective.
\end{lemma}
\begin{proof}
    % By \cref{morphism of complete algebra surjective iff cotangent surjective},
    % \[\O\llbracket X_i\mid i\in I \rrbracket\twoheadrightarrow A\iff t^\vee_{\O\llbracket X_i\mid i\in I \rrbracket}\twoheadrightarrow t^\vee_A,\]
    % and \[t^\vee_{\O\llbracket X_i\mid i\in I \rrbracket} = (\m_\O, \{X_i\}_{i\in I})/(\m_\O, \{X_i^2, X_iX_j\}_{i,j\in I})\]
    % is onto $\m_A/(\m_A^2  +\m_\O)$.
    Cannot use \cref{morphism of complete algebra surjective iff cotangent surjective} because Noetherianity of $A$ is the goal!
\end{proof}

\begin{corollary}
    The ring $R_V^\oblong$ is Noetherian if $H^1(G, \ad V)$ is $k$-finite-dimensional.
\end{corollary}
\begin{proof}
    Combine the lemmata above.
\end{proof}
This completes the proof of \cref{functor of framed deformation and deformation are pro-representable under finiteness condition} (a).

\subsubsection{Quotient by group action and the representability of \texorpdfstring{$D_V$}{DV}}

Result is $\spf R_V  = \spf R_V^\oblong\bigg/\widehat{\mathrm{PGL}_d}$.

\subsubsection{Presentation of the universal deformation ring \texorpdfstring{$R_V$}{RV}}



\section{Taylor-Wiles Patching}
Keep the notations $\O = \O_L$ for $L/\Q_p$,
and let $k = \O/\m_\O$ and $\varpi\in\O$ be a uniformizer.


Fix a continuous absolutely irreducible modular representation $\rho : \gal_{\Q, \{p, \infty\}}\to \GL_2(k)$ with determinant $\bar\varepsilon^{-1}$.






\end{document}