\documentclass{article}
\documentclass{article}
\usepackage{amsmath, amssymb, amsthm, amsbsy, mathrsfs, stmaryrd}
\usepackage{enumitem}
\usepackage[colorlinks,
linkcolor=cyan,
anchorcolor=blue,
citecolor=blue,
]{hyperref}
\usepackage[capitalize]{cleveref}
\usepackage[margin = 1in, headheight = 12pt]{geometry}
\usepackage{bbm}
\usepackage{tikz-cd}

\newtheorem{theorem}{Theorem}

\theoremstyle{definition}
\newtheorem{definition}{Definition}
\newtheorem{exercise}{Exercise}[section]
\newtheorem{problem}{Problem}
\newtheorem{example}{Example}
\newtheorem{proposition}{Proposition}[section]
\newtheorem{lemma}{Lemma}[section]
\newtheorem{corollary}{Corollary}[section]

\theoremstyle{remark}
\newtheorem*{remark}{Remark}

\renewcommand{\Re}{\mathop{\mathrm{Re}}}
\renewcommand{\Im}{\mathop{\mathrm{Im}}}

% 新命令
% 数学对象
    \newcommand{\R}{\mathbb{R}}
    \newcommand{\C}{\mathbb{C}}
    \newcommand{\Q}{\mathbb{Q}}
    \newcommand{\Z}{\mathbb{Z}}
    \DeclareMathOperator{\GL}{GL}
    \DeclareMathOperator{\SL}{SL}
    \newcommand{\p}{\mathfrak{p}}
    \renewcommand{\P}{\mathbb{P}}
    \newcommand{\A}{\mathbb{A}}
% 集合
    \newcommand{\sminus}{\smallsetminus} %(集合)差
% 范畴
    \newcommand{\op}[1]{{#1}^{\mathrm{op}}} %反范畴
    \DeclareMathOperator{\enom}{End} %自态射
    \DeclareMathOperator{\isom}{Isom} %同构
    \DeclareMathOperator{\aut}{Aut} %自同构
    \DeclareMathOperator{\im}{im} %像
    \newcommand{\Set}{\mathbf{Set}} %集合范畴
    \newcommand{\Abel}{\mathbf{Ab}} %群范畴
    \newcommand{\Ring}{\mathbf{Ring}}
    \newcommand{\Cring}{\mathbf{CRing}}
    \newcommand{\Alg}{\mathbf{Alg}}
    \newcommand{\Mod}{\mathbf{Mod}}
    \DeclareMathOperator{\Id}{id}
%向量空间, 矩阵
    \DeclareMathOperator{\rank}{rank} %秩
    \DeclareMathOperator{\tr}{Tr} %迹
    \newcommand{\tran}[1]{{#1}^{\mathrm{T}}} %转置
    \newcommand{\ctran}[1]{{#1}^{\dagger}} %共轭转置
    \newcommand{\itran}[1]{{#1}^{-\mathrm{T}}} %逆转置
    \newcommand{\ictran}[1]{{#1}^{-\dagger}} %逆共轭转置
    \DeclareMathOperator{\codim}{codim} %余维数
    \DeclareMathOperator{\diag}{diag} %对角阵
    \newcommand{\norm}[1]{\left\| #1\right\|} %范数
    \DeclareMathOperator{\lspan}{span} %张成
    \DeclareMathOperator{\sym}{\mathfrak{Y}}
% 群
    \DeclareMathOperator{\inn}{Inn} %(群)内自同构
    \newcommand{\nsg}{\vartriangleleft} %正规子群
    \newcommand{\gsn}{\vartriangleright} %正规子群
    \DeclareMathOperator{\ord}{ord} %元素的阶
    \DeclareMathOperator{\stab}{Stab} %稳定化子
    \DeclareMathOperator{\sgn}{sgn} %符号函数
% 环, 域
    \DeclareMathOperator{\cha}{char} %特征
    \DeclareMathOperator{\spec}{Spec} %素谱
    \DeclareMathOperator{\maxspec}{MaxSpec} %极大谱
    \DeclareMathOperator{\gal}{Gal}
% 微积分
    % \newcommand*{\dif}{\mathop{}\!\mathrm{d}} %(外)微分算子
% 流形
    \DeclareMathOperator{\lie}{Lie}
%代数几何
    \DeclareMathOperator{\proj}{Proj}
%多项式
    \DeclareMathOperator{\disc}{disc} %判别式
    \DeclareMathOperator{\res}{res} %结式

% 结构简写
    \newcommand{\pdfrac}[2]{\dfrac{\partial #1}{\partial #2}} %偏微分式
    \newcommand{\isomto}{\stackrel{\sim}{\rightarrow}} %有向同构
    \newcommand{\gene}[1]{\left\langle #1 \right\rangle} %生成对象
% 文字缩写
    \newcommand{\opin}{\;\mathrm{open\;in}\;}
    \newcommand{\st}{\;\mathrm{s.t.}\;}
    \newcommand{\ie}{\;\mathrm{i.e.,}\;}

% 重定义命令
\renewcommand{\hom}{\mathop{Hom}}
\renewcommand{\vec}{\boldsymbol}
\renewcommand{\and}{\;\text{and}\;}

% 编号
\newcommand{\cnum}[1]{$#1^\circ$} %右上角带圆圈的编号
\newcommand{\rmnum}[1]{\romannumeral #1}


\newcommand{\myit}{$\diamond$}

\title{}
\author{}
\date{}

\begin{document}
\maketitle

\end{document}

\usepackage[ruled, vlined]{algorithm2e}


\newcommand{\F}{\mathbb{F}}
\renewcommand{\O}{\mathcal{O}}

\theoremstyle{definition}
\newtheorem{question}{Question}[section]

\title{DM 1}
\author{Lei Bichang}


\begin{document}
\maketitle

\section{Factoring and Modular Square Roots}

\begin{question}
    \textit{Decription of an algorithm}.
    First, use the extended Euclidean division to find $u, v\in\Z$
    s.t. $uM + vN = 1$.
    Then $x := vNa + uMb$ does the job:
    \[x\equiv vNa\equiv a\bmod M,
    \quad
    x\equiv uMb\equiv b\bmod N.\]
    Denote by ExtGCD a realization of the extended Euclidean division algorithm.\\
    \begin{algorithm}[H]
        \caption{CongruenceModTwoIntegers}\label{CongruenceModTwoIntegers}
        \KwIn{$(M, N, a, b) \in \Z^4$, $\gcd(M, N) = 1$}
        \KwOut{$x\in\Z$, $x\equiv a\bmod M$, $x\equiv b\bmod N$}
        $(u, v)\gets \mathrm{ExtGCD}(M, N)$, s.t. $uM + vN = 1$\;
        \Return{$vNa + uMb$}
    \end{algorithm}

    \noindent\textit{Analysis of the algorithm}.
    This algorithm involves only a fixed number of steps more than ExtGCD.
    Hence its complexity is $O(\log(M)\log(N))$.
\end{question}

\begin{question}
    Assume that we have an oracle $\mathcal{O}_F$ for the problem Factoring.
    Let $N = pq$ with $p, q$ two distinct odd primes.
    Since $(\Z/N\Z)^\times\simeq\F_p^\times\times\F_q^\times$ via $x\mapsto (x\bmod p, x\bmod q)$,
    an integer $x$ is a square modulo $N$ if and only if it is a square modulo $p$ and $q$.
    In addition, if we have found integers $a$ and $b$ s.t. $x = a^2\bmod p$ and $x = b^2\bmod q$,
    then an integer $y$ satisfying \[y\equiv a\bmod p,\quad y\equiv b\bmod q\]
    would satisfy $x\equiv y^2\bmod p$ and mod $q$,
    and thus $y^2\equiv x\bmod N$.

    For a prime $p$ and an integer $x$, denote by Sq$(p, x)$
    an algorithm that provides an integer that is a square root of $x$ modulo $p$.
    By factoring the polynomial $X^2 - x\in \F_p[X]$,
    Sq$(p, x)$ can be realized as a probabilistic polynomial time algorithm.
    This provides us with the following probabilistic polynomial time algorithm with access to the oracle $\O_F$.\\
    \begin{algorithm}[H]
        \caption{SquareRootMod-$\mathcal{O}_F$}\label{SquareRootMod to Factoring}
        \KwIn{$N$ : RSA integer, $x\in\Z$}
        \KwOut{$y\in\Z$ s.t. $x = y^2\bmod N$ or False if $x$ is not a square modulo $N$}
        $(p, q)\gets \mathcal{O}_F(N)$\;
        \eIf{$x^{\frac{p-1}{2}} = 1$ and $x^{\frac{q-1}{2} = 1}$}{
            $a\gets \mathrm{Sq}(p, x)$, $b\gets\mathrm{Sq}(q, x)$\;
            $y\gets \mathrm{CongruenceModTwoIntegers}(p, q, a, b)$\;
            \Return{$y$}
        }{\Return{False}}
    \end{algorithm}
\end{question}


\begin{question}
    Let $N = pq$ be the prime factorization of $N$.

    Assume first that $2\nmid N$.
    Let $y\in(\Z/N\Z)^\times$ be a square root of $x^2$,
    then $y = \varepsilon x$ with \[\varepsilon\in\mu_2(\Z/N\Z) = \{a\in\Z/N\Z\mid a^2 = 1\} \stackrel{f}{\simeq}\{\pm 1\}\times\{\pm 1\}\]
    where $f$ is the restriction of the isomorphism $(\Z/N\Z)^\times\to \F_p^\times\times\F_q^\times$ obtained from Chinese remainder theorem.
    So we have \[x - y\equiv x - \pm x = \begin{cases}
        0\bmod p, & \varepsilon\equiv 1\bmod p,\\
        x\bmod p, & \varepsilon\equiv -1\bmod p,
    \end{cases}\]
    i.e. $p\mid x - y\iff \varepsilon\equiv 1\bmod p$.
    A similar result holds for $q$.
    Hence
    \[\gcd(x - y, N) = \begin{cases}
        N, & f(\varepsilon) = \{1, 1\},\\
        p, & f(\varepsilon) = \{1, -1\}, \\
        q, & f(\varepsilon) = \{-1, 1\},\\
        1, & f(\varepsilon) = \{-1, -1\}.
    \end{cases}\]
    If $y$ is uniformly random,
    $\gcd(x - y, N)$ is also uniformly random in $\{1, p, q, N\}$.

    Now assume that $2\mid N$, say $q = 2$ and $N = 2p$.
    In this case $\mu_2(\Z/N\Z)\simeq \{\pm 1\}$,
    so if $y$ is a uniformly random square root of $x$ in $\Z/N\Z$,
    then $\gcd(x - y, N)$ is uniformly random in \[\{\gcd(0, N),\ \gcd(2x, N)\} = \{N, 2\}.\]

\end{question}

\begin{question}
    Let $\O_S$ be an oracle for SquareRootMod.

    Pick $x\in(\Z/N\Z)^\times$ uniformly at random.
    First, consider $y := \O_S(x^2, N)$.
    If $y \not\equiv \pm x\bmod N$,
    $\gcd(x - y, N)$ is a non-trivial factor of $N$ by the discussion above.
    If $y\equiv \pm x\bmod N$,
    we try another random $x$ and repeat this procedure.\\
    \begin{algorithm}[H]
    \caption{Factoring-$\O_S$}
    \KwIn{$N$ : RSA integer}
    \KwOut{$p, q$ : primes s.t.\ $N = pq$}
    \Repeat{}{
        $x\gets$ uniformly random in $(\Z/N\Z)^\times$\;
        $y\gets \O_S(x^2, N)$\;
        \If{$y\not\equiv \pm x$}{
            $p\gets\gcd(x - y, N)$\;
            \Return{$(p, N/p)$}
        }
    }
    \end{algorithm}

    \textit{Analysis of the algorithm}.
    Choosing a uniformly random $x\in(\Z/N\Z)^\times$ is equivalent to choosing a uniformly random square $z\in ((\Z/N\Z)^\times)^2$ then choosing a uniformly random square root $x$ of $z$.
    Therefore,
    the probability that one loop successfully factors $N$ is the probability of the following event: for a uniformly random square $z$,
    a uniformly random square root of $z$ does not equal $\pm\O_S(z, N)$.

    % If $\O_S$ is deterministic,
    % i.e. $\O_S(z, N)$ always returns a fixed square root of $z$, then this probability is clearly $1/2$.
    % Thus the expected number of loops required to factor $N$ is $2$.
    % Since every loop costs polynomial time,    we get a probabilistic polynomial time algorithm to factor $N$.

    Let $\Omega = ((\Z/N\Z)^\times, \mathcal{B}, \mathbb{P})$ be the probability space where $\mathcal{B}$ is the powerset of $(\Z/N\Z)^\times$ and $\mathbb{P}$ is the uniform distribution.
    For each $a\in (\Z/N\Z)^\times$,
    fix a square root $\sqrt{a}\in (\Z/N\Z)^\times$.
    Let $X$ (resp.\ $Z$, $E$) be the inclusion maps from $(\Z/N\Z)^\times$ (resp.\ $((\Z/N\Z)^\times)^2$, $\mu_2(\Z/N\Z)$) to $(\Z/N\Z)^\times$.
    Each of them is a uniformly random variable on its domain\footnote{
        As a probability subspace of $\Omega$.
    } with value in $\Omega$,
    and the chance for one loop to successfully factor $N$ is \[\mathbb{P}(X\not\neq \pm\O(X^2))
    = \mathbb{P}\left(E\sqrt{Z}\neq \pm\O\left(\left(E\sqrt{Z}\right)^2\right)\right)
    = \mathbb{P}\left(E\sqrt{Z}\neq \pm\O\left(Z\right)\right)
    = \mathbb{P}\left( E\ne\frac{\O(Z)}{\sqrt{Z}} \right).\]
    % = \mathbb{P}\left( E\neq \pm f(Z) \right).\]
    Since the random variables $E$ and $f(Z) := \mathcal{O}(Z)/\sqrt{Z}$ are independent,
    \begin{align*}
        % \mathbb{P}\left( \left.E\ne\pm\frac{\O(Z)}{\sqrt{Z}}\right|E = \varepsilon \right)
        \mathbb{P}\left(E\ne\pm f(Z)\right)
        &= 1 - \sum_{\varepsilon\in\mu_2(\Z/N\Z)}
            \mathbb{P}(E = \varepsilon)
            \left( \mathbb{P}(f(Z) = \varepsilon) + \mathbb{P}(f(Z) = - \varepsilon) \right)\\
        &= 1 - \frac{1}{4}\sum_{\varepsilon\in\mu_2(\Z/N\Z)}
            \left( \mathbb{P}(f(Z) = \varepsilon) + \mathbb{P}(f(Z) = - \varepsilon) \right)\\
        &= 1 - \frac{1}{4}\cdot 2 = \frac{1}{2}.
    \end{align*}
    Therefore, each loop has an $\frac{1}{2}$ chance to factor $N$, ragardless of how $\mathcal{O}_S$ works.
    The expected number of loops required to factor $N$ is $2$.
    Since every loop costs polynomial time,    we get a probabilistic polynomial time algorithm to factor $N$.
\end{question}

\section{Evaluating an isogeny of degree \texorpdfstring{$\ell^n$}{ln} from its kernel}
\setcounter{question}{1}

\begin{question}
Since $\ell$ is a prime, it suffices to show that $g_n^i(P)\ne 0$ and $\ell\cdot g_n^i(P) = 0$.
Because $\varphi_{\ell^{n-i}P}$ is a group homomorphism from $E$ with kernel generated by $\ell^{n-i}P$,
we have
\[\ell g_n^i(P) = \ell \varphi_{\ell^{n-i}P}(\ell^{n-1-i}P) = \varphi_{\ell^{n-i}P}(\ell^{n-i}P) = 0.\]
If $g_n^i(P) = 0$,
then $\ell^{n-1-i}P = a\ell^{n-i}P$ for some $a\in\Z$.
Because $\ell^{n-i-1}P$ has order $i + 1$,
it cannot be a multiple of $\ell^{n-i}P$ which has order $i$. So $g_n^i(P) = \varphi_{l^{n-i}P}(\ell^{n-1-i}P)\ne 0$. 
% Since $P$ has order $\ell^n$,
% we have $\ell^n\mid \ell^{n-1-i}(1 - al)$,
% which is impossible for $0\le i\le n-1$.
\end{question}


\begin{question}
% Note that:
% \begin{lemma}
%     Let $P\in E$ be of order $\ell^n$, and $1\le i\le j\le n$,
%     then \[E/\!\gene{\ell^{n - j}P}\simeq \left( E/\!\gene{\ell^{n-i} P} \right)\bigg/\]
% \end{lemma}
% \begin{proof}
%     For $i\le j$, \[\ker\left( E/\!\gene{\ell^{n - j} P}\to E/\!\gene{\ell^{n - i} P} \right) = \gene{\varphi_{\ell^{n - j} P}(\ell^{n - i} P)}\]
% \end{proof}

Let $P$ be of order $\ell^n$.
Note that for $0\le i\le n - 1$,
we have $g_n^i(P)\in E/\gene{\ell^{n-i}P}$,
and \[\ker\left( E/\gene{\ell^{n-i}P}\to E/\gene{\ell^{n-i-1}P} \right) = \gene{\varphi_{\ell^{n-i}P}(\ell^{n-i-1}P)} = \gene{g_n^i(P)}.\]
Hence \[\left( E/\gene{\ell^{n-i}P} \right)/\gene{g_n^i(P)}\simeq E/\gene{\ell^{n-i-1}P}.\]
Thereofre, we can evaluate $\varphi_P(Q)$ along the path
% % https://q.uiver.app/#q=WzAsMTIsWzEsMCwiIGdfbl4wKFApXFxpbiBFIl0sWzEsMSwiZ19uXjEoUClcXGluIEUvXFxnZW5le1xcZWxsXntuLTF9UH0iXSxbMiwxLCJFL1xcZ2VuZXtnX25eMChQKX0iXSxbMSwyLCJnX25eMihQKVxcaW4gRS9cXGdlbmV7XFxlbGxee24tMn1QfSJdLFsxLDMsIlxcdmRvdHMiXSxbMSw1LCJnX25ee24tMX0oUClcXGluIEUvXFxnZW5le1xcZWxsIFB9Il0sWzIsMiwiXFxsZWZ0KEUvXFxnZW5le1xcZWxsXntuLTF9UH1cXHJpZ2h0KS9cXGdlbmV7Z19uXjEoUCl9Il0sWzIsMywiXFx2ZG90cyJdLFsyLDUsIlxcbGVmdChFL1xcZ2VuZXtcXGVsbF57bi0yfVB9XFxyaWdodCkgLyBcXGdlbmV7Z19uXntuLTJ9KFApfSJdLFswLDVdLFsxLDQsImdfbl57bi0yfShQKVxcaW4gRS9cXGdlbmV7XFxlbGxeMiBQfSJdLFsyLDQsIlxcbGVmdChFL1xcZ2VuZXtcXGVsbF57bi0yfVB9XFxyaWdodCkgLyBcXGdlbmV7Z19uXntuLTJ9KFApfSJdLFswLDIsIlxcdGV4dHtWXFwnZWx1fSIsMV0sWzIsMSwiXFxzaW1lcSIsMV0sWzEsNiwiXFx0ZXh0e1ZcXCdlbHV9IiwxXSxbNiwzLCJcXHNpbWVxIiwxXSxbOCw1LCJcXHNpbWVxIiwxXSxbMyw3LCJcXHRleHR7VlxcJ2VsdX0iLDFdLFs0LDExLCJcXHRleHR7VlxcJ2VsdX0iLDFdLFsxMCw4LCJcXHRleHR7VlxcJ2VsdX0iLDFdLFsxMSwxMCwiXFxzaW1lcSIsMV1d
% \[\begin{tikzcd}
% 	& { g_n^0(P)\in E} \\
% 	& {g_n^1(P)\in E/\gene{\ell^{n-1}P}} & {E/\gene{g_n^0(P)}} \\
% 	& {g_n^2(P)\in E/\gene{\ell^{n-2}P}} & {\left(E/\gene{\ell^{n-1}P}\right)/\gene{g_n^1(P)}} \\
% 	& \vdots & \vdots \\
% 	& {g_n^{n-2}(P)\in E/\gene{\ell^2 P}} & {\left(E/\gene{\ell^{n-2}P}\right) / \gene{g_n^{n-2}(P)}} \\
% 	{} & {g_n^{n-1}(P)\in E/\gene{\ell P}} & {\left(E/\gene{\ell^{n-2}P}\right) / \gene{g_n^{n-2}(P)}}
% 	\arrow["{\text{V\'elu}}"{description}, from=1-2, to=2-3]
% 	\arrow["{\text{V\'elu}}"{description}, from=2-2, to=3-3]
% 	\arrow["\simeq"{description}, from=2-3, to=2-2]
% 	\arrow["{\text{V\'elu}}"{description}, from=3-2, to=4-3]
% 	\arrow["\simeq"{description}, from=3-3, to=3-2]
% 	\arrow["{\text{V\'elu}}"{description}, from=4-2, to=5-3]
% 	\arrow["{\text{V\'elu}}"{description}, from=5-2, to=6-3]
% 	\arrow["\simeq"{description}, from=5-3, to=5-2]
% 	\arrow["\simeq"{description}, from=6-3, to=6-2]
% \end{tikzcd}\]
% https://q.uiver.app/#q=WzAsMTEsWzEsMCwiRSJdLFsyLDEsIkUvXFxnZW5le1xcZWxsXntuLTF9UH0iXSxbMSwxLCJFL1xcZ2VuZXtnX25eMChQKX0iXSxbMSwyLCJFL1xcZ2VuZXtcXGVsbF57bi0yfVB9Il0sWzEsNSwiRS9cXGdlbmV7XFxlbGwgUH0iXSxbMiwyLCJcXGxlZnQoRS9cXGdlbmV7XFxlbGxee24tMX1QfVxccmlnaHQpL1xcZ2VuZXtnX25eMShQKX0iXSxbMiw1LCJcXGxlZnQoRS9cXGdlbmV7XFxlbGxee24tMn1QfVxccmlnaHQpIC8gXFxnZW5le2dfbl57bi0yfShQKX0iXSxbMCw1XSxbMiw0LCJFL1xcZ2VuZXtcXGVsbF4yIFB9Il0sWzEsNCwiXFxsZWZ0KEUvXFxnZW5le1xcZWxsXntuLTJ9UH1cXHJpZ2h0KSAvIFxcZ2VuZXtnX25ee24tMn0oUCl9Il0sWzEsNiwiXFxsZWZ0KEUvXFxlbGwgUFxccmlnaHQpL2dfbl57bi0xfShQKTo9RS9cXGdlbmV7UH0iXSxbMCwyLCJcXHRleHR7VlxcJ2VsdX0iLDFdLFsyLDEsIlxcc2ltZXEiLDFdLFsxLDUsIlxcdGV4dHtWXFwnZWx1fSIsMV0sWzUsMywiXFxzaW1lcSIsMV0sWzYsNCwiXFxzaW1lcSIsMV0sWzgsNiwiXFx0ZXh0e1ZcXCdlbHV9IiwxXSxbOSw4LCJcXHNpbWVxIiwxXSxbNCwxMCwiXFx0ZXh0e1ZcXCdlbHV9IiwxXSxbMyw5LCJcXHZkb3RzIiwxXV0=
\[\begin{tikzcd}
	& E \\
	& {E/\gene{g_n^0(P)}} & {E/\gene{\ell^{n-1}P}} \\
	& {E/\gene{\ell^{n-2}P}} & {\left(E/\gene{\ell^{n-1}P}\right)/\gene{g_n^1(P)}} \\
	\\
	& {\left(E/\gene{\ell^{n-2}P}\right) / \gene{g_n^{n-2}(P)}} & {E/\gene{\ell^2 P}} \\
	{} & {E/\gene{\ell P}} & {\left(E/\gene{\ell^{n-2}P}\right) / \gene{g_n^{n-2}(P)}} \\
	& {\left(E/\ell P\right)/g_n^{n-1}(P):=E/\gene{P}}
	\arrow["{\text{V\'elu}}"{description}, from=1-2, to=2-2]
	\arrow["\simeq"{description}, from=2-2, to=2-3]
	\arrow["{\text{V\'elu}}"{description}, from=2-3, to=3-3]
	\arrow["\text{\rotatebox{90}{$\cdots$}}"{description}, from=3-2, to=5-2]
	\arrow["\simeq"{description}, from=3-3, to=3-2]
	\arrow["\simeq"{description}, from=5-2, to=5-3]
	\arrow["{\text{V\'elu}}"{description}, from=5-3, to=6-3]
	\arrow["{\text{V\'elu}}"{description}, from=6-2, to=7-2]
	\arrow["\simeq"{description}, from=6-3, to=6-2]
\end{tikzcd}\]
where horizontal arrows are isomrophisms and vertical arrows are $\ell$-isogenies.
The elliptic curves $E$ and $E/\gene{\ell^j(P)}$'s are given by $f_n(P)$, and every other elliptic curve will be computed as the target of an $\ell$-isogeny.
Since every curve is stored as a Weierstrass equation, the isomorphisms can be computed in $O(1)$-times $\bar{\F_q}$-operations.
% In addition, we also need to compute $\ell^iP$ for $1\le i\le  n - 1$ inductively.

Consequently, this algorithm requires $\ell$-isogeny $n$-times and no multiplication-by-$\ell$. 
% $(n-1)$-times.

\end{question}


\begin{question}

We have
\[g_m^i(\ell^{n-m}P) = \varphi_{\ell^{m-i}\ell^{n-m}P}(\ell^{m-i - 1}\ell^{n-m}P) = \varphi_{\ell^{n-i}P}(\ell^{n - i -1}P) = g_n^i(P).\]
By an argument similar to that in Question 2.3,
\[\left( E/\gene{\ell^{n-m}P} \right)\big/\gene{\varphi_{\ell^{n-m}P}(\ell^{n-m-i}P)}\simeq E/\gene{\ell^{n-m-i}P},\]
so \begin{align*}
    g_{n-m}^i(\varphi_{\ell^{n-m}P}(P))&= 
\varphi_{\ell^{n-m-i}\varphi_{\ell^{n-m}P}(P)}\left( \ell^{n-m-i-1}\varphi_{\ell^{n-m}P}(P) \right) \\ &=
\varphi_{\varphi_{\ell^{n-m}P}(\ell^{n-m-i}P)} \left( \varphi_{\ell^{n-m}P}(\ell^{n-m-i-1}P) \right) \\ &=
\varphi_{\ell^{n-m-i}P}(\ell^{n-m-i-1}P) = g_n^{m + i}(P).
\end{align*}
Hence
\[f_m(\ell^{n-m}P) = \left(
    g_m^0(\ell^{n-m}P),\dots,
    g_m^{m-1}(\ell^{n-m}P) \right) = (g_n^0(P), \dots, g_n^{m-1}(P)),\]
\[f_{n-m}(\varphi_{\ell^{n-m}P}(P)) = \left( 
    g_{n-m}^0(\varphi_{\ell^{n-m}P}(P)),\dots,
    g_{n-m}^{n-m-1}(\varphi_{\ell^{n-m}P}(P))
 \right)  = (g_n^m(P), \dots, g_n^{n-1}(P))\]
as desired.
\end{question}

\begin{question}
The list $f_0(P)$ is empty for every $P$ of order $\ell^0 = 1$ (namely $P = O$).
The complexity is $O(1)$.

The list $f_1(P)$ contains only one element $g_1^0(P) = \varphi_{\ell P}(P) = \varphi_O(P) = P$ is also trivial for $P$ of order $\ell$.
The complexity is $O(1)$.

% {\color{red}(Really?)}
\end{question}


\begin{question}
% To begin with, we describe an algorithm that compute $f_n(P)$ for $P\in E$ of order $\ell^n$.
For $P\in E[\ell]\setminus\{O\}$ and $Q\in E$,
let $\mathrm{Velu}(-, -)$ be an algorithm such that $\mathrm{Velu}(P, Q) = \varphi_P(Q)$.

For $P\in E[\ell^n]\setminus E[\ell^{n-1}]$
and $Q\in E$,
let $\mathrm{Compute}(-, -,-,-)$\footnote{
    This realization requires redundant arguments to make it more clear for me.
} be the algorithm described in Question 2.3.,
so that $\mathrm{Compute}(n, P, f_n(P), Q) = \varphi_{P}(Q)$.
In particular, if $n = 1$, then $\mathrm{Compute}(1, P, f_1(P), Q) = \mathrm{Velu}(P, Q)$.
 
To compute $\varphi_P(Q)$, we use the following algorithm $\mathrm{Aux}(-, -)$ to find $f_n(P)$ so that we can use Compute.

\begin{algorithm}[H]
    \caption{Aux}\label{compute the list f}
    \KwIn{$n\in\Z_{\ge 0}$, $P\in E(\F_q)$ of order $\ell^n$}
    \KwOut{$f_n(P)$}
    \eIf{$n = 0$}{
        \Return{$\varnothing$}
    }{\eIf{$n = 1$}{
        \Return{$(P)$}
    }{
        $m \gets \lfloor n/2\rfloor$\;
        $R\gets \ell^{n-m}P$\;
        $f \gets \mathrm{Aux}(m, R)$\;
        $\varphi_{R}(P) \gets \mathrm{Compute}(m, R, f ,P)$\;
        \Return{$f\sqcup \mathrm{Aux}(n - m, \varphi_{R}(P))$}}}
\end{algorithm}

% (make it persistent i.e. memorize all the Aux(-, -) appeared?)

\textit{Analysis of the algorithm.}
\newcommand{\floor}[1]{\left\lfloor #1 \right\rfloor}
Let $T_a(n)$ (resp.\ $T_c(n)$) be the numbers of multiplications by $\ell$ and $\ell$-isogeny evaluations required for $\mathrm{Aux}(n, -)$ (resp.\ $\mathrm{Compute}(n, -, f_n(-), -)$ when the list $f_n(-)$ is given).
By Question 2.3 and Question 2.5,
$T_a(0) = T_a(1) = 0$,
$T_c(n) = n$.
By definition of the algorithm Aux, \begin{align*}
    T_a(n)
&= T_a(\floor{n/2}) + T_a(n - \floor{n/2}) + T_c(\floor{n/2}) + (n - \floor{n/2})\\
&= T_a(\floor{n/2}) + T_a(n - \floor{n/2}) + n
\end{align*}
% To analyse $T_a$, we first note the following easy truth.
% \begin{lemma}
% If $\{a_n\}_{n\ge 0}$ is a sequence such that
% \[a_n = 2a_{n-1} + b\cdot 2^n, \quad\forall n\ge 1,\]
% then \[a_n = 2^n(bn + a_0),\quad\forall n\ge 0.\]
% \end{lemma}
% \begin{proof}
%     This can be verified directly.
% \end{proof}
% For $n = 2^rb$,
% we have \[T_a(2^rb) = 2T_a(2^{r-1}b) + 2^rb,\]
% so by the formula above,
% \[T_a(2^rb) = 2^r(br + T_a(b)).\]
% Now write \[n = 2^{r_1}(1 + s_1),\ b_1 = 2^{r_2}(1 + s_2), \dots, b_{i - 1} = 2^{r_{i - 1}}(1 + b_{i - 1}),\ b_i = 2^i.\]
% Then \begin{align*}
%     T_a(n) &= 2^{r_1}(r_1(1 + s_1) + T_a(1 + b_1))\\ &
%     =
% \end{align*}

\begin{lemma}
    The function $T_a$ on $\Z_{\ge 0}$ is non-decreasing. 
\end{lemma}
\begin{proof}
    We show that $T_a(n + 1)\ge T_a(n)$ by induction.
    Clearly, that $T_a(2) = 2 > T(1) = T(0)$.
    Assume that $T_a(m + 1)\ge T_a(m)$ for all $m < n$, and set $r := \floor{n/2}$.
    If $n$ is even, i.e. $n = 2r$,
    \begin{align*}
T_a(n + 1) - T_a(n)
= T_a(r) + T_a(r + 1) - 2T_a(r) + 1
= T_a(r+1) - T_a(r) + 1 > 0
    \end{align*}
    by induction hypothesis.
    If $n$ is odd, i.e. $n = 2r + 1$,
    \begin{align*}
T_a(n + 1) - T_a(n)
= 2T_a(r + 1) - (T_a(r) + T_a(r + 1)) + 1
= T_a(r+1) - T_a(r) + 1 > 0
    \end{align*}
    by induction hypothesis.
    Hence $T$ is non-decreasing.
\end{proof}
Now for $n = 2^r$ with $r\in\Z_{\ge 0}$,
we have
\[T_a(2^r) = 2T_a(2^{r-1}) + 2^r,\]
and it is plain to show that
\[T_a(2^r) = 2^r(r + T(0)) = r\cdot 2^r.\]
For general $n$,
let $r := \floor{\log_2n}$,
so that $2^r\le n < 2^{r+1}$,
then \[r\cdot 2^r\le T(n)\le (r + 1)2^{r+1}\]
by the above lemma,
which implies that $T(n) = O(n\log n)$.

Therefore, computing $\varphi_P(Q)$ takes
\[T_a(n) + T_c(n) = T_a(n) + n = O(n\log n)\]
multiplications by $\ell$ and $\ell$-isogeny evaluations.

    
\end{question}

\begin{question}
    Please see the other file.
    Unlike what we described in Question 2.3,
    we do not need to compute the isomorphisms
    \[\left( E/\gene{\ell^{n-i}P} \right)/\gene{g_n^i(P)}\simeq E/\gene{\ell^{n-i-1}P}\]
    in the actual function Compute,
    probably because the list $f_n(P)$ is computed recursively when running the functions,
    and every isogeny is computed via V\'elu's formula. But I chose to keep the isomorphism-computation part for more generality.
\end{question}

\begin{question}
    Please see the other file.
\end{question}

\end{document}
