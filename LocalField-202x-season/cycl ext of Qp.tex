\documentclass{article}
\usepackage{amsmath, amssymb, amsthm, amsbsy, mathrsfs}
\usepackage{enumitem}
\usepackage[capitalize]{cleveref}
\usepackage[margin = 1in, headheight = 12pt]{geometry}
\usepackage{bbm}
\usepackage{tikz-cd}

\newtheorem{theorem}{Theorem}

\theoremstyle{definition}
\newtheorem{definition}{Definition}
\newtheorem{exercise}{Exercise}[section]
\newtheorem{problem}{Problem}
\newtheorem{example}{Example}
\newtheorem{proposition}{Proposition}[section]
\newtheorem{lemma}{Lemma}[section]
\newtheorem{corollary}{Corollary}[section]

\theoremstyle{remark}
\newtheorem*{remark}{Remark}

\DeclareMathOperator{\gal}{Gal}

\renewcommand{\Re}{\mathop{\mathrm{Re}}}
\renewcommand{\Im}{\mathop{\mathrm{Im}}}
\renewcommand{\bar}{\overline}
\renewcommand{\tilde}{\widetilde}

% 新命令
% 数学对象
    % 基础空间
    \newcommand{\R}{\mathbb{R}}
    \newcommand{\C}{\mathbb{C}}
    \newcommand{\Q}{\mathbb{Q}}
    \newcommand{\Z}{\mathbb{Z}}
    % 常量
    % \newcommand{\e}{\mathrm{e}} %自然底数
    % 群
    \DeclareMathOperator{\GL}{GL}
    \DeclareMathOperator{\SL}{SL}
% 算符&映射&函子
    % 集合
    \newcommand{\sminus}{\smallsetminus} %(集合)差
    % 范畴
    \newcommand{\op}[1]{{#1}^{\mathrm{op}}} %反范畴
    \DeclareMathOperator{\enom}{End} %自态射
    \DeclareMathOperator{\isom}{Isom} %同构
    \DeclareMathOperator{\aut}{Aut} %自同构
    \DeclareMathOperator{\im}{im} %像
    %向量空间, 矩阵
    \DeclareMathOperator{\rank}{rank} %秩
    \DeclareMathOperator{\tr}{tr} %迹
    \newcommand{\tran}[1]{{#1}^{\mathrm{T}}} %转置
    \newcommand{\ctran}[1]{{#1}^{\dagger}} %共轭转置
    \newcommand{\itran}[1]{{#1}^{-\mathrm{T}}} %逆转置
    \newcommand{\ictran}[1]{{#1}^{-\dagger}} %逆共轭转置
    \DeclareMathOperator{\codim}{codim} %余维数
    \DeclareMathOperator{\diag}{diag} %对角阵
    \newcommand{\norm}[1]{\left\| #1\right\|} %范数
    \DeclareMathOperator{\spec}{Spec} %谱
    \DeclareMathOperator{\lspan}{span} %张成
    \DeclareMathOperator{\sym}{\mathfrak{Y}}
    % 群
    \DeclareMathOperator{\inn}{Inn} %(群)内自同构
    \newcommand{\nsg}{\vartriangleleft} %正规子群
    \newcommand{\gsn}{\vartriangleright} %正规子群
    \DeclareMathOperator{\ord}{ord} %元素的阶
    \DeclareMathOperator{\stab}{Stab} %稳定化子
    \DeclareMathOperator{\sgn}{sgn} %符号函数
    % 环, 域
    \DeclareMathOperator{\cha}{char} %特征
    % \DeclareMathOperator{\spec}{Spec} %素谱
    \DeclareMathOperator{\spm}{Spm} %极大谱
    \DeclareMathOperator{\Frac}{Frac}
    % 微积分
    % \newcommand*{\dif}{\mathop{}\!\mathrm{d}} %(外)微分算子
    % 流形
    \DeclareMathOperator{\lie}{Lie}
% 结构简写
    \newcommand{\pdfrac}[2]{\dfrac{\partial #1}{\partial #2}} %偏微分式
    \newcommand{\isomto}{\stackrel{\sim}{\rightarrow}} %有向同构
    \newcommand{\gene}[1]{\left\langle #1 \right\rangle} %生成对象
% 文字缩写
    \newcommand{\opin}{\;\mathrm{open\;in}\;}
    \newcommand{\st}{\;\mathrm{s.t.}\;}
    \newcommand{\ie}{\;\mathrm{i.e.,}\;}

% 重定义命令
\renewcommand{\hom}{\mathop{Hom}}
\renewcommand{\vec}{\boldsymbol}
\renewcommand{\and}{\;\text{and}\;}

% 编号
\newcommand{\cnum}[1]{$#1^\circ$} %右上角带圆圈的编号
\newcommand{\rmnum}[1]{\romannumeral #1}

\title{Notes on Local Fields}
\author{}
\date{}

\begin{document}
\maketitle

\section{Review: Galois theory}
Let $L/K$ be an algebraic extension. It is called: \begin{enumerate}
    \item [$\diamond$]\textbf{normal}, if every polynomial $f\in K[T]$ with a root in $L$ splits in $L$, $\iff$ $L$ is the splitting field of a bunch of polynomials over $K$;
    \item [$\diamond$]\textbf{separable}, if for every element in $L$, its minimal polynomial over $K$ has no multiple roots in its splitting field;
    \item [$\diamond$]\textbf{Galois}, if it is normal and separable, i.e., $L$ is the splitting field of a bunch of \textit{inseperable} polynomial over $K$. We put $\gal(L/K) := \aut_K(L)$.
\end{enumerate}
\begin{remark} {}
\begin{enumerate}
    \item For a finite \textit{normal} extension $L/K$, $|\aut_K(L)| \le [L:K]$, where the equality holds $\iff L/K$ is separable, i.e. Galois. This is because a $K$-automorphism of $L = K[T]/(f)$ just maps a root of $f$ to another.
    \item Normality is NOT transitive. As an example, take $\Q\subset\Q(\sqrt{2})\subset\Q(\sqrt[4]{2})$.
\end{enumerate}
\end{remark}
Now let $L/K$ be a Galois extension. Equip $\gal(L/K)$ with the following \textbf{Krull topology}: $\forall\sigma\in\gal(L/K)$, a basis of nbhd is given by\[\sigma\gal(L/F),\quad F/K < \infty\text{ \& Galois}.\]
This topology is the discrete topology when $L/K$ finite, and is profinite when $L/K$ infinite, whence \[\gal(L/K) \simeq \lim_{\stackrel{\longleftarrow}{F/K < \infty\text{ \& Galois}}}\gal(F/K). \]

The Galois theory says that the intermediate fields of $L/K$ corresponds to the closed subgroups of $\gal(L/K)$ bijectively and $\gal(L/K)$-equivariantly.
\begin{enumerate}
    \item [$\rightarrow$:] For an intermediate field $F$, it gives $\gal(L/F)\subset \gal(L/K)$. Note that $L/F$ is Glaois, but $F/K$ is NOT always Galois.
    The Galois group acts on $\{\text{intermediate field of } L/K\}$ by $(\sigma, F) \mapsto \sigma F = \sigma(F)$.
    \item [$\leftarrow$:] For a subgroup $H < G$, it fixes a subfield $L^H \subset L$. The Galois group act on $\{H : H < \gal(L/K)\}$ by conjugation, i.e., $(\sigma, H) \mapsto \sigma H\sigma^{-1}$.
\end{enumerate}
In particular,\begin{enumerate}
    \item [$\diamond$] Galois extensions correspond to normal closed subgroups,
    \item [$\diamond$] Finite extensions correspond to open subgroups.
\end{enumerate}

\section{DVR and Dedeking domains}

\subsection{Simple Extensions}
Let $A$ be a local ring with ($\mathfrak{m}$, $k$), $f\in A[X]$ a monic polynomial of deg $n$.
We consider the extension $A \to B_f := A[X]/f$.

Let $\bar{f}$ be the image of $f$ in $k[X] \simeq A[X]/\mathfrak{m}$ with decomposition \[\bar{f} = \prod_{i}\bar{g_i}^{e_i},\ g_i\in A[X],\ \bar{g_i}\in k[X]\text{ irreducible.}\]
and \[\bar{B_f} := B_f/\mathfrak{m}B_f \simeq A[X]/(\mathfrak{m}, f) \simeq k[X]/(\bar{f}).\]
\begin{lemma}
    $\mathfrak{m}_i := (\mathfrak{m},\ g_i\bmod f)\subset B_f$ are all the distinct maximal ideals of $B_f$.
\end{lemma}
\begin{proof}
    Denote $\pi : B_f\to\bar{B_f}$. We have $B_f/\mathfrak{m}_i \simeq \bar{B_f}/(\bar{g_i})$, so $\mathfrak{m}_i$'s are maximal.
    Note that $\mathfrak{m}_i = \pi^{-1}(\bar{g}_i)$.

    Take $\mathfrak{n}\in\spm B_f$.
    If $\mathfrak{n}\supset\mathfrak{m}$, then $\mathfrak{n} = \pi^{-1}\pi\mathfrak{n}$,
    and goes to a maximal ideal in $\bar{B_f}$ (because $\bar{B_f}/\pi\mathfrak{n} \simeq B_f/\mathfrak{n}$),
    so $\mathfrak{n} = \pi^{-1}(\bar{g}_i) = \mathfrak{m}_i$.

    So assume that $\mathfrak{m}\not\subset\mathfrak{n}$, then $\mathfrak{n} + \mathfrak{m}B_f = B_f$.
    (In this case $\mathfrak{n}/(\mathfrak{n\cap m})\simeq \bar{B_f}$ as $B_f$-module, and thus $\pi^{-1}\pi\mathfrak{n} = B_f$.)
    Therefore \[\frac{B_f}{\mathfrak{n}} = \frac{\mathfrak{n}+\mathfrak{m}B_f}{\mathfrak{n}} \simeq \frac{\mathfrak{m}B_f}{\mathfrak{n}}.\]
    Since $A$ is local and $B_f$ is a f.g. $A$-mod, by Nakayama's lemma, we see $\mathfrak{n} = B_f$. Contradiction.


\end{proof}

Now take $A$ to be a DVR with $\mathfrak{m} = (\varpi)$ and $K = \Frac A$. Put $L := K[X]/(f)$.
We give two cases where $B_f$ is a DVR.

\subsubsection*{Unramified case}
Let $\bar{f}\in k[X]$ be irreducible. Then $B_f$ is a DVR with maximal ideal $\mathfrak{m}B_f$.
\begin{corollary}
    $f\in A[X]$ is also irreducible, so $L$ is a field.
    Moreover, $B_f$ is the integral closure of $A$ in $L$, and $L/K$ is unramified if $\bar{f}$ is separable.
\end{corollary}
\begin{proof}
    $L = K[X]/f \simeq \left( A[X]/f \right)\otimes_{A} K = B_f\otimes_A K$.
    As $B_f$ is a domain, $L$ is a field and $L = \Frac B_f$.
    It left to prove that $B_f$ is integrally closed, ???????
\end{proof}
\subsubsection*{Totally ramified case}
Let $f\in A[X]$ be an \textbf{Eisenstein polynomial}, i.e., \[f = X^n + a_{n-1}X^{n-1} + \cdots  + a_0,\ a_i\in\mathfrak{m},\ a_0\notin\mathfrak{m}^2.\]
\begin{proposition}
    $B_f$ is a DVR, with maximal ideal generated by image of $X$ and residue field $k$.
\end{proposition}
\begin{proof}
    Let $x$ be the image of $X$ in $B_f$.
    We have $\bar{f} = X^n$, so $B_f$ is a local ring with maximal ideal $(\mathfrak{m}, x)$.
    Observe that $a_0\in\mathfrak{m\setminus m^2}$, hence it uniformizes $\mathfrak{m}\subset A$, and $-a_0\bmod f = x^n + \cdots + (a_1\bmod f)x$, we have $(\mathfrak{m}, x) = (x)$.
\end{proof}
Similarly, we have $f$ irreducible and $L$ is a field with $B_f$ the integral closure of $A$ in $L$.









\end{document}