\documentclass{article}

% \documentclass{article}
\usepackage{fontspec}
\usepackage{amsmath, amssymb, amsthm, amsbsy, mathrsfs}
\usepackage{stmaryrd}
\usepackage{enumerate}

\usepackage{xr-hyper} % 支持跨文件的超链接
\usepackage[colorlinks,
linkcolor=cyan,
anchorcolor=blue,
citecolor=blue,
]{hyperref}
\usepackage[capitalize]{cleveref}

\externaldocument[UND-]{../Undergrad/something}
\externaldocument[ANT-]{../ANT-2022-summer/note_summer_ANT}
\externaldocument[EC-]{../EC-2024/notes_elliptic_curves}
\externaldocument[LF-]{../LocalField-2024/notes_cft}
\externaldocument[MF-]{../ModularForm-2024-fall/note_modularForm}

\usepackage[margin = 1in, headheight = 12pt]{geometry}
\usepackage{bbm}
\usepackage{tikz-cd}

% \setmainfont{Arial}

\linespread{1.2}

\theoremstyle{definition}

\newtheorem{theorem}{Theorem}

% This environment is terrible
\newenvironment{theoremprime}[1]
  {\renewcommand{\thetheorem}{\ref{#1}$'$}%
   \addtocounter{theorem}{-1}%
   \begin{theorem}}
  {\end{theorem}}

\newtheorem{definition}{Definition}
\newtheorem{propdef}[definition]{Prop-Def}
\newtheorem{exercise}{Exercise}[section]
\newtheorem{problem}{Problem}[section]
\newtheorem{example}{Example}[section]
\newtheorem{proposition}{Proposition}[section]
\newtheorem{lemma}{Lemma}[section]
\newtheorem{corollary}{Corollary}[section]

\theoremstyle{remark}
\newtheorem*{remark}{Remark}

\renewcommand{\Re}{\mathop{\mathrm{Re}}}
\renewcommand{\Im}{\mathop{\mathrm{Im}}}
\renewcommand{\bar}{\overline}
\renewcommand{\tilde}{\widetilde}
\renewcommand{\hat}{\widehat}

% 新命令
% 数学对象
    \newcommand{\R}{\mathbb{R}}
    \newcommand{\C}{\mathbb{C}}
    \newcommand{\Q}{\mathbb{Q}}
    \newcommand{\Z}{\mathbb{Z}}
    \DeclareMathOperator{\GL}{GL}
    \DeclareMathOperator{\SL}{SL}
    \newcommand{\p}{\mathfrak{p}}
    \renewcommand{\P}{\mathbb{P}}
    \newcommand{\A}{\mathbb{A}}
    \newcommand{\Ga}{{\mathbb{G}_{\mathrm{a}}}}
    \newcommand{\Gm}{{\mathbb{G}_{\mathrm{m}}}}
% 集合
    \newcommand{\sminus}{\smallsetminus} % (集合)差
    \newcommand{\inject}{\hookrightarrow} % 单射
    \newcommand{\surject}{\twoheadrightarrow}
% 范畴
    \newcommand{\op}[1]{{#1}^{\mathrm{op}}} % 反范畴
    \DeclareMathOperator{\Hom}{Hom}
    \DeclareMathOperator{\enom}{End} % 自态射
    \DeclareMathOperator{\End}{End} % 自态射
    \DeclareMathOperator{\isom}{Isom} % 同构
    \DeclareMathOperator{\aut}{Aut} % 自同构
    \DeclareMathOperator{\im}{im} % 像
    \newcommand{\Set}{\mathbf{Set}} % 集合范畴
    \newcommand{\Abel}{\mathbf{Ab}} % 群范畴
    \newcommand{\Ring}{\mathbf{Ring}}
    \newcommand{\Cring}{\mathbf{CRing}}
    \newcommand{\Alg}{\mathbf{Alg}}
    \newcommand{\Mod}{\mathbf{Mod}}
    \DeclareMathOperator{\Id}{id}
%向量空间, 矩阵
    \DeclareMathOperator{\rank}{rank} % 秩
    \DeclareMathOperator{\tr}{Tr} % 迹
    \newcommand{\tran}[1]{{#1}^{\mathrm{T}}} % 转置
    \newcommand{\ctran}[1]{{#1}^{\dagger}} % 共轭转置
    \newcommand{\itran}[1]{{#1}^{-\mathrm{T}}} % 逆转置
    \newcommand{\ictran}[1]{{#1}^{-\dagger}} % 逆共轭转置
    \DeclareMathOperator{\codim}{codim} % 余维数
    \DeclareMathOperator{\diag}{diag} % 对角阵
    \newcommand{\norm}[1]{\left\| #1\right\|} % 范数
    \DeclareMathOperator{\lspan}{span} % 张成
% 群
    \DeclareMathOperator{\inn}{Inn} % (群)内自同构
    \newcommand{\nsg}{\vartriangleleft} % 正规子群
    \newcommand{\gsn}{\vartriangleright} % 正规子群
    \DeclareMathOperator{\ord}{ord} % 元素的阶
    \DeclareMathOperator{\stab}{Stab} % 稳定化子
    \DeclareMathOperator{\sgn}{sgn} % 符号函数
% 环, 域
    \DeclareMathOperator{\cha}{char} % 特征
    \DeclareMathOperator{\spec}{Spec} % 素谱
    \DeclareMathOperator{\maxspec}{MaxSpec} % 极大谱
    \newcommand{\spm}{\maxspec} %极大谱
    \DeclareMathOperator{\gal}{Gal} % Galois群
    \DeclareMathOperator{\Gal}{Gal} % Galois群
    \DeclareMathOperator{\Frac}{Frac} % 分式环
% 同调代数
    \DeclareMathOperator{\ext}{Ext} 
    \DeclareMathOperator{\Tor}{Tor}
% 微积分
    \newcommand*{\dif}{\mathop{}\!\mathrm{d}} %(外)微分算子
% 流形
    \DeclareMathOperator{\lie}{Lie}
%代数几何
    \DeclareMathOperator{\proj}{Proj} % 射影谱
%多项式
    \DeclareMathOperator{\disc}{disc} % 判别式
    \DeclareMathOperator{\res}{res} % 结式
% 结构简写
    \newcommand{\pdfrac}[2]{\dfrac{\partial #1}{\partial #2}} % 偏微分式
    \newcommand{\isomto}{\stackrel{\sim}{\rightarrow}} % 有向同构
    \newcommand{\gene}[1]{\left\langle #1 \right\rangle} % 生成对象
% 文字缩写
    \newcommand{\opin}{\;\mathrm{open\;in}\;}
    \newcommand{\st}{\;\mathrm{s.t.}\;}
    \newcommand{\ie}{\;\mathrm{i.e.,}\;}

% 重定义命令
\renewcommand{\hom}{\mathop{\mathrm{Hom}}}
\renewcommand{\vec}{\boldsymbol}
\renewcommand{\and}{\;\text{and}\;}

% 编号
\newcommand{\cnum}[1]{$#1^\circ$} %右上角带圆圈的编号
\newcommand{\rmnum}[1]{\romannumeral #1}
\newcommand{\myit}{$\diamond$}

% \tikzcdset{scale cd/.style={every label/.append style={scale=#1},
%     cells={nodes={scale=#1}}}}


\newcommand{\F}{\mathbb{F}}
\newcommand{\perf}{\mathrm{perf}}
\newcommand{\sep}{\mathrm{sep}}
\renewcommand{\O}{\mathcal{O}}
\newcommand{\m}{\mathfrak{m}}
\DeclareMathOperator{\pic}{Pic}
\DeclareMathOperator{\frob}{Frob}
\newcommand{\llpar}{(\!(}
\newcommand{\rrpar}{)\!)}
% todo: define (())
\newcommand{\ab}{\mathrm{ab}}
\DeclareMathOperator{\lc}{lc}

\title{Notes on Explicit CFT for Function Fields}
\author{Lei Bichang}

\begin{document}
\maketitle

\section{Review of CFT}
Let $F$ be a global field, $C_F = \A_F^\times/F^\times$ be its idele class group, and $F^\ab$ be its maximal abelian extension inside a separable closure in a fixed algebraic closure $\bar F$.
The class field theory asserts that the Artin map
\[\theta_F : C_F\to\gal(F^\ab/F)\]
is a continuous group homomorphism with dense image,
establishing a bijection
\[\{\text{finite abelian extensions of }F\}\longleftrightarrow \{\text{finite index open subgroups of }C_F\}.\]
The direction ``$\to$'' is computable: for a finite abelian $L/F$,
the corresponding open subgroup of $C_F$ is the kernel $U$ of $C_F\stackrel{\theta_F}{\to}\gal(F^\ab/F)\to\gal(L/F)$,
which can be computed as $U = N_{L/F}(C_L)$\footnote{$N_{L/F} : C_L\to C_F$ is the norm map.
    The norm for an idele is just the multiplication of the norm at every places.}.
% But the other direction ``$\gets$'' is not known in general:
% given a finite index open subgroup of $C_F$,
% the Artin map $\theta_F$ doesn't produce the generators of the corresponding extension $L/F$.
% It neither gives an explicit description of $F^\ab$.

The goal of explictit class field theory is to find the construciton ``$\gets$'', and to describe $F^\ab$.
Known cases for number fields inlcude $\Q$ and imaginary quadratic fields, and they all uses torsion points of some geometric object ($\Gm$ and CM elliptic curves, respectively).
In the article \cite{zywina2011explicitclassfieldtheory}, the author constructed the inverse of Artin map for function fields using one distinguished ``place at infinity'' with a sign function as well as Drinfeld modules, a characteristic $p$ analogue for $\Gm$ and elliptic curves.
In the end, he described explicitly the structure of $k(t)^\ab$, the maximal abelian extension of the field of rational functions over a finite field $k$.
Most of the proofs for general fact about Drinfeld modules can be found in \cite{goss2012basic}, and those specific for function fields can be found in \cite{Ha74} and \cite{zywina2011explicitclassfieldtheory}.
% \subsubsection*{Example: explicit class field theory for \texorpdfstring{$\Q$}{Q}}
% Recall that for each integer $n\ge 1$, we have the mod-$n$ cyclotomic character $\chi_n : \gal_\Q\to (\Z/n\Z)^\times$ from the $\gal_\Q$-action on $\mu_n(\bar{\Q})$.
% These characters patches to a continuous representation \[\chi : \gal_\Q^\ab\to \hat\Z^\times,\]
% which, by Kronecker-Weber, is an isomophism.
% So $\Q^\ab = \bigcup_{n}\Q(\mu_n)$.

% Let $\R^+ := \{x\in \R^\times\mid x > 0\}$.
% The quotient map $\R^+\times\hat\Z^\times\inject \A_\Q^\times\surject C_\Q$ is an isomophism,
% and taking completion gives us an isomophism $\hat\Z^\times\simeq \hat C_\Q$.
% Composing this map with the cyclotomic character $\chi$,
% we obtain an isomophism \[\rho : \gal_\Q^\ab\to \hat C_\Q\]
% such that $\sigma\mapsto\rho(\sigma)^{-1}$ is the inverse of the Artin map $\theta_\Q : \hat C_\Q\to \gal_\Q^\ab$.

% There are two important ingredients invloved:\begin{enumerate}[(a)]
%     \item a $\Z$-module structure on the separable closure $\bar\Q$ different from the usual one (given by $\Gm$), whose torsion points give rise to Galois representations;
%     \item a subgroup $\R^+ = \ker(\sgn)$ of $\R^\times$, where $\sgn : \R^\times\to\{\pm 1\}$ is the sign function.
% \end{enumerate}

% For function fields, we will replace $\Gm$ by some Drinfeld modules, and distinguish a place $\infty$ with a fixed sign function.
% These data will enable us to construct the inverse of the Artin map,
% and we shall describe more precisely the case of rational function field.

\section{Function Fields and Drinfeld Modules}
Let $k = \F_q$ be a finite field,
$F$ be a global function field with a fixed place\footnote{
    A \textbf{place} of a function field is a valuation subring, or equivalently, an equivalence class of discrete valuations.
    Note that there are no archimedean places.
} $\infty$, and with field of constants $k$,
i.e.\! $F$ is a finite extension of the field of rational functions $k(t)$ over $k$.

If $\lambda$ is a place of $F$,
we denote by $F_\lambda$ the completion at $\lambda$, by $\C_\lambda$ the completion of $\overline{F}_\lambda$,
by $\O_\lambda\subset F_\lambda$ the valuation ring,
by $\F_\lambda := \O_\lambda/\m_\lambda$ the residue field at $\lambda$, and by $\ord_\lambda$ the normalized valuation on $F_\lambda$ with value group $\Z$.
We regard $\F_\lambda\subset\O_\lambda\subset F_\lambda$ as a subfield via the Teichm\"uller lifting.
% This is the same as picking a uniformizer $\pi$ so that $F_\lambda = \F_\infty\llpar \pi\rrpar$ and then identifying $\F_\infty$ as the field of constants.
% any uniformizer $t\in\m_\lambda$ yields an isomorphism
% \[\O_\lambda\simeq \F_\lambda[t],\]
% giving us an embedding $\F_\infty\hookrightarrow \O_\lambda$.


For any extension $L$ of $k$, we denote by $\bar{L}$ an algebraic closure. Let $L^\sep$ be the separable closure of $L$ in $\bar L$,
$\gal_L = \gal(L^\sep/L)$ be the absolute Galois group.
% \subsection{Some constructions in function fields}
% Here are some facts about function fields that are different from number fields.

\subsection{The holomorphy rings}
Let $A := \{x\in F\mid \ord_\lambda(x)\ge 0, \forall \lambda\ne \infty\}$, the ring of functions that are regular away from $\infty$.
% It is a holomorphy ring.
By the general theory of holomorphy rings, $A$ is a Dedekind domain with fractional field $\Frac(A) = F$, and there is a 1-1 correspondence between maximal ideals of $A$ and the places of $F$ except for $\infty$.

% \subsubsection{holomorphy ring}
% There are no ``ring of integers'' in function fields.
% Instead, we consider the holomorphy rings.
% A \textbf{holomorphy ring} is a ring of the form
% \[\O^S := \bigcap_{\lambda\notin S} \O_\lambda = \{x\in F\mid \ord_\lambda(x)\ge 0,\ \forall\lambda\notin S\},\]
% where $S$ is a non-empty set of (not all the) places of $F$.
% % The ring $\O_S$ is the subring of $F$ consisting of elements regular away from $S$.
% \begin{proposition}
% Let $S$ be a finite set of places of $F$. Then $\O^S$ is a Dedekind domain with fractional field $\Frac(\O^S) = F$,
% and there is a bijection between maximal ideals of $\O^S$ and places of $F$ that are not in $S$.
% % Moreover, there is a bijection \[\{\text{place of }F\text{ not in }S\}\longleftrightarrow \spm \O^S\] giving by $\lambda\mapsto \m_\lambda\cap\O^S$, which induces isomorphisms $\F_\lambda = \O_\lambda/\m_\lambda\simeq \O^S/(\m_\lambda\cap\O^S)$.
% \end{proposition}
% We will consider $A := \O^{\{\infty\}}$, so that every place $\lambda\ne\infty$ of $F$ cooresponds to a maximal ideal of $A$.

\subsection{The Weil group}
Let $L$ be an extension of $k$.
The algebraic closure $\bar k$ of $k$ in $\bar F$ is contained in $L^\sep$,
and the absolute Galois group $\gal_L = \gal(L^\sep/L)$ stabilizes $\bar k$.
% Hence we have
% an exact sequence of topological groups
% \[1\longrightarrow\gal(L^\sep/L\bar{k})\longrightarrow \gal_L\stackrel{\deg}{\longrightarrow}\hat\Z\to 0,\]
% where $\deg : \gal_L\to \gal_k\simeq\hat\Z$ is defined by
% $\sigma(x) = \frob_q^{\deg(\sigma)}(x)$ for $\sigma\in \gal_L,\ x\in\bar{k}$.
Therefore, we can construct Weil group for $L$ just like for local fields.
The \textbf{Weil group} is the subgroup $W_L$ of $\gal_L$ of elements $\sigma$ that acts on $\bar{k}$ by an integral power of the Frobenius-$q$, i.e. $\sigma(x) = x^{q^{\deg(\sigma)}}$ for $\sigma\in W_L,\ x\in\bar{k}$.
The kernel of the map $\deg : W_L\to\Z$ is still $\gal(L^\sep/L\bar{k})$.
We endow $W_L$ with the weakest topology for which
\[1\longrightarrow \gal(L^\sep/L\bar{k})\longrightarrow W_L\stackrel{\deg}{\longrightarrow} \Z\longrightarrow0\]
is an exact sequence of topological groups,
where $\gal(L^\sep/L\bar{k})$ has its usual profinite topology and $\Z$ has discrete topology\footnote{This is not the topology induced from $\Z\subset \hat\Z$.}.
% \begin{itemize}
%     \item $\gal(L^\sep/L\bar{k})$ has its usual profinite topology,
%     \item $\Z$ has discrete topology\footnote{This is not the topology induced from $\Z\subset \hat\Z$.}.
% \end{itemize}
The inclusion $W_L\hookrightarrow \gal_L$ is still continuous with dense image.

% \subsection{Drinfeld modules}
\subsection{Drinfeld modules and isogenies}

Let $L$ be an extension of $k$, $L[T]$ be the ring of polynomial over $L$.
Consider the Frobenius-$q$ map \[\tau : L[T]\to L[T]\quad \sum_{i = 0}^n a_iT^i\mapsto \sum_{i = 0}^n a_i^qT^{iq}.\]
This is a $k$-linear endomorphism of $L[T]$,
and we denote by $L[\tau]$ the sub-$L$-algebra of $\End_{k}(L[T])$ generated by $\tau$.
The ring $L[\tau]$ is a ring of \textbf{twisted polynomials},
because it is non-commutative: $\tau a = a^q\tau$, $\forall a\in L$.


Recall that $A = \{x\in F\mid \ord_\lambda(x)\ge 0, \forall \lambda\ne\infty\}$.
Let $L$ be an extension of $F$.
A \textbf{Drinfeld $A$-module\footnote{
    There is a more general definition, but we only need and consider Drinfeld modules of this kind.
} over $L$} is a homomorphism
\[\phi : A\to L[\tau]\quad x\mapsto \phi(x) =: \phi_x\]
of $k$-algebras, such that
$\phi(A)$ is \textit{not contained} in $L\subset L[\tau]$, and
the map \[A\to L[\tau]\to L\quad x\mapsto \phi_x = a_0 + a_1\tau + \cdots + a_n\tau^n\mapsto a_0\] is the restriction of the inclusion map $F\hookrightarrow L$ to $A$.
In particular, $\phi : A\inject L[\tau]$ is injective.


Let $\phi$ and $\phi'$ be two Drinfeld modules $A\to L[\tau]$, $M$ be an extension of $L$.
An \textbf{isogeny} over $M$ from $\phi$ to $\phi'$
is an $f\in M[\tau]\sminus\{0\}$ such that \[f\phi_a = \phi_a'f,\quad\forall a\in A.\]
An \textbf{isomorphism} over $M$ from $\phi$ to $\phi'$ is an invertible isogeny, namely an isogeny $f\in M[\tau]^\times$.



% \subsubsection{The rank}

% Let $L^\perf$ be the purely inseparable closure of $L$ in $\bar L$,
% then $L^\perf(\!(\tau^{-1})\!)$ is a well-defined\footnote{
%     We need to have all $p$-th root,
%     so that $\tau^{-1}a = a^{1/q}\tau$ is always valid.
% } skew-field, containing $L[\tau]$ as a subring.
% The injection $\phi : A\hookrightarrow L[\tau]$ extends uniquely to a embedding
% $\phi : F\hookrightarrow L^{\perf}(\!(\tau^{-1})\!)$.
% The function \[v_\phi : F\to \Z\cup\{\infty\}\quad x\mapsto \ord_{\tau^{-1}}(\phi_x) \]
% is a nontrivial valuation as $\phi(A)\not\subset L$, and $v_\phi(x)\le 0$ for all $x\in A\sminus\{0\}$.
% Therefore $v_\phi$ is equivalent to the valuation $\ord_\infty$ attached to the place $\infty$.
% We define the \textbf{rank of $\phi$} to be $r\in\Q$ such that
% \begin{equation}\label{eq: def of rank}
%     \ord_{\tau^{-1}}(\phi_x) = rd_\infty\ord_\infty(x),
% \end{equation}
% for $x\in F$, where $d_\infty = [\F_\infty : k]$.
% % is the inertia degree of $F$ at $\infty$.
% The rank $r$ is always an integer, and it is an isogeneous invariant.

\subsubsection{Torsion submodules and the rank}
A Drinfeld module $\phi : A\to L[\tau]$
defines an $A$-module structure on $\bar L$ by\[x\cdot b := \phi_x(b),\quad\forall x\in A, b\in\bar L.\]
Every $\phi_x$ acts by a polynomial $\phi_x(T) = a_0T + a_1T^q + \cdots + a_nT^{q^n}$ with $a_i\in L$.
This polynomial is separable, because $x\mapsto \phi_x\mapsto a_0$ is injective.
Therefore $\phi$ gives an $A$-module structure on $L^\sep$.

For an ideal $\mathfrak{a}$ of $A$,
we define the $\mathfrak{a}$-torsion submodule to be\[\phi[\mathfrak{a}] :=
\left\{ b\in\bar L\mid \phi_x(b) = 0,\forall x\in\mathfrak{a} \right\},\]
an $A$-submodule of $L^\sep$ with $A$-module structure from $\phi$,
carrying a natrual $\gal_L$-action.

Similar to elliptic curves,
$\phi[\mathfrak{a}]$ is a finite free $A/\mathfrak{a}$-module, whose rank $r\in \Z$ is the same for all ideals $\mathfrak{a}\subset A$.
We call this number $r$ the \textbf{rank} of the Drinfeld module $\phi$.
It is an isogeneous invariant.

\subsection{The sign functions and the \texorpdfstring{$\varepsilon$}{epsilon}-normalized Drinfeld modules}
A \textbf{sign function for $F_\infty$} is a
group homomorphism $F_\infty^\times\to\F_\infty^\times$ such that $\varepsilon|_{\F_\infty^\times} = \Id_{\F_\infty^\times}$, and we write \[F_\infty^+ := \{x\in F_\infty^\times\mid \varepsilon(x) = 1\} = \ker(\varepsilon : F_\infty\to \F_\infty^\times).\]
Such a function $\varepsilon$ is determined by its value on any uniformizer\footnote{
    Choosing a uniformizer $\pi$ of $F_\infty$ yields a decomposition $F_\infty^\times\simeq \F_\infty^\times\times (1 + \m_\infty)\times \pi^{\Z}$.
    The value of $\varepsilon$ on $\F_\infty^\times$ is fixed,
    and it must be trivial on the pro-$q$ group $1 + \m^\infty$.
}.

We will fix a sign function $\varepsilon$ for $F_\infty$ and require our Drinfeld modules to be \textbf{$\varepsilon$-normalized}. This is a technical condition we don't need to worry much,
because every Drinfeld module over $L$ is isomorphic to some $\varepsilon$-normalized Drinfeld module of the \textit{same rank} over the algebraic closure $\bar L$.

\subsection{Hayes modules and group actions on it}

Fix a sign function $\varepsilon : F_\infty^\times\to\F_\infty^\times$ for $F_\infty$.
A \textbf{Hayes module for $\varepsilon$}
is a $\varepsilon$-normalized Drinfeld module $\phi : A\to \C_\infty[\tau]$ of rank $1$.
The Drinfeld modules of rank $1$ over $\C_\infty$ exist and can be constructed analytically.
Since $\C_\infty$ is algebraically closed, the Hayes modules must exist.

Let $X_\varepsilon$ be the set of Hayes modules for $\varepsilon$.
There is a natrual action of the group $\mathcal{I}_A$ of fractional ideals of $A$ on $X_\varepsilon$, denoted by \[(\mathfrak{a}, \phi)\mapsto\mathfrak{a} * \phi, \quad \mathfrak{a}\in\mathcal{I}_A,\ \phi\in X_\varepsilon.\]
This action has the following properties.
\begin{enumerate}[(i)]
    \item If $\mathfrak{a}\subset A$ is an integral ideal, then there is a unique $\phi_\mathfrak{a}\in L[\tau]$, and $\mathfrak{a} * \phi$ is the unique Drinfeld module making $\phi_\mathfrak{a}$ and isogeny $\phi\to\mathfrak{a} * \phi$. 
    In particular, $\phi_A = 1$ and $A*\phi = \phi$.
    These isogenies are important in later construcitons.
    \item The subgroup $\mathcal{P}_A^+ := \{(x)\mid x\in F^\times\cap F_\infty^+\}$ of $\mathcal{I}_A$
    % consisting of principal fractional ideals generated by $x\in F^\times\cap F_\infty^+$ 
    acts trivially on $X_\varepsilon$.
\end{enumerate}
We call $\pic^+(A) := \mathcal{I}_A/\mathcal{P}^+_A$ the \textbf{narrow class group}, so that $X_\varepsilon$ is a $\pic^+(A)$-set.
\begin{proposition}\label{action of Pic+ on the set of Hayes modules}
    % The narrow class group $\pic^+(A)$ acts \textit{freely} and \textit{transitively} on $X_\varepsilon$.
    The set $X_\varepsilon$ is a principal homogeneous space for $\pic^+(A)$, i.e. $\pic^+(A)$ acts freely and transitively on $X_\varepsilon$.
\end{proposition}
The group $\pic^+(A)$ will be realized as the Galois group for an ``almost'' unramified extension.
Define the \textbf{narrow Hilbert class field} or the \textbf{normalizing field for $(F, \infty, \varepsilon)$} to be the extension \[H_A^+ := F\left( \{\text{coefficient of }\phi_x\mid \phi\in X_\varepsilon, x\in A\} \right)\]
of $F$ in $\C_\infty$. This is the minimal extension of $F$ on which all Hayes modules for $\varepsilon$ are defined.
\begin{proposition}\label{narrow Hilbert class field is unramified outside infty}
    The extension $H_A^+/F$ is finite abelian, and it is unramified away from $\infty$.
\end{proposition}
There is thus a natrual action of $\gal_F$ on $X_\varepsilon$ through $\gal(H_A^+/F)$, given by \[\sigma(\phi)_x := \sigma(\phi_x)\footnotemark,\quad \forall\sigma\in\gal_F,\ \phi\in X_\varepsilon,\ x\in A.\]
\footnotetext{
    $\gal_F$ acts on $\bar{F}[\tau]$ by acting on the coefficients. It is direct to check that $\gal_F$ stabilizes $X_\varepsilon$
    by definition.
}
Any $\phi\in X_\varepsilon$, by \cref{action of Pic+ on the set of Hayes modules}, induces an injective group homomorphism
\[\Psi : \gal(H_A^+/F)\inject \pic^+(A),\]
such that $\sigma(\phi) = \Psi(\sigma) * \phi$ for all $\sigma\in\gal_F$.

\begin{proposition}\label{the isomophism from gal to Pic+ and the value of Frobenius under this isomophism}
    $\Psi : \gal(H_A^+/F)\to \pic^+(A)$ is an isomorphism, independent of the choice of $\phi$.
    For each non-zero prime $\mathfrak{p}$ of $A$,
    the class of $\Psi(\frob_\mathfrak{p})$ in $\pic^+(A)$ equals the class of $\mathfrak{p}$.
\end{proposition}


\section{Construction of the Inverse to the Artin Map}
We fix the tuple $(F, \infty, \varepsilon)$
and a Hayes module $\phi\in X_\varepsilon$.
\subsection{\texorpdfstring{$\lambda$}{lambda}-adic representation}
Let $\lambda$ be a place of $F$.
Take $\sigma\in\gal_F$. By \cref{the isomophism from gal to Pic+ and the value of Frobenius under this isomophism}, pick an ideal $\mathfrak{a}$ of $A$ such that $\sigma(\phi) = \mathfrak{a} * \phi$.
\begin{itemize}
\item $\lambda\ne \infty$.
Regarding $\lambda$ as a prime ideal of $A$, we consider the rank $1$ free $A/\lambda^e$-module $\phi[\lambda^e]$ for $e\in\Z_{\ge 1}$.
Define the \textbf{$\lambda$-adic Tate module} to be\begin{align*}
    &T_\lambda(\phi) := \Hom_A(F_\lambda/\O_\lambda,\ \phi[\lambda^\infty]),
\end{align*}
which is a free $\O_\lambda$-module of rank $1$.
Hence $V_\lambda(\phi) := T_\lambda(\phi)\otimes_{\O_\lambda} F_\lambda$ is an $1$-dimensional $F_\lambda$-vector space.
We have the following two isomophisms between vector spaces.
\begin{itemize}
    \item $\sigma$ induces $\phi[\lambda^e]\simeq (\sigma (\phi))[\lambda^e]$ for all $e\in\Z_{\ge 1}$,
    patching to an isomophism $V_\lambda(\sigma) : V_\lambda(\phi)\simeq V_\lambda(\sigma(\phi))$.
    \item The isogeny $\phi_\mathfrak{a} : \phi\to\mathfrak{a} * \phi$ induces an isomorphism\footnote{
        Since $\phi$ has rank $1$, it is equivalent to that $V_\lambda(\phi_\mathfrak{a})$ is non-zero.
        This is true, because, parallel to elliptic curves, taking Tate module is a faithful functor; see \cite{goss2012basic}, \S 4.10.
        % i.e. for any two Drinfeld modules $\phi$ and $\phi'$ over $L$, the map
        % \[\Hom_{L}(\phi, \phi')\hookrightarrow \Hom_{\O_\lambda}(T_\lambda(\phi), T_\lambda(\phi'))\]
        % is injective.
    } $V_\lambda(\phi_\mathfrak{a}) : V_\lambda(\phi)\simeq V_\lambda(\mathfrak{a}*\phi)$.
\end{itemize}
As $\mathfrak{a} * \phi = \sigma(\phi)$, we obtain an element $V_\lambda(\phi_\mathfrak{a})^{-1}\circ V_\lambda(\phi)\in\GL_{F_\lambda}(V_\lambda(\sigma)) = F_\lambda^\times\cdot\Id$,
corresponding to an element $\rho_\lambda^\mathfrak{a}(\sigma)\in F_\lambda^\times$.

\item $\lambda = \infty$.
If $\sigma\in W_F$, the next \cref{lem: to define infty-adic representation} provides a unique element $\rho_\infty^\mathfrak{a}(\sigma)\in F_\infty^+$.
\begin{lemma}\label{lem: to define infty-adic representation}
    There exists some series $u\in F^\sep\llbracket\tau^{-1}\rrbracket^\times$, such that
    $u^{-1}\phi(F_\infty)u\subset \bar{k}\llpar\tau^{-1}\rrpar.\footnotemark$
    \footnotetext{Any Drinfeld module $\phi : A\to H_A^+[\tau]$ extends to an injective homomorphism $\phi : F_\infty\to \left( H_A^+ \right)^\perf\llpar\tau^{-1}\rrpar$.
    }
    For such a series $u$, if $\sigma\in W_F$,
    then there is a unique element $\rho_\infty^\mathfrak{a}(\sigma)\in F_\infty^+$,
    such that \[\phi_\mathfrak{a}^{-1}\cdot\sigma(u)\cdot\tau^{\deg(\sigma)}\cdot u^{-1} = \phi(\rho_\infty^\mathfrak{a}(\sigma)).\]
\end{lemma}

\end{itemize}

These elements $\rho_\lambda^\mathfrak{a}(\sigma)$ has the following properties.

\begin{lemma}\label{basic properties of adic representations}
    Let $\lambda$ be a place of $F$, $\sigma, \gamma\in\gal_F$ (in $W_F$ if $\lambda = \infty$) and $\mathfrak{a}, \mathfrak{b}$ be ideals of $A$.
    \begin{enumerate}[(i)]
        \item If $\sigma(\phi) = \mathfrak{a}*\phi$ and $\gamma(\phi) = \mathfrak{b}*\phi$,
        then $(\sigma\gamma)(\phi) = (\mathfrak{a}\mathfrak{b}) * \phi$, and $\rho_{\lambda}^{\mathfrak{ab}}(\sigma\gamma) = \rho_\lambda^\mathfrak{a}(\sigma)\rho_\lambda^\mathfrak{b}(\gamma)$.
        \item If $\sigma(\phi) = \mathfrak{a}*\phi = \mathfrak{b}*\phi$,
        then $\rho_\lambda^\mathfrak{a}(\sigma)\rho_\lambda^\mathfrak{b}(\sigma)^{-1}\in F^\times\cap F_\infty^+$ and $\mathfrak{b}^{-1}\mathfrak{a}$ is generated by $\rho_\lambda^\mathfrak{a}(\sigma)\rho_\lambda^\mathfrak{b}(\sigma)^{-1}$
        \item  If $\lambda\ne\infty$, and $\sigma(\phi) = \mathfrak{a} * \phi$,
        then $\ord_\lambda(\rho_\lambda^\mathfrak{a}(\sigma)) = -\ord_\lambda(\mathfrak{a})$, the largest power of $\lambda$ dividing $\mathfrak{a}$.
    \end{enumerate}
\end{lemma}

If $\sigma\in\gal_{H_A^+}$,
then $\sigma(\phi) = \phi = A * \phi$.
By \cref{basic properties of adic representations} (i),
we obtain homomorphisms
\[\rho_\lambda : \gal_{H_A^+}\to \O_\lambda^\times\quad \sigma\mapsto \rho_\lambda^A(\sigma)\]
for $\lambda\ne\infty$, and the homomorphism
\[\rho_\infty : W_{H_A^+}\to F_\infty^+,\quad \sigma\mapsto \rho_\infty^A(\sigma).\]
In particular, $\phi_A = 1$, so the representation $\rho_\lambda$ is the representation of $\gal_{H_A^+}$ on $T_\lambda(\phi)$ and hence it takes value in $\O_\lambda^\times$.
These representations $\rho_\lambda$ are continuous and unramified at all places of $H_A^+$ not over $\lambda$ or $\infty$.

% \begin{lemma}\label{lem: non-infty-adic representation at narrow Hilbert class field is almost unramified}
%     $\rho_\lambda : \gal_{H_A^+}\to\O_\lambda^\times$ is continuous and unramified at all places of $H_A^+$ not over $\lambda$ or $\infty$.
% \end{lemma}
% \begin{lemma}\label{lem: infty-adic representation at Weil group of narrow Hilbert class field is almost unramified}
%     $\rho_\infty : W_{H_A^+}\to F_\infty^+$ is continuous and unramified at all places of $H_A^+$ not over $\infty$.
% \end{lemma}


\subsection{The inverse of the Artin map}
For each $\sigma\in W_F$,
fix an ideal $\mathfrak{a}_\sigma$ of $A$,
such that $\sigma(\phi) = \mathfrak{a}_\sigma * \phi$.
By \cref{basic properties of adic representations}, $\left( \rho_\lambda^{\mathfrak{a}_\sigma}(\sigma) \right)_\lambda$ is an idele of $F$, whose class $\rho(\sigma)$ in $C_F$ is independent of the choice of $\mathfrak{a}_\sigma$,
and the map \[\rho : W_F\to C_F,\quad \sigma\mapsto\rho(\sigma)\]
is a group homomorphism.
The restriction of $\rho : W_F\to C_F$ to $W_{H_A^+}$ is
\[W_{H_A^+}\stackrel{\prod_{\lambda}\rho_\lambda}{\longrightarrow} F_\infty^+\times \prod_{\lambda\ne\infty} \O_\lambda^\times\hookrightarrow\A_F^\times\surject C_F.\]
This homomorphism is continuous since all $\rho_\lambda$ are continuous.
The group $W_{H_A^+}$ has finite index in $W_F$,
so $\rho$ is continuous on $W_F$.
% The group $C_F$ is abelian,
% so $\rho$ factors through the maximal abelian quotient $W_F^\ab$,
% and we denote this map by $\rho : W_F^\ab\to C_F$ again.
Taking profinite completion yields a continuous homomorphism
\[\hat\rho : \gal_F\to \hat C_F.\]
that factors through the maximal abelian quotient $\gal_F^\ab = \gal(F^\ab/F)$.

\begin{theorem}\label{inverse of Artin map}
    The map $\hat\rho : \gal(F^\ab/F)\to \hat{C}_F$ is a topological isomophism depends only on $F$,
    % independent of the choice of $\infty$, $\varepsilon$ and $\phi$,
    and the map \[\gal(F^\ab/F)\to \hat{C}_F\quad \sigma\mapsto \hat\rho(\sigma)^{-1}\]
    is the inverse of the Artin map $\hat\theta_F : \hat C_F\to \gal(F^\ab/F)$.
\end{theorem}
\begin{proof}[Sketch of the proof]
% First, we need a lemma describing the action of $\rho_\lambda$ on Frobenius elements.
% \begin{lemma}\label{value of local representation at Frobenius different from itself and infty}
%     Let $\lambda$ be a place of $F$, $\mathfrak{p}$ be another place of $F$ that is not $\lambda$ or $\infty$.
%     Then $\rho_\lambda^\mathfrak{p}(\frob_\mathfrak{p}) = 1$.
% \end{lemma}
% \begin{remark}[Explaination to the notation $\rho_\lambda^\mathfrak{p}(\frob_\mathfrak{p})$]
%     Let $\lambda$ and $\mathfrak{p}$ be places of $F$ with $\mathfrak{p}\ne \infty$.
%     By \cref{narrow Hilbert class field is unramified outside infty} and \cref{the isomophism from gal to Pic+ and the value of Frobenius under this isomophism},
%     the extension $H_A^+/F$ is unramified at all places $\ne\infty$,
%     and
%     the unique $\frob_\mathfrak{p}\in\gal(H_A^+/F)$
%     satisfies $\frob_\mathfrak{p}(\phi) = \mathfrak{p} * \phi$.
%     The $\gal_F$-action on $X_\varepsilon$ factors through $\gal(H_A^+/F)$,
%     hence
%     any $\frob_\p\in \gal_F$ satisfies $\frob_\p(\phi) = \mathfrak{p} * \phi$.
% \end{remark}

% Now we begin the construciton.
Let $U < C_F$ be an open subgroup of finite index.
Consider the finite abelian extension $L_U := (F^\ab)^{\rho^{-1}(U)}$ of $F$ fixed by $\rho^{-1}(U) < W_F^\ab$,
so that
% $\gal_{L_U}^\ab = \text{the closure of }\rho^{-1}(U)$ in $\gal_F^\ab$.
we have an injective continuous homomorphism
\[\rho_U : \gal(L_U/F) \simeq\gal_F^\ab/\gal_{L_U}^\ab \simeq W_F^\ab/\rho^{-1}(U)\hookrightarrow C_F/U.\]
Using weak approximation and the description of $\rho_\lambda$ on (almost all) Frobenius elements,
one can show that there is a finite set of places $S_U$ containing $\infty$ and all places ramified in $L_U/F$, such that:
\begin{itemize}
    \item for each $\p\notin S_U$, $\rho_U$ sends $\frob_\p$ to the class of $(\cdots, 1, \pi_\p, 1, \cdots)$, where $\pi_\p$ is an uniformizer of $F_\p$;
    \item $\rho_U : \gal(L_U/F)\to C_F/U$ is surjective and thus an isomophism.
\end{itemize}
Therefore the pointwise inverse of $\rho_U^{-1}$ is
$C_F/U\to \gal(L_U/F),\ \alpha\mapsto (\rho_U^{-1}(\alpha))^{-1} = \theta_F(\alpha)|_{L_U}$, the Artin map.

% If $L$ is a finite abelian extension of $F$,
% then the corresponding open subgroup $U_L$ of $C_F$,
% according to class field theory,
% is the kernel of
% $C_F\stackrel{\theta_F}{\to} \gal(L/F)$.
% Hence $L = L_{U_L}$,
% and
The result above together with class field theory shows that $F^\ab = \bigcup_{U}L_U$.
Passing to the limit of these compatible isomophisms $\{\rho_U\}_U$, we get back to $\hat\rho : \gal_F^\ab\to C_F$ and see that it is an isomophism,
whose inverse is the point-wise inverse of the Artin map $\hat\theta_F$.
\end{proof}

% \begin{corollary}
%     The homomorphism $\rho : W_F^\ab\to C_F$ is a topological isomophism, and the map \[W_F^\ab\to C_F\quad \sigma\mapsto \rho(\sigma)^{-1}\]
%     is the inverse of the Artin map $\theta_F : C_F\to W_F^\ab$.
% \end{corollary}

\section{Example: the Rational Function Field}
Let $F = k(t)$.
We consider the usual place $\infty$, so that $A = k[t]$,
$F_\infty = k\llpar t^{-1}\rrpar$,
$\F_\infty = k$,
$\m_\infty = t^{-1}k\llbracket t^{-1}\rrbracket$,
$\ord_{\infty}(t^{-1}) = 1$.
Let $\varepsilon : F_\infty^\times\to k^\times$ be the sign function defined by $\varepsilon(t^{-1}) = 1$,
so that $F_\infty^+ = t^{\Z}\cdot(1 + \m_\infty)$.

% \subsection{The Carlitz module}

The \textbf{Carlitz module} $\phi$ is a Hayes module for $\varepsilon$ defined by \[\phi : A = k[t]\to F[\tau]\quad t\mapsto \phi_t := t + \tau.\]
The normalizing field for $(F, \infty, \varepsilon)$ is $H_A^+ = F$,
so $\phi$ is the only Hayes module for $\varepsilon$.

We have defined the representations $\rho_\lambda : W_F^\ab\to F_\lambda^\times$.
As a corollary of \cref{inverse of Artin map},
\[W_F^\ab \stackrel{\prod_{\lambda}\rho_\lambda}{\longrightarrow} F_\infty^+\times\prod_{\lambda\ne\infty}\O_\lambda^\times\to C_F\]
is an isomophism.
Similar to $\Q$, the second arrow above is an isomophism\footnote{
Let $x\in \A_F^\times$,
Every place $\lambda\ne\infty$ has a ``canonical'' uniformizer $\p\in k[t]$, namely the unique monic irreducible polynomial,
and we write $x_\p = u_\p\p^{n_\p}$ with $u_\p\in \O_\p^\times$.
Put $f := a_\infty\prod_{\p}\p^{n_\p}\in k(t)^\times$.
We have $f^{-1}x_\infty = a_\infty t^{n} + {}$ terms with lower degree in $t$ for some $a_\infty\in k$.
Then $(a_\infty f)^{-1}x\in F_\infty^+\times\prod_{\lambda\ne\infty}\O_\lambda^\times$.
This decomposition of ideles provides the desired isomophism.
}, and thus the first arrow
\[W_F^\ab\stackrel{\prod_{\lambda}\rho_\lambda}{\longrightarrow} \prod_{\lambda\ne\infty}\O_\lambda^\times \times t^{\Z}\times (1 + \m_\infty)\]
is also an isomophism.
Taking profinite completion, we got a decomposition
\[\gal(F^\ab/F)
% \simeq \prod_{\lambda\ne\infty}\O_\lambda^\times \times \hat F_\infty^+
\simeq \prod_{\lambda\ne\infty}\O_\lambda^\times \times t^{\hat\Z}\times (1 + \m_\infty)\]
of $\gal_F^\ab$, corresponding to three disjoint abelian extension of $F$ whose compositum is $F^\ab$.

\subsubsection*{The ``cyclotomic'' extension \texorpdfstring{$K_\infty$}{K_infty}}
For $\lambda\ne\infty$, the representation $\rho_\lambda : \gal_F\to \O_\lambda^\times$ is precisely the Galois representation on $T_\lambda(\phi)$, where $\phi$ is the Carlitz module.
The representation \[\chi := \prod_{\lambda\ne\infty}\rho_\lambda : \gal_F\to \prod_{\lambda\ne\infty}\O_\lambda = \hat A^\times\]
is the inverse limit of $\chi_m : \gal_F\to (A/(m))^\times$, which are from the $\gal_F$-action on $\phi[m]$ for monic $m\in k[t]$, ordered by divisibility.
Hence the field fixed by $\ker(\chi)$ is $K_\infty = \bigcup_{m} F(\phi[m])$.
The extension $K_\infty/F$ is a geometric extension\footnote{A \textbf{geometric extension} is an extension of function fields that doesn't extend the field of constants.}, tamely ramified at $\infty$\footnote{
    The ramification indexes are all $q - 1$; see \cite{Ha74}, \S 3.
}.

\subsubsection*{The extension of constants \texorpdfstring{$\bar{k}(t)$}{bark(t)}}
For each $\sigma\in W_F$, the factor in $t^{\Z}\simeq \Z$ is
$\ord_{t}(\rho_\infty(\sigma))$.
One can show that this number is $\deg(\sigma)$.
%  which equals $-\ord_{\tau^{-1}}(\phi(\rho_\infty(\sigma)))$ by (\ref{eq: def of rank}).
% By \cref{lem: to define infty-adic representation}, $\phi(\rho_\infty(\sigma)) = \sigma(u)\tau^{\deg(\sigma)}u^{-1}$,
% so $-\ord_{\tau^{-1}}(\phi(\rho_\infty(\sigma))) = \deg(\sigma)$.
% so the projection $W_F\to \Z$ is precisely the map $\deg$.
The field fixed by (the closure of) $\ker(\deg)$ is $\bar{k}(t)$,
and the extension $\bar{k}(t)/k(t)$ is the maximal constant field extension.


\subsubsection*{The wildly ramified extension \texorpdfstring{$L_\infty$}{L_\infty}}
By discussion above, the projection onto $1 + \m_\infty$ is
% (the multiplicative inver of)
\[W_F\to 1 + \m_\infty\quad \sigma\mapsto \rho_\infty(\sigma)/\ord_t(\rho_\infty(\sigma)) = \rho_\infty(\sigma)/\deg(\sigma).\]
Let $\beta : \gal_F\to 1 + \m_\infty$ be its profinite completion.
Denote by $L_\infty$ the fixed field of $\ker(\beta)$.
The extension $L_\infty/F$ is a geometric extension, unramified away from $\infty$ and wildly ramified at $\infty$.

To describe this field explicitly, we need to look at the construciton of $\rho_\infty$.
Choose recursively a sequence of elements $\{a_i\}_{i\ge 0}\subset F^\sep$ by\[a_0 := 1;\quad a_i^q - a_i = -ta_{i-1},\ i\ge 1.\]
Then $u := \sum_{i\ge 0}a_i\tau^{-i}$ verifies the condition of $u$ in \cref{lem: to define infty-adic representation}.
For $\sigma\in W_F$, $\rho_\infty(\sigma)\in F_\infty^+$ is characterized by $\phi(\rho_\infty(\sigma)) = \sigma(u)\tau^{\deg(\sigma)} u^{-1}$.
Every $\sigma\in\gal(L_\infty/F)$ has representatives in $W_F$ with $\deg = 0$ since it acts trivially on $\bar{k}$.
Hence $\phi(\beta(\sigma)) = \sigma(u)u^{-1}$,
which shows that\footnote{Recall that $\phi$ is injective.} $\beta(\sigma) = 1$ if and only if $\sigma(u) = u$,
and thus $L_\infty = F\left( \{a_i\}_{i\ge 0} \right)$.

\bibliography{refer}


\end{document}