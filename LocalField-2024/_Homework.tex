\documentclass{article}
\usepackage{amsmath, amssymb, amsthm, amsbsy, mathrsfs, stmaryrd}
\usepackage{enumitem}
\usepackage[colorlinks,
linkcolor=cyan,
anchorcolor=blue,
citecolor=blue,
]{hyperref}
\usepackage[capitalize]{cleveref}
\usepackage[margin = 1in, headheight = 12pt]{geometry}
\usepackage{bbm}
\usepackage{tikz-cd}

\newtheorem{theorem}{Theorem}

\theoremstyle{definition}
\newtheorem{definition}{Definition}
\newtheorem{exercise}{Exercise}[section]
\newtheorem{problem}{Problem}
\newtheorem{example}{Example}
\newtheorem{proposition}{Proposition}[section]
\newtheorem{lemma}{Lemma}[section]
\newtheorem{corollary}{Corollary}[section]

\theoremstyle{remark}
\newtheorem*{remark}{Remark}

\renewcommand{\Re}{\mathop{\mathrm{Re}}}
\renewcommand{\Im}{\mathop{\mathrm{Im}}}

% 新命令
% 数学对象
    \newcommand{\R}{\mathbb{R}}
    \newcommand{\C}{\mathbb{C}}
    \newcommand{\Q}{\mathbb{Q}}
    \newcommand{\Z}{\mathbb{Z}}
    \DeclareMathOperator{\GL}{GL}
    \DeclareMathOperator{\SL}{SL}
    \newcommand{\p}{\mathfrak{p}}
    \renewcommand{\P}{\mathbb{P}}
    \newcommand{\A}{\mathbb{A}}
% 集合
    \newcommand{\sminus}{\smallsetminus} %(集合)差
% 范畴
    \newcommand{\op}[1]{{#1}^{\mathrm{op}}} %反范畴
    \DeclareMathOperator{\enom}{End} %自态射
    \DeclareMathOperator{\isom}{Isom} %同构
    \DeclareMathOperator{\aut}{Aut} %自同构
    \DeclareMathOperator{\im}{im} %像
    \newcommand{\Set}{\mathbf{Set}} %集合范畴
    \newcommand{\Abel}{\mathbf{Ab}} %群范畴
    \newcommand{\Ring}{\mathbf{Ring}}
    \newcommand{\Cring}{\mathbf{CRing}}
    \newcommand{\Alg}{\mathbf{Alg}}
    \newcommand{\Mod}{\mathbf{Mod}}
    \DeclareMathOperator{\Id}{id}
%向量空间, 矩阵
    \DeclareMathOperator{\rank}{rank} %秩
    \DeclareMathOperator{\tr}{Tr} %迹
    \newcommand{\tran}[1]{{#1}^{\mathrm{T}}} %转置
    \newcommand{\ctran}[1]{{#1}^{\dagger}} %共轭转置
    \newcommand{\itran}[1]{{#1}^{-\mathrm{T}}} %逆转置
    \newcommand{\ictran}[1]{{#1}^{-\dagger}} %逆共轭转置
    \DeclareMathOperator{\codim}{codim} %余维数
    \DeclareMathOperator{\diag}{diag} %对角阵
    \newcommand{\norm}[1]{\left\| #1\right\|} %范数
    \DeclareMathOperator{\lspan}{span} %张成
    \DeclareMathOperator{\sym}{\mathfrak{Y}}
% 群
    \DeclareMathOperator{\inn}{Inn} %(群)内自同构
    \newcommand{\nsg}{\vartriangleleft} %正规子群
    \newcommand{\gsn}{\vartriangleright} %正规子群
    \DeclareMathOperator{\ord}{ord} %元素的阶
    \DeclareMathOperator{\stab}{Stab} %稳定化子
    \DeclareMathOperator{\sgn}{sgn} %符号函数
% 环, 域
    \DeclareMathOperator{\cha}{char} %特征
    \DeclareMathOperator{\spec}{Spec} %素谱
    \DeclareMathOperator{\maxspec}{MaxSpec} %极大谱
    \DeclareMathOperator{\gal}{Gal}
% 微积分
    % \newcommand*{\dif}{\mathop{}\!\mathrm{d}} %(外)微分算子
% 流形
    \DeclareMathOperator{\lie}{Lie}
%代数几何
    \DeclareMathOperator{\proj}{Proj}
%多项式
    \DeclareMathOperator{\disc}{disc} %判别式
    \DeclareMathOperator{\res}{res} %结式

% 结构简写
    \newcommand{\pdfrac}[2]{\dfrac{\partial #1}{\partial #2}} %偏微分式
    \newcommand{\isomto}{\stackrel{\sim}{\rightarrow}} %有向同构
    \newcommand{\gene}[1]{\left\langle #1 \right\rangle} %生成对象
% 文字缩写
    \newcommand{\opin}{\;\mathrm{open\;in}\;}
    \newcommand{\st}{\;\mathrm{s.t.}\;}
    \newcommand{\ie}{\;\mathrm{i.e.,}\;}

% 重定义命令
\renewcommand{\hom}{\mathop{Hom}}
\renewcommand{\vec}{\boldsymbol}
\renewcommand{\and}{\;\text{and}\;}

% 编号
\newcommand{\cnum}[1]{$#1^\circ$} %右上角带圆圈的编号
\newcommand{\rmnum}[1]{\romannumeral #1}


\newcommand{\myit}{$\diamond$}

\title{Homework}
\author{Lei Bichang}
\date{Sep 15, 2024}

\begin{document}
\maketitle

\begin{proof}
If $\{x_n\}_n$ is eventually periodic, we may assume that $\{x_n\}_n$ is periodic; that is, $\exists t\ge 0$ s.t. \[x_{n + t} = x_n,\ \forall n\in\Z, n\ge 0.\]
Otherwise we may just subtract the non-periodic part, which is an integer and doesn't affect the rationality of $x$.
Let $a := \sum_{j=0}^{t-1}x_jp^j\in\Z$,
then \begin{align*}
    x = \sum_{i\ge 0}\sum_{j=0}^{t-1} x_{it+j}p^{it}p^j
    = \sum_{i\ge 0}p^{it}\sum_{j=0}^{t-1}x_jp^j 
    = \frac{a}{1-p^t}\in\Q.
\end{align*}

Conversely, suppose that $x = \frac{a}{b}\in\Q$, where $a, b\in\Z$ are coprime and $b\ge 1$.
Because $x\in\Z_p$, we have $p\nmid b$, and thus there is an integer $t\ge 1$ s.t. $b \mid p^t-1$.
Write $c := \frac{1-p^t}{b}$\footnote{I feels that a simpler way to proof that $\{x_n\}$ is eventually periodic by showing that:
(1) $x$ has an eventually periodic expansion implies that \begin{align*}
            -x &= p^k - p^k -x\\
            &= p^k + \sum_{n\ge k}(p-1 -x_n)p^n\\
            &= (p - x_k)p^k + \sum_{n\ge k + 1}(p - 1 - x_n)p^n
        \end{align*} does,
and (2) $x$ has an eventually periodic expansion implies that
\[x + \frac{1}{1-p^t} = x + \sum_{n\ge 0}p^{nt}\] does (I think the period of $x + \frac{1}{1-p^t}$ should divides $\mathrm{lcm} (T, t)$, where $T$ is the period of $x$.). But, I have completed this tedious proof below so didn't write another...},
then \[x = \frac{ac}{1-p^t},\]
and \[ac = x(1-p^t) = \sum_{n\ge 0}x_np^n - \sum_{n\ge 0}x_{n}p^{n+t} = \sum_{i=0}^{t-1}x_ip^i + \sum_{n\ge t}(x_n-x_{n-t})p^n.\quad.\]
It suffices to show that $x_n - x_{n-t} = 0$ for all $n$ large enough.
Note that either \[\sum_{n\ge t}(x_n-x_{n-t})p^n = ac - \sum_{i=1}^{t-1}x_ip^i\] or \[\sum_{n\ge t}(x_{n-t}-x_{n})p^n = \sum_{i=1}^{t-1}x_ip^i - ac\] is a positive integer, and thus have a fintie expansion in base $p$.

% First, we prove the following statement: 
% \begin{itemize}
%     \item [(*)] If $x$ has an eventually periodic expansion in base $p$, so is $-x$.
%     \item \textit{Proof of the statement} (*).
%     Let $x = \sum_{n\ge 0}x_np^n$ be the expansion of $x$.
%     Let $k$ be the smallest integer s.t. $x_k\ne 0$, then \begin{align*}
%         -x &= p^k - p^k -x\\
%         &= p^k + \sum_{n\ge k}(p-1 -x_n)p^n\\
%         &= (p - x_k)p^k + \sum_{n\ge k + 1}(p - 1 - x_n)p^n
%     \end{align*}
%     is the expansion of $-x$, which is periodic if $x$ is.
% \end{itemize}

Consider first the case of \begin{equation}\label{sum x p eq sum y p}
    \sum_{n\ge t}(x_n-x_{n-t})p^n = \sum_{i=0}^r y_ip^i
\end{equation} being postive, where $y_i\in\{0, 1, \dots, p-1\}$.
\begin{enumerate}
    \item [(A)]
    If $r < t$, then \[v_p\left( \sum_{n\ge t}(x_n-x_{n-t})p^n \right)\ge t\]
    and $$v_p\left(\sum_{i=0}^r y_ip^i\right)\le r < t,$$
    if $\sum_{i=0}^r y_ip^i\ne 0$. Therefore \[\sum_{n\ge t}(x_n-x_{n-t})p^n = \sum_{i=0}^r y_ip^i = 0.\]
    Since \[1-p\le x_{n-t}-x_n\le p-1,\quad 
    v_p\left((x_n-x_{n-t})p^n\right) = \begin{cases}
        n, & x_n \ne x_{n-t},\\
        \infty, & x_n = x_{n-t}
    \end{cases}\] for all $n \ge t$,
    we have \[v_p\left( \sum_{n\ge t}(x_n-x_{n-t})p^n \right) = \min \{n\ge t | x_n \ne x_{n-t}\},\] where $\min\varnothing = \infty$, and thus $x_n = x_{n-t}$ for all $n\ge t$.

    \item [(B)]
    If $r \ge t$,
    then \[\sum_{i=0}^{t-1} y_ip^i + \sum_{j=t}^r (y_j - x_j + x_{j-t})p^j + \sum_{n\ge r + 1}(x_{n-t} - x_n)p^n = 0.\]
    Again by computing $p$-adic valuation, we see that $y_0 =  \dots = y_{t-1} = 0$, and
    \[p^t\left(\sum_{j=t}^r (y_j - x_j + x_{j-t})p^{j-t} + \sum_{n\ge r + 1}(x_{n-t} - x_n)p^{n-t}\right) = 0.\]
    Hence,\begin{equation}\label{eq when r ge t}
        \sum_{j=t}^r (y_j - x_j + x_{j-t})p^{j-t} + \sum_{n\ge r + 1}(x_{n-t} - x_n)p^{n-t} = 0.
    \end{equation}

    For simplicity, put \[a_n := \begin{cases}
        y_n - x_n + x_{n-t}, & t\le n\le r,\\
        x_{n-t} - x_n, & n\ge r + 1.
    \end{cases}\]
    We have \[1-p\le y_j-x_j +x_{j-t}\le 2p-2,
    \]for all $t\le j\le r$, and
    \[1-p\le x_{n-t}-x_n\le p-1,
    \] for all $n \ge r+1$.
    
    % By a similar argument on valuation as in (A), it suffices to show that \[\sum_{n\ge N}(x_{n-t} - x_n)p^{n-t} = 0\]for some $N \ge r + 1$.
    % This is true when $a_t = \dots = a_r = 0$,
    % so assume now that $\exists k\in \{t, \dots, r\}$ s.t. $a_k \ne 0$ and $a_j = 0$ whenever $t\le j\le k$.

    % WLOG, we may assume that $k = t$ by dividing $p^{k-t}$ on both sides of \cref{eq when r ge t}.
    From \cref{eq when r ge t}, we see that \[ a_t = -\sum_{n \ge t+1} a_np^{n-t} = -p\sum_{n\ge t +1}a_jp^{n-t-1}\] is divided by $p$ in $\Z_p$.
    Hence $a_t = 0$ or $p$, so \[a_{t + 1} + \sum_{n\ge t + 2}a_np^{n-t-1} = 0\] or \[(a_{t + 1} + 1) + \sum_{n\ge t + 2}a_np^{n-t-1} = 0.\]
    
    This procedure continues. More precisely, define $b_t := a_t$ and \[b_j := \begin{cases}
        a_j,\ b_{j-1} = 0,\\
        a_j + 1,\ b_{j-1} = p,
    \end{cases}\quad j\ge t + 1. \]
    Then we can show inductively that \begin{equation}\label{bj + sum eq 0}
        b_j + \sum_{n\ge j + 1}a_np^{n-j} = 0
    \end{equation} and $b_j\in\{0, p\}$ for all $j\ge t$.

    Consider $j \ge r+1$.
    If $b_j = 0$, then $b_{j + 1} = a_{j+1}\in\{1-p, \dots, p-1\}$, so $b_{j+1} = 0$. Therefore, we get recursively that if $b_N = 0$ for some $N\ge r + 1$, then $a_{j + 1} = b_{j+1} = 0$, i.e., $x_{j - t} = x_j$ for all $j> N$.

    Now suppose that $b_j = p$ for all $j \ge r + 1$. Then \[x_{j-t} - x_j = a_j = b_j - 1 = p -1\] for all $j\ge r  +2$.
    However, $x_{j-t} -x_j = p - 1$ if and only if $x_{j-t} = p-1$ and $x_j = 0$,
    which cannot be true for all $j\ge r  +2$.
    
\end{enumerate}
Therefore, we proved that $\{x_n\}_n$ is eventually periodic given $\sum_{n\ge t}(x_n-x_{n-t})p^n > 0$.
But there is no hard to construct a similar proof in the case of $\sum_{n\ge t}(x_{n-t}-x_{n})p^n > 0$.
\end{proof}

\end{document}