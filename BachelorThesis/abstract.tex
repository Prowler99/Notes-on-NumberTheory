\renewcommand{\abstractname}{\textbf{\Large \heiti 摘要}}
\renewenvironment{abstract}{%
    \par\small
    \noindent\mbox{}\hfill{\bfseries \abstractname}\hfill\mbox{}\par
    \vskip 2.5ex}{\par\vskip 2.5ex} 
%中文摘要及关键词放在扉页一、外文摘要及关键词放在扉二,页码编排为Ⅰ,Ⅱ,设置页眉
\begin{abstract}
%1.5倍行距
\begin{onehalfspace}
设$D$是在素数$p$处分歧的$\Q$上的四元数代数. Cerednik-Drinfeld定理给出了相应于$D(\A_f)^\times$的紧开子群$U$的志村曲线的$p$-进单值化.
此定理最初由Cerednik证明; 而Drinfeld利用他的 ``基本定理'', 即$p$-进半平面$\Omega = \P^1_{\Q_p} - \P^1(\Q_p)$的一个形式模型$\hat{\Omega}$参数化了一族$p$-可除群,
重新导出了Cerednik的原始结果. 
为了证明Drinfeld的定理, 首先需要利用Deligne与Drinfeld的函子给出$\hat{\Omega}$的一个模诠释, 然后利用形式模的Cartier理论构造出Drinfeld定理中所需的同构.
这篇文章将主要跟随\cite{BC91}, 构造$p$-进半平面及其形式模型, 并补充其中部分细节, 随后参照\cite{Zi84}给出Cartier理论的主要内容, 最后陈述Drinfeld定理.
\\[12pt]
\textbf{\textbf{\heiti 关键词:}}$p$-进半平面\quad Cartier理论 \quad Drinfeld定理
\end{onehalfspace}
\end{abstract}
\setcounter{page}{1}
\pagenumbering{Roman}
\cfoot{\footnotesize \thepage}
\newpage


% 外文摘要及关键词:
% Abstract,单倍行距,段前1行,段后1行,16号Times New Roman字,加粗,居中。
% 内容使用12号Times New Roman字,起行空两格,回行顶格,两倍行距。
% Key Words,英文摘要内容下隔一行,左端顶格,12号Times New Roman字,
% 加粗,后面加“:”。
% 关键词内容,直接放在“:”之后,互相之间间隔3格,12号Times New Roman字。

\newcommand{\enabstractname}{\textbf{\Large Abstract}}
\newenvironment{enabstract}{%
    \par\small
    \noindent\mbox{}\hfill{\bfseries \enabstractname}\hfill\mbox{}\par
    \vskip 2.5ex}{\par\vskip 2.5ex}  
\begin{enabstract}
\begin{doublespace}
Let $D$ be a quaternion algebra over $\Q$, ramified at a prime $p$.
The Cerednik-Drinfeld theorem gives the $p$-adic uniformisation of Shimura curves associated to compact open subgroups $U$ of $D(\A_f)^\times$. This theorem was first proved by Cerednik. Then Drinfeld reinterpreted Cerednik's original result in another way using his theorem using his ``fundamental theorem'', which states that a formal model $\hat{\Omega}$ of the $p$-adic half plane $\Omega = \P^1_{\Q_p} - \P^1(\Q_p)$ parameterised a family of $p$-divisible groups.

To prove the Drinfeld's  theorem, the first step is to give a modular discription of $\hat{\Omega}$ through Deligne's and Drinfeld's functor. Then with the help of the Cartier theory on formal modules, one can construct the isomorphism in Drinfeld's theorem.
Following mainly to \cite{BC91}, this article will construct the $p$-adic half plane and its formal model with more details depicted. Then, the main contents of Cartier theory will be given with reference to \cite{Zi84}.
Finally, we state Drinfeld's theorem.
\\[12pt]
\textbf{Keywords:}$p$-adic half plane \quad  Cartier theory  \quad Drinfeld's theorem
\end{doublespace}
\end{enabstract}
\cfoot{\thepage}
\newpage
\cfoot{}
