\section{Group Cohomology}
In this section we fix a commutative ring $\K$.
\subsection{Cohomology}
Let $G$ be a group. A \textbf{$G$-module} with coefficients in $\K$ is a $\K$-module together with a $\K$-linear \textit{left} $G$-action. Hence the category of $G$-modules with coefficients in $\K$ is isomorphic to the category of $\K[G]$-modules.
\begin{remark}
    In particular, a $G$-module with coefficients in $\Z$ is an abelian group with additive left $G$-action. 
\end{remark}

\begin{example}
We list some important constructions of $G$-modules here.
\begin{enumerate}[(a)]
    \item The \textbf{trivial $G$-module} is $\K$ with the trivial $G$-action.
    \item The group ring $\K[G]$ is a $G$-module with $G$ acting by left-multiplication.
    \item Direct sum and product. Both direct sums and products for $G$-modules as $\K$-modules can be lifted to $G$-modules, by giving $G$-action diagonally, i.e, \[g((m_i)_i) := \left( (gm_i)_i \right).\]
    \item Tensor products. For $M, N\in \Mod_G$, define $M\otimes N\in \Mod_G$ to be $M\otimes_\K N$ with the diagonal $G$-action \[g(x\otimes y) := gx\otimes gy,\quad x\in M, y\in N.\]
    \item Hom module. For $M, N\in\Mod_G$, define $\hom(M, N)\in \Mod_G$ to be $\hom_{\K}(M, N)$ with $G$ acting ``by conjugation'': \[(gf)(x) := g f(g^{-1}x),\quad f\in\Hom_\K(M, N), x\in M.\]
    \begin{itemize}
        \item We have \[\Hom_G(M, N) = \Hom(M, N)^G\] as $G$-modules.
        \item The adjoint $L\otimes_\K (-) \dashv \Hom_\K(L, -)$ in $\Mod_\K$ holds in $\Mod_G$, i.e,
\[\begin{tikzcd}
	{\Hom(L\otimes M,N)} & {\Hom(L, \Hom(M, N))} \\
	\varphi & {x\mapsto y\mapsto\varphi(x\otimes y)} \\
	{\left( x\otimes y\mapsto \psi(x)(y) \right)} & \psi
	\arrow["\sim", leftrightarrow, from=1-1, to=1-2]
	\arrow[maps to, from=2-1, to=2-2]
	\arrow[maps to, from=3-2, to=3-1]
\end{tikzcd}\]
    are isomorphisms of $G$-modules.
    \begin{remark}
        The $K$-modules $M\otimes_\K N$ and $\hom_\K(M, N)$ with their $G$-module structures are \textit{NOT} the tensor product or $\Hom$-set in $\K[G]$-module.
    \end{remark}
    \end{itemize}
    \item Induced module. Let $H < G$ be a subgroup, $N$ a $H$-module. Then $\Ind_H^G N$ is the $K$-module of $H$-invariant functions $G\to N$, i.e., \[\Ind_H^G N := \{\varphi : G \to N\mid \varphi(hg) = h\varphi(g),\;\forall h\in H, g\in G\}\simeq\Hom_H(\K[G], N).\]
    The group $G$ acts on $\Ind_H^G N$ from the left by \[(g\varphi)(x) := \varphi(xg).\]
    We obtain a functor $\Ind_H^G : \Mod_H\to\Mod_G$ by sending $\alpha : N\to N'$ to \[\alpha_* : \Ind_H^G N\to\Ind_H^G N' := \varphi\mapsto\alpha\circ\varphi.\]
    \begin{itemize}
        \item $\Ind_H^G$ is \textit{right adjoint to the forgetful functor} $\Mod_G\to\Mod_H$. The isomorphism is given by
        \[\begin{tikzcd}
            {\Hom_G\left(M, \Ind_H^GN\right)} & {\Hom_H(M, N)} \\
            \alpha & {x\mapsto \alpha(x)(1_G)} \\
            {\left[ x\mapsto (g\mapsto \beta(gx) \right]} & \beta
            \arrow["\sim", leftrightarrow, from=1-1, to=1-2]
            \arrow[maps to, from=2-1, to=2-2]
            \arrow[maps to, from=3-2, to=3-1]
        \end{tikzcd}\]
        where $M\in \Mod_G$, $N\in \Mod_H$.
        \item $\Ind_H^G$ is an exact funtor.
        \item For any $\K$-module $M$, we define \[\Ind^G M := \Ind_{\{1\}}^G M =\{\varphi : G\to M\}.\] An \textbf{induced module} is a $G$-module of the form $\Ind^GM$ for some $\K$-module $M$.
        \item Let $M$ be a $G$-module. Define $M_* := \Ind^G M$, then we have an embedding \[M\hookrightarrow M_* := x \mapsto [g\mapsto gx]\]of $G$-modules. The exact sequence \begin{equation}
            0\to M\to M_*\to M_\dagger \to 0
        \end{equation} in $\Mod_G$, where $M_\dagger := M_*/M$, will be used many times in ``dimensional shifting''.
    \end{itemize}
\end{enumerate}
\end{example}

Let $M$ be a $G$-module, $r\ge 0$ a natural number.
We define the \textbf{ $r$-th cohomology groups of $G$ with coefficients in $M$} to be the value of the $r$-th right derived functor of the left-exact functor \[(-)^G\simeq \Hom_G(\K, -) : \Mod_G\to\Mod_K\] at $M$.
But for this definition to make sense, we need to show that:
\begin{lemma}\label{Mod G has enough injectives}
    The category $\Mod_G$ has enough injectives.
\end{lemma}
\begin{proof}
    The category $\Abel$ has enough injectives. Let $M \in\Mod_G$, $I\in\Abel$ injective with $M\hookrightarrow I$. Applying the exact functor $\Ind^G$ gives \[M\hookrightarrow M_* := \Ind^G M \hookrightarrow\Ind^G I. \] So it remains to show that
    \begin{itemize}
        \item the functor $\Ind^G$ preserves injectives,
    \end{itemize}
    which follows from $\Hom_G(-, \Ind^G I)\simeq\Hom_\Z(-, I)$.
\end{proof}


\begin{proposition}[Shapiro's lemma]\label{Shapiro's lemma}
Let $H < G$ be a subgroup.
The isomorphism\[(-)^H \simeq \Hom_H(\Z, -) \simeq \Hom_G\left(\Z, \Ind^G_H(-)\right)\simeq \left( \Ind_H^G(-) \right)^G\]induces a canonical isomorphism \[H^\bullet \left(G, \Ind^G_H (-)\right)\simeq H^\bullet (H, -),\]
which is compatible with the long exact sequence.
\end{proposition}
\begin{proof}
    
\end{proof}

\begin{corollary}\label{induced modules have trivial cohomology}
    If $M$ is an induced $G$-module, then $H^r(G, M) = 0$ for all $r\ge 1$.\qed
\end{corollary}


\subsection{Compute Cohomology via cochains}

Homological algebra tells us that \[H^r(G, M) = R^r\Hom_G(\Z, -)(M) = \ext^r(\Z, M) = R^r\Hom_G(-, M)(\Z),\]
so we can use the projective resolution of $\Z\in\Mod_G$ to compute $H^\bullet(G, M)$.

Denote by $P_r$ the free $\Z$-module with basis $G^{r+1} = G\times\dots\times G$ and endow $P_r$ with the $G$-action
\[g(g_0, g_1,\dots, g_r) := (gg_0, gg_1, \dots, gg_r).\]
Define $d_r : P_r\to P_{r-1}$ by \[d_r(g_0, \dots, g_r) := \sum_{i=0}^r(-1)^i(g_0,\dots, \hat{g}_i, \dots, g_r).\]
Then \[\dots\to P_1\stackrel{d_1}{\to} P_0\stackrel{d_0}{\to} \Z\] is exact, i.e., a projective resolution of $\Z$.

Note that $\varphi\in\Hom_G(P_r, M)$ is equivalent to a function $\varphi : G^{r+1}\to M$ s.t. \[\varphi(gg_0, \dots, gg_r) = g\varphi(g_0, \dots, g_r),\]
which is thus determined by its value on the set $\{(1, g_1, \dots, g_r) : g_i\in G\}$. Therefore we consider the abelian group\footnote{The group structure on $C^r(G, M)$ is point-wise addition.} $C^r(G^r, M) := \{\varphi : G \to M\}$. Note that $G^0 = 1$ and thus $C^0(G, M) = M$.
Define a homomorphism \[d^r : C^r(G, M) \to C^{r+1}(G, M)\] by $(d^r\varphi)(g_1, \dots, g_{r+1})$\begin{align}
    := g_1\varphi(g_2, \dots, g_{r+1}) + \sum_{j=1}^r(-1)^j\varphi(g_1, \dots, \hat{g}_j, \dots, g_r) + (-1)^{r+1}\varphi(g_1, \dots, g_r).
\end{align}
Let \[Z^r(G, M) := \ker d^r,\ B^r(G, M) := \im d^{r-1}.\]
One can prove that $d^r\circ d^{r-1} = 0$, and \[H^r(G, M) = Z^r(G, M)/B^r(G, M).\]


\subsection{The Inflation-Restriction Exact Sequence}



\subsection{Homology}

For $M\in \Mod_G$, define its \textbf{coinvariant} to be the quotient
\[M_G := M\big/ \gene{gm - m\mid g\in G, m\in M} = M/(G - \Id)M\in \Abel.\]

\begin{lemma}
    The assignment $M\mapsto M_G$ defines a right-exact functor \[(-)_G\simeq \Z\otimes_{\Z[G]}(-) : \Mod_{\Z[G]}\to\Abel\]
\end{lemma}
\begin{proof}
Consider the augmentation map $\Z[G]\to\Z$ which is an additive homomorphism sending all $g\in G$ to $1\in\Z$.
Its kernel $I_G$ is called the \textbf{augmentation ideal}.
Note that:\begin{itemize}
    \item $I_G\subset\Z[G]$ is the free abelian subgroup with basis $\{g - 1 \mid g\in G, g\ne 1\}$.
\end{itemize}
Therefore \[M_G = M/I_GM\simeq \Z[G]/I_G\otimes_{\Z[G]} M\simeq \Z\otimes_{\Z[G]} M.\qedhere\]
\end{proof}
We define the \textbf{$r$-th homology groups of $G$ with coefficients in $M\in\Mod_G$} to be the value of the $r$-th left derived functor of the right-exact functor $(-)_G$.

\subsection{The Tate cohomology groups}

In this subsection, let $G$ be a \textit{finite} group.

Recall that the norm $N_G : M \to M$ for a $G$-module $M$ is defined by \[N_G(x) := \sum_{g\in G}gx,\quad x\in G.\]
One observes that \[\im N_G\subset M^G,\quad I_GM\subset \ker N_G.\]
Therefore $N_G$ factors as \[M\twoheadrightarrow M/I_GM = M_G\to M^G\hookrightarrow M,\] and we got an exact sequence \[0\to\ker N_G/I_GM\to M_G\to M^G\to M^G/\im N_G\to 0.\]

The map $H_0(G, M)\to H^0(G, M)$ induced by the norm map on $M$ connects homologies and cohomologies. We define the \textbf{Tate cohomology groups} by \[\hat H^r (G, M) := \begin{cases}
    H^r(G, M), &r\ge 1,\\
    M^G/N_G(M), &r = 0,\\
    \ker (N_G : M\to M)/I_GM, & r = -1,\\
    H_{-r-1} (G, M), & r\le -2.
\end{cases}\]

\begin{proposition}
    If $M$ is induced, then $\hat H^\bullet(G, -) = 0$.
\end{proposition}

(connecting $H^r$ to $H^{r + 2}$.)

\begin{example}\label{eg: H1 to H-1 for finite cyclic group}
    Let $G$ be a finite cyclic group generated by $\sigma$.
    Then \[I_G = \gene{\sigma^n m - m\mid m\in M, n\in\Z } = \gene{\sigma m - m\mid m\in M},\]
    \[\hat H^{-1}(G, M) = \ker(N_G)/(\sigma - 1)M.\]
    In this case, choosing a generator $\sigma$ of $G$ defines an explicit isomorphism
    \begin{align*}
        \hat H^1(G, M) &\to \hat H^{-1}(G, M)\\
        \varphi&\mapsto \varphi(\sigma).
    \end{align*}
    Indeed, crossed homomorphisms $G\to M$ are defined by their value on generators of $G$, and for $\varphi : G\to M$ a crossed homomorphism, \[
    \varphi(\sigma^n) = \sigma^{n-1}\varphi(\sigma) + \sigma^{n-2}\varphi(\sigma) + \cdots + \sigma\varphi(\sigma) + \varphi(\sigma),\ \forall \sigma\in G.\]
    Therefore, if $G\simeq \Z/n\Z$ is generated by $\sigma$ of order $n$,
    then \[\varphi\text{ is a crossed homomorphism }\iff x := \varphi(\sigma) \text{ verifies } N_Gx = \sum_{g\in G} gx = x + \sigma x + \cdots + \sigma^{n-1}x =0.\]
    \[\varphi \text{ is principal }\iff \varphi(\sigma) \in \left( \sigma - 1 \right)M.\]
    As $Z^1(G, M)\to M,\ \varphi\to \varphi(\sigma)$ is a group homomorphism, we get the isomorphism.
\end{example}


\subsection{Cohomology of \texorpdfstring{$L$ and $L^\times$}{PDFstring}}
In this subsection, let $L/K$ be a \textit{finite} Galois extension, $G := \gal(L/K)$. Then both $L$ and $L^\times$ have natural $G$-module structures.

\subsubsection{Hilbert's Theorem 90 and \texorpdfstring{$H^1(G, L^\times)$}{H1(G, L cross)}}

\begin{theorem}[Dedekind-Artin]\label{dedekind theorem on linearly independence of characters}
    Let $\Gamma$ be a monoid, $E$ be a integral domain, and $\Hom_{\times}(\Gamma, E)$ the set of monoid homomorphisms $\Gamma\to E$.
    \footnote{
        The set $\Hom_{\times}(\Gamma, E)$ admits a $E$-module structure defined point-wisely.
        The elements in $\Hom_{\times}(\Gamma, E)$ are sometimes called characters.
    }
    Then $\Hom_{\times}(\Gamma, E)$ is a linearly independent set over $E$; i.e, for $a_\chi\in E$,
    \[\sum_{\chi\in\Hom_{\times}(\Gamma, E)} a_\chi\chi(\cdot) = 0\text{ on }E
    \implies a_\chi = 0,\forall\chi.\]
\end{theorem}
\begin{proof}
    Suppose that $J := \{\chi\in\Hom_{\times}(\Gamma, E)\mid a_\chi\ne 0\}\ne\varnothing$.
    The idea is to {\color{blue} take $(a_\chi)_\chi$ s.t.
    $J = J((a_\chi)_\chi)$ is nonempty but minimal}.

    Since $\chi(1) = 1\ne 0\in E$, we have $\# J > 1$.
    Let $\xi, \eta$ be two different characters $\Gamma\to E$. Then $\exists g\in\Gamma$ s.t. $\xi(g)\ne \eta(g)$.
    Note that \[\sum_{\chi\in J} a_\chi \chi(g)\chi(\cdot) = \sum_{\chi\in J} a_\chi\chi(g\,\cdot) = 0,\]
    \[\sum_{\chi\in J}a_\chi\xi(g)\chi(\cdot) = \xi(g)\sum_{\chi\in J}a_\chi\chi(\cdot) = 0,\]
    and subtracting these two identities yields
    \[\sum_{\chi\in J\sminus\{\xi\}} a_\chi(\chi(g) - \xi(g))\chi(\cdot) = 0.\]
    This new identity is nontrivial sicne $\eta(g) - \chi(g)\ne 0$, but concerns strictly lesser characters than $J$. Contradiction.
\end{proof}

\begin{proposition}\label{Hilbert 90 - multiplicative - cohomology}
    $H^1(\gal(L/K), L^\times) = 0$.\par
    In other words, if $\varphi : G\to L^\times$ is a crossed homomorphism, i.e., \[\varphi(gh) = g\varphi(h)\varphi(g),\ \forall g, h\in G,\]
    then $\exists b_\varphi\in L^\times$ s.t. \[\varphi(g) = \frac{g b_\varphi}{b_\varphi},\ \forall g\in G.\]
\end{proposition}
\begin{proof}
    Take $a\in L^\times$ and define \[b := \sum_{g\in G}\varphi(g)\cdot ga\in L.\]
    Then \begin{align*}
        hb = \sum_{g\in G} h\varphi(g)\cdot hga
        = \sum_{g\in G}\frac{\varphi(hg)}{\varphi(h)} hga = \frac{b}{\varphi(h)}.
    \end{align*}
    Hence if $b\ne 0$, we would have $\varphi(g) = b/gb = g(b^{-1})/b^{-1}$.
    By \cref{dedekind theorem on linearly independence of characters},
    $\gal(L/K)\subset \Hom_\times(L, L)$ is linearly independent over $L$,
    so $\sum_{g\in G}\varphi(g)g(\cdot) : L\to L$ is a non-zero function, and thus can we find $a\in L$ with $b\ne 0$.
\end{proof}

\begin{corollary}[Hilbert 90]\label{Hilbert 90 - multiplicative}
    Let $L/K$ be a finite cyclic extension, $\sigma$ a generator of $G = \gal(L/K)$, and $a\in L$.
    If $N_{L/K}a = 1$, then $\exists b\in L^\times$ s.t. $a = \sigma b/b$.
\end{corollary}
\begin{proof}
    For the $G$-module $L^\times$, the norm map \[N_G = N_{L/K} : x\mapsto\prod_{g\in G}  gx.\]
    So \[\dfrac{\ker(N_{L/K})}{(\sigma(\cdot) / \Id(\cdot))L^\times} = \hat H^{-1}(G, L^\times) \simeq H^1(G, L^\times) = 0.\qedhere\]
\end{proof}

\subsubsection{Normal Basis and \texorpdfstring{$H^r(G, L)$}{Hr(G, L)}}

\begin{theorem}[Normal basis theorem]\label{normal basis theorem for fintie galois}
    Any finite Galois extension $L/K$ admits a normal basis; i.e, $\exists x\in L$ s.t. $\{\sigma x\mid \sigma\in\gal(L/K)\}$ forms a $K$-basis of $L$.
\end{theorem}
We prove this in two cases: 1) $K$ is infinite
and 2) $L/K$ is finite cyclic.
\begin{proof}[Proof in case $K$ infinite]
    (T.B.C.)
\end{proof}

\begin{proof}[Proof in case $G$ cyclic]
    (T.B.C.)
\end{proof}

\begin{proposition}
    $L$ is an induced $G = \gal(L/K)$-module, hence
    $H^r(G, L) = 0$ for all $r \ge 1$.
\end{proposition}
\begin{proof}
    By \cref{normal basis theorem for fintie galois},
    we choose $x\in L$ with $L = \bigoplus_{g\in G} Kgx$, giving an isomorphism \[K[G]\to L,\quad \sum_{g\in G}a_gg\to\sum_{g\in G}a_ggx\]
    as $G$-modules. Hence as a $G$-module,
    $L\simeq K[G]\simeq K\otimes_\Z \Z[G]\simeq\Ind^G(K)$.
\end{proof}

\begin{corollary}\label{Hilbert 90 - additive}
    Let $L/K$ be a finite cyclic extension, $\sigma$ a generator of $G$, and $a\in L$.
    If $\tr_{L/K}a = 0$,
    then $\exists b\in L$ s.t. $a = \sigma b - b$.
\end{corollary}
\begin{proof}
    For the $G$-module $L$, the norm map \[N_G = \tr_{L/K} : x\mapsto \sum_{g\in G}gx.\]
    Now use $H^{1}(G, L) \simeq \hat H^{-1}(G, L)$.
\end{proof}

\subsubsection{Kummer Theory}



