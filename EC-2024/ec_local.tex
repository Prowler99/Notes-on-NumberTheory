\section{Elliptic Curves over Local Fields}
In this section, let $K$ be a perfect local field with valuation $v$
and value group $\Z$. Let $R = \O_K$, $\pi$ a uniformizer,
and $k = R/\pi$ the residue field.
Denote the quotient map $\O_K\to k$ by $t\mapsto \tilde{t}$.
\subsection{Minimal Weierstrass Equation}

\begin{definition}
    Let $A$ be a ring, \[E : 
    y^2 + a_1xy + a_3y = x^3 + a_2x^2 + a_4x + a_6,\quad a_i\in A\]
    an Weierstrass equation over $A$.
    We define many quantities in $\Z[a_1, a_2, a_4, a_3, a_4, a_6]$:
    \[b_2,\ b_4,\ b_6,\ b_8,\ \Delta,\ c_4,\ c_6.\]
    Among them, $\Delta$ is the most important one, called \textbf{discriminant}.
\end{definition}
\begin{remark}
\begin{itemize}
\item Under the change of variable \[y = \frac{1}{2}(y' - a_1x - a_3),\]
the long Weierstrass equation is converted to\footnote{
    Careful!! I am not very clear about the notion of ``change of variable''.}
\[
y'^2 = 4x^3 + b_2x + 2b_4x + b_6.\]
\item Under the change of variable \[(u; r, s, t) := \quad
\begin{cases}
    x = u^2x' + r,\\ y = u^3y' + u^2sx' + t,
\end{cases}\]
the discriminant is converted to \[\Delta' = u^{-12}\Delta.\]
\item For a short Weierstrass equation\[
y^2 = x^3 + ax + b,\]
we have \begin{align*}
    \Delta &= -16(4a^3 + 27b^2),\\
    c_4 &= -48a,\\
    c_6 &= -864b.
\end{align*}
\end{itemize}

\begin{proposition}
    If $E$ is defined over a field, then $E$ is nonsingular $\iff \Delta \ne 0$.
\end{proposition}


\end{remark}
Back to our setting that $K = $ local field.
We need some particular integral equation that modulo $\pi$ gives something nontrvial.
\begin{definition}
    A Weierstrass equation $E$ over $K$ is called \textbf{minimal}, if it is integral (i.e, all coefficients $a_i\in \O_K$), and the valuation of the discriminant $v(\Delta)$ is minimized.
\end{definition}
\begin{lemma}
    Every elliptic curve $E/K$ admits some minimal Weierstrass equation.\par
    Moreover, an integral Weierstrass equation with $v(\Delta) < 12$ is minimal.
\end{lemma}
\begin{proof}
    First, we need to find Weierstrass over $\O_K$.
    Starting from any equation over $K$,
    the change of variable $(u; 0, 0, 0)$ sends $a_i$ to \[a_i' = u^ia_i.\]
    Hence for $v(u)\gg 0$, it will give an integral equation.

    For integral Weierstrass equations, $\Delta\in \Z[\{a_i\}_i]\subset \O_K$, so it can be minimized.

    Since $v(\Delta)\equiv v(\Delta')\bmod 12$
    for any $(u; r, s, t)$, an integral equation with $v(\Delta)  <12$ must be minimal.
\end{proof}

\begin{remark}
    \begin{itemize}
        \item Minimal Weierstrass equations are not unique.
        \item For a similar reason, \[v(c_4) < 4\text{ or } v(c_6) < 6\]
        are also sufficient for being minimal.
        \item If $\cha K \ne 2, 3$,
        then \[\text{minimal }\iff v(\Delta) < 12\text{ or } v(c_4) < 4.\]
    \end{itemize}
\end{remark}

\subsection{Reduction and Reduction Type}
Let \[E : 
    y^2 + a_1xy + a_3y = x^3 + a_2x^2 + a_4x + a_6,\quad a_i\in \O_K\]
be a Weierstrass equation for an elliptic curve over $K$.
Then it reduces modulo $\pi$ to a projective curve $\tilde{E}$, defined also by Weierstrass equation,
over $k$.
\begin{itemize}
    \item The discriminant of $\tilde{E}$ is $\tilde{\Delta} = \Delta\mod\pi$.
    In particular,\[\tilde{E}\text{ is smooth }\iff \Delta\in\O_K^\times.\]
\end{itemize}

\begin{proposition}
    Let $E$ be an elliptic curve over $K$.
    Then the reductions $\tilde{E}/k$ are isomorphic for all change of variable between minimal Weierstrass equations.
\end{proposition}
\begin{proof}
    Suppose that $(u; r, s, t)$ changes a minimal equation to another, then $u\in\O_K^\times$, so it is an isomorphism over $K$.
    It is left (and necessary!) to prove that we can take $r, s, t\in \O_K$.
\end{proof}

Then we classify different kind of reductions.
We say that $E$ has \textbf{good reduction} if $\tilde{E}$ is an elliptic curve, namely $\tilde{E}$ is smooth or $\Delta\in\O_K^\times$; other wise it has \textbf{bad reduction}.

Assume that $E$ has bad reduction.
Then $\tilde{E}$ has a \textit{unique} singular point.
(T.B.C.)


\subsection{Reduction of Rational Poitns}
Let $E$ be an elliptic curve over $K$.
For all $P \in E(K)$,
we can write $P = [x : y : z]$ with \begin{itemize}
    \item $x, y, z\in\O_K$, and
    \item at least one of the coordinates is in $\O_K^\times$.
\end{itemize}
$\leadsto \tilde{P} := [\tilde{x}, \tilde{y}, \tilde{z}]\in \tilde{E}(k)$.
This is independent to the choice of coordinate,
giving a map \[\pi : E(K)\to\tilde{E}(k).\]

\begin{definition}
    Denote by $\tilde{E}_\ns$ the set of non singular points of $\tilde{E}$, and $\tilde{E}_\ns(k) = \tilde{E}_\ns\cap\tilde{E}(k)$.
    Let
    \begin{align*}
        E_0(K) &:= \{P\in E(K)\mid \pi(P)\in \tilde{E}_{\ns}(k)\},\\ 
        E_1(K) &:= \{P\in E(K)\mid \pi(P) = \tilde{O}\}
    \end{align*}
\end{definition}

\begin{proposition}\label{reduction is a group homomorphism on E0}
    We have an exact sequence of \textit{groups}:
    \[0\to E_1(K)\to E_0(K)\stackrel{\pi}{\to} \tilde{E}_\ns(k)\to 0.\]
\end{proposition}
\begin{proof}[Proof (when $E$ has good reduction)]
    In case $\tilde{E}$ smooth, $\tilde{E}_\ns = \tilde{E}$ is an elliptic curve, and $E_0(K) = E(K)$.
    \begin{itemize}
\item $\pi$ is a homomorphism := by \par
Note that $\pi$ sends a line to a line. A real proof needs to take care of the tangent lines.
\item $\pi$ is surjective := by \par
Hensel's lemma. Choose a minimal (integral) equation $E : F(x, y) = 0$.
Let $(\tilde{x}, \tilde{y})\in \tilde{E}_\ns(k)$.
Smoothness means
\[\pdfrac{\tilde{F}}{X}(\tilde{x}, \tilde{y})\ne 0\text{
or }\pdfrac{\tilde{F}}{Y}(\tilde{x}, \tilde{y})\ne 0.\]
Hence, say, $\pdfrac{F}{X}(x, y) \in \O_K^\times$.
By Hensel's lemma, there is a unique $x'\in\O_K$ s.t. $F(x', y) = 0$.\qedhere
    \end{itemize}
\end{proof}
\begin{remark}
    In case $\tilde{E}$ singular, \cref{reduction is a group homomorphism on E0} still holds. In particular,
    \begin{itemize}
\item $\tilde{E}_\ns$ is a group, with $\tilde{E}_\ns(k)$ being a subgroup.
    \end{itemize}
\end{remark}

\begin{proposition}
    $E_0(K)$ has finite index in $E(K)$.
\end{proposition}


\subsection{Torsion points}
Let $E$ be an elliptic curve over the local field $K/\Q_p < \infty$ defined by a Weierstrass equation.

The first step is to look at the group \[E_1(K) = \ker \left( E_0(K)\to\tilde{E}_\ns(k) \right).\]
A point $P = (x, y) = [x : y : 1]\in E(K)$ reduces to $\tilde{P} = \tilde{O}\in \tilde{E}(k)$,
iff \[v(y) < \min\{0 = v(1), v(x)\}.\]
If $v(x)\ge 0$, then $v(y^2 + a_1xy + a_3y)\ge 0 $;
hence $v(x) < 0$.
Meanwhile,\[v(x^3 + a_2x^2 + a_4x + a_6) =v(y^2 + a_1xy + a_3y).\]
As $v(x), v(y) < 0$, we have
\[3v(x) = 2v(y).\]
So \[v(x) = -2n,\ v(y) = -3n,\quad n\ge 1.\]
Conversely, if $(x, y)\in E(K)$ and $v(x) < 0$,
the above computation holds.
So \[E_1(K) = \{(x, y)\in E(K)\mid v(x)\le -2\}.\]
This leads to the following filtration on $E_1(K)$.
\begin{definition}
    For $n\in\Z_{\ge 1}$,
    set \[E_n(K) := \{(x, y)\in E(K)\mid v(x)\le -2n, v(y)\le -3n\}\cup\{O\}.\]
\end{definition}

\begin{lemma}
    $E_n(K)$ are subgroups of $E_1(K)$.
\end{lemma}
\begin{proof}
    Consider the change of variable \[\begin{cases}
        x = \pi^{-2n}x', \\ y = \pi^{-3n}y'
    \end{cases}\] on $E$.
    This gives a non-minimal integral equation,
    which reduces mod $\pi$ to \[\tilde{E}_n : y'^2 = x'^3\] and gives a map \[\pi_n : E(K)\to \tilde{E}_n(k).\]
    One verifies that:
\begin{itemize}
    \item $(0, 0)$ is the only singular point of $\tilde{E}_n$, and $\tilde{E}_{n, \ns}(k)\simeq k$ as additive groups.
    \item $E_n(K) = \pi_n^{-1}(\tilde{E}_{n, \ns}(k))$.
    \item $E_{n+1}(K) = \ker\left( \pi_n : E(K)\twoheadrightarrow \tilde{E}_{n, \ns}(k)\simeq k \right)$.\qedhere
\end{itemize}
\end{proof}

\begin{proposition}
$E_1(K)[m] = 0$ for $m\in\Z$ prime to $p$.
\end{proposition}
\begin{proof}
    Let $P\in E_1(K)[m]$.
    Then \[ 0 = \pi_1(mP) = m(\pi_1(P))\in k.\]
    As $p\nmid m$, this implies $\pi_1(P) = 0$,
    i.e, $P\in  E_2(P)$.
    Inductively, we see that $P\in E_n(P)$ for all $n$, so $P = O$.
\end{proof}
    
\begin{corollary}
    If $E/K$ has good reduction, and $p\nmid m$,
    then \[E(K)[m]\hookrightarrow \tilde{E}(k)\] injectively.
\end{corollary}

\subsubsection*{Bounding the denominators of torsion points}

\begin{theorem}
    Let $P\in E(K)$ be a torsion point of order $m\ge 2$.
    \begin{enumerate}
\item [(1)] If $m$ is not a power of $p$,
then $x(P), y(P)\in\O_K$.
\item [(2)] If $m = p^n$,
then \[\pi^{2r}x(P), \pi^{3r}y(P)\in\O_K,\quad r = \left\lfloor  \frac{v(p)}{p^n - p^{n-1}}\right\rfloor\ge 0.\]
    \end{enumerate}
\end{theorem}

\begin{theorem}
    Let\[E : y^2 = x^3 + ax + b,\quad a, b\in\Z,\ \Delta := 4a^3 + 27b^2\ne 0.\]
    If $P = (x, y)\in E(\Q)$ is a torsion point,
    then $x, y\in\Z$, and either $y = 0$ or $y\mid \Delta$\footnote{Note that this $\Delta$ is the discriminant of a polynomial, rather than discriminant of a elliptic curve.}.
\end{theorem}

