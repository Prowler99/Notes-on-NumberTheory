\section{Schemes}
In this section, all rings are assumed to be commutative.

\subsection{Some Properties of Schemes}

\subsubsection{Reduced Schemes}

A ring $A$ is reduced if its nilradical
$\nil(A) = 0$.

A scheme $X$ is said to be \textbf{reduced at} $x\in X$,
if the local ring $\O_{X, x}$ is reduced.
We say that $X$ is \textbf{reduced} if it is reduced at every point.

\begin{proposition}\label{reduced is Zariski local}
    Let $X$ be a scheme. Then TFAE:
    \begin{enumerate}
    \item [(a)] $X$ is reduced.
    \item [(b)] $\forall U\subset X$ open, $\O_X(U)$ is reduced.
    \item [(c)] $\exists$ affine cover $\{U_i\}$ of $X$, s.t. all $\O_X(U_i)$ are reduced.
    \end{enumerate}
\end{proposition}

To prove this, we consider an ideal sheaf $\mathcal{N}_X$ on $X$ defined by
\[\mathcal{N}_X(U) := \left\{s\in \O_X(U)\mid s_x\in\nil(\O_{X, x})\right\},\]
so that $X$ is reduced $\iff\mathcal{N}_X = 0$


\begin{proposition}\label{reduced subscheme structure on closed subset}
    Let $Z$ be a closed subset of the scheme $X$.
    Then there is a unique structure of reduced scheme on $Z$, making it a closed subscheme of $X$.

    In particular, there is a unique closed subscheme $i : X_{\red}\hookrightarrow X$
    with the same topological space as $X$.
    More precisely, $X_{\red} = V(\mathcal{N}_X) = (X, \O_X/\mathcal{N}_X)$.
    This subscheme $X_{\red}$ verifies the following property:
    \begin{itemize}
        \item If $Y$ is a reduced scheme,
        then any morphism $Y\to X$ factors through $Y\to X_{\red}\to X$.
    \end{itemize}
\end{proposition}

\begin{proof}[Proof of \cref{reduced is Zariski local} and \cref{reduced subscheme structure on closed subset}]
    Let $U\subset X$ be an affine open.
    Then since taking radical commutes with localization,
    we have $\mathcal{N}_X(U) = \nil(\O_X(U))$.
    This proves \cref{reduced is Zariski local}.
    We also see that $|V(\mathcal{N}_X)| = |X|$.

    Next, we verify that the closed ringed subspace $X_{\red} := (X, \O_X/\mathcal{N}_X)$
    of $X$ is a scheme (and, of course, reduced); that is, it is locally affine. So assume $X = \spec A$.
    Let $N = \nil(A)$ and $i : \spec A/N\to \spec A$ the closed embedding.
    For any $g\in A$,
    we know that the kernel of
    \[A_g = \O_{\spec A}(D(g))\to \O_{\spec A/N}(D(\bar g)) = (A/N)_{\bar g}\]
    is $N\otimes_A A_g$,
    which $ = \nil(A_g) = \mathcal{N}_X(D(g))$;
    i.e, $\ker i^\# = \mathcal{N}_X$.
    Hence $X_{\red} \simeq \spec A/N$ is a scheme.

    Now let $Z$ be a closed subset of $X$.
    We first show that the reduced closed subscheme $(Z, \O_Z)$ is unique if it exists.
    For an affine open $U\subset X$,
    $(Z\cap U, \O_Z|_{Z\cap U})$ is a closed subscheme of $U$,
    so it is isomorphic to $\spec \O_X(U)/I$
    for some ideal $I\subset \O_X(U)$.
    As a subset of $U = \spec \O_X(U)$,
    $Z\cap U = V(I)$.
    Since $Z$ is reduced, $\nil(\O_X(U)/I) = 0$,
    namely $I$ is radical.
    If there is another reduce closed subscheme structure of $X$ on $Z$,
    it must also be given by a radical ideal $J\subset\O_X(U)$ on $U$,
    and $V(I) = V(J)$. But for radical ideals, this means $I = J$.

    To make $Z$ a reduced closed subscheme,
    take affine cover $\{U_i = \spec A_i\}_i$ of $X$
    and ideals $I_i\subset A_i$ s.t. $Z\cap U_i = V(I_i)$.
    We endow $Z\cap U_i$ with the reduced scheme structure of $\spec A_i/\sqrt{I_i}$ and use uniqueness to glue them.
\end{proof}

\subsection{Irreducible Components}
A topological space $X$ is \textbf{irreducible}
if it is \textit{not} the union of proper closed subsets.
\begin{itemize}
    \item The closure of a irreducible subset is irreducible.
    \item The set of irreducible subsets in a space $X$ admits maximal w.r.t.\! inclusions,
    called \textbf{irreducible components}.
\end{itemize}

\subsection{Integral Schemes}
A scheme $X$ is said to be \textbf{integral at} $x\in X$, if the local ring $\O_{X, x}$ is an integral domain.

\subsection{Normal Schemes}
An integral domain $A$ is normal if it is integrally closed in its fraction field.

\subsection{Regular Schemes}



A scheme $X$ is said to be \textbf{normal at}
$x\in X$, if the local ring $\O_{X, x}$ is (an integral domain and) normal.

\section{Some Properties of Morphisms}

\subsection{Seperated Morphisms}

\subsection{Quasi-Compact Morphisms}
A morphism $f: X\to Y$ of schemes is said to be
\textbf{quasi-compact}, if the inverse image $f^{-1}(V)$ of any affine open $V\subset Y$ is quasi-compact.
\begin{proposition}
    Let $f : X\to Y$ be a morphism of schemes.
    \begin{enumerate}
    \item [(1)] If $f$ is a closed immersion, then $f$ is quasi-compact.
    \item [(2)] If $f$ is an open immersion, and $Y$ is locally Noetherian, then $f$ is quasi-compact.
    \end{enumerate}
\end{proposition}


\subsection{Morphisms of Finite Type}
A ring homomorphism $A\to B$ is of finite type if it make $B$ a finitely generated $A$-algebra. 

A morphism $f: X\to Y$ of schemes is said to be
\textbf{of finite type}, if
\begin{itemize}
    \item $f$ is quasi-compact, and
    \item For any affine open $V\subset Y$ and affine open $U\subset f^{-1}(V)$, the map $\O_Y(V)\to\O_X(f^{-1}V)\to\O_X(U)$ is of finite type.
\end{itemize}

\subsection{Proper Morphisms}

\subsection{Flat Morphisms}
A morphism $f : X\to Y$ of schemes is said to be
\textbf{flat at} $x\in X$, if $f^\#_x : \O_{Y, f(x)}\to \O_{X, x}$ is flat.
We say that $f$ is \textbf{flat} if it is flat at every point.

\subsection{Unramified Morphisms}

\subsection{\'Etale Morphism}

\subsection{Smooth Morphisms}

