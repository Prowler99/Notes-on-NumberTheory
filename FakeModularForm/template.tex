\documentclass[11pt,english]{smfart}
%%In the above, write english or french according to the language you'll be using (this affects, for instance, the name of statements: the environment theo produces Theorem or Théorème according to what you have chosen here
\usepackage[T1]{fontenc}

%\usepackage{bull} Activate this package by removing the % sign, if you want to submit to the Bulletin de la SMF

%\usepackage{rhm} Activate this package by removing the % sign, if you want to submit to the Revue d'Histoire de Mathématique. This comes with a bibliography style too, discussed below on line 85: it conflicts with other styles, so if you activate the package you also need to modify the bibliography style.

\usepackage[english,francais]{babel}
\usepackage{amssymb,url,xspace,smfthm}

%%%%%%%%%%%%%%%%%%%%%%
\usepackage{amsmath, amssymb, amsthm, amsbsy}
\usepackage{mathrsfs}
\usepackage{tikz-cd}
\usepackage{enumitem}
\usepackage{color}  
\usepackage[colorlinks,
            linkcolor=blue,      
            anchorcolor=blue,  %%修改此处为你想要的颜色
            citecolor=blue,        %%修改此处为你想要的颜色,例如修改blue为red
            ]{hyperref}
\usepackage[capitalize]{cleveref}
\usepackage{mathtools}

\newtheorem{theorem}{Theorem}
\newtheorem{proposition}{Proposition}[section]
\newtheorem{lemma}{Lemma}[section]
\newtheorem{corollary}{Corollary}[section]

\theoremstyle{definition}
\newtheorem{definition}{Definition}
\newtheorem{exercise}{Exercise}[section]
\newtheorem{example}{Example}
\allowdisplaybreaks[4]
\theoremstyle{remark}
\newtheorem*{remark}{Remark}
\newenvironment{myproof}{\begin{proof}[\indent\it Proof]}{\end{proof}}
\newcommand*{\dif}{\mathop{}\!\mathrm{d}}
\newcommand{\cnum}[1]{$#1^\circ$} %右上角带圆圈的编号
%%%%%%%%%%%%%%%%%%%%%%


\tolerance 400

\pretolerance 200

%%%Below here are the main information about your paper: DO NOT MODIFY THEIR ORDER.
\title[FakeModularForm]{HAhaHAHAAH}%As you see, in square brackets goes a shorter title intended to appear at top of odd pages
%\date {January 1, 1000}

\author[A.T.]{A. U. Thor}%In square brackets goes a shorter name intended to appear at top of even pages
%\address{Paradise, Universe}
%\email{someone@some.where} 
%\urladdr{http://one.two.three}
%\thanks{I wish to thank many people}
%\keywords{Good News, Santa Claus}

\begin{document}

\begin{abstract}
This is a template for the class to be used when publishing in a review from the Soci\'et\'e Ma\-th\'e\-Ma\-ti\-que de France
\end{abstract}


%\begin{altabstract}
%Ici un r\'esum\'e alternatif (meaning: here you put an alternative abstract, probably in French).
%\end{altabstract}
\maketitle

\tableofcontents

\section{Introduction}
Here goes some text that you can replace with yours. But in what follows a couple of hints, should you need them, are provided.

This Overlaf project contains, a part from two classes (file ending with \texttt{.cls}) , two packages (file ending with \texttt{.sty}) and two bibliographical styles (file ending with \texttt{.bst}), also two .tex files (one in English, one in French) which contain very useful information. Below we describe some parts of it, but you are encouraged to read them anyhow.

\subsection{About the numbering}
By default, theorems and theorem-like statements are numbered according to the section in which they appear:
\begin{theo} You see what I mean.
\end{theo}
Some theoremlike environments are defined. They use one and the same counter.
\par\nobreak
\begin{center}\begin{tabular}{lccc}
\noalign{\hrule height .08em\vskip.65ex}
Style & {\tt Macro} \LaTeX & Nom fran\c{c}ais & English name\quad \\
\noalign{\vskip .4ex \hrule height 0.05em\vskip.65ex}
\it plain & \tt theo & Th\'eor\`eme & \it Theorem \\
 & \tt prop & Proposition & \it Proposition \\
 & \tt conj & Conjecture & \it Conjecture \\
 & \tt coro & Corollaire & \it Corollary \\
 & \tt lemm & Lemme & \it Lemma \\
\noalign{\vskip .4ex \hrule height 0.05em\vskip.65ex}
\it definition & \tt defi & D\'efinition & \it Definition \\
\noalign{\vskip .4ex \hrule height 0.05em\vskip.65ex}
\it remark & \tt rema & Remarque & \it Remark \\
 & \tt exem & Exemple & \it Example \\
\noalign{\vskip .4ex \hrule height 0.08em\vskip.65ex}
\end{tabular}\end{center}
The way of numbering the statements is described in detail in the \texttt{$\ast$-doc.tex} file.

\subsubsection{Generic statement}

This is a \verb|\subsubsection|, and you see the difference. 
\begin{enonce}{Question}
Do you see the difference?
\end{enonce}
The above nice statement ``which looks like a Theorem although is a Question'' can be introduced by using \texttt{enonce}: look in the code to see the example, and in the \texttt{$\ast$-doc.tex} file for more details.

\section{And now: the bibliography!}
We have been very happy to read \cite{GMP81} as well as \cite[Theorem~2.5]{BMM94} or possibly the series of two papers \cite{GS,GS82}.

The references are included in the \texttt{bibtemplate} file: you can modify it keeping the same formatting.\\
The references are included in the \texttt{bibtemplate} file: you can modify it keeping the same formatting.\\
The references are included in the \texttt{bibtemplate} file: you can modify it keeping the same formatting.\\
The references are included in the \texttt{bibtemplate} file: you can modify it keeping the same formatting.\\
The references are included in the \texttt{bibtemplate} file: you can modify it keeping the same formatting.
The references are included in the \texttt{bibtemplate} file: you can modify it keeping the same formatting.
The references are included in the \texttt{bibtemplate} file: you can modify it keeping the same formatting.
The references are included in the \texttt{bibtemplate} file: you can modify it keeping the same formatting.
The references are included in the \texttt{bibtemplate} file: you can modify it keeping the same formatting.

\bibliography{bibtemplate}%This it the name of the file where you put all your entries
\bibliographystyle{smfalpha} %This is the style of your bibliography: you choose it among smfalpha, smfplain or—if publishing for the Revue d'Histoire des Mathématiques—rhmunsrtnat. When changing styles, you might get some errors by the server: select "recompile from scratch" a couple of times before getting mad.

\end{document}