\documentclass{article}
\usepackage{fontspec}
\usepackage{amsmath, amssymb, amsthm, amsbsy, mathrsfs}
\usepackage{stmaryrd}
\usepackage{enumerate}
\usepackage[colorlinks,
linkcolor=cyan,
anchorcolor=blue,
citecolor=blue,
]{hyperref}
\usepackage[capitalize]{cleveref}
\usepackage[margin = 1in, headheight = 12pt]{geometry}
\usepackage{bbm}
\usepackage{tikz-cd}

\setmainfont{Arial}
\linespread{1.2}

\theoremstyle{definition}
\newtheorem{theorem}{Theorem}

\newtheorem{definition}{Definition}
\newtheorem{exercise}{Exercise}[section]
\newtheorem{problem}{Problem}
\newtheorem{example}{Example}
\newtheorem{proposition}{Proposition}[section]
\newtheorem{lemma}{Lemma}[section]
\newtheorem{corollary}{Corollary}[section]

\theoremstyle{remark}
\newtheorem*{remark}{Remark}

\renewcommand{\Re}{\mathop{\mathrm{Re}}}
\renewcommand{\Im}{\mathop{\mathrm{Im}}}

% 新命令
% 数学对象
    \newcommand{\R}{\mathbb{R}}
    \newcommand{\C}{\mathbb{C}}
    \newcommand{\Q}{\mathbb{Q}}
    \newcommand{\Z}{\mathbb{Z}}
    \DeclareMathOperator{\GL}{GL}
    \DeclareMathOperator{\SL}{SL}
    \newcommand{\p}{\mathfrak{p}}
    \renewcommand{\P}{\mathbb{P}}
    \newcommand{\A}{\mathbb{A}}
% 集合
    \newcommand{\sminus}{\smallsetminus} %(集合)差
% 范畴
    \newcommand{\op}[1]{{#1}^{\mathrm{op}}} %反范畴
    \DeclareMathOperator{\enom}{End} %自态射
    \DeclareMathOperator{\isom}{Isom} %同构
    \DeclareMathOperator{\aut}{Aut} %自同构
    \DeclareMathOperator{\im}{im} %像
    \newcommand{\Set}{\mathbf{Set}} %集合范畴
    \newcommand{\Abel}{\mathbf{Ab}} %群范畴
    \newcommand{\Ring}{\mathbf{Ring}}
    \newcommand{\Cring}{\mathbf{CRing}}
    \newcommand{\Alg}{\mathbf{Alg}}
    \newcommand{\Mod}{\mathbf{Mod}}
    \DeclareMathOperator{\Id}{id}
%向量空间, 矩阵
    \DeclareMathOperator{\rank}{rank} %秩
    \DeclareMathOperator{\tr}{Tr} %迹
    \newcommand{\tran}[1]{{#1}^{\mathrm{T}}} %转置
    \newcommand{\ctran}[1]{{#1}^{\dagger}} %共轭转置
    \newcommand{\itran}[1]{{#1}^{-\mathrm{T}}} %逆转置
    \newcommand{\ictran}[1]{{#1}^{-\dagger}} %逆共轭转置
    \DeclareMathOperator{\codim}{codim} %余维数
    \DeclareMathOperator{\diag}{diag} %对角阵
    \newcommand{\norm}[1]{\left\| #1\right\|} %范数
    \DeclareMathOperator{\lspan}{span} %张成
    \DeclareMathOperator{\sym}{\mathfrak{Y}}
% 群
    \DeclareMathOperator{\inn}{Inn} %(群)内自同构
    \newcommand{\nsg}{\vartriangleleft} %正规子群
    \newcommand{\gsn}{\vartriangleright} %正规子群
    \DeclareMathOperator{\ord}{ord} %元素的阶
    \DeclareMathOperator{\stab}{Stab} %稳定化子
    \DeclareMathOperator{\sgn}{sgn} %符号函数
% 环, 域
    \DeclareMathOperator{\cha}{char} %特征
    \DeclareMathOperator{\spec}{Spec} %素谱
    \DeclareMathOperator{\maxspec}{MaxSpec} %极大谱
    \DeclareMathOperator{\gal}{Gal}
% 微积分
    % \newcommand*{\dif}{\mathop{}\!\mathrm{d}} %(外)微分算子
% 流形
    \DeclareMathOperator{\lie}{Lie}
%代数几何
    \DeclareMathOperator{\proj}{Proj}
%多项式
    \DeclareMathOperator{\disc}{disc} %判别式
    \DeclareMathOperator{\res}{res} %结式

% 结构简写
    \newcommand{\pdfrac}[2]{\dfrac{\partial #1}{\partial #2}} %偏微分式
    \newcommand{\isomto}{\stackrel{\sim}{\rightarrow}} %有向同构
    \newcommand{\gene}[1]{\left\langle #1 \right\rangle} %生成对象
% 文字缩写
    \newcommand{\opin}{\;\mathrm{open\;in}\;}
    \newcommand{\st}{\;\mathrm{s.t.}\;}
    \newcommand{\ie}{\;\mathrm{i.e.,}\;}

% 重定义命令
\renewcommand{\hom}{\mathop{Hom}}
\renewcommand{\vec}{\boldsymbol}
\renewcommand{\and}{\;\text{and}\;}

% 编号
\newcommand{\cnum}[1]{$#1^\circ$} %右上角带圆圈的编号
\newcommand{\rmnum}[1]{\romannumeral #1}


\newcommand{\myit}{$\diamond$}

\title{Something Something}
\author{Fmoc}

\begin{document}
\maketitle
There are something I should have learnt back in my first two years as an undergraduate.

\section{Polynomials}

\subsection{Resultant and Discriminant}
Let $K$ be a field.
We want to know when are two polynomials $f, g\in K[X]$ coprime.
\begin{lemma}\label{not coprime over field iff fu = gv}
    $(f, g)\ne 1\iff \exists u, v\in K[X]\setminus \{0\}$ s.t. \(\begin{cases}
        fu = gv,\\ 
        \deg u <\deg g,\ \deg v < \deg f.
    \end{cases}\)
\end{lemma}
\begin{proof}
    If $(f, g) \ne 1$, then put $u = g/(f, g)$, $v = f/(f, g)$.

    If $(f, g) = 1$ and $fu = gv$, then $u\mid g$, $v\mid f$, so $g/u = f/v$ divides $(f, g) = 1$,
    meaning $u = g, v = f$.
\end{proof}

Now assume $fu = gv$ for some $u, v\in K[X]$ with $\deg u  < \deg g, \deg v < \deg f$.
\cref{not coprime over field iff fu = gv} shows that, $(f, g)\ne 1$ iff $fu = gv$ has nonzero solution. This is a linear equation in the $K$-vector space $K\oplus KX\oplus\dots\oplus KX^{m+n-1}$, and it has a nonzero solution iff and only if the discriminant is zero.
\begin{definition}
    Let $A$ be a commutative ring, $f, g\in A[X]$.
    We define the \textbf{resultant} of $f = \sum_{i=0}^{n}a_iX^i$ and $g = \sum_{j = 0}^m b_jX^j$ to be\footnote{Of course, we require $\deg f = n$ and $\deg g = m$.} \[\res_X(f, g) := 
\begin{vmatrix}
    m\left\{\begin{matrix}
        a_n & \cdots & a_0 &  &  &  & \\
         & a_n & \cdots &a_0 &  &  & \\ 
         & & & \ddots & & & \\ 
         & &  &  & a_n & \cdots & a_0
    \end{matrix}\right.\\
    n\left\{\begin{matrix}
        b_m & \cdots & b_0 &  &  &  & \\
         & b_m & \cdots &b_0 &  &  & \\ 
         & & & \ddots & & & \\ 
         & &  &  & b_m & \cdots & b_0
    \end{matrix}\right.
\end{vmatrix},\]
    a determinant of an $(n + m)\times (n + m)$-matrix over $A$.
\end{definition}
So we can rephrase \cref{not coprime over field iff fu = gv} into:
$f, g\in K[X]$ are coprime if and only if their resultant $\res_X(f, g) \ne 0$.

Now assume that both $f$ and $g$ split in $K$.
Then $(f, g)\ne 1\iff f$ and $g$ share at least one same root. This suggests that $\res_X(f, g)$ should be divided by \textit{all} $x - y$, where $x$ is a root of $f$ and $y$ is a root of $g$; multiplicity are considered here.
\begin{theorem}\label{resultant and roots}
    If $f = \sum_{i=0}^{n}a_iX^i = \prod_{i=1}^n(X-x_i)$ and $g = \sum_{j = 0}^m b_jX^j = \prod_{j = 1}^m (X - y_j)$,
    are polynomials that splits in $K$,
    then \[\res_X(f, g) = a_n^mb_m^n\prod_{i=1}^n\prod_{j=1}^m (x_i - y_j).\]
\end{theorem}

In particular, we can study if a polynomial has multiple roots (in its splitting field) using resultant.
\begin{definition}
    Let $A$ be a commutative ring and $f(X) = a_nX^n + \dots + a_0\in A[X]$.
    The \textbf{discriminant} of $f$ is\[\disc(f) := \frac{(-1)^{\frac{1}{2}n(n-1)}}{a_n}\res_X(f, f')\in A,\]
    where $f'(X) = na_nX^{n - 1} + \dots + a_1$ is the derivative of $f$.

    Note that $\res_X(f, f')$ is a multiple of $a_n$,
    because its first column is $^t(a_n\ 0\ \cdots\ 0\ na_n\ 0\ \cdots\ 0)$, and we require $a_n\ne 0$. Thus $\disc(f)$ is well-defined.
\end{definition}
So $f$ has multiple roots iff $\disc(f) = 0$.
\begin{example}
    \begin{enumerate}
        \item [(1)] If $f(X) = aX^2 + bX + c$,
        then $\disc(f) = -\dfrac{\res_X(f, f')}{a} =b^2 - 4ac$.
        \item [(2)] If $f(X) = X^3 + pX + q$,
        then $\disc(f) = -\res_X(f, f') = -(4p^3 + 27q^2)$.
    \end{enumerate}
    \end{example}
    
\begin{proposition}\label{discriminant and roots}
    Let $f(X) = a_nX^n + \dots + a_0\in K[X]$,
    then \[\disc(f) = a_n^{2n-2}\prod_{1\le i < j\le n}(x_i - x_j)^2,\]
    where $x_1, \dots, x_n$ are all the roots of $f$ in a fixed splitting field with multiplicity counted.
\end{proposition}
\begin{proof}
    By \cref{resultant and roots},
    \[\res_X(f, g) = a_n^m\prod_{i = 1}^n g(x_i).\]
    Use this to compute.
\end{proof}


\end{document}